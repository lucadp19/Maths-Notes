\section{Divisibilità nei domini}

Considereremo $A$ dominio di integrità per il resto della sezione.

\begin{definition}
    [Divisione esatta]
    Siano $a, b \in A$ con $a \neq 0$. Si dice che $a$ \strong{divide} $b$ (e lo si indica con $a \divides b$) se esiste $c \in A$ tale che $b = ac$.
\end{definition}

Questa definizione può anche essere data in termini di ideali: $a \divides b$ è equivalente a $\ideal{b} \subseteq \ideal{a}$. Infatti $a \divides b$ significa che $b = ac$ per qualche $c \in A$, da cui $b \in \ideal{a}$. Ma allora per ogni $x \in A$ vale che $xb \in \ideal{a}$ (per la proprietà di assorbimento di $\ideal{a}$) e quindi tutto l'ideale generato da $b$ deve essere incluso nell'ideale generato da $a$.

\begin{definition}
    [Elementi associati]
    Due elementi non nulli $a, b \in A$ si dicono \strong{associati} (e si scrive $a \sim b$) se esiste un elemento $u \in \units A$ tale che $a = ub$.
\end{definition}

Osserviamo che la relazione di associazione tra elementi di un dominio è una relazione di equivalenza: infatti 
\begin{itemize}
    \item $a = 1 \cdot a$, da cui $a \sim a$;
    \item se $a = ub$ con $u \in \units A$ allora $b = u\inv a$, ovvero $b \sim a$;
    \item se $a \sim b$ e $b \sim c$ (ovvero se esistono $x, y \in \units A$ tali che $a = xb$ e $b = yc$) allora $a = xyc$ e $xy \in \units{A}$, da cui $a \sim c$.
\end{itemize}

\begin{proposition}
    [Caratterizzazione degli elementi associati]
    \label{prop:caratt_associati}
    Siano $a, b \in A \setminus \set{0}$. Le seguenti affermazioni sono equivalenti.
    \begin{enumerate}[label=({\roman*})]
        \item $a, b$ sono associati.
        \item $a \divides b$ e $b \divides a$.
        \item $\ideal{a} = \ideal{b}$.
    \end{enumerate}
\end{proposition}
\begin{proof}
    Mostriamo la catena di implicazioni \[
        (i) \implies (ii) \implies (iii) \implies (i).    
    \]
    \begin{description}
        \item[$(i) \implies (ii)$] Sicuramente $b \divides a$ in quanto $a = ub$. Inoltre moltiplicando entrambi i membri per l'inverso di $u$ (che esiste poiché $u \in \units A$) segue che $b = u\inv a$, ovvero $a \divides b$.
        \item[$(ii) \implies (iii)$] Abbiamo mostrato sopra che la divisibilità equivale all'inclusione tra ideali, ovvero \[
            a \divides b \implies \ideal{a} \subseteq \ideal{b}, \qquad b \divides a \implies \ideal{b} \divides \ideal{a},    
        \] da cui $\ideal{a} = \ideal{b}$.
        \item[$(iii) \implies (i)$] Siccome $\ideal{a} = \ideal{b}$ segue che $a \in \ideal{b}$ e $b \in \ideal{a}$. Dalla prima uguaglianza otteniamo che esiste $x \in A$ tale che $a = xb$, mentre dalla seconda otteniamo che esiste $y \in A$ tale che $b = ya$. Sostituendo questa uguaglianza nella prima si ottiene che \[
            a = xya \;\implies\; xy = 1,
        \] da cui in particolare $x \in \units{A}$ e quindi $a \sim b$.
    \end{description}
\end{proof}

Possiamo quindi estendere il concetto di massimo comun divisore a domini generici.
\begin{definition}
    [Massimo comun divisore]
    Siano $a, b \in A$ non entrambi nulli. Si dice che $d \in A$ è un \strong{massimo comun divisore} per $a$ e $b$ se \begin{enumerate}[label={(\roman*)}]
        \item $d \divides a$ e $d \divides b$,
        \item per ogni $x \in A$, se $x \divides a$ e $x \divides b$ allora $x \divides d$.
    \end{enumerate}
\end{definition}

Notiamo che in genere il massimo comun divisore non è unico, tuttavia se $d$ e $d'$ sono due massimi comuni divisori di $a$ e $b$, allora $d \sim d'$.

\begin{definition}
    [Elementi primi ed irriducibili]
    Sia $x \in A$, $x$ non invertibile e non nullo. \begin{itemize}
        \item $x$ si dice \strong{primo} se per ogni $a, b \in A$ vale che \[
            x \divides ab \implies x \divides a \text{ oppure } x \divides b. 
        \]
        \item $x$ si dice \strong{irriducibile} se per ogni $a, b \in A$ vale che \[
            x = ab \implies a \in \units{A} \text{ oppure } b \in \units{A}.    
        \]
    \end{itemize}
\end{definition}

Come nel caso dei numeri interi vale che ogni elemento primo è irriducibile, tuttavia non vale necessariamente il viceversa.

\begin{proposition}
    [Relazione tra elementi e ideali]
    \label{prop:elems_vs_ideals}
    Sia $x \in A$ non invertibile e non nullo. Valgono le seguenti affermazioni.
    \begin{enumerate}[label={(\roman*)}]
        \item $x$ è primo se e solo se $\ideal{x}$ è un ideale primo (non nullo).
        \item $x$ è irriducibile se e solo se $\ideal{x}$ è massimale nell'insieme degli ideali principali.
    \end{enumerate}
\end{proposition}
\begin{proof}
    La prima proposizione è ovvia, dunque dimostriamo entrambi i versi dell'implicazione.

    \begin{description}
        \item[($\implies$)] Supponiamo che $x$ sia irriducibile e sia $y \in A$ tale che $\ideal{x} \subseteq \ideal{y} \subsetneq A$. Allora esiste $z \in A$ tale che $x = yz$; inoltre necessariamente $y \notin \units{A}$, altrimenti l'ideale generato da $y$ sarebbe tutto l'anello $A$.
        
        Tuttavia $x$ è irriducibile, dunque uno tra $z$ e $y$ deve essere invertibile, ma per l'osservazione appena sopra sappiamo che $y \notin \units{A}$, dunque $z$ è invertibile. Da questo segue che $x \sim y$, da cui per la \autoref{prop:caratt_associati} $\ideal{x} = \ideal{y}$, ovvero $\ideal{x}$ è massimale tra gli ideali principali.
        \item[($\impliedby$)] Supponiamo che $x$ sia riducibile, ovvero $x = yz$ per qualche $y, z \in A$ entrambi non invertibili. Allora \[
            \ideal{x} \subsetneq \ideal{y} \subsetneq A,    
        \] dove il primo $\subsetneq$ viene dal fatto che $z$ non è invertibile (poiché se gli ideali fossero uguali allora $z \in \units{A}$), mentre il secondo viene dal fatto che $y$ non è invertibile.
    \end{description}
\end{proof}