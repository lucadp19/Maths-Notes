\section{Azioni di gruppo}

\begin{definition}
    {Azione di un gruppo su un insieme}{} Sia $G$ un gruppo e $X$ un insieme qualunque. Si dice \strong{azione di $G$ su $X$} un omomorfismo di gruppi \begin{align*}
        \phi : G &\to \Sym{X}\\
        g &\mapsto \phi_g.
    \end{align*}
\end{definition}

Altre notazioni che useremo per la permutazione degli elementi di $X$ definita da $g$ sono $g \cdot x$ e $x^g$.

\begin{example}
    Se $X = G$ un possibile esempio è dato dal coniugio per $g$: l'applicazione $g \mapsto \phi_g$ dove $\phi_g(x) = gxg\inv$ è un omomorfismo tra il gruppo $G$ e il gruppo delle permutazioni degli elementi di $G$, dunque è un'azione di $G$ su $G$.
\end{example}
\begin{example}
    Sia $V$ un $\K$-spazio vettoriale. Allora l'applicazione \begin{align*}
        \phi : \units{\K} &\to \Sym{V}\\
        \lambda &\mapsto \begin{aligned}[t]
            \phi_\lambda : V &\to V\\
            \vec v &\mapsto \lambda\vec v
        \end{aligned}
    \end{align*} è un'azione del gruppo moltiplicativo degli scalari $\units{\K}$ sullo spazio vettoriale. Più in generale, potremmo definire uno spazio vettoriale come un gruppo abeliano additivo su cui è definita un'azione di $\units{\K}$ su $V$.
\end{example}

\subsection{Orbite e stabilizzatori}

Sia $\phi : G \to \Sym{X}$ un'azione di gruppo. $\phi$ definisce su $X$ la seguente relazione: \begin{equation}\label{eq:eq_rel_group_action}
    x \sim y \iff \exists g \in G \text{ tale che } \phi_g(x) = y.    
\end{equation}

\begin{proposition}{}{}
    La relazione definita da un'azione di gruppo è una relazione di equivalenza.
\end{proposition}
\begin{proof}
    Sia $G$ un gruppo, $X$ l'insieme su cui $G$ agisce. Mostriamo che la relazione $\sim$ definita nella \eqref{eq:eq_rel_group_action} è una relazione di equivalenza.
    \newthought{Riflessività} Sia $x \in X$. Siccome $\phi$ è un omomorfismo di gruppi segue che $\phi(e_G) = \phi_e = \id$, da cui \[
        \phi_e(x) = \id(x) = x.    
    \]
    \newthought{Simmetria} Siano $x, y \in X$ tali che $x \sim y$, ovvero $\phi_g(x) = y$ per qualche $g \in G$. Mostriamo che $\phi_{g\inv}(y) = x$: applicando $\phi_{g\inv}$ ad entrambi i membri otteniamo \begin{align*}
        \phi_{g\inv}(y) &= \phi_{g\inv}\parens[\big]{\phi_g(x)} \\
        &= (\phi_{g\inv} \circ \phi_g)(x)\\
        &= (\phi(g\inv) \circ \phi(g))(x)\\
        &= (\phi(g)\inv \circ \phi(g))(x)\\
        &= x,
    \end{align*} da cui segue $y \sim x$.
    \newthought{Transitività} Siano $x, y, z \in X$ tali che $x \sim y$ e $y \sim z$, ovvero $\phi_g(x) = y$ e $\phi_h(y) = z$ per qualche $g, h \in G$. Allora vale che \begin{align*}
        z &= \phi_h\parens[\big]{\phi_g(x)}\\
        &= (\phi_{h} \circ \phi_g)(x)\\
        &= (\phi(h) \circ \phi(g))(x)\\
        &= \phi(hg)(x)\\
        &= \phi_{hg}(x),
    \end{align*} da cui segue che $x \sim z$.
\end{proof}

\begin{remark}
    Notiamo che siccome $\phi$ è un omomorfismo di gruppi, se $\phi_g$ e $\phi_h$ sono le azioni di $g$ e $h$ sull'insieme $X$, allora la loro composizione sarà l'azione \[
        \phi_g \circ \phi_h = \phi(g) \circ \phi(h) = \phi(gh) = \phi_{gh}.    
    \] Invece, data l'azione $\phi_g$ di $g$ su $X$, segue che la sua inversa è $\phi_{g\inv}$: \begin{gather*}
        \phi_{g\inv} \circ \phi_g = \phi(g\inv) \circ \phi(g) = \phi(g)\inv \circ \phi(g) = \id.\\   
        \phi_g \circ \phi_{g\inv} = \phi(g) \circ \phi(g\inv) = \phi(g) \circ \phi(g)\inv = \id.\\ 
    \end{gather*}
\end{remark}

Introduciamo ora i concetti fondamentali di \strong{orbita} e \strong{stabilizzatore}.

\begin{definition}{Orbita}{}
    Sia $G$ un gruppo che agisce sull'insieme $X$. Dato $x \in X$ si dice \strong{orbita di $x$} l'insieme \[
        \orb{x} \deq \set*{\phi_g(x) \given g \in G} \subseteq X.    
    \]
\end{definition}

\begin{remark}
    L'orbita di $x$ è esattamente la classe di equivalenza data dalla relazione di equivalenza definita in \eqref{eq:eq_rel_group_action}. In particolare se $R$ è un insieme di rappresentanti vale che \[
        X = \bigsqcup_{x \in R} \orb{x}.    
    \]
\end{remark}

\begin{definition}{Stabilizzatore}{}
    Sia $G$ un gruppo che agisce sull'insieme $X$. Dato $x \in X$ si dice \emph{stabilizzatore di $x$} l'insieme \[
        \Stab_{G}{x} \deq \set*{g \in G \given \phi_g(x) = x} \subseteq G.    
    \]
\end{definition}

\begin{proposition}
    {Lo stabilizzatore è un sottogruppo}{}
    Sia $G$ un gruppo che agisce sull'insieme $X$; sia inoltre $x \in X$. Allora vale che \[
        \Stab_{G}{x} \sgr G.    
    \]
\end{proposition}
\begin{proof}
    Innanzitutto $e_G \in \Stab_{G}{x}$ in quanto $\phi_e(x) = x$ (l'azione dell'identità è sempre l'identità).

    \newthought{Chiusura per inversi} Supponiamo che $g \in \Stab_{G}{x}$, ovvero $\phi_g(x) = x$: mostriamo che anche $g\inv \in \Stab_{G}{x}$, ovvero $\phi_{g\inv} \in \Stab_{G}{x}$. Applichiamo ad entrambi i membri l'azione $(\phi_g)\inv$, ottenendo \begin{align*}
        (\phi_g)\inv(x) = (\phi_g)\inv\parens[\big]{\phi_g(x)} = x.
    \end{align*} Come abbiamo osservato precedentemente, $(\phi_g)\inv = \phi_{g\inv}$, da cui segue che $x = \phi_{g\inv}(x)$ e quindi $g\inv \in \Stab_{G}{x}$.

    \newthought{Chiusura per prodotti} Supponiamo infine che $g, h \in \Stab_{G}{x}$ e mostriamo che $hg \in \Stab_{G}{x}$. Infatti \begin{align*}
        \phi_{hg}(x) &= (\phi_h \circ \phi_g)(x)\\
        &= \phi_h\parens[\big]{\phi_g(x)}\\
        &= \phi_h(x)\\
        &= x.
    \end{align*}

    Dunque $\Stab_{G}{x}$ è un sottogruppo di $G$.
\end{proof}

Consideriamo un'azione generica $\phi$ di un gruppo $G$ su un insieme $X$: sia $x \in X$ e siano $g, h \in G$ tali che $\phi_g(x) = \phi_h(x)$. Allora 
\begin{align*}
    &\phi_g(x) = \phi_h(x)\\
    \iff &(\phi_{h\inv} \circ \phi_g)(x) = x\\
    \iff &\phi_{h\inv g}(x) = x\\
    \iff &h\inv g = \Stab_{G}{x}\\
    \iff &g\Stab_{G}{x} = h\Stab_{G}{x}.
\end{align*}

Esiste dunque una bigezione tra l'orbita di un elemento $x \in X$ e le classi laterali di $x$ in $G$:
\begin{align*}
    \orb{x} &\biject \quot{G}{\Stab_{G}{x}}\\
    \phi_g(x) &\mapsto g\Stab_{G}{x}.
\end{align*}

Questa corrispondenza è \begin{description}
    \item[ben definita:] se $\phi_g(x) = \phi_h(x)$ allora $g\Stab_{G}{x} = h\Stab_{G}{x}$;
    \item[iniettiva:] se $g\Stab_{G}{x} = h\Stab_{G}{x}$ sicuramente $\phi_g(x) = \phi_h(x)$;
    \item[surgettiva:] le classi laterali di $\Stab_{G}{x}$ sono tutte e solo della forma $g\Stab_{G}{x}$ al variare di $g \in G$, e per ogni $g \in G$ segue che $\phi_g(x) \in \orb{x}$.
\end{description}


Segue quindi la seguente proposizione.
\begin{proposition}{Lemma Orbita-Stabilizzatore}{lem_orb-stab}
    Sia $G$ un gruppo che agisce su un insieme $X$. Se $G$ è finito, allora per ogni $x \in X$ vale che \begin{equation}
        \card{G} = \card{\orb{x}} \cdot \card{\Stab_{G}{x}}.
    \end{equation}
    In particolare quindi $\card{\orb{x}}$ divide $\card{G}$.
\end{proposition}
\begin{proof}
    Per la bigezione mostrata sopra, la cardinalità dell'orbita di $x$ è uguale al numero di classi laterali di $\Stab_{G}{x}$ in $G$, ovvero \[
        \card{\orb{x}} = \GrpIndex{G : \Stab_{G}{x}} = \frac{\card G}{\card{\Stab_{G}{x}}},    
    \] da cui segue la tesi.
\end{proof}

\subsection{Azione di coniugio}
Sia $G$ un gruppo che agisce su se stesso tramite l'azione di coniugio: ovvero \begin{align*}
    \phi : G &\to \Sym{G}\\
           g &\mapsto 
    \begin{aligned}[t]
        \phi_g : G &\to G\\
                 x &\mapsto gxg\inv.
    \end{aligned}
\end{align*}

Abbiamo già osservato che questa è un'azione. Sia ora $x \in G$ qualunque. Allora l'orbita di $x$ è data da
\begin{align*}
    \orb{x} 
    = \set{\phi_g(x) \given g \in G}
    = \set{gxg\inv \given g \in G}
    = \Cl{x},
\end{align*}
dove $\Cl{x}$ rappresenta la classe di coniugio di $x$.

Invece lo stabilizzatore di $x$ in $G$ è:
\begin{align*}
    \Stab_{G}{x} 
    = \set{g \in G \given \phi_g(x) = x}
    = \set{g \in G \given gxg\inv = x}
    = {g \in G \given gx = xg}
    = \Zentr_{G}{x},
\end{align*}
ovvero il centralizzatore di $x$ in $G$.

Per il \nameref{prop:lem_orb-stab}, segue che, se $G$ è finito: \[
    \card{G} = \card{\Cl{x}} \cdot \card{\Zentr_{G}{x}},  
\] ovvero $\card{\Cl{x}} \divides \card{G}$.

Da questo segue un'altra importante proprietà dei gruppi normali.
\begin{proposition}{I gruppi normali sono unione di classi di coniugio}{H_normal_iff_union_conj}
    Sia $G$ un gruppo, $H \sgr G$. Allora $H \normal G$ se e solo se $H$ è unione di intere classi di coniugio.
\end{proposition}
\begin{proof}
    Mostriamo entrambi i versi dell'implicazione.
    \begin{description}
        \item[\boximpl] Se $H \normal G$ allora per ogni $g \in G$ vale che $gHg\inv \subseteq H$, ovvero per ogni $g \in G, h \in H$ vale che $ghg\inv \in H$, ovvero per ogni $h \in H$ vale che $\set{ghg\inv \given g \in G} = \Cl{h} \subseteq H$, ovvero $H$ è unione di intere classi di coniugio.
        \item[\boximplby] Supponiamo $H$ sia un sottogruppo di $G$ dato dall'unione di intere classi di coniugio. Allora per ogni $h \in H$ segue che $\Cl{h} \in H$, ovvero per ogni $g \in G$ vale che $gHg\inv \subseteq H$, cioè $H \normal G$. \qedhere 
    \end{description}
\end{proof}

\subsection{Coniugio di sottogruppi}
Sia $G$ un gruppo e $X$ l'insieme di tutti i suoi sottogruppi. Definiamo la seguente azione di $G$ su $X$:
\begin{align*}
    \phi : G &\to \Sym{X}\\
           g &\mapsto 
    \begin{aligned}[t]
        \phi_g : X &\to X\\
                 H &\mapsto gHg\inv.
    \end{aligned}
\end{align*}

Mostriamo innanzitutto che $\phi$ rappresenta effettivamente un'azione:
\newthought{Omomorfismo} Siano $g, h \in G$. Allora per ogni $H \in X$ vale che \[
    \phi_{gh}(H) = (gh)H(gh)\inv = g(hHh\inv)g\inv = (\phi_g \circ \phi_h)(H).
\]
\newthought{Bigettività} Sia $g \in G$ qualunque. Mostriamo che $\phi_g$ è una bigezione e $\phi_{g\inv}$ è la sua inversa: per ogni $H \in X$ vale che \begin{align*}
    &(\phi_{g\inv} \circ \phi_g)(H) = \phi_{g\inv}(gHg\inv) = g\inv gHg\inv g = H.\\
    &(\phi_{g} \circ \phi_{g\inv})(H) = \phi_{g}(g\inv Hg) = gg\inv Hgg\inv = H.
\end{align*}

Segue quindi che $\phi$ è un'azione di $G$ sui suoi sottogruppi. 

Sia $H \sgr G$. L'orbita di $H$ rispetto a questa azione è \[
    \orb{H} = \set{\phi_g(H) \given g \in G} = \set{gHg\inv \given g \in G},
\] ovvero è l'insieme dei sottogruppi di $G$ coniugati ad $H$. Invece lo stabilizzatore di $H$ è \[
    \Stab_{G}{H} = \set{g \in G \given \phi_g(H) = H}
    = \set{g \in G \given gHg\inv = H} = \Normaliser_{G}{H},
\] ovvero è il normalizzatore del sottogruppo $H$ in $G$.

Osserviamo che, per il \nameref{prop:lem_orb-stab}, il numero di coniugati di $H$ è dato da \[
    \card{\orb{H}} = \frac{\card{G}}{\card{\Normaliser_{G}{H}}}    
\]

\begin{proposition}{Sottogruppi normali e azione di coniugio per sottogruppi}{}
    Sia $G$ un gruppo e $H \sgr G$. 
    Consideriamo l'azione di $G$ sull'insieme dei suoi sottogruppi data dal coniugio.
    Le seguenti affermazioni sono equivalenti:
    \begin{enumerate}[label={(\roman*)}]
        \item $H \normal G$.
        \item $\orb{H} = \set{H}$.
        \item $\Stab_{G}{H} = G$.
    \end{enumerate}
\end{proposition}
\begin{proof}
    Dimostriamo la catena di implicazioni \[
        (i) \implies (ii) \implies (iii) \implies (i).    
    \]
    \begin{description}
        \item[($(i) \implies (ii)$)] Se $H \normal G$ allora $gHg\inv = H$ per ogni $g \in G$, da cui $\orb{H} = \set{H}$.
        \item[($(ii) \implies (iii)$)] Supponiamo che \[
            \orb{H} = \set{gHg\inv \given g \in G} = \set{H}.    
        \] Questo significa che per ogni $g \in G$ vale che $gHg\inv = H$, da cui $\Stab_{G}{H} = G$.
        \item[($(iii) \implies (i)$)] Supponiamo $\Stab_{G}{H} = G$. Allora per ogni $g \in G$ vale che $gHg\inv = H$, da cui $H \normal G$.
    \end{description}
\end{proof}

\subsection{Formula delle classi}
Sia $G$ un gruppo; consideriamo l'azione $\phi$ di $G$ su se stesso data dal coniugio.

Ricordiamo che, dato $x \in G$, la classe di coniugio di $x$ mediante $\phi$ è \[
    \Cl{x} \deq \orb{x} = \set{\phi_g(x) \given g \in G} = \set{gxg\inv \given g \in G}.    
\]

Sicuramente $x \in \orb{x}$ in quanto $x = \phi_{e_G}(x)$; inoltre possiamo notare che $\Cl{x} = \set{x}$ se e solo se per ogni $g \in G$ vale che $gxg\inv = x$, ovvero $x$ è un elemento del centro di $G$.

Più in generale se $G$ è finito vale il \nameref{prop:lem_orb-stab}, da cui $\card{G} = \card{\Cl{x}}\cdot \card{\Zentr_{G}{x}}$. Allora vale che $\Cl{x} = \set{x}$ se e solo se $\card{\Cl{x}} = 1$, da cui $\card{G} = \card{\Zentr_{G}{x}}$, ovvero $G = \Zentr_{G}{x}$ (poiché $G$ è finito), da cui $x \in \Zentr{G}$.

Siccome le classi di coniugio formano le classi di equivalenza della relazione data dall'azione di coniugio, dato un insieme di rappresentanti $R$ segue che \[
    G = \bigdisjunion_{x \in R} \orb{x} = \bigdisjunion_{x \in R} \Cl{x}.
\] Se $G$ è finito, passando alle cardinalità si ottiene \[
    \card{G} = \sum_{x \in R} \card*{\Cl{x}}. 
\] Siccome abbiamo notato prima che gli elementi del centro formano classi di coniugio con un solo elemento possiamo separarle dalle altre, ottenendo
\begin{align*}
    \card{G} &= \sum_{x \in R} \card*{\Cl{x}}\\
    &= \sum_{x \in \Zentr{G}} \card*{\Cl{x}} + \sum_{x \in R \setminus \Zentr{G}} \card*{\Cl{x}}\\
    &= \sum_{x \in \Zentr{G}} 1 + \sum_{x \in R \setminus \Zentr{G}} \frac{\card*{G}}{\card*{\Zentr_{G}{x}}}\\
    &= \card*{\Zentr{G}} + \sum_{x \in R \setminus \Zentr{G}} \frac{\card*{G}}{\card*{\Zentr_{G}{x}}}.
\end{align*}

Vale quindi la seguente formula.
\begin{theorem}
    {Formula delle classi}{}
    Sia $G$ un gruppo finito e sia $R$ un insieme di rappresentanti delle classi di coniugio di $G$.
    Allora \begin{equation}
        \label{eq:class_formula}
        \card*{G} = \card*{\Zentr{G}} + \sum_{x \in R \setminus \Zentr{G}} \frac{\card*{G}}{\card*{\Zentr_{G}{x}}}.
    \end{equation}
\end{theorem}

Osserviamo che la formula delle classi non vale solo per $G$, ma anche per tutti i sottogruppi normali di $G$. Infatti per la \Cref{prop:H_normal_iff_union_conj} segue che \[
    H = \bigunion_{x \in R \inters H} \Cl{x},    
\] dunque se $H$ è finito si ha \begin{align*}
    \card*{H} &= \sum_{x \in R \inters H} \Cl{x}\\
    &= \sum_{x \in \Zentr{G} \inters H} 1 + \sum_{{x \in (R \setminus \Zentr{G}) \inters H}} \Cl{x}\\
    &= \card*{\Zentr{G} \inters H} + \sum_{{x \in (R \setminus \Zentr{G}) \inters H}} \Cl{x}. 
\end{align*}

\section{Alcuni risultati che derivano da azioni di gruppo}

Elenchiamo in questa sezione alcuni risultati importanti che possono essere dimostrati considerando particolari azioni di gruppo.

\subsection{$p$-Gruppi}

\begin{definition}{$p$-gruppo}{}
    Sia $p \in \Z$ primo. Si dice \emph{$p$-gruppo} un gruppo finito di ordine $p^k$ per qualche $k \in \N$.
\end{definition}

Come vedremo in seguito, i $p$-gruppi sono fondamentali nello studio di gruppi più complicati. Vediamo alcune loro proprietà di base.

\begin{proposition}
    {Il centro di un $p$-gruppo è non banale}{p-group_center_is_nontrivial}
    Sia $G$ un $p$-gruppo di ordine $p^n$. Allora $\Zentr{G} \neq \set{e_G}$.
\end{proposition}
\begin{proof}
    Per la formula delle classi vale che \[
        p^n = \card*{G} = \card*{\Zentr{G}} + \sum_{x \in R \setminus \Zentr{G}} \frac{\card*{G}}{\card*{\Zentr_{G}{x}}}.
    \]
    Notiamo che se $x \in R \setminus \Zentr{G}$ allora $\Cl{x} = \dfrac{\card*{G}}{\card*{\Zentr_{G}{x}}} > 1$, in quanto le uniche classi di coniugio formate da un singolo elemento sono date dagli elementi del centro di $G$. Segue quindi che per ogni $x \in R \setminus \Zentr{G}$ vale che \[
        p \divides \frac{\card*{G}}{\card*{\Zentr_{G}{x}}},
    \] da cui $p$ divide la somma di questi rapporti.

    Per differenza segue dunque che $p \divides \card*{\Zentr{G}}$, da cui $\Zentr{G}$ è non banale.
\end{proof}

\begin{proposition}{}{}
    Un gruppo di ordine $p^2$ è necessariamente abeliano.
\end{proposition}
\begin{proof}
    Sia $G$ un gruppo di ordine $p^2$: siccome è un $p$-gruppo per la \Cref{prop:p-group_center_is_nontrivial} il centro di $G$ è non banale, da cui $\Zentr{G}$ ha ordine $p$ o $p^2$.

    Se per assurdo $\Zentr{G}$ avesse ordine $p$ allora $\quot{G}{\Zentr{G}}$ ha ordine $p$, ovvero è ciclico. Tuttavia questo (per la \Cref{prop:G/Z(G)_cyclic=>abelian_G}) implica che $G$ è abeliano, il che è assurdo in quanto abbiamo assunto che il suo centro fosse diverso dall'intero gruppo.

    Segue quindi che $\card*{\Zentr{G}} = p^2$, ovvero $G = \Zentr{G}$ da cui $G$ è abeliano.
\end{proof}

\subsection{Teorema di Cauchy (caso non abeliano)}

\begin{theorem}
    {Teorema di Cauchy}{cauchy_not_abelian}
    Sia $p \in \Z$ primo e $G$ un gruppo finito. Se $p \divides \card{G}$ allora esiste $g \in G$ tale che \[
        \ord_G{g} = p.
    \]
\end{theorem}
\begin{proof}
    Sia $\card{G} = pn.$ Dimostriamo il Teorema per induzione su $n$.

    \newthought{Caso base} Supponiamo $n = 1$: allora $pn = p$, dunque $G \isomorph \Zmod{p}$ e un qualsiasi generatore di $\Zmod{p}$ ha ordine $1$.

    \newthought{Passo induttivo} Supponiamo che la tesi sia vera per tutti i gruppi di ordine $pm$ con $1 \leq m < n$ e mostriamola per un gruppo $G$ di cardinalità $pn$. Dividiamo la dimostrazione in due casi.

    \begin{itemize}
        \item Supponiamo che esista un $H \sgr G$, $H \neq G$ di ordine multiplo di $p$. Allora $\card{H} = pk$ per qualche $k < n$, dunque per ipotesi induttiva esiste $g \in H$ tale che $\ord_H{g} = p$, da cui $\ord_G{g} = p$.
        \item Se invece non esiste alcun sottogruppo di ordine multiplo di $p$ per la \nameref{eq:class_formula} si ha che \[
            pn = \card{G} = \card{\Zentr{G}} + \sum_{x \in R\setminus \Zentr{G}} \frac{\card{G}}{\card{\Zentr_G{x}}}.
        \] Siccome $\Zentr_G{x} \sgr G$ segue che $p \ndivides \card{\Zentr_G{x}}$; inoltre $p \divides G$, dunque segue necessariamente che $p$ divide ogni termine della sommatoria, da cui \[
            p \divides \sum_{x \in R \setminus \Zentr{G}} \frac{\card{G}}{\card{\Zentr_G{x}}}.
        \] Ma allora per differenza $\Zentr{G}$ ha cardinalità multipla di $p$, da cui segue che non può essere un sottogruppo proprio di $G$.

        Segue quindi che $G$ è abeliano, e la tesi segue dal Teorema di Cauchy per gruppi abeliani. \qedhere
    \end{itemize}
\end{proof}

\subsection{Teorema di Cayley}

\begin{theorem}
    {Teorema di Cayley}{cayley}
    Sia $G$ un gruppo. Allora $G$ è isomorfo ad un sottogruppo di un gruppo di permutazioni.

    In particolare se $G$ è un gruppo finito di ordine $n$, $G$ è isomorfo ad un sottogruppo di $\Sym_n$. 
\end{theorem}
\begin{proof}
    Per mostrare che $G$ è isomorfo ad un sottogruppo di un gruppo di permutazioni è sufficiente mostrare che esiste un omomorfismo iniettivo (ovvero un'\emph{immersione}) di $G$ in tale gruppo $\cS$: infatti se $\lambda : G \embeds \cS$ è un'immersione, per il \nameref{th:first_iso} si ha che \[
        \quot{G}{\set*{e_G}} \isomorph G \isomorph \Imm \lambda \sgr \cS.
    \]

    Poniamo dunque $\cS \deq \Sym{G}$ e consideriamo l'azione di $G$ su se stesso per \emph{moltiplicazione a sinistra}: allora \begin{align*}
        \lambda : G &\to \Sym{G}\\
        g &\mapsto (x \mapsto gx)
    \end{align*} è effettivamente un omomorfismo di gruppi iniettivo.

    \newthought{Buona definizione} La funzione $\lambda_g = x \mapsto gx$ è una bigezione (e quindi appartiene a $\Sym{G}$) poiché è iniettiva (per la legge di cancellazione sinistra se $gx = gy$ allora $x = y$) ed è surgettiva (basta osservare che $\lambda_g(g\inv y) = gg\inv y = y$ per ogni $y \in G$).

    \newthought{Omomorfismo} Infatti per ogni $x \in G$ \[
        \lambda(g_1g_2)(x) = g_1g_2x = \lambda(g_1)\parens[\big]{\lambda(g_2)(x)} = \parens[\big]{\lambda(g_1) \circ \lambda(g_2)}(x),
    \] ovvero $\lambda(g_1g_2) = \lambda(g_1) \circ \lambda(g_2)$, come volevamo.

    \newthought{Iniettività} Si ha che \[
        \ker \lambda = \set*{g \in G \given \lambda(g) = \id} = \set*{g \in G \given \lambda(g)(x) = gx = x,\ \forall x \in G} = \set*{e_G},
    \] da cui $\lambda$ è iniettiva.

    Si ha quindi che $G \embeds \Sym{G}$, come volevamo. In particolare se $\card{G} = n$ avremo che $\Sym{G} \isomorph \Sym_n$, da cui il secondo punto.
\end{proof}