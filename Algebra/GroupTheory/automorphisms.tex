\section{Automorfismi di un gruppo}

\begin{definition}
    [Automorfismo] Sia $(G, \cdot)$ un gruppo. Si dice \emph{automorfismo di $G$} un isomorfismo da $G$ in $G$.
    Inoltre si indica con $\Aut{G}$ l'insieme di tutti gli automorfismi di $G$.
\end{definition}

\begin{proposition}
    [Gli automorfismi formano un gruppo] Sia $(G, \cdot)$ un gruppo. Allora $(\Aut{G}, \circ)$ è un gruppo;
    in particolare  $\Aut{G} \sgr \SS(G)$.
\end{proposition}
\begin{proof}
    Innanzitutto l'identità $\id : G \to G$ è un automorfismo di $G$, dunque $\id \in \Aut{G}$.

    Sia $\phi$ un automorfismo di $G$: essendo un isomorfismo, esso ammette un inverso $\phi\inv$. 
    Siccome $\phi\inv$ è ancora un isomorfismo da $G$ in $G$ segue che $\phi\inv$ è un automorfismo di $G$.

    Infine siano $\phi, \psi$ due automorfismi di $G$ : allora la composizione $\phi \circ \psi$ è ancora un automorfismo di $G$.
    Infatti la composizione è ancora un isomorfismo da $G$ in $G$, dunque è un automorfismo.

    Il fatto che $\Aut{G}$ è un sottogruppo di $\SS(G)$ segue banalmente dal fatto che 
    $\Aut{G}$ è contenuto nell'insieme delle bigezioni da $G$ in $G$ 
    insieme con il fatto che $\Aut{G}$ è un gruppo con la stessa operazione di $\SS{G}$.
\end{proof}
