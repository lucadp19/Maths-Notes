\section{Estensioni normali}

Iniziamo questa sezione con un paio di esempi.

\begin{example}
    Consideriamo il polinomio $x^3 - 2$ in $\Q[x]$. Questo polinomio è monico e irriducibile su $\Q$ (poiché di Eisenstein) e pertanto \[
        \mu_{\!\!\sqrt[3]{2}} = x^3 - 2.
    \] Per comodità scriviamo $\alpha \deq \!\sqrt[3]{2}$. Costruiamo allora tutte le possibili estensioni $\phi : \Q(\alpha) \embeds \closure\Q$ tali che $\phi\restrict{\Q} = \id$.
    
    Per quanto visto in precedenza ne esistono tante quante i coniugati di $\alpha$, che sono \[
        \alpha,\; \alpha\zeta_3,\; \alpha\zeta_3^2
    \] dove $\zeta_3$ è ovviamente una radice primitiva terza dell'unità.

    Segue quindi che \[
        \phi\parens[\big]{\Q(\alpha)} = \Q\parens[\big]{\phi(\alpha)} = \begin{cases}
            \Q(\alpha)\\
            \Q(\alpha\zeta_3)\\
            \Q\parens*{\alpha\zeta_3^2}.
        \end{cases}
    \]

    Tali estensioni sono distinte tra loro: ad esempio $\Q(\alpha)$ è un sottocampo di $\R$ mentre le altre due contengono anche numeri complessi immaginari.
\end{example}

\begin{example}
    Sia $p$ un primo e consideriamo il campo $\Q(\zeta_p)$ ottenuto aggiungendo a $\Q$ una radice primitiva $p$-esima dell'unità.

    Il polinomio minimo di $\zeta_p$ su $\Q$ è \[
        \mu_{\zeta_p}(x) = \frac{x^p - 1}{x - 1} = x^{p-1} + \dots + x + 1.
    \] In effetti questo polinomio è irriducibile poiché è traslato di un polinomio di Eisenstein ed è monico, quindi deve essere il polinomio minimo di $\zeta_p$.
    
    I coniugati di $\zeta_p$ sono quindi \[
        \zeta_p, \zeta_p^2, \dots, \zeta_p^{p-1}.
    \] Le estensioni dell'identità a $\Q(\zeta_p)$ dovranno dunue essere della forma \begin{align*}
        \phi_i : \Q(\zeta_p) &\embeds \closure\Q\\
        \Q &\mapsto \Q\\
        \zeta_p &\mapsto \zeta_p^i
    \end{align*} con $i = 1, \dots, p-1$. Osserviamo ora che per ogni $i$ \[
        \phi_i\parens[\big]{\Q(\zeta_p)} = \Q\parens[\big]{\phi_i(\zeta_p)} = \Q(\zeta_p^i) = \Q(\zeta).
    \] In effetti l'inclusione $\Q(\zeta_p) \supseteq \Q(\zeta_p^i)$ è ovvia in quanto in $\Q(\zeta_p)$ c'è l'elemento $\zeta_p^i$ e dunque deve esserci tutto il campo da essa generato. Ma a questo punto possiamo costruire la torre di estensioni \[
        \begin{tikzcd}            
            \Q \arrow[r, dash, "p-1"] \arrow[rr, dash, bend left, "p-1"]
            & \Q\parens[\big]{\zeta_p^i} \arrow[equal, r]
            & \Q(\zeta_p)
        \end{tikzcd}
    \] da cui segue che il grado dell'estensione $\ext{\Q(\zeta_p) / \Q\parens[\big]{\zeta_p^i}}$ è $1$, dunque i due campi devono essere lo stesso campo.
\end{example}

\begin{definition}
    {Estensione normale di campi}{}
    Sia $\ext{\F / \K}$ un'estensione algebrica di campi. $\ext{\F / \K}$ si dice \strong{normale} se per ogni immersione $\phi : \F \embeds \closure\K$ tale che $\phi\restrict{\K} = \id$ si ha \[
        \phi(\F) = \F.
    \]  
\end{definition}

Ad esempio per quanto visto sopra $\ext{\Q(\zeta_p) / \Q}$ è normale, mentre $\ext{\Q\parens*{!\sqrt[3]{2}} / \Q}$ non lo è.

\begin{proposition}
    {Caratterizzazione delle estensioni normali}{caratt_normal_ext}
    Sia $\ext{\F / \K}$ un'estensione algebrica (e finita). I seguenti fatti sono equivalenti.
    \begin{enumerate}
        \item $\ext{\F / \K}$ è normale.
        \item Ogni polinomio $f \in \K[x]$ che ha una radice in $\F$ ha tutte le sue radici in $\F$.
        \item $\F$ è il campo di spezzamento di una famiglia di polinomi di $\K[x]$.  
    \end{enumerate}
\end{proposition}
\begin{remark}
    I primi due punti della \Cref{prop:caratt_normal_ext} valgono anche per le estensioni algebriche infinite, ma noi la useremo e dimostreremo solo nel caso finito.
\end{remark}
\begin{proof}
    Dimostriamo la catena di implicazioni \[
        \text{(1)} \implies \text{(2)} \implies \text{(3)}\implies \text{(1)}.
    \]
    \begin{description}
        \item[\fcolorbox{Black}{White}{(1) $\implies$ (2)}]
        Sia $f \in \K[x]$ e siano $\alpha_1, \dots, \alpha_n \in \closure\K$ le radici di $f$. Supponiamo senza perdita di generalità che $\alpha_1$ appartenga ad $\F$ e dimostriamo che da ciò segue che $\alpha_2, \dots, \alpha_n \in \F$.
        
        Siccome $\alpha_1 \in \F$ certamente $\K(\alpha_1) \subseteq \F$. Allora per ogni $i = 1, \dots, n$ definiamo \begin{align*}
            \phi_i : \K(\alpha_1) &\embeds \K(\alpha_i) \subseteq \closure\K \\
            \alpha_1 &\mapsto \alpha_i
        \end{align*} tale che $\phi_i\restrict{\K} = \id$.
        
        Dato che $\ext{\F / \K}$ è finita a maggior ragione lo sarà $\ext{\F / \K(\alpha_1)}$. Per il \Cref{th:finite_ext_embedding} possiamo allora estendere ogni $\phi_i$ ad un'immersione definita su $\F$: \[
            \tilde\phi_i : \F \embeds \closure\K
        \] tale che $\tilde\phi_i\restrict{\K(\alpha_1)} = \phi_i$, dunque in particolare $\tilde\phi_i\restrict{\K} = \id$.

        Possiamo quindi applicare l'ipotesi che l'estensione $\ext{\F / \K}$ è normale: $\tilde\phi_i(\F) = \F$ per ogni $i$. In particolare dato che $\alpha_1 \in \K(\alpha_1)$ si ha che \[
            \tilde\phi_i(\alpha_1) = \phi_i(\alpha_1) = \alpha_i \in \F,
        \] e dunque $\F$ contiene tutte le radici di $f$. 
        \item[\fcolorbox{Black}{White}{(2) $\implies$ (3)}]
        Consideriamo il campo di spezzamento su $\K$ della famiglia \[
            \FF \deq \set*{\mu_\alpha \given \alpha \in \F},
        \] dove $\mu_\alpha$ indica il polinomio minimo di $\alpha$ su $\K$. Chiamiamo tale campo $\F_0$.
        
        Certamente $\F \subseteq \F_0$: ogni elemento di $\F$ è radice del proprio polinomio minimo, e tale polinomio minimo fa parte della famiglia $\FF$. D'altra parte per definizione \[
            \F_0 = \K\ParensGen*{\beta \given \beta \text{ radice di qualche } \mu_\alpha \in \FF}.
        \] Allora ogni $\beta \in \F_0$ è radice di un polinomio $\mu_\alpha$ che ha almeno una radice in $\F$ (ovvero $\alpha$ stessa, per definizione della famiglia $FF$), dunque per ipotesi $\beta \in \F_0$ e quindi $\F_0 \subseteq \F$.
        
        Segue quindi che $\F = \F_0$ ed è dunque il campo di spezzamento di una famiglia di polinomi di $\K[x]$. 
        \item[\fcolorbox{Black}{White}{(3) $\implies$ (1)}]
        Sia $\F$ il campo di spezzamento su $\K$ di una famiglia $\FF$ di polinomi di $\K[x]$. Siccome l'estensione $\ext{\F / \K}$ è finita, la famiglia è necessariamente finita: \[
            \FF = \set*{f_1, \dots, f_k}.
        \]

        Dunque in particolare siano $\set*{\alpha_{ij} \given j = 1, \dots, n_i}$ le radici del polinomio $f_i$: il campo di spezzamento $\F$ può essere descritto come \[
            \F = \K\ParensGen*{\alpha_{ij} \given i = 1, \dots, k,\; j = 1, \dots, n_i}.
        \]

        Sia allora $\phi : \F \embeds \closure\K$ tale che $\phi\restrict{\K} = \id$ e mostriamo che $\phi(\F) = \F$. 
        
        Osserviamo che per ogni scelta di $i, j$ il polinomio minimo $\mu_{\alpha_{ij}}$ di $\alpha_{ij}$ su $\K$ dovrà dividere $f_i$, poiché $f_i$ è un polinomio che si annulla in $\alpha_{ij}$. Essendo $\phi$ un'immersione di $\ext{\F / \K}$, $\phi$ dovrà mandare $\alpha_{ij}$ in un'altra radice del suo polinomio minimo, che sarà ancora una volta una radice di $f_i$ (poiché il polinomio minimo di $\alpha_{ij}$ divide $f$).
        
        Segue quindi che \[
            \phi(\alpha_{ij}) = \alpha_{i,j'} \in \F
        \] per un qualche $j' \in \set*{1, \dots, n_i}$.
        
        Ma allora $\phi$ manda i generatori di $\F$ in elementi di $\F$, dunque \[
            \phi(\F) \subseteq \F.
        \] Siccome $\phi$ è iniettivo segue che $\phi(\F)$ e $\F$ hanno la stessa dimensione finita su $\K$ come spazi vettoriali, dunque $\ExtDegree{\F : \K} = \ExtDegree{\phi(\F) : \K}$, da cui $\phi(\F) = \F$ e quindi $\ext{\F / \K}$ è un'estensione normale.   \qedhere
    \end{description}
\end{proof}

\begin{proposition}
    {Estensione di grado $2 \implies$ normale}{}
    Sia $\K$ un campo con $\FieldChar \K \neq 2$ e sia $\ext{\F / \K}$ di grado $2$. Allora $\ext{\F / \K}$ è normale.   
\end{proposition}
\begin{proof}
    Sia $\gamma \in \F \setminus \K$. Per quanto dimostrato dalla \Cref{prop:quad_ext=>gen_sqrt} si ha che $\F = \K(\gamma)$. Sia allora \[
        \mu_\gamma(x) = x^2 + b_1x + b_0 \in \K[x]
    \]il polinomio minimo di $\gamma$ su $\K$ e siano $\gamma_1, \gamma_2$ le sue radici.
    
    Per l'\Cref{rem:quad_ext=>sqrt_discr} sappiamo che \[
        \F = \K\parens[\Big]{\sqrt{\Delta}}
    \] dove $\Delta$ è il discriminante dell'equazione \[
        \mu_\gamma(x) = x^2 + b_1x + b_0 = 0, 
    \] ovvero $\Delta \deq b_1^2 - 4b_0$. Per la formula risolutiva delle equazioni di secondo grado allora \[
        \gamma_{1/2} = \frac{-b_1 \pm \sqrt{\Delta}}{2},
    \] ovvero $\gamma_{1/2} \in \K\parens[\Big]{\sqrt{\Delta}} = \F$.
    
    Segue quindi che $\F$ è il campo di spezzamento di $\mu_\gamma$, dunque per la \Cref{prop:caratt_normal_ext} $\ext{\F / \K}$ è normale.  
\end{proof}

\subsection{Proprietà delle estensioni normali}

Studiamo ora come si comportano le estensioni normali rispetto alle solite operazioni che compiamo sui campi, ovvero l'intersezione, il composto e le torri di estensioni.

\paragraph{Composto ed intersezione}

\begin{proposition}
    {}{}
    Dato il diagramma di estensioni di campo \[
        \begin{tikzcd}[every arrow/.append style={dash}]
                & \F\bL & \\
            \F \arrow[ur] & &\bL \arrow[ul] \\
                & \F \inters \bL \arrow[ur] \arrow[ul] &\\
                & \K \arrow[uul, blue, "\text{norm.}"]
                     \arrow[uur, blue, labels=below right,"\text{norm.}"]
                     \arrow[u, Maroon]
                     \arrow[uuu, dashed, Maroon, bend left]
        \end{tikzcd}
    \] dove $\ext{\F / \K}$ e $\ext{\bL / \K}$ (segnate in blu) sono estensioni normali. Allora $\ext{\F\bL / \K}$ e $\ext{\F \inters \bL / \K}$ (segnate in rosso) sono normali.
\end{proposition}
\begin{proof}
    Dimostriamo i due casi separatamente.
    \newthought{Composto} Sia $\phi : \F\bL \embeds \closure\K$ tale che $\phi\restrict{\K} = \id$. Per definizione \[
        \phi(\F\bL) = \phi\parens[\big]{\F(\bL)} = \phi(\F)\parens[\big]{\phi(\bL)}.
    \] Ma $\phi$ è un'immersione che ristretta a $\K$ fa $\id$, dunque per normalità di $\F$ e di $\bL$ si ha che \[
        \phi(\F\bL) = \F\bL,
    \] ovvero $\ext{\F\bL / \K}$ è normale.
    \newthought{Intersezione} Sia $\psi : \F \inters \bL \embeds \closure\K$ tale che $\psi\restrict{\K} = \id$. Sia $\alpha \in \F \inters \bL$: allora $\phi(\alpha)$ è certamente un elemento di $\phi(\F)$ e di $\phi(\bL)$, dunque \[
        \phi(\F \inters \bL) \subseteq \phi(\F) \inters \phi(\bL) = \F \inters \bL.
    \] D'altro canto però $\F \inters \bL$ e $\phi(\F \inters \bL)$ hanno lo stesso grado su $\K$ in quanto $\phi$ è iniettivo.
    
    Segue quindi che $\F \inters \bL = \phi(\F \inters \bL)$, ovvero $\ext{\F \inters \bL / \K}$ è un'estensione normale.
\end{proof}

\paragraph{Torri}

\begin{proposition}
    {}{}
    Siano $\K \embeds \F \embeds \bL$ campi tali che $\ext{\bL / \K}$ è normale. Allora $\ext{\bL / \F}$ è normale.
\end{proposition}

Il relativo diagramma delle estensioni in questo caso è \[
    \begin{tikzcd}[every arrow/.append style={dash}]
        \K \arrow[r] \arrow[rr, bend left, blue, "\text{norm.}"]
        &\F \arrow[r, Maroon]
        &\bL.
    \end{tikzcd}
\]

\begin{proof}
    Sia $\phi : \bL \embeds \closure\K$ tale che $\phi\restrict{\F} = \id$. Allora in particolare $\phi\restrict{\K} = \id$, in quanto $\K \subseteq \F$. Per normalità di $\ext{\bL / \K}$ segue quindi che $\phi(\bL) = \bL$, ovvero $\ext{\bL / \F}$ è normale.    
\end{proof}

\begin{remark}
    Non possiamo dedurre nessuna altra relazione! Ad esempio la torre di estensioni \[
        \begin{tikzcd}[every arrow/.append style={dash}]
            \Q \arrow[r] 
            & \Q\parens*{\sqrt[3]{2}} \arrow[r]
            & \Q\parens*{\sqrt[3]{2}, \zeta_3}
        \end{tikzcd}
    \] è tale che $\ext{\Q\parens*{\sqrt[3]{2}, \zeta_3} / \Q}$ è normale, ma $\ext{\Q\parens*{\sqrt[3]{2}} / \Q}$ non lo è (in quanto contiene una radice di $x^3 - 2 \in \Q[x]$ ma non tutte le sue radici).

    Allo stesso modo la torre \[
        \begin{tikzcd}[every arrow/.append style={dash}]
            \Q \arrow[r, "2"] 
            & \Q\parens*{\sqrt{2}} \arrow[r, "2"]
            & \Q\parens*{\sqrt[4]{2}}
        \end{tikzcd}
    \] è formata da due estensioni di grado $2$ in sequenza, che sono quindi normali, ma $\ext{\Q\parens*{\sqrt[4]{2}}}$ non è normale poiché $\Q\parens*{\sqrt[4]{2}}$ contiene una radice di $x^4 - 2$ ma non tutte.   
\end{remark}