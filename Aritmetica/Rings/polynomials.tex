\section{Anello dei polinomi}

\begin{definition}
    [Polinomi a coefficienti in un anello]
    Sia $(A, +, \cdot)$ un anello commutativo con identità e consideriamo una successione $\left(a_i\right)$ di elementi di $A$ che sia definitivamente nulla, ovvero tale che esista un $n \in \N$ tale che \[
        a_m = 0 \quad \text{per ogni } m > n.    
    \]
    Allora si dice \emph{polinomio nell'indeterminata $X$} la scrittura formale \[
        p = p(X) = \sum_{i = 0}^{\infty} a_iX^i.    
    \] Gli $a_i$ si dicono \emph{coefficienti del polinomio}.

    L'insieme dei polinomi a coefficienti in $A$ si indica con $A[X]$.
\end{definition}

Dato che la successione che definisce il polinomio è definitivamente nulla, possiamo scrivere il polinomio come una sequenza finita di termini: basta prendere i termini fino al massimo indice per cui $a_i$ è diverso da $0$. Diamo però alcune definizioni preliminari.

Innanzitutto d'ora in avanti $(A, +, \cdot)$ è un anello commutativo con identità a meno di ulteriori specifiche.

\begin{definition}
    [Polinomio nullo]
    Si dice \emph{polinomio nullo in $A[X]$} il polinomio definito dalla successione costantemente nulla, e lo si indica come $p(X) = 0_{A[X]}$.
\end{definition}

\begin{definition}
    [Grado di un polinomio]
    Sia $p \in A[X]$, $p(X) \neq 0_{A[X]}$. Allora si dice grado di $p$ il numero \[
        \deg p = \max \set{n \in \N \suchthat a_n \neq 0}.    
    \] Il polinomio $0_{A[X]}$ non ha grado.
\end{definition}

Notiamo che i polinomi di grado $0$ sono tutti e solo della forma $p(X) = a_0$ per qualche $a_0 \in A$; ovvero sono tutte e sole le costanti dell'anello $A$. Possiamo quindi considerare l'anello $A$ come un sottoinsieme dell'insieme dei polinomi $A[X]$.

\begin{definition}
    [Uguaglianza tra polinomi]
    Siano $p, q \in A[X]$. Allora i polinomi $p$ e $q$ sono uguali se e solo se tutti i loro coefficienti sono uguali.
\end{definition}

Definiamo ora le operazioni di somma e prodotto tra polinomi.

\begin{definition}
    [Somma tra polinomi]
    Siano $p, q \in A[X]$.
    Allora definisco l'operazione di somma \begin{align*}
        + : A[X] \times A[X] &\to A[X]\\
        (p, q) &\mapsto p + q
    \end{align*} nel seguente modo: \begin{align*}
        p(X) = \sum_{i = 0}^\infty a_iX^i, \quad q(X) = \sum_{i = 0}^\infty b_iX^i\\
        \implies (p + q)(X) \deq \sum_{i = 0}^\infty (a_i + b_i)X^i.
    \end{align*}
\end{definition}

\begin{definition}
    [Prodotto tra polinomi]
    Siano $p, q \in A[X]$.
    Allora definisco l'operazione di prodotto tra polinomi \begin{align*}
        \cdot : A[X] \times A[X] &\to A[X]\\
        (p, q) &\mapsto p \cdot q
    \end{align*} nel seguente modo: \begin{align*}
        p(X) = \sum_{i = 0}^\infty a_iX^i, \quad q(X) = \sum_{j = 0}^\infty b_jX^j \\
        \implies (p \cdot q)(X) \deq \sum_{i = 0}^\infty \sum_{j = 0}^\infty a_ib_jX^{i+j}.
    \end{align*}
\end{definition}

\begin{theorem}
    [L'insieme dei polinomi è un anello]
    La struttura $(A[X], +, \cdot)$ è un anello commutativo con identità (dove l'identità è il polinomio $1_{A[X]}(X) = 1_A$).
\end{theorem}
\begin{proof}
    Basta verificare tutti gli assiomi degli anelli.
\end{proof}

\begin{proposition}
    [Grado della somma e del prodotto]
    \label{prop:deg_somma_prod_polinomi}
    Siano $p, q \in A[X] \setminus \set{0_{A[X]}}$. Allora vale che \begin{enumerate}[label={(\roman*)}]
        \item $\deg (p + q) \leq \max \set{\deg p, \deg q}$.
        \item se $A$ è un dominio, allora $\deg (pq) = \deg p + \deg q$.
    \end{enumerate}
\end{proposition}
\begin{proof}
    Siano i due polinomi \[
        p(X) = \sum_{i = 0}^\infty a_iX^i, \quad q(X) = \sum_{i = 0}^\infty b_iX^i.   
    \] e siano $n = \deg p$, $m = \deg q$.

    \paragraph{Grado della somma} Sia $k = \max {n, m}$. Allora per ogni $i > k$ varrà che $a_i = b_i = 0$, ovvero $a_i + b_i = 0$, da cui $\deg (p + q) \leq k$.

    \paragraph{Grado del prodotto} Il termine di grado massimo di $(pq)(X)$ deve essere quello in posizione $n + m$. 
    
    Mostriamo che per ogni $i > n$, $j > m$ vale che il coefficiente del termine di grado $i + j$ è uguale a $0$.
    Infatti per definizione di grado segue che $a_i, b_j = 0$ se $i > n$ o $j > m$, dunque il prodotto $a_i \cdot b_j$ sarà $0$, ovvero il coefficiente di grado $i + j$ sarà nullo. Da ciò segue che $\deg (pq) \leq n + m$.

    Inoltre essendo $A$ un dominio il termine $a_nb_m$ deve essere diverso da $0$, in quanto altrimenti uno tra $a_n$ e $b_m$ dovrebbe essere $0$, contro la definizione di grado.

    Dunque $\deg (pq) = \deg p + \deg q$.
\end{proof}

\begin{corollary}
    Se $A$ è un dominio, allora $A[X]$ è un dominio.
\end{corollary}
\begin{proof}
    Siano $p, q \in A[X] \setminus \set{0_{A[X]}}$, con $\deg p = n \geq 0$, $\deg q = m \geq 0$. Allora per la \autoref{prop:deg_somma_prod_polinomi} vale che \[
        \deg (pq) = \deg p + \deg q = n + m \geq 0.    
    \] Dunque il polinomio $(pq)(X)$ non può essere il polinomio nullo (che non ha grado), da cui segue che in $A[X]$ non vi sono divisori dello zero.
\end{proof}

\begin{corollary}
    Se $A$ è un dominio, allora gli invertibili di $A[X]$ sono tutti e soli gli elementi invertibili di $A$, ovvero \[
        \invertible{A[X]} = \invertible{A}. 
    \]
\end{corollary}
\begin{proof}
    Sia $p \in \invertible{A[X]}$ e sia $q \in A[X]$ il suo inverso, ovvero tale che $(pq)(X) = 1_A$.

    Notiamo che $p, q \neq 0_{A[X]}$. Infatti se uno dei due fosse il polinomio nullo per la \autoref{prop:prop_anelli:per_0} il loro prodotto dovrebbe essere il polinomio nullo e non l'unità. Allora esistono $\deg p, \deg q \geq 0$ e vale che \[
        \deg (pq) = \deg p + \deg q \seteq \deg 1 = 0.    
    \]

    Dato che i gradi di $p$ e $q$ sono positivi o nulli, il grado del prodotto è $0$ se e solo se entrambi i polinomi $p$ e $q$ sono di grado zero, ovvero se e solo se sono elementi dell'anello $A$.

    Siano $a, b \in A$ tali che $f(X) = a$e $q(X) = b$. Allora $(pq)(X) = a\cdot b = 1$, ovvero $a$ è invertibile, cioè $a \in \invertible{A}$.
\end{proof}

\subsection{Polinomi a coefficienti in un campo}

In questa sezione studieremo l'anello $\K[X]$, dove $\K$ è un campo generico. Questo anello ha una relazione molto stretta con l'insieme $\Z$ dei numeri interi, soprattutto per quanto riguarda le proprietà di divisibilità.

\begin{theorem}
    [Esistenza e unicità della Divisione Euclidea]
    Siano $f, g \in \K[X]$ con $f(X) \neq 0_{\K[X]}$. Allora esistono e sono unici due polinomi $q, r \in \K[X]$ tali che \[
        g(X) = q(X)f(X) + r(X),
    \] con $r(X) = 0_{\K[X]}$ oppure $0 \leq \deg r \leq \deg f$.
\end{theorem}
\begin{proof}[Dimostrazione dell'esistenza]
    Se $g(X) = 0_{\K[X]}$ allora posso scegliere $q(X) = 0_{\K[X]}$ e $r(X) = q(X) = 0_{\K[X]}$.
    Altrimenti procedo per induzione su $n \deq \deg g$.
    \begin{description}
        \item[Caso base] Supponiamo $\deg g = 0$, ovvero $g(X) = g_0$. Abbiamo due casi: \begin{itemize}
            \item se $\deg f = 0$, ovvero $f(X) = f_0 \in \K$, allora \[
                q(X) = g_0{f_0}\inv, \; r(X) = \bm 0;
            \]
            \item se $\deg f > \deg g$ allora \[
                q(X) = \bm 0, \; r(X) = g(X).    
            \]
        \end{itemize}
        \item[Passo induttivo] Sia $m \deq \deg f$. Come nel caso base, se $\deg f > \deg g$ basta scegliere $q$ uguale al polinomio nullo, $r(X) = g(X)$.
        
        Supponiamo invece che $\deg f \leq \deg g$. Possiamo scrivere i due polinomi come \[
            f(X) = \sum_{i = 0}^m a_iX^i, \; g(X) = \sum_{i = 0}^n b_iX^i.    
        \]
        Sia $g_1 \in \K[X]$ il seguente polinomio: \begin{align*}
            g_1[X] &\deq g(X) - \frac{b_n}{a_m}X^{n-m}f(X)\\  
            &= g(X) - b_nX^n + \dots 
        \end{align*}
        dove i puntini indicano termini di grado inferiore al termine di grado massimo (ovvero $n$).

        Il polinomio $g_1$ ha sicuramente grado inferiore al polinomio $g$, in quanto il termine di grado $n$ (ovvero $b_nX^n$) è stato eliso.

        Segue quindi per ipotesi induttiva che esistono $q_1, r_1 \in \K[X]$ tali che \[
            g_1(X) = q_1(X)f(X) + r_1(X)    
        \] con $r_1 = 0_{\K[X]}$ oppure $0 \leq r_1 \leq \deg f$.

        Dunque possiamo ricavare un'espressione per $g$ dalla definizione di $g_1$:
        \begin{align*}
            g(X) &= g_1(X) + \frac{b_n}{a_m}x^{n-m}f(X)\\
            &= q_1(X)f(X) + r_1(X) + \frac{b_n}{a_m}x^{n-m}f(X)\\
            &= (q_1(X) + \frac{b_n}{a_m}x^{n-m})f(X) + r_1(X).
        \end{align*}
        Dunque scegliendo $q(X) = q_1(X) + \frac{b_n}{a_m}x^{n-m}$ e $r(X) = r_1(X)$ ottengo la divisione euclidea tra $f$ e $g$.
    \end{description}
\end{proof}
\begin{proof}[Dimostraizone dell'unicità]
    Siano $q_1, r_1, q_2, r_2 \in \K[X]$ tali che \begin{align*}
        g(X) = q_1(X)f(X) + r_1(X) = q_2(X)f(X) + r_2(X)
    \end{align*} con $r_1 = 0_{\K[X]}$ oppure $0 \leq \deg r_1 \leq \deg f$, $r_2 = 0_{\K[X]}$ oppure $0 \leq \deg r_2 \leq \deg f$.

    Riarrangiando i termini ottengo \[
        (q_1(X) - q_2(X))f(X) = r_2(X) - r_1(X).    
    \] Se $r_1 = r_2$ segue che $q_1 = q_2$ (per differenza), dunque supponiamo per assurdo $r_1 \neq r_2$.
\end{proof}