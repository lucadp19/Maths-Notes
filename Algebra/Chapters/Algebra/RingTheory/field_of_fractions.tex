\section{Anello delle frazioni}

In questa sezione $A$ sarà un dominio di integrità.

\begin{definition}
    {Parte moltiplicativa}{}
    Sia $S \subseteq A$ un sottoinsieme di $A$ tale che \begin{itemize}
        \item $0 \notin S$,
        \item $1 \in S$,
        \item se $a, b \in S$ allora $ab \in S$.
    \end{itemize}
    $S$ si dice \strong{parte moltiplicativa} di $A$.
\end{definition}

Consideriamo l'insieme $S\inv A$ dato da \[
    S\inv A \deq \quot{A \times S}{\sim},    
\] dove la relazione $\sim$ è definita da $(a, s) \sim (b, t)$ se e solo se $at = bs$.

Mostriamo che $\sim$ è una relazione di equivalenza.
\begin{description}
    \item[Riflessività] Ovviamente $(a, s) \sim (a, s)$.
    \item[Simmetria] Se $(a, s) \sim (b, t)$ allora $at = bs$, ovvero $bs = at$, cioè $(b, t) \sim (a, s)$.
    \item[Transitività] Supponiamo che $(a, s) \sim (b, t)$ e $(b, t) \sim (c, u)$: mostriamo che $(a, s) \sim (c, u)$.
    
    Le due ipotesi ci dicono che $at = bs$ e $bu = tc$; per verificare che $au = cs$ moltiplichiamo entrambi i membri della prima relazione per $u$, ottenendo \[
        aut = bus = cts,
    \] dove la seconda uguaglianza viene dalla seconda relazione. A questo punto raccogliendo $t$ si ottiene che \[
        t(au - cs) = 0,    
    \] dunque siccome $A$ è un dominio dovrà valere che $t = 0$ oppure $au = cs$. Tuttavia $t \in S$, dunque per definizione di parte moltiplicativa $t \neq 0$, da cui la tesi.
\end{description}

Indicheremo $\dfrac{a}{s}$ la classe di equivalenza della coppia $(a, s)$. Vale il seguente risultato.

\begin{proposition}{}{}
    $S\inv A$ con le operazioni definite da \[
        \frac{a}{s} + \frac{b}{t} = \frac{at + bs}{st}, \qquad \frac{a}{s} \cdot \frac{b}{t} = \frac{ab}{st},    
    \] è un anello commutativo con identità.
\end{proposition}
\begin{proof}
    Mostriamo innanzitutto che le operazioni sono ben definite. Siano $\frac{a}{s} = \frac{a'}{s'}$ e $\frac{b}{t} = \frac{b'}{t'}$ elementi di $S\inv A$ e mostriamo che \[
        \frac{a}{s} + \frac{b}{t} = \frac{a'}{s'} + \frac{b'}{t'}, \qquad \frac{a}{s} \cdot \frac{b}{t} = \frac{a'}{s'} \cdot \frac{b'}{t'}.
    \]

    Per definizione di somma vale che \[
        \frac{a}{s} + \frac{b}{t} = \frac{at + bs}{st}, \qquad \frac{a'}{s'} + \frac{b'}{t'} = \frac{a't' + b's'}{s't'};       
    \] queste due frazioni sono uguali se e solo se \begin{align*}
        &(at + bs)s't' = (a't' + b's')st\\
        \iff &att's' + bss't' = a't'ts + b's'st
    \end{align*} e quest'uguaglianza è verificata poiché $as' = a's$ e $bt' = b't$.

    Analoga dimostrazione per la buona definizione del prodotto. Il resto delle verifiche è standard.
\end{proof}

L'anello $S\inv A$ viene chiamato \strong{anello delle frazioni} oppure \strong{localizzato di $A$ ad $S$}.

\begin{proposition}{}{}
    L'anello $A$ si immerge naturalmente in $S\inv A$ tramite l'omomorfismo iniettivo \begin{align*}
        \iota: A &\embeds S\inv A\\
        a &\mapsto \frac{a}{1}.
    \end{align*}
\end{proposition}
\begin{proof}
    Mostriamo prima che $\iota$ è un omomorfismo e poi che è iniettivo. Infatti: \begin{align*}
        \iota(a + b) = \frac{a + b}{1} = \frac{a}{1} + \frac{b}{1} = \iota(a) + \iota(b),\\
        \iota(ab) = \frac{ab}{1} = \frac{a}{1}\cdot \frac{b}{1} = \iota(a) \cdot \iota(b). 
    \end{align*}
    Inoltre siccome $A$ è un dominio vale automaticamente che $\iota(1) = \nicefrac{1}{1}$.

    Inoltre il nucleo di $\iota$ è \[
        \ker \iota = \set*{a \in A \given \frac{a}{1} = \frac{0}{1}} = \set*{a \in A \given a = 0} = \set*{0},    
    \] da cui $\iota$ è iniettivo.
\end{proof}

Osserviamo che se $A$ è un dominio l'insieme $S \deq A \setminus \set{0}$ è una parte moltiplicativa di $A$:
\begin{itemize}
    \item $0 \notin A \setminus \set{0}$,
    \item $1 \in A \setminus \set{0}$,
    \item siccome $A$ è un dominio, se $x, y \in A \setminus \set{0}$ allora anche $xy \in A \setminus \set{0}$.
\end{itemize}

In questo caso chiamiamo \strong{campo dei quozienti} di $A$ il localizzato di $A$ ad $S$ e lo indichiamo con $Q(A)$. Vale la seguente proposizione.
\begin{proposition}{}{}
    Il campo dei quozienti $Q(A)$ di un dominio $A$ è un campo ed in particolare è il più piccolo campo che contiene $A$.
\end{proposition}

\subsection{Ideali di $S\inv A$}

Sia $I \subseteq A$ un ideale di $A$. Definiamo \[
    S\inv I \deq \set*{\frac{x}{s} \in S\inv A \given x \in I, s \in S}.    
\]

\begin{proposition}{}{}
    Sia $A$ un dominio, $I \subseteq A$ un suo ideale e $S$ una parte moltiplicativa di $A$. Valgono le seguenti affermazioni.
    \begin{enumerate}
        \item $S\inv I$ è un ideale di $S\inv A$.
        \item Per ogni $J \subseteq S\inv A$ ideale di $S\inv A$ esiste un ideale $I$ di $A$ tale che $J = S\inv I$.
        \item $S\inv I$ è un ideale proprio di $S\inv A$ se e solo se $S \inters I = \varnothing$.
        \item Sia $P \subseteq A$ un ideale primo di $A$, $S \inters P = \varnothing$. Allora $S\inv P$ è un ideale primo di $S\inv A$.
    \end{enumerate}
\end{proposition}

Osserivamo che i primi due punti ci dicono che gli ideali di $S\inv A$ sono tutti e soli della forma $S\inv I$ al variare di $I$ tra gli ideali di $A$.