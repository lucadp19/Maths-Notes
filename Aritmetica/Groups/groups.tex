\chapter{Gruppi}

\section{Introduzione ai gruppi}

\begin{definition}
    [Gruppo] \label{def:gruppo}
    Sia $G \neq \varnothing$ un insieme e sia $*$ un'operazione su $G$, ovvero \begin{align} \begin{split}
        * : G  \times  G &\to       G  \\
            (a, b)       &\mapsto  a*b.
    \end{split} \end{align}

    Allora la struttura $(G, *)$ si dice \emph{gruppo} se valgono i seguenti assiomi: \begin{enumerate}[label={(G\arabic*)}]
        \item \label{def:gruppo:ass} L'operazione $*$ è \emph{associativa}:
        
        per ogni $a, b, c \in G$ vale che $a * (b * c) = (a * b) * c$.
        \item Esiste un elemento $e_G \in G$ che fa da \emph{elemento neutro} rispetto all'operazione $*$:
        
        per ogni $a \in G$ vale che $a * e_G = e_G * a = a$.
        \item Ogni elemento di $G$ è \emph{invertibile} rispetto all'operazione $*$:
        
        per ogni $a \in G$ esiste $a\inv \in G$ tale che $a * a\inv = a\inv * a = e_G$.
        
        Tale $a\inv$ si dice \emph{inverso di $a$}.
    \end{enumerate}
\end{definition}

\begin{definition}
    [Gruppo abeliano] \label{def:gruppo abeliano}
    Sia  $(G, *)$ un gruppo.
    Allora $(G, *)$ si dice \emph{gruppo abeliano} se vale inoltre \begin{enumerate}[label={(G\arabic*)}, start=4]
        \item l'operazione $*$ è commutativa, ovvero \begin{align*}
            \forall a, b \in G \quad a * b = b * a.
        \end{align*}
    \end{enumerate}
\end{definition}

L'elemento neutro di $G$ si può rappresentare come $e_G$, $\operatorname{id}_G$, $1_G$ o semplicemente $e$ nel caso sia evidente il gruppo a cui appartiene.

Possiamo rappresentare un gruppo in \emph{notazione moltiplicativa}, come abbiamo fatto finora, oppure in \emph{notazione additiva}, spesso usata quando si studiano gruppi abeliani. 

In notazione additiva, ovvero considerando un gruppo $(G, +)$ gli assiomi diventano \begin{enumerate}[label={(G\arabic*)}]
    \item l'operazione $+$ è associativa, ovvero \begin{align*}
        \forall a, b, c \in G. \quad a + (b + c) = (a + b) + c
    \end{align*}
    \item esiste un elemento $e_G \in G$ che fa da elemento neutro rispetto all'operazione $+$: \begin{align*}
        \forall a \in G. \quad a + e_G = e_G + a = a 
    \end{align*}
    \item ogni elemento di $G$ è invertibile rispetto all'operazione $+$: \begin{align*}
        \forall a \in G \;\;
        \exists (-a) \in G.
        \quad a + (-a) = (-a) + a = e_G.
    \end{align*}
    Per semplicità spesso si scrive $a - b$ per intendere $a + (-b)$.
    \item l'operazione $+$ è commutativa, ovvero \begin{align*}
        \forall a, b \in G \quad a + b = b + a.
    \end{align*}
\end{enumerate}

Facciamo alcuni esempi di gruppi.
\begin{example}
    Sono gruppi abeliani $(\Z, +)$ e le sue estensioni $(\Q, +)$, $(\R, +)$, $(\C, +)$, come è ovvio verificare.
\end{example}
\begin{example}
    $(\Zmod{n}, +)$ è un gruppo, definendo l'operazione di somma rispetto alle classi di resto.
\end{example}
\begin{example}
    è un gruppo la struttura $(\mu_n, \cdot)$ dove \[
        \mu_n \deq \set{x \in \C \suchthat x^n = 1}.    
    \]
\end{example}
\begin{proof}
    Infatti \begin{enumerate}[label={(G\arabic*)}, start=0]
        % \setcounter{enumi}{-1}
        \item $\cdot$ è un'operazione su $\mu_n$. Infatti se $x, y \in \mu_n$, ovvero \[
            x^n = y^n = 1    
        \] allora segue anche che \[
            (xy)^n = x^ny^n = 1    
        \] da cui $xy \in \mu_n$;
        \item $\cdot$ è associativa in $\C$, dunque lo è in $\mu_n \subseteq \C$;
        \item $1 \in \C$ è l'elemento neutro di $\cdot$ e $1 \in \mu_n$ in quanto $1^n = 1$;
        \item ogni elemento di $\mu_n$ ammette inverso. Infatti sia $x \in \mu_n$, dunque $x \neq 0$ (altrimenti $x^n = 0 \neq 1$) e sia $x\inv \in \C$ il suo inverso. Allora \[
            (x\inv)^n = (x^n)\inv = 1\inv = 1    
        \] ovvero $x\inv \in \mu_n$;
        \item inoltre $\cdot$ è commutativa in $\C$, dunque lo è anche in $\mu_n$.
    \end{enumerate}
    Da ciò segue che $\mu_n$ è un gruppo abeliano.
\end{proof}
\begin{example}
    $(\invertible{\Z}, \cdot)$ dove \[
        \invertible{\Z} \deq \set{n \in \Z \suchthat n \text{ è invertibile rispetto a } \cdot} = \set{\pm 1}
    \] è un gruppo abeliano;
\end{example}
\begin{example}
    $(\invertible{\Zmod{n}}, \cdot)$ dove \[
        \invertible{\Zmod{n}} \deq \set{\eqclass n \in \Zmod{n} \suchthat \eqclass n \text{ è invertibile rispetto a } \cdot}
    \] è un gruppo abeliano.
\end{example}
\begin{proof}
    Infatti \begin{enumerate}[label={(G\arabic*)}, start=0]
    % \setcounter{enumi}{-1}
        \item $\cdot$ è un'operazione su $\Zmod{n}$. Infatti se $\eqclass x, \eqclass y \in \Zmod{n}$ allora segue anche che $\eqclass{xy}$ è invertibile in $\Zmod{n}$ e il suo inverso è $\eqclass{x\inv} \cdot \eqclass{y\inv}$, da cui $\eqclass{xy} \in \Zmod n$;
        \item $\cdot$ è associativa in $\Zmod{n}$, dunque lo è in $\invertible{\Zmod{n}} \subseteq \Zmod{n}$;
        \item $1 \in \Zmod{n}$ è l'elemento neutro di $\cdot$ e $1 \in \invertible{\Zmod{n}}$ in quanto $1$ è invertibile e il suo inverso è $1$;
        \item ogni elemento di $\invertible{\Zmod{n}}$ ammette inverso per definizione;
        \item inoltre $\cdot$ è commutativa in $\Zmod{n}$, dunque lo è in $\invertible{\Zmod{n}} \subseteq \Zmod{n}$.
    \end{enumerate}
    Da ciò segue che $\Zmod{n}$ è un gruppo abeliano.
\end{proof}
\begin{example}
    Se $X$ è un insieme e $\SS(X)$ è l'insieme \[
        \SS(X) \deq \set{f : X \to X \suchthat f \text{ è bigettiva}}    
    \] allora $(\SS(X), \circ)$ è un gruppo (dove $\circ$ è l'operazione di composizione tra funzioni).
\end{example}
\begin{proof}
    Infatti \begin{enumerate}[label={(G\arabic*)}, start=0]
        % \setcounter{enumi}{-1}
        \item se $f, g \in \SS(X)$ allora $f \circ g : X \to X$ è bigettiva, dunque $f \circ g \in \SS(X)$;
        \item l'operazione di composizione di funzioni è associativa;
        \item la funzione \begin{align*}
            \operatorname{id} : X &\to X\\
            x &\mapsto x
        \end{align*} è bigettiva ed è l'elemento neutro rispetto alla composizione;
        \item Se $f \in \SS(X)$ allora $f$ è invertibile ed esisterà $f\inv : X \to X$ tale che $f \circ f\inv = \operatorname{id}$. Ma allora $f\inv$ è invertibile e la sua inversa è $f$, dunque $f\inv$ è bigettiva e quindi $f\inv \in \SS(X)$.
    \end{enumerate}
    Dunque $\SS(X)$ è un gruppo (non necessariamente abeliano).
\end{proof}

Esempi di strutture che non rispettano le proprietà di un gruppo sono invece:
\begin{itemize}
    \item $(\N, +)$ poichè nessun numero ha inverso ($-n \notin \N$);
    \item $(\Z, \cdot)$, $(\Q, \cdot)$, $(\R, \cdot)$ e $(\C, \cdot)$ non sono gruppi in quanto $0$ non ha inverso moltiplicativo;
    \item l'insieme \[
        \set{x \in \C \suchthat x^n = 2}    
    \] in quanto il prodotto due elementi di questo insieme non appartiene più all'insieme.
\end{itemize}

Definiamo ora alcune proprietà comuni a tutti i gruppi.

\begin{proposition}
    [Proprietà algebriche dei gruppi] \label{prop:prop_grp}
    Sia $(G, \cdot)$ un gruppo. Allora valgono le seguenti affermazioni:
    \begin{enumerate}[label={(\roman*)}, ref={\theproposition: (\roman*)}]
        \item \label{prop:prop_grp:e_unico} l'elemento neutro di $G$ è unico;\
        \item \label{prop:prop_grp:inv_unico} $\forall g \in G$ l'inverso di $g$ è unico;
        \item \label{prop:prop_grp:inv_inv} $\forall g \in G \;\; (g\inv)\inv = g$;
        \item \label{prop:prop_grp:inv_prod} $\forall h, g \in G \;\; (hg\inv)\inv = g\inv h\inv$; 
        \item \label{prop:prop_grp:canc} Valgono le \emph{leggi di cancellazione}: $\forall a, b, c \in G$ vale che \begin{align}
            ab = ac \iff b = c &\tag{sx} \label{prop:prop_grp:canc:sx}\\
            ba = ca \iff b = c &\tag{dx} \label{prop:prop_grp:canc:dx}
        \end{align}
    \end{enumerate}
\end{proposition}
\begin{proof}
    \begin{enumerate}[label={(\roman*)}]
        \item Siano $e_1, e_2 \in G$ entrambi elementi neutri. Allora \[
            e_1 = e_1 \cdot e_2 = e_2    
        \] dove il primo uguale viene dal fatto che $e_2$ è elemento neutro, mentre il secondo viene dal fatto che $e_1$ lo è.
        \item Siano $x, y \in G$ entrambi inversi di qualche $g \in G$. Allora per definizione di inverso \[
            xg = gx = e = gy = yg.    
        \]

        Ma allora segue che \begin{align*}
            &x \tag*{(el. neutro)}\\
            =\ &x \cdot e \tag*{($e = gy$)}\\
            =\ &x(gy) \tag*{(per (G1))}\\
            =\ &(xg)y \tag*{($xg = e$)}\\
            =\ &e \cdot y \tag*{(el. neutro)}\\
            =\ &g
        \end{align*} ovvero $x = y = g\inv$.
        \item Sappiamo che $gg\inv = g\inv g = e$. Sia $x$ l'inverso di $g\inv$, ovvero \[
            g\inv x = xg\inv = e.    
        \]
        
        Dunque $g$ è un inverso di $g\inv$, ma per \ref{prop:prop_grp:inv_unico} l'inverso è unico e quindi $(g\inv)\inv = g$.
        \item Sia $(hg)\inv$ l'inverso di $hg$. Allora per (G3) sappiamo che \begin{align*}
            &(hg)(hg)\inv = e \tag*{(moltiplico a sx per $h\inv$)} \\
            \iff\ &h\inv hg(hg)\inv = h\inv \tag*{(per (G3))}\\
            \iff\ &g(hg)\inv = h\inv \tag*{(moltiplico a sx per $g\inv$)} \\
            \iff\ &g\inv g(hg)\inv = g\inv h\inv \tag*{(per (G3))}\\
            \iff\ &(hg)\inv = g\inv h\inv.
        \end{align*}
        \item Legge di cancellazione sinistra: \begin{align*}
            &ab = ac \tag*{(moltiplico a sx per $a\inv$)} \\
            \iff\ &a\inv ab = a\inv ac \tag*{(per (G3))} \\
            \iff\ &b = c.
        \end{align*}

        Legge di cancellazione destra: \begin{align*}
            &ba = ca \tag*{(moltiplico a dx per $a\inv$)} \\
            \iff\ & ba a\inv = ca a\inv \tag*{(per (G3))} \\
            \iff\ &b = c. \tag*{\qedhere}
        \end{align*}
    \end{enumerate}
\end{proof}