\section{Gruppi e generatori}

Nella prima parte abbiamo studiato gruppi generati da un solo elemento (i gruppi \emph{ciclici}). Un gruppo può però essere generato da più di un singolo elemento: in particolare possiamo considerare un gruppo generato da un suo sottoinsieme:

\begin{definition}
    {Gruppo generato da un suo sottoinsieme}{}
    Sia $G$ un gruppo e sia $S \subseteq G$ un suo sottoinsieme.
    
    $G$ si dice \strong{generato da $S$}, oppure si dice che $S$ è un \strong{insieme di generatori per $G$} (e si indica con $G = \gen*{S}$), se \[
        G = \set*{s_1\dots s_n \given n \in \N, s_i \in S \union S\inv},
    \] dove $S\inv$ è l'insieme degli inversi degli elementi di $S$.
\end{definition}

Parleremo più nel dettaglio di generatori quando introdurremo il gruppo libero su un insieme.

\begin{remark}
    $s_1 \dots s_n$ rappresenta tutte le parole di lunghezza finita formate da elementi di $S$ o dai loro inversi: siccome $G$ è un gruppo (ed è quindi chiuso per prodotto) e $S, S\inv \subseteq G$ segue che la parola $s_1 \dots s_n \in G$, dunque $\gen{S} \subseteq G$.
\end{remark}

\begin{remark}
    Se $S = \set*{g}$ allora \[
        G = \set*{g^{\eps_1}\dots g^{\eps_n} \given n \in \N, \eps_i = \pm 1} = \set*{g^{\sum \eps_i}} = \gen{g}.
    \]
\end{remark}

\begin{remark}
    Se il gruppo $G$ è finito è sufficiente che $s_i \in S$ (non serve considerare $S\inv$).
    \begin{proof}
        Siccome $G$ è finito ogni suo sottogruppo è finito; in particolare se $s \in S$ allora $\gen{s} \sgr G$ è un sottogruppo finito, e sarà della forma \[
            \gen{s} = \set*{e_G, s, s^2, \dots, s^m},    
        \] dove $m \deq \ord_{G}{s}$.
        Siccome $\gen s$ è un sottogruppo di $G$ segue che $s\inv \in \gen s$, dunque $s\inv = s^k$ per qualche $0 \leq k < m$.
        Dunque ogni occorrenza di $s\inv$ in una parola può essere sostituita con $s^k$ che è ottenibile dai soli elementi di $S$. 
    \end{proof}
\end{remark}

\begin{example}
    Mostriamo che $\Zmod{2} \times \Zmod{2} = \gen{(1, 0), (0, 1)}$.

    Come abbiamo osservato in precedenza l'inclusione $\supseteq$ è banale, dunque basta far vedere che $\Zmod{2} \times \Zmod{2}$ è un sottoinsieme di $\gen{(1, 0), (0, 1)}$.
    \[
        \Zmod 2 \times \Zmod 2 = \set*{(1, 0), (0, 1), \overbrace{(1, 0) + (0, 1)}^{= (1, 1)}, \overbrace{(1, 0) + (1, 0)}^{= (0, 0)}} \subseteq \gen{(1, 0), (0, 1)}.
    \]
\end{example}

Al contrario degli spazi vettoriali non è detto che esista una \emph{dimensione} del gruppo: due insiemi di generatori minimali possono avere cardinalità diverse.

\begin{example}
    Sappiamo già che $\Z = \gen{1} = \gen{-1}$. Mostriamo che $\Z = \gen{2, 3}$ e che $\set*{2, 3}$ è un insieme minimale di generatori.

    È sufficiente mostrare che $\Z \subseteq \gen{2, 3}$, ovvero che per ogni $n \in \Z$ esistano $a, b \in \Z$ tali che \[
        n = a\cdot 2 + b \cdot 3.    
    \] Per l'identità di Bézout sappiamo che esistono $a_0, b_0 \in \Z$ tali che \[
        a_0\cdot 2 + b_0 \cdot 3 = \gcd{2}{3} = 1,    
    \] dunque moltiplicando tutto per $n$ otteniamo la tesi.

    Inoltre $\gen{2} = 2\Z$, $\gen 3 = 3\Z$, dunque $\set*{2, 3}$ è un insieme minimale di generatori.
\end{example}

Un tipo particolare di gruppi sono i \strong{gruppi finitamente generati}.

\begin{definition}
    {Finitamente generato}{}
    Sia $G$ un gruppo. $G$ si dice \strong{finitamente generato} se $G$ ammette un insieme finito di generatori.
\end{definition}

\begin{proposition}{}{}
    Se $G$ è finitamente generato, allora ogni suo insieme minimale di generatori ha cardinalità finita.
\end{proposition}
\begin{proof}
    Siccome $G$ è finitamente generato esisterà un insieme di generatori \[
        S = \set*{s_1, \dots, s_n}    
    \] tale che $G = \gen S$.

    Sia $X$ un insieme di generatori per $G$ di cardinalità infinita. Dato che $S \subseteq G$ ogni elemento di $S$ è esprimibile come una parola finita formata da elementi di $X$ o da loro inversi: per ogni $s_i \in S$ esisteranno quindi $k_i$ elementi di $X \union X\inv$ tali che \[
        s_i = x_{1i}\dots x{k_ii}.
    \] Segue quindi che \[
        S = \set*{x_{11}\dots x{k_{11}}, \dots, x_{1n}\dots x{k_{nn}}}.
    \] Dato che $S$ è un insieme di generatori per $G$ segue che gli elementi $x_{ij}$ generano il gruppo $G$, in quanto sono sufficienti per generare i generatori di $G$. Siccome essi sono in numero finito segue che $X$ non è minimale, da cui la tesi.
\end{proof}

\section{Gruppo diedrale}

Uno degi gruppi più importanti in algebra è il gruppo delle isometrie dell'$n$-agono regolare (ovvero delle trasformazioni che mandano l'$n$-agono regolare in sé), detto \strong{gruppo diedrale}.

\begin{definition}
    {Gruppo diedrale}{}
    Se $n \geq 3$, si dice \strong{gruppo diedrale} $D_n$ l'insieme delle isometrie del piano che mandano in sé l'$n$-agono regolare insieme all'operazione di composizione. 
\end{definition}

Per mostrare che il gruppo diedrale è effettivamente un gruppo dobbiamo mostrare che la composizione di isometrie è un'isometria (il che è ovvio) e che ogni isometria ammette un'inversa. Ma l'inversa di una trasformazione $\sigma$  è semplicemente l'isometria che \emph{annulla} l'effetto di $\sigma$, dunque $D_n$ è un gruppo.

Per studiare la struttura del gruppo diedrale, numeriamo i vertici dell'$n$-agono regolare da $1$ a $n$.

\begin{proposition}
    {Cardinalità del gruppo diedrale}{} 
    La cardinalità di $D_n$ è $2n$ per ogni $n \geq 3$.
\end{proposition}
\begin{proof}
    Mostriamo inizialmente che $\#D_n \leq 2n$.

    Sia $x \in D_n$. Questa isometria manderà ogni vertice dell'$n$-agono in un altro vertice, ed ogni lato in un altro lato. 
    
    Sia quindi $i \deq x(1)$, ovvero $i$ è il vertice in cui viene mandato il vertice $1$. A questo punto il lato $(1, 2)$ dovrà essere mandato in un altro lato, dunque segue che $x(2) = i+1$ oppure $i-1$. 
    
    Dopo aver fatto queste due scelte, l'isometria $x$ è fissata: se $x(2) = i+1$ allora $x(3) = i+2$, $x(4) = i+3$ eccetera; se $x(2) = i-1$ allora $x(3)=i-2$ eccetera. Abbiamo quindi $n$ possibili scelte per $x(1)$ e $2$ possibili scelte per $x(2)$, dunque il numero di isometrie distinte è al più $2n$. 

    Mostriamo ora che queste scelte sono tutte distinte, ovvero che $\#D_n = 2n$. Innanzitutto l'$n$-agono ammette $n$ rotazioni distinte, di cui una è la rotazione banale $\id$; inoltre vi sono $n$ assi di simmetria:
    \begin{itemize}
        \item se $n$ è pari essi congiungono i vertici con i vertici opposti e le metà dei lati con le metà dei lati opposti;
        \item se $n$ è dispari, essi congiungono i vertici con le metà dei lati opposti ai vertici.
    \end{itemize}
    Inoltre ogni simmetria non è una rotazione, in quanto le simmetrie invertono l'orientazione dei vertici mentre le rotazioni la mantengono. Dunque vi sono almeno $2n$ elementi in $D_n$, da cui segue che $\#D_n = 2n$.
\end{proof}

Abbiamo quindi visto che vi sono due tipi distinti di elementi: le \strong{rotazioni} e le \strong{simmetrie}. Chiamiamo $r$ la rotazione attorno al centro di $\frac{2\pi}{n}$: le altre rotazioni saranno date da \[
    \id = r^0, r, r^2, \dots, r^{n-1}
.\] Le simmetrie saranno invece $s_1, s_2, \dots, s_n$. Tuttavia essendo $D_n$ un gruppo segue che $s_ir, s_ir^2, \dots, s_ir^{n-1}$ sono anch'essi tutti elementi di $D_n$ per qualsiasi $i$: dobbiamo capire quali di questi siano uguali.

\begin{proposition}{}{}
    Sia $r$ la rotazione di $\frac{2\pi}{n}$ radianti attorno all'origine e sia $s$ una simmetria qualunque dell'$n$-agono regolare. Allora \[
        D_n = \set*{\id, r, \dots, r^{n-1}, s, sr, \dots, sr^{n-1}}.    
    .\] 
\end{proposition}
\begin{proof}
    Sappiamo già che le rotazioni sono distinte tra loro e che le simmetrie non sono rotazioni. 

    Mostriamo che $sr^i$ è una simmetria, ovvero non è una rotazione. Se per assurdo lo fosse, allora sarebbe uguale a $r^j$ per qualche $j \in \Z$, $0 \leq j < n$. Allora abbiamo tre possibilità:
    \begin{enumerate}
        \item se $i = j$ allora $s = \id$, da cui $s$ è una rotazione, il che è assurdo;
        \item se $i > j$ allora $sr^{i-j} = \id$, da cui $s$ è l'inversa di una rotazione e quindi è una rotazione, il che è assurdo;
        \item se $i < j$ allora $s = r^{j-1}$, da cui $s$ è una rotazione, il che è assurdo.
    \end{enumerate}
    Dunque $sr^i$ è una simmetria.

    Mostriamo che le simmetrie sono distinte fra loro: siano $sr^i$, $s r^j$ due simmetrie con $i \neq j$ e mostriamo che $sr^i \neq sr^j$. Per la legge di cancellazione da ciò segue che $r^i = r^j$; tuttavia questo è assurdo in quanto le rotazioni sono distinte tra loro.
\end{proof}

Possiamo quindi esprimere $D_n$ tramite una \emph{presentazione di gruppo}: \[
    D_n \deq \gen[\big]{r, s \given r^n = \id, s^2 = \id, sr = r\inv s}.
\]
Questo modo di scrivere il gruppo mette in evidenza:
\begin{itemize}
    \item i generatori del gruppo, ovvero $r$ e $s$;
    \item gli ordini dei generatori: $\ord{r} = n$ e $\ord{s} = 2$;
    \item le relazioni tra i generatori: come mostreremo tra poco vale che $sr = r\inv s$.
\end{itemize}

La rotazione $r$ ha ovviamente ordine $n$: siccome è una rotazione di $\frac{2\pi}{n}$ radianti, ripetendola $n$ volte otteniamo l'$n$-agono originale.
Per lo stesso motivo la simmetria $s$ ha ordine $2$.

Per mostrare che $sr = r\inv s$ basta mostrare che l'immagine di tutti i vertici mediante le due isometrie è la stessa. 
% Infatti vale che \begin{align*}
%     sr(1) = s(2) 
% \end{align*}

\subsection{Sottogruppi del gruppo diedrale}

Studiamo ora i sottogruppi del gruppo diedrale $D_n$.

Iniziamo studiando $\gen r$: siccome $\ord r = n$ segue che $\GrpIndex{D_n : \gen r} = 2$, da cui per la \Cref{prop:norm_se_indice2} segue che $\gen r \normal D_n$.

Tuttavia possiamo anche mostrare che per ogni $j = 0, \dots, n-1$ il gruppo $\gen*{r^j}$ è normale in $D_n$. Osserviamo inizialmente che $\gen r$ è l'unico sottogruppo di $D_n$ di ordine $n$: esso infatti contiene tutte le rotazioni e, siccome tutte le simmetrie hanno ordine $2$, non possono esserci altri sottogruppi ciclici di ordine $n$.
Inoltre, essendo un gruppo ciclico, per la \Cref{cor:sgr_gruppo_ciclico} esso ha uno e un solo sottogruppo di ordine $d$ per ogni $d$ che divide $n$.

Mostriamo alcuni risultati intermedi.
\begin{proposition}{}{}
    $\gen*{r^{\frac{n}{d}}}$ è l'unico sottogruppo ciclico di $D_n$ di ordine $d$ per ogni $d > 2$.
\end{proposition}
\begin{proof}
    Innanzitutto $\ord*[D_n]{r^{\frac{n}{d}}} = d$ in quanto \[
        \parens*{r^{\frac{n}{d}}}^d = r^n = \id.
    \] Inoltre esso contiene tutti gli elementi di ordine $d$ poiché è l'unico sottogruppo ciclico di $\gen r$ di ordine $d$ e gli elementi che non appartengono a $\gen r$ hanno ordine $2$ (sono simmetrie); da questo segue che è l'unico sottogruppo ciclico di ordine $d$ di $D_n$.
\end{proof}

\begin{proposition}{}{}
    Sia $G$ un gruppo. Se $H$ è l'unico sottogruppo di ordine $d$ di $G$, allora $H \normal G$.
\end{proposition}
\begin{proof}
    Per ogni $g \in G$ vale che $gHg\inv$ è un sottogruppo di $G$ di ordine $d$, dunque siccome $H$ è l'unico sottogruppo con queste proprietà segue che $gHg\inv = H$, da cui la tesi.
\end{proof}

\begin{corollary}{}{}
    Sia $G$ un gruppo. Se $H$ è l'unico sottogruppo ciclico di ordine $d$ di $G$, allora $H \normal G$.
\end{corollary}
\begin{proof}
    Se $H = \gen h$ per qualche $h \in G$ allora segue che il coniugato $gHg\inv$ è generato dall'elemento $ghg\inv$, dunque anche esso è ciclico. Tuttavia l'unico sottogruppo di $G$ di ordine $d$ e ciclico è $H$, da cui segue che $gHg\inv = H$, ovvero $H$ è normale in $G$.
\end{proof}

Sfruttando le due proposizioni precedenti segue che per ogni $d$ che divide $n$ ($d > 2$) il sottogruppo $\gen*{r^{\frac{n}{d}}}$ è normale in $D_n$.

Questo ragionamento non ci permette di mostrare che $\gen*{r^{\frac{n}2}}$ è normale in $D_n$; tuttavia possiamo dimostrarlo studiando il centro di $D_n$.

\begin{proposition}{Centro di $D_n$}{}
    \[
        \Zentr{D_n} = \begin{cases}
            \set*{\id}, &\text{se } n \text{ è dispari}\\[3pt]
            \gen*{r^{\frac{n}2}}, &\text{se $n$ è pari}.
        \end{cases}   
    \]
\end{proposition}
\begin{proof}
    Per definizione di centro di un gruppo, un elemento è nel centro se e solo se commuta con tutti gli elementi del gruppo; è dunque sufficiente mostrare che un elemento commuta con i generatori del gruppo. Segue quindi che \[
        \Zentr{D_n} = \set*{s^\eps r^j \in D_n \given s^\eps r^j \cdot r = r \cdot s^\eps r^j, s^\eps r^j \cdot s = s \cdot s^\eps r^j}.
    \]

    Se $s^\eps r^j$ soddisfa la seconda condizione, allora \begin{align*}
        &s^\eps r^j \cdot s = s \cdot s^\eps r^j \\
        \iff &s^\eps sr^{-j} = s \cdot s^\eps r^j  \\
        \iff &s^{\eps + 1}r^{-j} = s^{\eps + 1} r^j\\
        \iff &r^{-j} = r^j.
    \end{align*} Dunque segue che $j \congr -j \mod{n}$, ovvero $2j \congr 0 \mod{n}$. Abbiamo quindi due casi:
    \begin{itemize}
        \item Se $n$ è dispari questo significa che $j \congr 0 \mod{n}$, ovvero $j = 0$. 
        Le possibili scelte sono quindi $\id$ ed $s$; tuttavia $s$ non commuta con $r$, dunque l'unico elemento che rispetta entrambe le condizioni è $\id$ e quindi \[
            \Zentr{D_n} = \set*{id}.    
        \]
        \item Se $n$ è pari questo implica $j \congr 0 \mod{\nicefrac{n}2}$, da cui segue che $j = 0, \nicefrac{n}2$. I quattro elementi che possono essere nel centro di $D_n$ sono quindi \[
            \id, r^{\frac{n}2}, s, sr^\frac{n}{2}.   
        \] Tuttavia $s$ e $sr^{\nicefrac{n}{2}}$ non commutano con $r$, in quanto \[
            sr = r\inv s, \quad sr^{\nicefrac{n}{2}} \cdot r = sr^{\frac{n}{2} + 1} \neq sr^{\frac{n}{2} - 1} = r \cdot sr^\frac{n}{2}.
    \]  Dunque gli unici elementi nel centro sono $\id, r^{\nicefrac{n}{2}}$, da cui segue che \[
        \Zentr{D_n} = \gen*{r^\frac{n}{2}}. \qedhere    
    \]
    \end{itemize}
\end{proof}

Siccome il centro di un gruppo è sempre un sottogruppo normale di quel gruppo (per la \Cref{prop:centro_normale}) segue che $\gen*{r^{\nicefrac{n}{2}}}$ è un sottogruppo normale di $D_n$.

Vale quindi il seguente Teorema.

\begin{theorem}
    {Sottogruppi generati dalle rotazioni sono normali}{normal_sgrps_Dn}
    Ogni sottogruppo di $D_n$ della forma $\gen{r^i}$ è normale in $D_n$. 
\end{theorem}