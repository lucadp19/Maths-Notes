\section{Divisibilità}

Consideriamo la relazione di divisibilità tra numeri interi:
\begin{definition}
    [Divisibilità]
    Siano $a, b \in \Z$. Allora si dice che \emph{$a$ divide $b$} (e si indica con $a \divides b$) se \[
        a = kb
    \] per qualche $k \in \Z$.
\end{definition}

\begin{proposition}
    [Divisibilità come relazione d'ordine]
    La relazione di divisibilità tra numeri interi è una relazione di ordine parziale su $\N \setminus \set*{0}$.
\end{proposition}
\begin{proof}
    Per definizione di relazione d'ordine dobbiamo mostrare tre cose:
    \begin{description}
        \item[Riflessività] Sia $a \in \N$ non nullo. Allora $a \divides a$ poiché $a = 1\cdot a$.
        \item[Simmetria] Siano $a, b \in \N$ non nulli e supponiamo che $a \divides b$ e $b \divides a$. Allora per definizione di divisibilità segue che \[
            a = kb, \quad b = ha    
        \]  per qualche $k, h \in \Z$. Sostituendo la seconda equazione nella prima otteniamo $a = kha$, ovvero $kh = 1$. Ma dato che $a, b \in \N$ segue che $k = h = 1$, dunque $a = b$.
        \item[Transitività] Siano $a, b, c \in \N$ non nulli tali che $a \divides b$ e $b \divides c$. Allora per definizione vale che \[
            a = kb, \quad b = hc    
        \] per qualche $k, h \in \Z$.

        Sostituendo la seconda nella prima ottengo quindi $a = khc$, ovvero $a \divides c$ in quanto $kh \in \Z$.
    \end{description}
\end{proof}

La relazione di divisibilità può essere pensata anche come un ordinamento su $\Z \setminus {0}$, ma l'antisimmetria è "a meno del segno", ovvero \[
    a \divides b, b \divides a \implies a = b \text{ oppure } a = -b.
\] In questi casi scriveremo più semplicemente $a = \pm b$ per indicare che $a$ può essere $b$ oppure il suo opposto.

\begin{definition}
    [Massimo comun divisore]
    Siano $a, b \in \Z$ non nulli. Si dice \emph{massimo comun divisore di $a,b$} il numero $d \in \Z$ tale che \begin{enumerate}[label={(\roman*)}]
        \item $d \divides a$ e $d \divides b$;
        \item se $c \divides a$ e $c \divides b$ allora $c \divides d$.
    \end{enumerate}

    Tale $d$ si indica anche con $\operatorname{mcd}(a, b)$, oppure con $\gcd(a, b)$ oppure anche con $\gcd{a}{b}$.
\end{definition}

\begin{theorem}
    [Esistenza ed unicità del massimo comun divisore]
    Siano $a, b \in \Z$ non nulli. Allora esiste ed è unico (a meno del segno) $d \in \Z$ tale che $d = \gcd{a}{b}$.
\end{theorem}
\begin{proof}
    Mostriamo sia l'esistenza che l'unicità del massimo comun divisore.
    \begin{description}
        \item[Esistenza] Sia $X$ il sottoinsieme di $\Z$ tale che \[
            X \deq \set*{ax + by \given x, y \in \Z}    
        \] e sia $Y \deq X \inters \N \setminus \set*{0}$. Notiamo che $Y \subseteq \N$ e $Y \neq \varnothing$, in quanto \begin{itemize}
            \item se $a > 0$ allora posso scegliere $x = 1 - b$, $y = a$ da cui segue che \[
                ax + by = a(1-b)+ab = a - ab + ab = a > 0
            \] cioè $a \in X$,
            \item se $a < 0$ posso scegliere $x = - 1 - b$ e $y = a$, da cui \[
                ax + by = a(-1-b)+ab = -a - ab + ab = -a > 0 
            \] da cui segue $-a \in X$.
        \end{itemize}

        Da ciò segue che per il \nameref{ax:min_intero} l'insieme $Y$ ammette minimo. Sia $d \deq \min Y$. Mostro ora che $d = \gcd{a}{b}$.

        Notiamo che siccome $d \in Y$ allora dovranno esistere $x_0, y_0 \in \Z$ tali che $d = ax_0 + by_0$.
        \begin{enumerate}[label={(\roman*)}]
            \item Dimostro che $d \divides a$; per simmetria dimostrare ciò dimostra automaticamente che $d \divides b$.
            
            Per la divisione euclidea scrivo \begin{equation}
                a = qd + r \quad \text{per qualche } 0 \leq r < \abs{a}.
            \end{equation} Allora vale che\begin{equation*}
                0 \leq r = a - qd = a - q(ax_0 + by_0) = a(1 - qx_0) + b(-qy_0)
            \end{equation*}
            Dunque $r = 0$ oppure $r \in Y$. Tuttavia abbiamo supposto che $d$ fosse il minimo di $Y$, dunque siccome $r < d$ la seconda opzione è impossibile. Quindi $r = 0$, da cui segue che $a = qd$, ovvero $d \divides a$.
            \item Dimostro ora che per ogni $c$ che divide sia $a$ che $b$ segue che $c \divides d$.
            Per definizione di divisibilità sappiamo che esistono $h, k \in \Z$ tali che \[
                a = hc, \quad b = kc.    
            \] Da ciò segue che \[
                d = ax_0 + by_0 = c(hx_0 + ky_0)    
            \] e siccome $hx_0 + ky_0 \in \Z$ segue che $c \divides d$.
        \end{enumerate}
        Dunque $d$ è il massimo comun divisore tra i numeri $a$ e $b$.
        \item[Unicità] Supponiamo che esistano $d, d^\prime \in \Z$ che siano entrambi massimi comun divisiori di $a$ e $b$. Dunque dovranno valere le seguenti proprietà: siccome $d$ è un massimo comun divisore dovrà valere che \begin{enumerate}[label={(\roman*)}]
            \item $d \divides a$ e $d \divides b$,
            \item per ogni $c$ che divide $a$ e $b$, allora $c \divides d$.
        \end{enumerate} Siccome anche $d^\prime$ è un massimo comun divisore, dovranno valere le seguenti: \begin{enumerate}[label={(\roman*')}]
            \item $d^\prime \divides a$ e $d^\prime \divides b$,
            \item per ogni $c$ che divide $a$ e $b$, allora $c \divides d^\prime$.
        \end{enumerate}

        Allora sfruttando la (i) e la (ii') otteniamo che \[
            d \divides a, d \divides b \implies d \divides d^\prime,    
        \] mentre sfruttando la (i') e la (ii) otteniamo che \[
            d^\prime \divides a, d^\prime \divides b \implies d^\prime \divides d.   
        \] Concludiamo quindi che $d$ e $d^\prime$ sono uguali (a meno del segno), ovvero che il massimo comun divisore è unico a meno del segno. \qedhere
    \end{description}
\end{proof}

\subsection{Algoritmo di Euclide}

Per trovare il massimo comun divisore di due numeri possiamo sfruttare il seguente algoritmo, detto \emph{algoritmo di Euclide}.

Siano $a, b \in \Z$ non entrambi nulli (supponiamo senza perdita di generalità $b \neq 0$). L'algoritmo di Euclide mappa la coppia $(a, b) \in \Z^2$ alla tripla $(d, x_0, y_0) \in \Z^3$ dove $d = \gcd{a}{b}$ e $x_0, y_0$ sono tali che \begin{equation}
    d = ax_0 + by_0. \label{eq:bézout}
\end{equation} Quest'ultima identità viene detta \emph{identità di Bézout}.

Il procedimento si basa su divisioni euclidee iterate: poniamo $r_0 \deq b$ e applichiamo la divisione euclidea sui resti ottenuti nel seguente modo.
\begin{align*}
    a &= q_1r_0 + r_1 \quad &&\text{con } 0 \leq r_1 < b \\
    r_0 &= q_2r_1 + r_2 \quad &&\text{con } 0 \leq r_2 < r_1 \\
    r_1 &= q_3r_2 + r_3 \quad &&\text{con } 0 \leq r_3 < r_2 \\
    &\vdots &&\vdots\\
    r_{n-1} &= q_{n+1}r_n + r_{n+1} \quad &&\text{con } 0 \leq r_{n+1} < r_n.
\end{align*}
Supponiamo che $r_n$ sia l'ultimo resto diverso da $0$ (ovvero $r_n \neq 0$, $r_{n+1} = 0$). Allora vale che $d = r_n$.

Prima di mostrare questo fatto, dimostriamo un lemma importante.

\begin{lemma}{mcd_diff}
    Siano $a, b \in \Z$ non entrambi nulli. Allora per ogni $k \in \Z$ vale che \begin{equation}
        \gcd{a}{b} = \pm\gcd{a}{b-ka}.    
    \end{equation}
\end{lemma}
\begin{proof}
    Siano $d, d^\prime \in \Z$ tali che \[
        d = \gcd{a}{b}, \quad d^\prime = \gcd{a}{b-ka}.    
    \] Mostriamo che $d \divides d^\prime$ e $d^\prime \divides d$.

    \begin{description}
        \item[($d \divides d^\prime$)] Siano $s, t \in \Z$ tali che \[
            a = ds, \quad b = dt.    
        \] Allora varrà che \[
            b - ka = dt - kds = d(t - ks),    
        \] dunque siccome $t - ks \in \Z$ segue che $d \divides b - ka$.
        Allora $d$ divide sia $a$ che $b - ka$, dunque dovrà dividere il loro massimo comun divisore $d^\prime$.
        \item[($d^\prime \divides d$)] Segue automaticamente per simmetria: sia $\beta = b - ka$, allora vale che \[
            d^\prime = \gcd{a}{\beta} \divides \gcd{a}{\beta + ka} = \gcd{a}{b} = d,  
        \] dove il secondo passaggio è giustificato dal punto precedente, sostituendo $-k$ al posto di $k$.
    \end{description}
    Concludiamo che $d = \pm d^\prime$, come volevasi dimostrare.
\end{proof}

\begin{theorem}
    [Correttezza e terminazione dell'algoritmo di Euclide]
    Siano $a, b \in \Z$ non entrambi nulli. Allora l'algoritmo di Euclide termina in un numero finito di passi e restituisce una terna $(d, x_0, y_0)$ tale che \begin{itemize}
        \item $d$ è il massimo comun divisore di $a$ e $b$,
        \item vale che $\gcd{a}{b} = ax_0 + by_0$.
    \end{itemize}
\end{theorem}
\begin{proof}
    Mostriamo separatamente che l'algoritmo termina e che produce la terna desiderata.
    \begin{description}
        \item[Terminazione] Consideriamo la successione $(r_i)_{i\geq 0}$. Essa è una successione strettamente decrescente di numeri naturali. Sia $r_n$ il minimo numero positivo della successione (esiste per il \nameref{ax:min_intero}), ovvero \[
            r_n = \min \set*{r_i \given i \geq 0, r_i > 0}.    
        \] Ma essendo $(r_i)$ strettamente decrescente avremo $r_{n+1} < r_n$, dunque dovrà valere che $r_{n+1} = 0$.
        \item[Correttezza] Dimostriamo la correttezza dell'algoritmo per induzione sul numero di passi $N$: se $r_n$ è l'ultimo resto non nullo, allora diciamo che il numero di passi necessari per eseguire l'algoritmo è $N \deq n+1$. 
        \begin{description}
            \item[Caso base] Se $N \deq 1$, ovvero $r_1 = 0$, allora $a = q_1b$, da cui segue che $\gcd{a}{b} = b$. Infatti
            \begin{enumerate}[label={\roman*}]
                \item $b \divides a$ e $b \divides b$ (ovvio)
                \item se $c \divides a$ e $c \divides b$ allora $c \divides \gcd{a}{b} = b$ (ovvio).
            \end{enumerate} 
            \item[Passo induttivo] Supponiamo che il caso con $N - 1$ passi termini e restituisca il risultato corretto.
            
            Siano $a, b \in \Z$ e siano $q_1, r_1$ rispettivamente il quoziente e il resto della divisione euclidea di $a$ per $b$, ovvero \[
                a = bq_1 - r_1.    
            \]
            
            Supponiamo che per calcolare $\operatorname{A.E.}(a, b)$ siano necessari $N$ passi: allora per calcolare $\operatorname{A.E.}(b, r_1)$ ne servono $N-1$, quindi per ipotesi induttiva l'algoritmo termina e dà come risultato la tripla \[
                (\gcd{b}{r_1}, x_1, y_1) 
            \] dove $\gcd{b}{r_1} = bx_1 + r_1y_1$(identità di Bézout).

            Per il \Cref{lem:mcd_diff} segue che \[
                \gcd{b}{r_1} = \gcd{b}{a - q_1b} = \gcd{b}{a} = \gcd{a}{b}.
            \] Inoltre segue che \begin{align*}
                \gcd{a}{b} &= \gcd{b}{r_1} \\
                &= bx_1 + r_1y_1 \\
                &= bx_1 + (a - q_1b)y_1 \\
                &= a(y_1) + b(x_1 - q_1y_1).
            \end{align*}

            Dunque la tripla \[
                (\gcd{b}{r_1}, y_1, x_1 - q_1y_1)    
            \] è il risultato dell'algoritmo di Euclide in $N$ passi, come volevasi dimostrare. \qedhere
        \end{description} 
    \end{description}
\end{proof}