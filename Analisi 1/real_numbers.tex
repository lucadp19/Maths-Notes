\section{Numeri reali}

\subsection{Proprietà algebriche dei numeri reali}

\begin{definition} [Campo]
    \label{def:campo}
    Sia $\K$ un insieme e siano $+$ (\emph{somma}), $\cdot$ (\emph{prodotto}) due operazioni su $\K$, ovvero \begin{align*}
        + : \K \times \K &\to \K, & \cdot : \K \times \K &\to \K. \\
        (a, b) &\mapsto a+b,      &             (a, b) &\mapsto a\cdot b.
    \end{align*} Allora la struttura $(\K, +, \cdot)$ si dice \emph{campo} se valgono i seguenti assiomi:
    \begin{enumerate}[label={(S\arabic*)}]
        \item \label{def:campo_sum:com} Vale la \emph{proprietà commutativa della somma}:
        
        per ogni $a, b \in \K$ vale che $a + b = b + a$.
        \item \label{def:campo_sum:ass} Vale la \emph{proprietà associativa della somma}:
        
        per ogni $a, b, c \in \K$ vale che $(a + b) + c = a + (b + c)$.
        \item \label{def:campo_sum:neu} Esiste un elemento $0 \in \K$ che è \emph{elemento neutro} per la somma:
        
        per ogni $a \in \K$ vale che $a + 0 = 0 + a = a$.
        \item \label{def:campo_sum:opp} Ogni elemento di $\K$ è \emph{invertibile} rispetto alla somma:
        
        per ogni $a \in \K$ esiste $(-a) \in \K$ (detto \emph{opposto di $a$}) tale che $a + (-a) = 0$.
    \end{enumerate}
    \begin{enumerate}[label={(P\arabic*)}]
        \item \label{def:campo_prod:com} Vale la \emph{proprietà commutativa del prodotto}:
        
        per ogni $a, b \in \K$ vale che $a \cdot b = b \cdot a$.
        \item \label{def:campo_prod:ass} Vale la \emph{proprietà associativa del prodotto}:
        
        per ogni $a, b, c \in \K$ vale che $(a \cdot b) \cdot c = a \cdot (b \cdot c)$.
        \item \label{def:campo_prod:neu} Esiste un elemento $1 \in \K$ che è \emph{elemento neutro} per il prodotto:
        
        per ogni $a \in \K$ vale che $a \cdot 1 = 1 \cdot a = a$.
        \item \label{def:campo_prod:inv} Ogni elemento di $\K$ è \emph{invertibile} rispetto al prodotto:
        
        per ogni $a \in \K$ esiste $a\inv \in \K$ (detto \emph{inverso di $a$}) tale che $a \cdot a\inv = 1$.
    \end{enumerate}
    \begin{enumerate}[label=(SP)]
        \item \label{def:campo:distr} Vale la \emph{proprietà distributiva del prodotto rispetto alla somma}:
         
        per ogni $a, b, c \in \K$ vale che $a(b + c) = ab + ac$.
    \end{enumerate}
\end{definition}

Spesso sottointendiamo il simbolo di prodotto, scrivendo $ab$ per $a \cdot b$.

\begin{definition} [Campo ordinato]
    \label{def:campo_ord}
    Sia $(\K, +, \cdot)$ un campo e sia $\leq$ una relazione di ordine totale su $\K$.
    
    Allora la struttura $(\K, +, \cdot, \leq)$ si dice \emph{campo ordinato} se valgono i seguenti assiomi:
    \begin{enumerate}[label={(CO\arabic*)}]
        \item \label{def:campo_ord:ord_sum} Per ogni $a, b, c \in \K$, se $a \leq b$ allora $a + c \leq b + c$.
        \item \label{def:campo_ord:ord_prod} Per ogni $a, b \in \K$, se $0 \leq a$ e $0 \leq b$ allora $0 \leq ab$.
    \end{enumerate}
\end{definition}

Esempi di campi sono $\Q$, $\R$ e $\C$ con le usuali operazioni di somma e prodotto. Di questi soltanto $\Q$ e $\R$ ammettono un ordinamento totale, dunque soltanto $\Q$ e $\R$ sono campi ordinati.

Inoltre nei campi ordinati introduciamo l'ordinamento totale $\geq$ tale che per ogni $a, b \in \K$ vale che \[
    a \geq b \iff b \leq a.    
\]

\begin{proposition}
    [Proprietà di un campo]

    Sia $(\K, +, \cdot)$ un campo. Allora valgono le seguenti:
    \begin{enumerate}[label={(\roman*)}]
        \item per ogni $a \in \K$ vale che $a \cdot 0 = 0$,
        \item per ogni $a, b \in \K$ se $ab = 0$ allora $a = 0$ oppure $b = 0$.
    \end{enumerate}

    Inoltre se $\leq$ è un ordinamento totale su $\K$ tale che $(\K, +, \cdot, \leq)$ è un campo ordinato, vale che:
    \begin{enumerate}[label={(\roman*)}, start=3]
        \item per ogni $a \in \K$, se $a \geq 0$ allora $-a \leq 0$,
        \item per ogni $a, b, c \in \K$, se $a \leq b$ e $c \leq 0$ allora $ac \geq bc$,
        \item per ogni $a \in \R$ vale che $0 \leq a^2$.
    \end{enumerate}
\end{proposition}
\begin{proof}
    Dimostriamo le cinque proprietà separatamente.
    \begin{enumerate}[label={(\roman*)}]
        \item Sia $a \in \K$ qualsiasi. Allora \begin{align*}
            a0 &= a(0 + 0) \tag{per \ref{def:campo_sum:neu}}\\
            &= a0 + a0 \tag{per \ref{def:campo_sum:neu}}\\
            \intertext{Per \ref{def:campo_sum:neu} esiste $(-a0) \in \K$, dunque}
            \iff a0 - a0 &= a0 + a0 - a0 \tag{per \ref{def:campo_sum:opp}}\\
            \iff 0 &= a0.
        \end{align*}
        COMPLETA
    \end{enumerate}
\end{proof}

\subsection{Valore assoluto}

\begin{definition}
    Sia $a \in \R$. Allora si definisce \emph{valore assoluto} di $a$ il massimo tra $a$ e $-a$, ovvero \begin{equation}
        \abs*{a} \deq \max \set{a, -a} = \begin{cases}
            a, &\text{se } a \geq 0\\
            -a, &\text{se } a < 0.
        \end{cases}
    \end{equation}
\end{definition}

\begin{proposition}
    [Proprietà del valore assoluto]
    Il valore assoluto gode delle seguenti proprietà:
    \begin{enumerate}
        \item Per ogni $a \in \R$ vale che $a \leq \abs*{a}$, $-a \leq \abs*{a}$.
        \item Per ogni $a, b \in \R$ vale che $\abs*{a + b} \leq \abs*{a} + \abs*{b}$.
    \end{enumerate}
\end{proposition}

\subsection{Completezza dei numeri reali}

\begin{definition} [Sezione]
    \label{def:sezione}
    Sia $\K$ un campo ordinato e siano $A, B \subseteq \K$ non vuoti. Allora si dice che $(A, B)$ è una sezione di $\K$ se \begin{enumerate}[label={(\roman*)}, ref={\thedefinition: (\roman*)}]
        \item \label{def:sezione:part} $A \inters B = \varnothing$ e $A \union B = \K$;
        \item \label{def:sezione:sx} $A$ sta a sinistra di $B$, ovvero per ogni $a \in A, b \in B$ vale che $a \leq b$.
    \end{enumerate}
\end{definition}

\begin{unnamed}[Assioma di Dedekind]
    \label{ax:dedekind}
    Sia $(A, B)$ una sezione di $\K$.

    Allora si dice che $\K$ è \emph{Dedekind-completo} (o semplicemente \emph{completo}) se esiste un unico elemento separatore per la sezione $(A, B)$, ovvero \[
        \exists! \xi \in \K. \quad a \leq \xi \leq b \qquad \forall a \in A, b \in B.
    \]
\end{unnamed}

\begin{theorem}
    [Esistenza dei numeri reali]
    Esiste una struttura $(\R, +, \cdot, \leq)$, detti \emph{numeri reali}, che è un campo ordinato completo.
\end{theorem}

\begin{theorem}
    [Unicità dei numeri reali]
    La struttura $(\R, +, \cdot, \leq)$ è unica a meno di isomorfismi; 
    
    ovvero se $(\R^\prime, +^\prime, \cdot^\prime, \leq^\prime)$ è un campo ordinato completo, allora esiste una bigezione $\phi : \R \to \R^\prime$ tale che per ogni $a, b \in \R$ \begin{enumerate}[label={(\roman*)}]
        \item $\phi(a + b) = \phi(a) +^\prime \phi(b)$,
        \item $\phi(a \cdot b) = \phi(a) \cdot^\prime \phi(b)$,
        \item se $a \leq b$ allora $\phi(a) \leq^\prime \phi(b)$.
    \end{enumerate}
\end{theorem}

\subsection{Proprietà dei numeri reali}

\begin{definition} [Maggioranti e minoranti]
    Sia $A \subseteq \R$, $A \neq \varnothing$. 
    
    Allora si dice che $M \in \R$ è un \emph{maggiorante} di $A$ se \[
        \forall a \in A. \quad M \geq a. 
    \] L'insieme dei maggioranti di un insieme $A$ si indica con $\MM(A)$.

    Ugualmente si dice che $m \in \R$ è un \emph{minorante} di $A$ se \[
        \forall a \in A. \quad m \leq a. 
    \] L'insieme dei minoranti di un insieme $A$ si indica con $\NN(A)$.
\end{definition}

\begin{definition} [Limitato superiormente o inferiormente]
    Sia $A \subseteq \R$, $A \neq \varnothing$. 
    
    Allora si dice che $A$ è \emph{limitato superiormente} se ammette almeno un maggiorante, ovvero se $\MM(A) \neq \varnothing$.

    Ugualmente si dice che $A$ è \emph{limitato inferiormente} se ammette almeno un minorante, ovvero se $\NN(A) \neq \varnothing$.

    Se $A$ è limitato sia superiormente che inferiormente allora si dice che $A$ è limitato.
\end{definition}

\begin{remark}
    Un insieme $A \subseteq \R$ è limitato se e solo se esiste un $M \in \R$ tale che \[
        \forall a \in A. \quad \abs*{a} \leq M    
    \]
\end{remark}

\begin{definition} [Massimo e minimo di un insieme]
    Sia $A \subseteq \R$, $A \neq \varnothing$. 

    Allora si dice che $M \in \R$ è il \emph{massimo} di $A$, e si scrive $M = \max A$, se
    \begin{enumerate}[label={(\roman*)}]
        \item $M$ è un maggiorante di $A$, ovvero per ogni $a \in A$ vale che $M \geq a$
        \item $M \in A$.
    \end{enumerate}
    
    Ugualmente si dice che $m \in \R$ è il \emph{minimo} di $A$, e si scrive $m = \min A$, se
    \begin{enumerate}[label={(\roman*)}]
        \item $m$ è un minorante di $A$, ovvero per ogni $a \in A$ vale che $m \leq a$
        \item $m \in A$.
    \end{enumerate}
\end{definition}

\begin{remark}
    Massimo e minimo possono non esistere (ad esempio l'insieme $(0, 1)$ non ha né massimo né minimo) ma se esistono allora sono unici.
\end{remark}

\begin{definition}
    [Estremo superiore ed estremo inferiore]
    Sia $A \subseteq \R$, $A \neq \varnothing$. 

    Allora si definisce l'estremo superiore di $A$, in simboli $\sup A$, nel seguente modo: \[
        \sup A = \begin{cases}
            +\infty,    &\text{se $A$ è illimitato superiormente,}\\
            \min \MM(A) &\text{altrimenti.}
        \end{cases}    
    \]
    
    Ugualmente si definisce l'estremo inferiore di $A$, in simboli $\inf A$: \[
        \inf A = \begin{cases}
            -\infty,    &\text{se $A$ è illimitato inferiormente,}\\
            \max \NN(A) &\text{altrimenti.}
        \end{cases}    
    \]
\end{definition}

\begin{theorem}
    [Esistenza dell'estremo superiore]
    Sia $A \subseteq \R$, $A \neq \varnothing$, $A$ limitato superiormente.

    Allora l'insieme dei maggioranti di $A$ ammette minimo.
\end{theorem}
\begin{proof}
    Siccome $A$ è limitato superiormente, $\MM(A) \neq \varnothing$. 
    
    Sia $B = \R \setminus \MM(A)$; mostriamo che $(B, \MM(A))$ è una sezione di $\R$.
    Ovviamente le proprietà in \ref{def:sezione:part} sono verificate, in quanto ogni $x \in \R$ è in $\MM(A)$ oppure in $B$.

    Siano ora $b \in B$ e $m \in \MM(A)$ e mostriamo che necessariamente $b < m$. Siccome $b$ non è un maggiorante di $A$ dovrà esistere un $a \in A$ tale che $b < a$. D'altra parte $m$ è un maggiorante di $A$, dunque $a \leq m$. Da ciò segue che $b < m$, che verifica la proprietà \ref{def:sezione:sx}.

    Per l'\nameref{ax:dedekind} esiste ed è unico l'elemento separatore. Sia quindi $\xi$ l'elemento separatore della sezione: mostriamo che $\xi = \min \MM(A)$.

    Abbiamo già mostrato che $\xi \leq m$ per ogni $m \in \MM(A)$, quindi dobbiamo soltanto mostrare che $\xi \in \MM(A)$. 
    
    Supponiamo per assurdo che $\xi$ non sia in $\MM(A)$, ovvero che $\xi$ non sia un maggiorante di $A$: allora dovrà esistere un $a \in A$ tale che $a > \xi$. Ma allora varrebbe che \[
        \xi < \frac{\xi + a}{2} < a.
    \] Essendo $\dfrac{\xi + a}{2} < a$ allora anche esso appartiene a $B$, da cui segue che $\xi$ non può essere l'elemento separatore di $(B, \MM(A))$, il che è assurdo.

    Dunque $\xi \in \MM(A)$, ovvero $\xi = \min \MM(A) = \sup A$.
\end{proof}

Si può dimostrare un analogo risultato per l'estremo inferiore:
\begin{theorem}
    [Esistenza dell'estremo inferiore]
    Sia $A \subseteq \R$, $A \neq \varnothing$, $A$ limitato inferiormente.

    Allora l'insieme dei minoranti di $A$ ammette massimo.
\end{theorem}

\begin{proposition}
    [Caratterizzazione dell'estremo superiore] \label{prop:caratt_sup}
    Sia $A \subseteq \R$, $A \neq \varnothing$, $A$ limitato superiormente.

    Allora se $L = \sup A$ segue che \begin{enumerate}[label={(\roman*)}, ref={caratterizzazione del sup: (\roman*)}]
        \item \label{prop:caratt_sup:magg} per ogni $a \in A$ vale che $a \leq L$;
        \item \label{prop:caratt_sup:min_magg} per ogni $\eps > 0$ esiste un $a \in A$ tale che $L - \eps < a$.
    \end{enumerate}
\end{proposition}
\begin{proof}
    Infatti la (i) dice che $L$ è un maggiorante di $A$; la (ii) dice che $L$ è il minimo maggiorante di $A$, ovvero che qualsiasi numero minore di $L$ non è maggiorante di $A$.
\end{proof}

\begin{proposition}
    [Caratterizzazione dell'estremo inferiore] \label{prop:caratt_inf}
    Sia $A \subseteq \R$, $A \neq \varnothing$, $A$ limitato inferiormente.

    Allora se $L = \inf A$ segue che \begin{enumerate}[label={(\roman*)}, ref={caratterizzazione dell'inf: (\roman*)}]
        \item \label{prop:caratt_inf:magg} per ogni $a \in A$ vale che $a \geq L$;
        \item \label{prop:caratt_inf:min_magg} per ogni $\eps > 0$ esiste un $a \in A$ tale che $L - \eps > a$.
    \end{enumerate}
\end{proposition}

\subsection{Retta reale estesa}

\begin{definition}
    [Retta reale estesa]
    Si definisce retta reale estesa l'insieme \[
        \closure{\R} = \R \union \set{-\infty, +\infty}    
    \] tali che $-\infty < a < +\infty$ per ogni $a \in \R$.
\end{definition}

La retta reale estesa non è un campo ordinato poiché non possiamo estendere le operazioni di somma e prodotto in modo che siano consistenti. Possiamo tuttavia definirle parzialmente in questo modo:
\begin{enumerate}
    \item $a + (+\infty) = +\infty$ per ogni $a \in \R$;
    \item $a + (-\infty) = -\infty$ per ogni $a \in \R$;
    \item $a(+\infty) = +\infty$ per ogni $a > 0$;
    \item $a(+\infty) = -\infty$ per ogni $a < 0$;
    \item $a(-\infty) = -\infty$ per ogni $a > 0$;
    \item $a(-\infty) = +\infty$ per ogni $a < 0$.
\end{enumerate}

I casi che non sono automaticamente determinati vengono chiamati \emph{forme indeterminate} oppure \emph{di indecisione}, e sono \begin{align*}
    &+\infty - \infty & \pm\infty \cdot 0
\end{align*} a cui si aggiungono $\dfrac{0}{0}$ e $\dfrac{\pm\infty}{\pm\infty}$ nello studio dei limiti.