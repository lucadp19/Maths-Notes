\section{Quozienti di anelli polinomiali}

In questa sezione studieremo i quozienti di anelli polinomiali. Innanzitutto abbiamo bisogno di una nozione equivalente a quella di gruppo normale.

\begin{definition}
    [Ideale] Sia $A$ un anello commutativo. Allora $I \subseteq A$ si dice \emph{ideale} se \begin{enumerate}
        \item $(I, +)$ è un sottogruppo di $(A, +)$;
        \item vale la \emph{proprietà di assorbimento}: per ogni $a \in A$, $x \in I$ vale che $ax \in I$.
    \end{enumerate}
\end{definition}

\begin{definition}
    [Ideale generato da un elemento]
    Sia $A$ un anello commutativo e sia $r \in A$. Allora si dice \emph{ideale generato da $r$} l'ideale \[
        \ideal{a} \deq \set*{ra \given a \in A}.    
    \]
\end{definition}

Nel caso degli anelli polinomiali, come $\K[X]$, gli ideali generati da un polinomio $f$ assumono la forma \[
    \ideal[\big]{f(X)} = \set*{f(X)\cdot a(X) \given a \in \K[X]}.    
\] Siccome $\K[X]$ forma un gruppo abeliano con l'operazione di somma abbiamo automaticamente che $\ideal[\big]{f(X)} \normal \K[X]$, dunque possiamo definire il gruppo quoziente \[
    \quot{\K[X]}{\ideal[\big]{f(X)}} \deq \set*{p(X) + \ideal[\big]{f(X)} \given p(X) \in \K[X]}.
\] Su questo gruppo è automaticamente definita un'operazione di somma \[
    p(X) + \ideal[\big]{f(X)} + q(X) + \ideal[\big]{f(X)} = p(X) + q(X) + \ideal[\big]{f(X)};
\] tuttavia, possiamo anche definire un'operazione di prodotto tra classi laterali: \[
    \left( p(X) + \ideal[\big]{f(X)} \right)\left( q(X) + \ideal[\big]{f(X)} \right) = p(X)q(X) + \ideal[\big]{f(X)}.
\] 

\begin{theorem}
    La struttura $\left( \quot{\K[X]}{\ideal[\big]{f(X)}}, +, \cdot \right)$ è un anello commutativo con identità.
\end{theorem}
\begin{proof}
    Basta verificare gli assiomi degli anelli. Lo zero dell'anello è dato da $\ideal[\big]{f(X)}$, mentre l'identità è data da $1 + \ideal[\big]{f(X)}$.
\end{proof}

Per semplicità definiamo $\eqcl+{a(X)} \deq a(X) + \ideal[\big]{f(X)}$, esattamente allo stesso modo come abbiamo fatto nel caso degli interi e le classi resto.
Prima di dimostrare alcune proprietà importanti di questo anello, mostriamo il seguente lemma:
\begin{lemma}{resto_in_ideale}
    Siano $f, r \in \K[X]$ con $r = 0_{\K[X]}$ oppure $\deg r < \deg f$. Allora $r \in \ideal[\big]{f(X)}$ se e solo se $r = 0_{\K[X]}$.
\end{lemma}
\begin{proof}
    I polinomi di $\ideal[\big]{f(X)}$ sono tutti e solo i multipli di $f(X)$, dunque se non sono nulli hanno grado maggiore o uguale al grado di $f$, da cui segue che $r$ deve essere il polinomio nullo.
\end{proof}

\begin{theorem}{caratt_quot_polinomi}
    Sia $f \in \K[X]$ e sia $n \deq \deg f$. Allora \begin{enumerate}[label={(\roman*)}]
        \item un insieme minimale di rappresentanti dell'anello quoziente $\quot{\K[X]}{\ideal[\big]{f(X)}}$ è dato dall'insieme di tutti i possibili resti delle divisioni per $f$, ovvero da tutti e soli i polinomi $r \in \K[X]$ tali che $r = 0_{\K[X]}$ oppure $0 \leq \deg r < n$;
        \item l'anello quoziente è un $\K$-spazio vettoriale di dimensione $n$ e in particolare una sua base è data da \[
            \left( \eqcl+{1}, \dots, \eqcl+{X^{n-1}} \right).
        \]
    \end{enumerate}
\end{theorem}
\begin{proof}
    Mostriamo innanzitutto che l'insieme dei possibili resti è un insieme di rappresentanti. Sia $a \in \K[X]$ un polinomio qualunque. Per il \hyperref[th:divisione_euclidea_KX]{Teorema di Divisione Euclidea} esisteranno due polinomi $q, r \in \K[X]$, con $r = 0_{\K[X]}$ oppure $0 \leq \deg r < n$ tali che \[
        a(X) = q(X)f(X) + r(X).    
    \] Ma allora vale che \[
        a(X) + \ideal[\big]{f(X)} = r(X) + \overbrace{q(X)f(X)}^{\in \ideal[\big]{f(X)}} +  \ideal[\big]{f(X)} = a(X) + \ideal[\big]{f(X)},
    \] ovvero $\eqcl+ a = \eqcl+ r$.

    Mostriamo inoltre che l'insieme dei resti è un insieme di rappresentanti minimale, ovvero che se due resti $r_1, r_2 \in \K[X]$ (con $\deg r_1 < n, \deg r_2 < n$) rappresentano la stessa classe di equivalenza, allora devono essere uguali.
    \begin{align*}
        &r_1(X) + \ideal[\big]{f(X)} = r_2 + \ideal[\big]{f(X)} \\
        \iff &r_1(X) - r_2(X) \in \ideal[\big]{f(X)} \tag{per il \Cref{lem:resto_in_ideale}}\\
        \iff &r_1(X) - r_2(X) = 0_{\K[X]}\\
        \iff &r_1(X) = r_2(X).
    \end{align*}

    Da questo segue direttamente che $\left( \eqcl+ 1, \dots, \eqcl+{X^{n-1}} \right)$ sono un insieme di generatori per $\quot{K[X]}{\ideal[\big]{f(X)}}$: infatti per ogni $\eqcl+ a \in \quot{\K[X]}{\ideal[\big]{f(X)}}$ segue che esiste un polinomio $r$ di grado minore di $n$ tale che $\eqcl+ a = \eqcl+ r$. Siccome $\deg r < n$ esso può essere espresso come combinazione lineare di $\left( \eqcl+ 1, \dots, \eqcl+{X^{n-1}} \right)$, da cui segue che \[
        \eqcl+ a = \eqcl+ r \in \Span*{\eqcl+ 1, \dots, \eqcl+{X^{n-1}}}.
    \] Inoltre questi vettori sono linearmente indipendenti. Per mostrarlo consideriamo una loro combinazione lineare e poniamola uguale a $\eqcl+ 0$: \begin{align*}
        \sum_{i = 0}^{n-1} a_i\eqcl+{X^i} = \eqcl+ 0.
    \end{align*} Sia $\eqcl+{r(X)} = \sum_{i = 0}^{n-1} a_i\eqcl+{X^i}$. Sicuramente $r = 0_{\K[X]}$ oppure $\deg r < n$, dunque per il \Cref{lem:resto_in_ideale} segue che $r = 0_{\K[X]}$, ovvero $a_1 = \dots = a_{n-1} = 0$, il che significa che i vettori $\eqcl+{X^i}$ sono indipendenti e dunque formano una base dello spazio vettoriale.
\end{proof}

\begin{proposition}
    [Divisori di zero e invertibili in $\quot{\K[X]}{\ideal[\big]{f(X)}}$]
    {invertibili_divzero_in_KX/f}
    Siano $f \in \K[X]$, $\eqcl+{a(X)} \in \quot{\K[X]}{\ideal[\big]{f(X)}}$.
    Allora \begin{enumerate}[label={(\roman*)}]
        \item $\eqcl+ a$ è invertibile se e solo se $\gcd{a(x)}{f(x)} = 1$;
        \item $\eqcl+ a$ è divisore di zero se e solo se $\gcd{a(x)}{f(x)} \neq 1$.
    \end{enumerate}
    In particolare ogni elemento di $\quot{\K[X]}{\ideal[\big]{f(X)}}$ è invertibile oppure divisore di zero.
\end{proposition}
\begin{proof}
    Dimostriamo separatamente le due affermazioni.
    \begin{enumerate}[label={(\roman*)}]
        \item Il massimo comun divisore tra $a$ e $f$ è $1$ se e solo se esistono due polinomi $h, k \in \K[X]$ tali che\[
            a(X)h(X) + f(X)k(X) = 1.    
        \] Riducendo tutto modulo $\ideal[\big]{f(X)}$ otteniamo \begin{align*}
            &\eqcl+{a(X)h(X)} + \eqcl+{f(X)k(X)} = \eqcl+ 1,\\
            \intertext{ma siccome $\eqcl+{f(X)k(X)} = \eqcl+ 0$ poiché $f(X)k(X) \in \ideal[\big]{f(X)}$}
            \iff &\eqcl+{a(X)h(X)} = \eqcl+ 1 \\
            \iff &\eqcl+{a(X)}\cdot\eqcl+{h(X)} = \eqcl+ 1,
        \end{align*} ovvero se e solo se $\eqcl+ a$ è invertibile.
        \item Supponiamo $\gcd{a(X)}{f(X)} = d(X)$ con $\deg d \geq 1$. Sia $b(X) \deq \frac{f(X)}{d(X)} \in \K[X]$ con $\deg b < \deg f$.
        
        Sicuramente $\eqcl+ b \neq 0_{\K[X]}$, tuttavia $\eqcl+{a(X)b(X)} = \eqcl+ 0$ poiché \[
            f(X) \divides a(X)b(X) = \frac{a(X)}{d(X)}f(X)
        \] e $\frac{a(X)}{d(X)} \in \K[X]$ poiché $d$ è un divisore di $a$.

        Viceversa se $\eqcl+ a$ è divisore di zero allora dovrà esistere $\eqcl+ b \in \quot{\K[X]}{\ideal[\big]{f(X)}}$, con $\eqcl+ b \neq \eqcl+ 0$, tale che \[
            \eqcl+{a(X)b(X)} = \eqcl+ 0.    
        \] Questo implica che $f(X) \divides a(X)b(X)$, ma siccome $f(X) \ndivides b(X)$ (altrimenti $b$ sarebbe nella classe di $0_{\K[X]}$) segue che $f(X) \divides a(X)$, ovvero $\eqcl+ a = \eqcl+ 0$.
    \end{enumerate}
\end{proof}

\begin{corollary}
    Sia $f \in \K[X]$. Allora vale che $\quot{\K[X]}{\ideal[\big]{f(X)}}$ è un campo se e solo se $f$ è irriducibile in $\K[X]$.
\end{corollary}
\begin{proof}
    Il quoziente $\quot{\K[X]}{\ideal[\big]{f(X)}}$ è un campo se e solo se tutti i suoi elementi non nulli sono invertibili, ovvero (per la \Cref{prop:invertibili_divzero_in_KX/f}) se e solo se per ogni polinomio $a \in \K[X]$ vale che $\gcd{a(X)}{f(X)} = 1$, ovvero se e solo se $f$ è irriducibile.
\end{proof}