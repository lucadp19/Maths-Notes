\section{Teoremi di Omomorfismo}

\subsection{Primo Teorema degli Omomorfismi}
\begin{theorem}
    [Primo Teorema degli Omomorfismi] \label{th:first_iso}
    Siano $(G, \cdot)$, $(G^\prime, *)$ due gruppi e sia $f : G \to G^\prime$ un omomorfismo di gruppi. Sia inoltre $N \normal G$, $N \subseteq \ker f$.

    Allora esiste un unico omomorfismo $\phi : \quot{G}{N} \to G^\prime$ per cui il seguente diagramma commuta:
    \begin{equation}
        \begin{tikzcd}
            G \arrow[d, swap, "\pi_N"] \arrow[r, "f"] & G^\prime \\
            \quot{G}{N} \arrow[ur, swap, "\phi"] &
        \end{tikzcd}
    \end{equation}
    Inoltre vale che \begin{align*}
        \Imm{f} = \Imm{\phi}, \quad \ker \phi = \quot{\ker f}{N}.
    \end{align*}
\end{theorem}
\begin{proof}
    Notiamo che se $\phi$ esiste allora è necessariamente unica. Infatti se $\phi$ rende il diagramma commutativo significa che $f = \phi \circ \pi_N$, da cui segue che per ogni $x \in G$ \begin{align*}
        f(x) &= (\phi \circ \pi_N)(x) \\
        &= \phi(\pi_N(x))\\
        &= \phi(xN).
    \end{align*}
    Questa equazione assegna a $\phi$ un valore per ogni elemento del dominio $\quot{G}{N}$, da cui segue l'unicità.

    Mostriamo dunque che la funzione \begin{align*}
        \phi : \quot{G}{N} &\to G^\prime \\
        gH &\mapsto f(g)
    \end{align*} è ben definita ed è un omomorfismo di gruppi. Inoltre verifichiamo le due proprietà dell'immagine e del nucleo.
    \begin{description}
        \item[Buona definizione] Siano $x, y$ tali che $xN = yN$. Dato che esse rappresentano classi di equivalenza, ciò significa che $x \in yN$. 
        
        Sia dunque $n \in N$ tale che $x = yn$. Allora vale che \begin{align*}
            f(x) &= f(yn) \tag{$f$ è omo.}\\
            &= f(y) * f(n) \tag{$N \subseteq \ker f$}\\
            &= f(y) * e^\prime \\
            &= f(y).
        \end{align*} Dunque segue che \begin{align*}
            \phi(xN) = f(x) = f(y) = \phi(yN),
        \end{align*} ovvero $\phi$ è ben definita.
        \item[Omomorfismo] Siano $xN, yN \in \quot{G}{N}$ e mostriamo che \[
            \phi(xN \cdot yN) = \phi(xN) * \phi(yN).    
        \] Infatti vale che \begin{align*}
            \phi(xN \cdot yN) &= \phi(xyN)\\
            &= f(xy) \tag{$f$ è omo.}\\
            &= f(x) * f(y)\\
            &= \phi(xN) * \phi(yN).
        \end{align*}
        \item[Proprietà delle immagini] Per definizione \begin{align*}
            \Imm \phi &= \set*{\phi(xN) \given xN \in \quot{G}{N}}\\
            &= \set*{f(x) \given xN \in \quot{G}{N}}.
            \intertext{Tuttavia, come abbiamo verificato nella parte relativa alla buona definizione di $\phi$, se $xN = yN$ allora $f(x) = f(y)$, dunque vale che}
            \Imm \phi &= \set*{f(x) \given x \in G}\\
            &= \Imm f.
        \end{align*}
        \item[Proprietà dei nuclei] Per definizione \begin{align*}
            \ker \phi &= \set*{xN \in \quot{G}{N} \given \phi(xN) = e^\prime}\\
            &= \set*{xN \in \quot{G}{N} \given f(x) = e^\prime}\\
            &= \set*{xN \in \quot{G}{N} \given x \in \ker f}\\
            &= \quot{\ker f}{N}. \tag*{\qedhere}
        \end{align*}
    \end{description}
\end{proof}

Nel caso particolare in cui $N = \ker f$ abbiamo che $\phi$ è iniettiva, come ci assicura il seguente corollario.

\begin{corollary}
    Siano $(G, \cdot)$, $(G^\prime, *)$ due gruppi e sia $f : G \to G^\prime$ un omomorfismo di gruppi. 
    
    Allora esiste un unico omomorfismo $\phi$ tale che il seguente diagramma commuta:
    \begin{equation}
        \begin{tikzcd}
            G \arrow[d, swap, "\pi_{\ker f}"] \arrow[r, "f"] & G^\prime \\
            \quot{G}{\ker f} \arrow[ur, hook, swap, "\phi"] &
        \end{tikzcd}
    \end{equation}

    In particolare $\phi$ è iniettivo, dunque ogni omomorfismo è fattorizzabile come composizione di un omomorfismo surgettivo e uno iniettivo.
\end{corollary}
\begin{proof}
    Siccome $\ker f \subseteq \ker f$ e $\ker f \normal G$ possiamo applicare il \nameref{th:first_iso}, da cui segue che esiste un unico omomorfismo $\phi$ tale che \[
        f = \phi \circ \pi_{\ker f}.    
    \]

    \paragraph{Iniettività di $\phi$} Per definizione di $\phi$ vale che $\phi(x\ker f) = e_{G^\prime}$ se e solo se $f(x) = e_{G^\prime}$, ovvero se e solo se $x \in \ker f$. Dunque il nucleo di $\phi$ è $\ker f$, che è l'elemento neutro del gruppo quoziente $\quot{G}{\ker f}$, da cui segue che $\phi$ è iniettiva. 

    Essendo inoltre $\pi_{\ker f}$ surgettivo segue la tesi.
\end{proof}

La fattorizzazione definita dal precedente corollario può essere resa ancora più precisa specificando un oggetto intermedio, l'immagine di $f$: l'omomorfismo $f$ viene quindi scomposto nella composizione di un omomorfismo surgettivo (la proiezione canonica modulo il kernel, ovvero $\pi_{\ker f}$), un isomorfismo e infine un omomorfismo iniettivo (l'inclusione canonica $\iota : \Imm f \to G^\prime$, $\iota(g) = g$). 

L'isomorfismo è proprio l'omomorfismo $\phi$ del \nameref{th:first_iso}: infatti per l'osservazione precedente $\phi$ è iniettivo; inoltre restringendo il codominio a $\Imm f$ e sapendo che $\Imm \phi = \Imm f$ segue che $\phi$ è anche surgettivo, rendendolo un isomorfismo.

Il seguente diagramma dunque commuta:
\begin{equation}
    \begin{tikzcd}
        G \arrow[r, swap, two heads, "\pi_{\ker f}"] \arrow[rrr, bend left, "f"] 
        & \quot{G}{\ker f} \arrow[r, hook, two heads, swap, "\phi"] 
        & \Imm f \arrow[r, hook, swap, "\iota"] 
        & G^\prime
    \end{tikzcd}
\end{equation}

Vale dunque il seguente corollario.
\begin{corollary}\label{cor:G/ker=Imm}
    Siano $(G, \cdot)$, $(G^\prime, *)$ due gruppi e sia $f : G \to G^\prime$ un omomorfismo di gruppi. Allora \begin{equation}
        \quot{G}{\ker f} \isomorph \Imm f.
    \end{equation}
\end{corollary}

\subsection{Secondo Teorema degli Omomorfismi}
\begin{theorem}
    [Secondo Teorema degli Omomorfismi] \label{th:second_iso}
    Sia $(G, \cdot)$ un gruppo e siano $H, K \normal G$, con $H \subseteq K$. Allora \begin{equation}
        \nicefrac{\quot{G}{H}}{\quot{K}{H}} \isomorph \quot{G}{K}.
    \end{equation}
\end{theorem}
\begin{proof}
    Consideriamo le proiezioni canoniche $\pi_H$ e $\pi_K$. Siccome $H \subseteq K = \ker \pi_K$ possiamo applicare il \nameref{th:first_iso} all'omomorfismo $\pi_K$ e al sottogruppo normale $H \normal G$ (tramite la proiezione $\pi_H$). Dunque esiste un unico omomorfismo \begin{align*}
        \phi : \quot{G}{H} &\to \quot{G}{K}\\
        gH &\mapsto gK
    \end{align*} che fa commutare il seguente diagramma:
    \begin{equation*}
        \begin{tikzcd}
            G \arrow[d, swap, "\pi_{H}"] \arrow[r, "\pi_K"] & \quot{G}{K} \\
            \quot{G}{H} \arrow[ur, two heads, swap, "\phi"] &
        \end{tikzcd}
    \end{equation*}
    Tale funzione è anche surgettiva, in quanto per il \nameref{th:first_iso} sappiamo che $\Imm \phi = \Imm \pi_K$, e $\pi_K$ è surgettiva. Inoltre \[
        \ker \phi = \quot{\ker \pi_K}{H} = \quot{K}{H}.    
    \]

    Consideriamo ora i gruppi $\quot{G}{H}$ e $\quot{G}{K}$ e il sottogruppo $\nicefrac{\quot{G}{H}}{\ker \phi}$, che corrisponde a $\nicefrac{\quot{G}{H}}{\quot{K}{H}}$. Per il \nameref{th:first_iso} esiste un unico omomorfismo \begin{align*}
        \tilde{\phi} : \nicefrac{\quot{G}{H}}{\quot{K}{H}} &\to \quot{G}{K}
    \end{align*} che fa commutare il seguente diagramma:
    \begin{equation*}
        \begin{tikzcd}
            \quot{G}{H} \arrow[r, two heads, "\phi"] \arrow[d, swap, "\pi_{\quot{K}{H}}"] & \quot{G}{K} \\
            \nicefrac{\quot{G}{H}}{\quot{K}{H}} \arrow[ur, hook, two heads, swap, "\widetilde{\phi}"] &
        \end{tikzcd}
    \end{equation*}

    $\widetilde{\phi}$ è un isomorfismo di gruppi: infatti essendo $\phi$ surgettivo anche $\widetilde{\phi}$ lo è; inoltre la proiezione $\pi_{\quot{K}{H}}$ porta il gruppo $\quot{G}{H}$ nel quoziente modulo $\ker \phi = \quot{K}{H}$, dunque l'omomorfismo $\widetilde{\phi}$ è iniettivo ed è dunque un isomorfismo di gruppi.

    Segue quindi che \[
        \nicefrac{\quot{G}{H}}{\quot{K}{H}} \isomorph \quot{G}{K}. \qedhere 
    \]
\end{proof}

\subsection{Terzo Teorema degli Omomorfismi}
\begin{theorem}
    [Terzo Teorema degli Omomorfismi] Sia $(G, \cdot)$ un gruppo e siano $H \sgr G$,$N \normal G$. 
    Valgono le seguenti affermazioni: \begin{itemize}
        \item $N$ è un sottogruppo normale di $HN$,
        \item $H \inters N$ è un sottogruppo normale di $H$,
        \item inoltre \begin{equation}
            \frac{H}{H \inters N} \isomorph \frac{HN}{N}.
        \end{equation}
    \end{itemize}
\end{theorem}
\begin{proof}
    Dimostriamo innanzitutto le due condizioni di normalità.
    \begin{description}
        \item[($N \normal HN$)] Mostriamo innanzitutto che $HN$ è un sottogruppo di $G$. Per la \autoref{prop:cond_prod_sgr_e'_sgr}, è sufficiente mostrare che $HN = NH$. 
        Siccome $N$ è normale in $G$ segue che $gN = Ng$ per ogni $g \in G$. Dato che $H \subseteq G$ segue che $hN = Nh$ per ogni $h \in H$, ovvero $HN = NH$.
        Dunque $HN$ è un sottogruppo di $G$.

        Notiamo inoltre che $N \subseteq HN$ (basta scegliere tutti gli elementi della forma $e_Gn$ al variare di $n \in N$), dunque essendo $N$ normale in $G$ segue che $N$ è normale in ogni sottogruppo di $G$ che lo contiene; in particolare $N \normal HN$.
        \item[($H \inters N \normal H$)] Sia $n \in H \inters N$ e sia $g \in H$.
        
        Ovviamente $gng\inv \in H$, in quanto $n$ ed $g$ sono entrambi elementi di $H$. Inoltre essendo $N$ un sottogruppo normale di $G$ segue che $gng\inv \in N$ per ogni $g \in G$, dunque a maggior ragione per ogni $g \in H \subseteq G$. 
        
        Dunque $gng\inv \in H \inters N$, da cui segue che $H \inters N$ è normale in $H$.
    \end{description}
    Consideriamo ora l'applicazione 
    \begin{align*}
        \begin{split}
            f : H &\to \quot{HN}{N} \\
            h &\mapsto hN.
        \end{split}
    \end{align*}  

    Quest'applicazione è una restrizione all'insieme $H \subseteq HN$ della proiezione canonica \[
        \pi_{N} : HN \to \quot{HN}{N};    
    \] questo ci garantisce che $f$ è ben definita e che è un omomorfismo di gruppi.
    
    Inoltre $f$ è surgettiva: basta mostrare che \begin{gather*}
        \Imm f = \quot{HN}{N}\\
        \intertext{il che equivale a}
        \set*{hN \in \quot{HN}{N} \given h \in H} = \set*{yN \in \quot{HN}{N} \given y \in HN}.
    \end{gather*}
    L'inclusione $\Imm f \subseteq \quot{HN}{N}$ è data dalla definizione; l'inclusione contraria viene dal fatto che se $yN \in \quot{HN}{N}$, ovvero $y = hn$ per qualche $hn \in HN$, allora $yN = hnN \in \set*{hN \given h \in H}$ in quanto $nN = N$.

    Inoltre \begin{align*}
        \ker f &= \set*{h \in H \given f(h) = N} \\
        &= \set*{h \in H \given hN = N} \\
        &= \set*{h \in H \given h \in N} \\
        &= H \inters N.
    \end{align*}

    Dunque per il \hyperref[cor:G/ker=Imm]{Corollario al Primo Teorema degli Omomorfismi} segue che \[
        \frac{H}{H \inters N} \isomorph \Imm f = \frac{HN}{N}. \qedhere    
    \]
\end{proof}

Prima di studiare il Teorema di Corrispondenza, introduciamo un lemma che ci sarà utile:
\begin{lemma}\label{lem:controimm_normal}
    Siano $(G, \cdot)$, $(G^\prime, \cdot)$ due gruppi e sia $f : G \to G^\prime$ un omomorfismo. Se $K \normal G^\prime$, allora $f\inv(K) \normal G$.

    Inoltre se $f$ è surgettivo e $H \normal G$ segue che \[
        f(H) \normal G^\prime = f(G).    
    \]
\end{lemma}

\begin{theorem}[Teorema di Corrispondenza tra Sottogruppi]
    Sia $(G, \cdot)$ un gruppo e $N \normal G$. Sia $\GG$ l'insieme dei sottogruppi di $G$ che contengono $N$ e $\NN$ l'insieme dei sottogruppi di $\quot{G}{N}$.

    Allora esiste una corrispondenza biunivoca tra $\GG$ e $\NN$ che preserva l'indice di sottogruppo e i sottogruppi normali, ovvero esiste una funzione \begin{align*}
        \begin{split}
            \psi : \GG &\to \NN\\
            A &\mapsto \quot{A}{N}
        \end{split}
    \end{align*}    
    tale che \begin{itemize}
        \item $\GrpIndex{G:A} = \GrpIndex{\quot{G}{N} : \quot{A}{N}}$,
        \item se $A \normal G$ allora $\quot{A}{N} \normal \quot{G}{N}$.
    \end{itemize}
\end{theorem}
Prima di iniziare la dimostrazione, osserviamo che siccome la proiezione canonica è un omomorfismo, vale che \[
    \pi(H) \sgr \quot{G}{N}, \qquad \pi\inv(K) \sgr G    
\] per ogni $H \sgr G$, $K \sgr \quot{G}{N}$.
\begin{proof}
    Siano $\alpha$ e $\beta$ le mappe date da:
    \begin{align*}
        X &\biject Y\\
        H &\xmapsto{\alpha} \quot{H}{N} = \pi_N(H)\\
        \pi_N\inv(K) &\xmapsfrom{\beta} K.
    \end{align*}

    \paragraph{Buona definizione} $\alpha$ è ben definita poiché l'immagine di un sottogruppo attraverso la proiezione canonica è un sottogruppo: \[
        \alpha(H) = \pi_N(H) = \quot{H}{N} \sgr \quot{G}{N}.    
    \] Mostriamo quindi che $\beta$ è ben definita: sia $K \sgr \quot{G}{N}$ e mostriamo che $\beta(K) = \pi_N\inv(K)$ è un sottogruppo di $G$ che contiene $N$. Siccome $\quot{G}{N}$ è il quoziente modulo $N$ la sua identità è $N = eN$; per definizione di sottogruppo ogni elemento di $\NN$ dovrà contenere l'identità del gruppo, ovvero $N$. Segue quindi che \[
        N = \pi_N\inv(N) \subseteq \pi_N\inv(K),
    \] da cui $\pi_N\inv(K) \in \GG$.

    \paragraph{Le due funzioni sono una l'inversa dell'altra} Mostriamo che $\alpha \circ \beta = \id$. Sia $K \in \NN$: allora \[
        (\alpha \circ \beta)(K) = \alpha\parens[\big]{\pi\inv(K)} = \pi\parens[\big]{\pi\inv(K)} = K,
    \] dove il penultimo passaggio viene dal fatto che $\pi$ è surgettiva, e quindi invertibile da destra.

    Mostriamo ora che $\beta \circ \alpha = \id$. Sia $H \in \GG$: allora
    \begin{align*}
           (\beta \circ \alpha)(H) 
        &= \beta(\pi(H))\\
        &= \beta\parens[\big]{\quot{H}{N}} \\
        &= \pi_N\inv\parens[\big]{\quot{H}{N}}\\
        &= \set*{x \in G \given \pi_N(x) \in \quot{H}{N}}\\
        &= \set*{x \in G \given xN \in \quot{H}{N}}\\
        &= \set*{x \in G \given x \in H}\\
        &= H.
    \end{align*}

    \paragraph{La bigezione preserva i sottogruppi normali} Sia $H \in \GG$; mostriamo che \[
        H \normal G \iff \quot{H}{N} \normal \quot{G}{N}.    
    \] \begin{description}
        \item[($\implies$)] Segue dal \nameref{th:second_iso}. Infatti siccome $N, H \normal G$ e $N \subseteq H$ segue che \[
            \frac{\quot{G}{N}}{\quot{H}{N}} \isomorph \quot{G}{H}.    
        \] Ma questo significa che $\dfrac{\quot{G}{N}}{\quot{H}{N}}$ è un gruppo, da cui segue che \[
            \quot{H}{N} \normal \quot{G}{N}.
        \]
        \item[($\impliedby$)] Segue dal \autoref{lem:controimm_normal}.
    \end{description}

    \paragraph{La bigezione conserva l'indice di sottogruppo} Sia $H \in \GG$: mostriamo che \[
        \GrpIndex{G : H} = \GrpIndex{\quot{G}{N} : \quot{H}{N}}.    
    \] Siano $x, y \in G$ qualsiasi. Mostriamo che le classi laterali $xH$ e $yH$ sono uguali se e solo se \[
        (xN)\quot{H}{N} = (yN)\quot{H}{N}.    
    \] Per definizione \[
        (xN)\quot{H}{N} = \set*{xNhN \given h \in H} = \set*{xhN \given h \in H};    
    \] allo stesso modo \[
        (yN)\quot{H}{N} = \set*{yhN \given h \in H}.    
    \] FINIRE
\end{proof}
