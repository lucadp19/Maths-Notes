\section{Divisibilità}

Consideriamo la relazione di divisibilità tra numeri interi:
\begin{definition}
    [Divisibilità]
    Siano $a, b \in \Z$. Allora si dice che \emph{$a$ divide $b$} (e si indica con $a \divides b$) se \[
        a = kb
    \] per qualche $k \in \Z$.
\end{definition}

\begin{proposition}
    [Divisibilità come relazione d'ordine]
    La relazione di divisibilità tra numeri interi è una relazione di ordine parziale su $\N \setminus \set{0}$.
\end{proposition}
\begin{proof}
    Per definizione di relazione d'ordine dobbiamo mostrare tre cose:
    \begin{description}
        \item[Riflessività] Sia $a \in \N$ non nullo. Allora $a \divides a$ poiché $a = 1\cdot a$.
        \item[Simmetria] Siano $a, b \in \N$ non nulli e supponiamo che $a \divides b$ e $b \divides a$. Allora per definizione di divisibilità segue che \[
            a = kb, \quad b = ha    
        \]  per qualche $k, h \in \Z$. Sostituendo la seconda equazione nella prima otteniamo $a = kha$, ovvero $kh = 1$. Ma dato che $a, b \in \N$ segue che $k = h = 1$, dunque $a = b$.
        \item[Transitività] Siano $a, b, c \in \N$ non nulli tali che $a \divides b$ e $b \divides c$. Allora per definizione vale che \[
            a = kb, \quad b = hc    
        \] per qualche $k, h \in \Z$.

        Sostituendo la seconda nella prima ottengo quindi $a = khc$, ovvero $a \divides c$ in quanto $kh \in \Z$.
    \end{description}
\end{proof}

La relazione di divisibilità può essere pensata anche come un ordinamento su $\Z \setminus {0}$, ma l'antisimmetria è "a meno del segno", ovvero \[
    a \divides b, b \divides a \implies a = b \text{ oppure } a = -b.
\] In questi casi scriveremo più semplicemente $a = \pm b$ per indicare che $a$ può essere $b$ oppure il suo opposto.

\begin{definition}
    [Massimo comun divisore]
    Siano $a, b \in \Z$ non nulli. Si dice \emph{massimo comun divisore di $a,b$} il numero $d \in \Z$ tale che \begin{enumerate}[label={(\roman*)}]
        \item $d \divides a$ e $d \divides b$;
        \item se $c \divides a$ e $c \divides b$ allora $c \divides d$.
    \end{enumerate}

    Tale $d$ si indica anche con $\operatorname{mcd}(a, b)$, oppure con $\gcd(a, b)$ oppure anche con $\mcd{a}{b}$.
\end{definition}

\begin{theorem}
    [Esistenza ed unicità del massimo comun divisore]
    Siano $a, b \in \Z$ non nulli. Allora esiste ed è unico (a meno del segno) $d \in \Z$ tale che $d = \mcd{a}{b}$.
\end{theorem}
\begin{proof}
    Mostriamo sia l'esistenza che l'unicità del massimo comun divisore.
    \begin{description}
        \item[Esistenza] Sia $X$ il sottoinsieme di $\Z$ tale che \[
            X \deq \set{ax + by \suchthat x, y \in \Z}    
        \] e sia $Y \deq X \inters \N \setminus \set{0}$. Notiamo che $Y \subseteq \N$ e $Y \neq \varnothing$, in quanto \begin{itemize}
            \item se $a > 0$ allora posso scegliere $x = 1 - b$, $y = a$ da cui segue che \[
                ax + by = a(1-b)+ab = a - ab + ab = a > 0
            \] cioè $a \in X$,
            \item se $a < 0$ posso scegliere $x = - 1 - b$ e $y = a$, da cui \[
                ax + by = a(-1-b)+ab = -a - ab + ab = -a > 0 
            \] da cui segue $-a \in X$.
        \end{itemize}

        Da ciò segue che per il Principio del Minimo l'insieme $Y$ ammette minimo. Sia $d \deq \min Y$. Mostro ora che $d = \mcd{a}{b}$.

        NOtiamo che siccome $d \in Y$ allora dovranno esistere $x_0, y_0 \in \Z$ tali che $d = ax_0 + by_0$.
        \begin{enumerate}[label={(\roman*)}]
            \item Dimostro che $d \divides a$; per simmetria è sufficiente dimostrare questa parte.
            
            Per la divisione euclidea scrivo \begin{equation}
                a = qd + r \quad \text{per qualche } 0 \leq r < \abs{a}.
            \end{equation} Allora \begin{equation*}
                0 \leq r = a - qd = a - q(ax_0 + by_0) = a(1 - qx_0) + b(-qy_0)
            \end{equation*}
            Dunque $r = 0$ oppure $r \in Y$. Tuttavia abbiamo supposto che $d$ fosse il minimo di $Y$, dunque siccome $r < d$ la seconda opzione è impossibile. Quindi $r = 0$, da cui segue che $a = qd$, ovvero $d \divides a$.
            \item Dimostro ora che per ogni $c$ che divide sia $a$ che $b$ segue che $c \divides d$.
            
            Per definizione di divisibilità sappiamo che esistono $h, k \in \Z$ tali che \[
                a = hc, \quad b = kc.    
            \] Da ciò segue che \[
                d = ax_0 + by_0 = c(hx_0 + ky_0)    
            \] e siccome $hx_0 + ky_0 \in \Z$ segue che $c \divides d$.
        \end{enumerate}
        Dunque $d$ è il massimo comun divisore tra i numeri $a$ e $b$.
        \item[Unicità] Supponiamo che esistano $d, d^\prime \in \Z$ che siano entrambi massimi comun divisiori di $a$ e $b$. Dunque dovranno valere le seguenti proprietà: siccome $d$ è un massimo comun divisore dovrà valere che \begin{enumerate}[label={(\roman*)}]
            \item $d \divides a$ e $d \divides b$
        \end{enumerate}
    \end{description}
\end{proof}