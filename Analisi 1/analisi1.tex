\documentclass[italian,oneside,headinclude,10pt]{scrbook}
% \documentclass[italian,oneside,headinclude,10pt]{book}
    \usepackage[utf8]{inputenc}
    \usepackage[italian]{babel}
    \usepackage[T1]{fontenc}
    \usepackage{textcomp, microtype}
    \usepackage{amsmath, amsthm, amssymb, cases, mathtools, bm}
    \usepackage{array, multicol}
    \usepackage{centernot}
    \usepackage{faktor}

    \usepackage{xparse}

    \usepackage{enumitem}
    \usepackage{float}
    \usepackage{letltxmacro}

    \LetLtxMacro\amsproof\proof
    \LetLtxMacro\amsendproof\endproof

    
    \usepackage[style=arsclassica, pdfspacing, eulermath]{classicthesis}
    
    % \usepackage{lmodern}

    % \usepackage{mathpazo}
    % \usepackage{eulervm}

    % \usepackage{eulerpx}
    
    \usepackage{thmtools}
    \usepackage[framemethod=TikZ]{mdframed}
    \usepackage{hyperref} % penultimo package da caricare!
    \usepackage{cleveref} % ultimo package da caricare!

\restylefloat{table}

\makeatletter %only needed in preamble
\renewcommand\large{\@setfontsize\large{11.5pt}{18}}
\makeatother

% \titleformat*{\chapter}{\LARGE\scshape}
% \titleformat*{\section}{\Large\scshape}

% \renewenvironment{proof}[1][\proofname]{{\scshape #1. }}{\qed\topsep}

\renewcommand*{\chapterformat}{%
\mbox{\chapappifchapterprefix{\nobreakspace}%
\scalebox{2}{\color{gray}\thechapter\autodot}\enskip}}

\makeatletter
\renewenvironment{proof}[1][\proofname]
  {\par\pushQED{\qed}%
   \normalfont \topsep6\p@\@plus6\p@\relax
   \list{}{\leftmargin=2em
          \rightmargin=\leftmargin
          \settowidth{\itemindent}{\itshape#1}%
          \labelwidth=\itemindent
          % the following line is not needed with amsart, but might be with other classes
          \parsep=0pt \listparindent=\parindent 
  }
   \item[\hskip\labelsep
        %  \scshape
         \bfseries
         #1\@addpunct{.\hspace{1em}}]\ignorespaces}
  {\popQED\endlist\@endpefalse}
\makeatother

\declaretheoremstyle[
    spaceabove=2\topsep, spacebelow=2\topsep,
    headindent=0pt,
    % headfont=\bfseries,
    % notefont=\bfseries, notebraces={ (}{)},
    headfont=\bfseries,
    notefont=\normalfont\normalsize\bfseries, notebraces={}{.},
    bodyfont=\itshape\normalsize,
    headformat={\llap{\smash{\parbox[t]{1.1in}{\centering \NAME\\ \NUMBER}}} \NOTE},
    headpunct={},
    % qed={$\triangleright$},
    postheadspace=10pt
]{thmstyle}
\declaretheorem[numberwithin=section, style=thmstyle]{principle}
\declaretheorem[name=Teorema, numberwithin=section, style=thmstyle]{theorem}
\declaretheorem[name=Corollario, sibling=theorem, style=thmstyle]{corollary}
\declaretheorem[name=Proposizione, sibling=theorem, style=thmstyle]{proposition}
\declaretheorem[name=Lemma, sibling=theorem, style=thmstyle]{lemma}

\declaretheoremstyle[
    spaceabove=2\topsep, spacebelow=2\topsep,
    headindent=0pt,
    % headfont=\bfseries,
    % notefont=\bfseries, notebraces={ (}{)},
    headfont=\bfseries,
    notefont=\bfseries, notebraces={}{},
    bodyfont=\itshape\normalsize,
    headformat={\llap{\smash{\parbox[t]{1.1in}{\centering \NUMBER\\ \NAME}}} \NOTE},
    headpunct={},
    % qed={$\triangleright$},
    postheadspace=10pt
]{unnamedstyle}
\declaretheorem[name={\ignorespaces}, sibling=theorem, style=unnamedstyle]{unnamed}

\declaretheoremstyle[
    spaceabove=2\topsep, spacebelow=2\topsep,
    headindent=0pt,
    headfont=\bfseries,
    notefont=\normalfont\normalsize\bfseries, notebraces={}{.},
    bodyfont=\normalfont\normalsize,
    headformat={\llap{\smash{\parbox[t]{1.1in}{\centering \NAME\\ \NUMBER}}} \NOTE},
    headpunct={},
    % qed={$\triangleright$},
    postheadspace=10pt
]{defstyle}
\declaretheorem[name=Definizione, sibling=theorem, style=defstyle]{definition}

\declaretheoremstyle[
    headfont=\scshape,
    notefont=\normalfont, notebraces={ - }{.},
    bodyfont=\normalfont,
    postheadspace=1em
]{exmplstyle}
\declaretheorem[name=Esempio, sibling=theorem, style=exmplstyle]{example}
\declaretheorem[name=Esercizio, sibling=theorem, style=exmplstyle]{exercise}


\declaretheoremstyle[
    headfont=\scshape,
    notefont=\normalfont, notebraces={(}{)},
    bodyfont=\normalfont,
    numbered=no,
    postheadspace=1em
]{remarkstyle}
\declaretheorem[name=Osservazione, style=remarkstyle]{remark}
\declaretheorem[name=Soluzione, style=remarkstyle]{solution}
\declaretheorem[name=Intuizione, style=remarkstyle]{intuition}

\newcolumntype{z}{r<{{}}}
\newcolumntype{o}{@{}>{{}}c<{{}}@{}}

% Set related symbols
\newcommand{\set}[1]{ \left\{\,#1\,\right\} }
\newcommand{\union}{\cup}
\newcommand{\bigunion}{\bigcup}
\newcommand{\inters}{\cap}
\newcommand{\biginters}{\bigcap}
\newcommand{\suchthat}{\,:\,} % oppure con {:}
\newcommand{\pset}[1]{\mathcal{P}\!\left(#1\right)}
\DeclareMathOperator{\tc}{\text{ tale che }}

\newcommand{\defUppBound}[1]{\mathcal{M}^{(#1)}}
\newcommand{\defLowBound}[1]{\mathcal{N}^{(#1)}}

% Topology related symbols
\newcommand{\interior}[1]{\mathring{#1}}
\newcommand{\closure}[1]{\eqclass{#1}}
\newcommand{\derived}[1]{\mathcal{D}\!\left(#1\right)}
\newcommand{\oB}[2][\varepsilon]{B_{#1}\!\left(#2\right)}
\newcommand{\cB}[2][\varepsilon]{\closure{B}_{#1}\!\left(#2\right)}
\newcommand{\diameter}{\operatorname{diam}}

\renewcommand{\epsilon}{\varepsilon}
\renewcommand{\theta}{\vartheta}
\renewcommand{\rho}{\varrho}
\renewcommand{\phi}{\varphi}
\newcommand{\eps}{\epsilon}

\let\oldre\Re
\let\oldim\Im
\renewcommand{\Re}[1]{\operatorname{Re}(#1)}
\renewcommand{\Im}[1]{\operatorname{Im}(#1)}
\newcommand{\conj}[1]{\eqclass{#1}}

\newcommand{\deq}{:=}
\newcommand{\iseq}{\overset{?}{=}}
\newcommand{\seteq}{\overset{!}{=}}
\newcommand{\divides}{\mid} % divide esattamente
\newcommand{\ndivides}{\not\mid} % non divide esattamente
\newcommand{\congr}{\equiv} % congruo 
\newcommand{\ncongr}{\not\congr} % non congruo

\newcommand{\Mod}[1]{\ \left(#1\right)}
\newcommand{\mcm}[2]{\operatorname{mcm}\!\left[#1, #2\right]}
\newcommand{\mcd}[2]{\operatorname{mcd}\!\left(#1, #2\right)}
\newcommand{\ord}[2]{\operatorname{ord}_{#2}\!\left( #1 \right)}

\renewcommand{\vec}[1]{\bm{#1}}

\DeclarePairedDelimiter{\abs}{\lvert}{\rvert}
\DeclarePairedDelimiter{\norm}{\lVert}{\rVert}
\DeclarePairedDelimiter{\ang}{\langle}{\rangle}

\NewDocumentCommand{\seqn}{sO{n}O{\N}m}{
    \left( {#4}_{#2} \right)
    \IfBooleanF{#1}
        {_{{#2} \in {#3}}}
}
% \newcommand{\seqn}[2][\N]{\left({#2}_n\right)_{n \in #1}}
\newcommand{\inv}{^{-1}}

\newcommand{\Imm}[1]{\operatorname{Im} #1}
% \newcommand{\ang}[1]{\left\langle #1 \right\rangle}
\newcommand{\invertible}[1]{(#1)^{\times}}
\newcommand{\compl}[1]{#1^C}

\newcommand{\N}{\mathbb{N}}
\newcommand{\Z}{\mathbb{Z}}
\newcommand{\Q}{\mathbb{Q}}
\newcommand{\R}{\mathbb{R}}
\newcommand{\C}{\mathbb{C}}
\newcommand{\K}{\mathbb{K}}

\newcommand{\BB}{\mathcal{B}}
\newcommand{\FF}{\mathcal{F}}
\newcommand{\MM}{\mathcal{M}}
\newcommand{\NN}{\mathcal{N}}
\newcommand{\TT}{\mathcal{T}}
\newcommand{\UU}{\mathcal{U}}
\newcommand{\VV}{\mathcal{V}}

\newcommand{\thmref}[1]{\hyperref[#1]{\nameref*{#1} (\ref*{#1})}}


\begin{document}

\author{Luca De Paulis}
\title{Analisi 1}
\maketitle

\tableofcontents

\chapter{Fondamentali}

\section{Relazioni}

\begin{definition}
    [Relazione su un insieme]
    Sia $X$ un insieme. Allora si dice \emph{relazione su $X$} un sottoinsieme $R \subseteq X \times X$. 
    
    Si indica che una coppia $(x, y)$ soddisfa $R$ scrivendo $xRy$. 
\end{definition}

\begin{definition}
    [Relazione di equivalenza]
    Sia $X$ un insieme e $\sim$ una relazione su $X$. Allora $\sim$ si dice \emph{relazione di equivalenza} se valgono i seguenti assiomi: \begin{enumerate}[label={(EQ\arabic*)}]
        \item La relazione $\sim$ è \emph{riflessiva}:
        
        per ogni $x \in X$ vale che $x \sim x$.
        \item La relazione $\sim$ è \emph{simmetrica}:
        
        per ogni $x, y \in X$, se $x \sim y$ allora necessariamente $y \sim x$.
        \item La relazione $\sim$ è \emph{transitiva}:
        
        per ogni $x, y, z \in X$, se $x \sim y$ e $y \sim z$ allora necessariamente $x \sim z$.
    \end{enumerate}
\end{definition}

\begin{definition}
    [Relazione di ordinamento]
    Sia $X$ un insieme e $\leq$ una relazione su $X$. Allora $\leq$ si dice \emph{relazione di ordinamento} se valgono i seguenti assiomi: \begin{enumerate}[label={(ORD\arabic*)}]
        \item La relazione $\leq$ è \emph{riflessiva}:
        
        per ogni $a \in \K$ vale che $a \leq a$.
        \item La relazione $\leq$ è \emph{antisimmetrica}:
        
        per ogni $a, b \in \K$, se $a \leq b$ e $b \leq a$ allora necessariamente $a = b$.
        \item La relazione $\leq$ è \emph{transitiva}:
        
        per ogni $a, b, c \in \K$, se $a \leq b$ e $b \leq c$ allora necessariamente $a \leq c$.
    \end{enumerate}

    In particolare l'ordinamento si dice \emph{totale} se vale anche che
    \begin{enumerate}[label={(O\arabic*)}, start=4]
        \item La relazione $\leq$ è \emph{totale}:
        
        per ogni $a, b \in \K$ vale che $a \leq b$ oppure $b \leq a$.
    \end{enumerate}
\end{definition}

\section{Funzioni}

\begin{definition}[Funzione (operativamente)]
    \label{def:funz_oper}
    Si dice funzione una terna $(A, B, f)$ formata da:
    \begin{itemize}
        \item un insieme $A$ detto \emph{dominio};
        \item un insieme $B$ detto \emph{codominio};
        \item una legge $f : A \to B$ che associa ad ogni $a \in A$ un elemento $f(a) \in B$.
    \end{itemize}
\end{definition}

\begin{definition}
    [Grafico di una funzione] \label{def:graph}
    Sia $f : A \to B$ una funzione. Allora si dice \emph{grafico della funzione} l'insieme \[
        \operatorname{graph}(f) = \set{(a, b) \in A \times B \suchthat b = f(a)}.    
    \]
\end{definition}

\begin{definition}
    [Funzione (rigorosamente)]
    Si dice funzione da $A$ a $B$ un qualunque sottoinsieme $G \subseteq A \times B$ tale che \[
        \forall a \in A.\quad \exists! b \in B. \quad (a, b) \in G.    
    \]

    Per ogni $a \in A$ si definisce quindi $f(a)$ come l'unico $b$ che rispetta la condizione sopra.
\end{definition}

\begin{definition}
    [Funzione composta]
    Siano $f : A \to B$, $g : B \to C$ due funzioni. Allora si dice \emph{funzione composta} la funzione $(g \circ f) : A \to C$ tale che per ogni $a \in A$ \[
        (g \circ f)(a) = g(f(a)).    
    \]
\end{definition}

\subsection{Iniettività, surgettività}

\begin{definition}
    [Funzione iniettiva]
    Sia $f : A \to B$ una funzione. La funzione $f$ si dice \emph{iniettiva} se \[
        \forall a_1, a_2 \in A. \quad a_1 \neq a_2 \implies f(a_1) \neq f(a_2)    
    \] o equivalentemente
    \[
        \forall a_1, a_2 \in A. \quad f(a_1) = f(a_2) \implies a_1 = a_2.  
    \]
\end{definition}

\begin{definition}
    [Funzione surgettiva]
    Sia $f : A \to B$ una funzione. Allora $f$ si dice \emph{surgettiva} se \[
        \forall b \in B. \quad \exists a \in A. \quad f(a) = b.    
    \]
\end{definition}

\begin{definition}
    [Funzione bigettiva]
    Sia $f : A \to B$ una funzione. Allora $f$ si dice \emph{bigettiva} se:
    \begin{enumerate}[label={(\roman*)}]
        \item $f$ è iniettiva;
        \item $f$ è surgettiva.
    \end{enumerate}

    In tal caso la funzione risulta \emph{invertibile}, ovvero esiste \[
        g : B \to A
    \] tale che per ogni $a \in A$, $b \in B$ vale che
    \begin{align*}
        g(f(a)) &= a, \\
        f(g(b)) &= b.
    \end{align*}
    Se l'inversa esiste, allora si indica con $f\inv : B \to A$.
\end{definition}

\begin{proposition}
    Siano $f : A \to B$, $g : B \to C$ due funzioni.
    Allora \begin{enumerate}[label={(\roman*)}]
        \item se $f$, $g$ sono iniettive, allora $(g \circ f)$ è iniettiva;
        \item se $f$, $g$ sono surgettive, allora $(g \circ f)$ è surgettiva;
        \item se $(g \circ f)$ è iniettiva, allora $f$ è iniettiva;
        \item se $(g \circ f)$ è surgettiva, allora $g$ è surgettiva.
    \end{enumerate}
\end{proposition}
\begin{proof}
    Dimostriamo le quattro affermazioni. \begin{enumerate}
        \item Siano $a_1, a_2 \in A$ con $a_1 \neq a_2$. Allora \[
            a_1 \neq a_2 \overbrace{\implies}^{f \text{ iniett.}} f(a_1) \neq f(a_2) \overbrace{\implies}^{g \text{ iniett.}}  g(f(a_1)) \neq g(f(a_2)).
        \]
    \end{enumerate}
\end{proof}

\subsection{Immagine e controimmagine}

\begin{definition}
    [Immagine di un sottoinsieme]
    Sia $f : A \to B$ una funzione e sia $E \subseteq A$. Allora si dice\emph {immagine di $E$ attraverso $f$} l'insieme\[
        f(E) \deq \set{f(a) \suchthat a \in E} \subseteq B    
    \]
    o equivalentemente \[
        f(E) \deq \set{b \in B \suchthat \exists a \in E \quad b = f(a)}.    
    \]
\end{definition}


\begin{definition}
    [Immagine di una funzione]
    Sia $f : A \to B$ una funzione. Allora si dice \emph{immagine di $f$} l'insieme \[
        \Imm{f} \deq f(A) = \set{f(a) \suchthat a \in A} \subseteq B.
    \]
\end{definition}

\begin{remark}
    Una funzione $f : A \to B$ è surgettiva se e solo se \[
        \Imm{f} = B.    
    \]
\end{remark}

\begin{definition}
    [Controimmagine di un sottoinsieme]
    Sia $f : A \to B$ una funzione e sia $F \subseteq B$. Allora si dice \emph{controimmagine di $F$ attraverso $f$} l'insieme \[
        f\inv(F) \deq \set{a \in A \suchthat f(a) \in F} \subseteq A.    
    \]
\end{definition}

\section{Numeri naturali e Induzione}

\begin{unnamed}
    [Assiomi di Peano]
    Si dice \emph{insieme dei numeri naturali} l'insieme $\N$ che rispetta i seguenti 5 assiomi: \begin{enumerate}[label={(P\arabic*)}, ref={(P\arabic*)}]
        \item Esiste il numero naturale $0$, ovvero $0 \in \N$.
        \item Esiste una funzione $\sigma : \N \to \N$, detta \emph{successore}, definita per ogni naturale.
        \item La funzione $\sigma$ è iniettiva, ovvero per ogni $n, m \in \N$, $n \neq m$ segue che $\sigma(n) \neq \sigma(m)$.
        \item Nessun numero ha $0$ come proprio successore, ovvero $\sigma(n) \neq 0$ per ogni $n \in \N$. 
        \item Se $A$ è un sottoinsieme di $\N$ tale che \begin{enumerate}
            \item $0 \in A$
            \item $n \in A$ implica $\sigma(n) \in A$
        \end{enumerate}
        allora $A = \N$.
    \end{enumerate}
\end{unnamed}

La terna $(\N, 0, \sigma)$ dove $\sigma(n) \deq n+1$ è l'unica terna che rispetta gli assiomi di Peano a meno di isomorfismi: ogni altra terna $(X, a, \Sigma)$ è isomorfa ad essa.

Il quinto assioma, detto \emph{Principio di Induzione}, può essere riformulato in modo da essere usato più comodamente nelle dimostrazioni. La formulazione alternativa è la seguente: 

\begin{unnamed}[Principio di Induzione]
    Sia $A = \set{n \in \N \suchthat n \geq n_0}$, $P : A \to \set{\TT, \FF}$ un predicato. Allora se \begin{enumerate}[label={(\roman*)}]
        \item vale $P(n_0)$
        \item $\forall n \geq n_0 \quad P(n) \implies P(n + 1)$
    \end{enumerate}
    segue che $P(n)$ vale per ogni $n \geq n_0$.
\end{unnamed}

\begin{example}
    Dimostrare che per ogni $n \geq 0$ \begin{align}
        &\sum_{k=0}^n k = \frac{n(n+1)}{2} \label{eq:somma_lineare}\\
        &\sum_{k=0}^n k^2 = \frac{n(n+1)(2n+1)}{6} \label{eq:somma_quadrati}\\
        &\sum_{k=0}^n a^k = \frac{a^{k+1}-1}{a-1} &&\forall a \neq 1. \label{eq:somma_geometrica}
    \end{align}
\end{example}
\begin{proof}
    Dimostriamo la $\ref{eq:somma_quadrati}$ per induzione su $n$.
    \begin{description}
        \item[Caso base] Sia $n = 0$. Allora \[
            \sum_{k=0}^0 k^2 = 0 = \frac{0 \cdot 1 \cdot 1}{2}.  
        \]
        \item[Passo induttivo] Supponiamo che valga la tesi per $n$ e dimostriamola per $n+1$.
        \begin{align*}
            \sum_{k=0}^{n+1} k^2 &= (n+1)^2 + \sum_{k=0}^n k^2 \\
            &= (n+1)^2 + \frac{n(n+1)(2n+1)}{6}\\
            &= \frac{(n+1)^2 + n(n+1)(2n+1)}{6}\\
            &= \frac{(n+1)(n+1 + 2n^2 + n)}{6}\\
            &= \frac{(n+1)(2n^2 + 2n + 1)}{6}\\
            &= \frac{(n+1)(n+2)(2n+2)}{6}.
        \end{align*} 
    \end{description}
    Dunque per il principio di induzione la tesi vale per ogni $n \in \N$.
\end{proof}

% \begin{proposition}
%     [Disuguaglianza di Bernoulli]\label{eq:bernoulli}
%     \[

%     \]
% \end{proposition}
\section{Numeri reali}

\subsection{Proprietà algebriche dei numeri reali}

\begin{definition} [Campo]
    \label{def:campo}
    Sia $\K$ un insieme e siano $+$ (\emph{somma}), $\cdot$ (\emph{prodotto}) due operazioni su $\K$, ovvero \begin{align*}
        + : \K \times \K &\to \K, & \cdot : \K \times \K &\to \K. \\
        (a, b) &\mapsto a+b,      &             (a, b) &\mapsto a\cdot b.
    \end{align*} Allora la struttura $(\K, +, \cdot)$ si dice \emph{campo} se valgono i seguenti assiomi:
    \begin{enumerate}[label={(S\arabic*)}]
        \item \label{def:campo_sum:com} Vale la \emph{proprietà commutativa della somma}:
        
        per ogni $a, b \in \K$ vale che $a + b = b + a$.
        \item \label{def:campo_sum:ass} Vale la \emph{proprietà associativa della somma}:
        
        per ogni $a, b, c \in \K$ vale che $(a + b) + c = a + (b + c)$.
        \item \label{def:campo_sum:neu} Esiste un elemento $0 \in \K$ che è \emph{elemento neutro} per la somma:
        
        per ogni $a \in \K$ vale che $a + 0 = 0 + a = a$.
        \item \label{def:campo_sum:opp} Ogni elemento di $\K$ è \emph{invertibile} rispetto alla somma:
        
        per ogni $a \in \K$ esiste $(-a) \in \K$ (detto \emph{opposto di $a$}) tale che $a + (-a) = 0$.
    \end{enumerate}
    \begin{enumerate}[label={(P\arabic*)}]
        \item \label{def:campo_prod:com} Vale la \emph{proprietà commutativa del prodotto}:
        
        per ogni $a, b \in \K$ vale che $a \cdot b = b \cdot a$.
        \item \label{def:campo_prod:ass} Vale la \emph{proprietà associativa del prodotto}:
        
        per ogni $a, b, c \in \K$ vale che $(a \cdot b) \cdot c = a \cdot (b \cdot c)$.
        \item \label{def:campo_prod:neu} Esiste un elemento $1 \in \K$ che è \emph{elemento neutro} per il prodotto:
        
        per ogni $a \in \K$ vale che $a \cdot 1 = 1 \cdot a = a$.
        \item \label{def:campo_prod:inv} Ogni elemento di $\K$ è \emph{invertibile} rispetto al prodotto:
        
        per ogni $a \in \K$ esiste $a\inv \in \K$ (detto \emph{inverso di $a$}) tale che $a \cdot a\inv = 1$.
    \end{enumerate}
    \begin{enumerate}[label=(SP)]
        \item \label{def:campo:distr} Vale la \emph{proprietà distributiva del prodotto rispetto alla somma}:
         
        per ogni $a, b, c \in \K$ vale che $a(b + c) = ab + ac$.
    \end{enumerate}
\end{definition}

Spesso sottointendiamo il simbolo di prodotto, scrivendo $ab$ per $a \cdot b$.

\begin{definition} [Campo ordinato]
    \label{def:campo_ord}
    Sia $(\K, +, \cdot)$ un campo e sia $\leq$ una relazione di ordine totale su $\K$.
    
    Allora la struttura $(\K, +, \cdot, \leq)$ si dice \emph{campo ordinato} se valgono i seguenti assiomi:
    \begin{enumerate}[label={(CO\arabic*)}]
        \item \label{def:campo_ord:ord_sum} Per ogni $a, b, c \in \K$, se $a \leq b$ allora $a + c \leq b + c$.
        \item \label{def:campo_ord:ord_prod} Per ogni $a, b \in \K$, se $0 \leq a$ e $0 \leq b$ allora $0 \leq ab$.
    \end{enumerate}
\end{definition}

Esempi di campi sono $\Q$, $\R$ e $\C$ con le usuali operazioni di somma e prodotto. Di questi soltanto $\Q$ e $\R$ ammettono un ordinamento totale, dunque soltanto $\Q$ e $\R$ sono campi ordinati.

Inoltre nei campi ordinati introduciamo l'ordinamento totale $\geq$ tale che per ogni $a, b \in \K$ vale che \[
    a \geq b \iff b \leq a.    
\]

\begin{proposition}
    [Proprietà di un campo]

    Sia $(\K, +, \cdot)$ un campo. Allora valgono le seguenti:
    \begin{enumerate}[label={(\roman*)}]
        \item per ogni $a \in \K$ vale che $a \cdot 0 = 0$,
        \item per ogni $a, b \in \K$ se $ab = 0$ allora $a = 0$ oppure $b = 0$.
    \end{enumerate}

    Inoltre se $\leq$ è un ordinamento totale su $\K$ tale che $(\K, +, \cdot, \leq)$ è un campo ordinato, vale che:
    \begin{enumerate}[label={(\roman*)}, start=3]
        \item per ogni $a \in \K$, se $a \geq 0$ allora $-a \leq 0$,
        \item per ogni $a, b, c \in \K$, se $a \leq b$ e $c \leq 0$ allora $ac \geq bc$,
        \item per ogni $a \in \R$ vale che $0 \leq a^2$.
    \end{enumerate}
\end{proposition}
\begin{proof}
    Dimostriamo le cinque proprietà separatamente.
    \begin{enumerate}[label={(\roman*)}]
        \item Sia $a \in \K$ qualsiasi. Allora \begin{align*}
            a0 &= a(0 + 0) \tag{per \ref{def:campo_sum:neu}}\\
            &= a0 + a0 \tag{per \ref{def:campo_sum:neu}}\\
            \intertext{Per \ref{def:campo_sum:neu} esiste $(-a0) \in \K$, dunque}
            \iff a0 - a0 &= a0 + a0 - a0 \tag{per \ref{def:campo_sum:opp}}\\
            \iff 0 &= a0.
        \end{align*}
        COMPLETA
    \end{enumerate}
\end{proof}

\subsection{Valore assoluto}

\begin{definition}
    Sia $a \in \R$. Allora si definisce \emph{valore assoluto} di $a$ il massimo tra $a$ e $-a$, ovvero \begin{equation}
        \abs*{a} \deq \max \set{a, -a} = \begin{cases}
            a, &\text{se } a \geq 0\\
            -a, &\text{se } a < 0.
        \end{cases}
    \end{equation}
\end{definition}

\begin{proposition}
    [Proprietà del valore assoluto]
    Il valore assoluto gode delle seguenti proprietà:
    \begin{enumerate}
        \item Per ogni $a \in \R$ vale che $a \leq \abs*{a}$, $-a \leq \abs*{a}$.
        \item Per ogni $a, b \in \R$ vale che $\abs*{a + b} \leq \abs*{a} + \abs*{b}$.
    \end{enumerate}
\end{proposition}

\subsection{Completezza dei numeri reali}

\begin{definition} [Sezione]
    \label{def:sezione}
    Sia $\K$ un campo ordinato e siano $A, B \subseteq \K$ non vuoti. Allora si dice che $(A, B)$ è una sezione di $\K$ se \begin{enumerate}[label={(\roman*)}, ref={\thedefinition: (\roman*)}]
        \item \label{def:sezione:part} $A \inters B = \varnothing$ e $A \union B = \K$;
        \item \label{def:sezione:sx} $A$ sta a sinistra di $B$, ovvero per ogni $a \in A, b \in B$ vale che $a \leq b$.
    \end{enumerate}
\end{definition}

\begin{unnamed}[Assioma di Dedekind]
    \label{ax:dedekind}
    Sia $(A, B)$ una sezione di $\K$.

    Allora si dice che $\K$ è \emph{Dedekind-completo} (o semplicemente \emph{completo}) se esiste un unico elemento separatore per la sezione $(A, B)$, ovvero \[
        \exists! \xi \in \K. \quad a \leq \xi \leq b \qquad \forall a \in A, b \in B.
    \]
\end{unnamed}

\begin{theorem}
    [Esistenza dei numeri reali]
    Esiste una struttura $(\R, +, \cdot, \leq)$, detti \emph{numeri reali}, che è un campo ordinato completo.
\end{theorem}

\begin{theorem}
    [Unicità dei numeri reali]
    La struttura $(\R, +, \cdot, \leq)$ è unica a meno di isomorfismi; 
    
    ovvero se $(\R^\prime, +^\prime, \cdot^\prime, \leq^\prime)$ è un campo ordinato completo, allora esiste una bigezione $\phi : \R \to \R^\prime$ tale che per ogni $a, b \in \R$ \begin{enumerate}[label={(\roman*)}]
        \item $\phi(a + b) = \phi(a) +^\prime \phi(b)$,
        \item $\phi(a \cdot b) = \phi(a) \cdot^\prime \phi(b)$,
        \item se $a \leq b$ allora $\phi(a) \leq^\prime \phi(b)$.
    \end{enumerate}
\end{theorem}

\subsection{Proprietà dei numeri reali}

\begin{definition} [Maggioranti e minoranti]
    Sia $A \subseteq \R$, $A \neq \varnothing$. 
    
    Allora si dice che $M \in \R$ è un \emph{maggiorante} di $A$ se \[
        \forall a \in A. \quad M \geq a. 
    \] L'insieme dei maggioranti di un insieme $A$ si indica con $\MM(A)$.

    Ugualmente si dice che $m \in \R$ è un \emph{minorante} di $A$ se \[
        \forall a \in A. \quad m \leq a. 
    \] L'insieme dei minoranti di un insieme $A$ si indica con $\NN(A)$.
\end{definition}

\begin{definition} [Limitato superiormente o inferiormente]
    Sia $A \subseteq \R$, $A \neq \varnothing$. 
    
    Allora si dice che $A$ è \emph{limitato superiormente} se ammette almeno un maggiorante, ovvero se $\MM(A) \neq \varnothing$.

    Ugualmente si dice che $A$ è \emph{limitato inferiormente} se ammette almeno un minorante, ovvero se $\NN(A) \neq \varnothing$.

    Se $A$ è limitato sia superiormente che inferiormente allora si dice che $A$ è limitato.
\end{definition}

\begin{remark}
    Un insieme $A \subseteq \R$ è limitato se e solo se esiste un $M \in \R$ tale che \[
        \forall a \in A. \quad \abs*{a} \leq M    
    \]
\end{remark}

\begin{definition} [Massimo e minimo di un insieme]
    Sia $A \subseteq \R$, $A \neq \varnothing$. 

    Allora si dice che $M \in \R$ è il \emph{massimo} di $A$, e si scrive $M = \max A$, se
    \begin{enumerate}[label={(\roman*)}]
        \item $M$ è un maggiorante di $A$, ovvero per ogni $a \in A$ vale che $M \geq a$
        \item $M \in A$.
    \end{enumerate}
    
    Ugualmente si dice che $m \in \R$ è il \emph{minimo} di $A$, e si scrive $m = \min A$, se
    \begin{enumerate}[label={(\roman*)}]
        \item $m$ è un minorante di $A$, ovvero per ogni $a \in A$ vale che $m \leq a$
        \item $m \in A$.
    \end{enumerate}
\end{definition}

\begin{remark}
    Massimo e minimo possono non esistere (ad esempio l'insieme $(0, 1)$ non ha né massimo né minimo) ma se esistono allora sono unici.
\end{remark}

\begin{definition}
    [Estremo superiore ed estremo inferiore]
    Sia $A \subseteq \R$, $A \neq \varnothing$. 

    Allora si definisce l'estremo superiore di $A$, in simboli $\sup A$, nel seguente modo: \[
        \sup A = \begin{cases}
            +\infty,    &\text{se $A$ è illimitato superiormente,}\\
            \min \MM(A) &\text{altrimenti.}
        \end{cases}    
    \]
    
    Ugualmente si definisce l'estremo inferiore di $A$, in simboli $\inf A$: \[
        \inf A = \begin{cases}
            -\infty,    &\text{se $A$ è illimitato inferiormente,}\\
            \max \NN(A) &\text{altrimenti.}
        \end{cases}    
    \]
\end{definition}

\begin{theorem}
    [Esistenza dell'estremo superiore]
    Sia $A \subseteq \R$, $A \neq \varnothing$, $A$ limitato superiormente.

    Allora l'insieme dei maggioranti di $A$ ammette minimo.
\end{theorem}
\begin{proof}
    Siccome $A$ è limitato superiormente, $\MM(A) \neq \varnothing$. 
    
    Sia $B = \R \setminus \MM(A)$; mostriamo che $(B, \MM(A))$ è una sezione di $\R$.
    Ovviamente le proprietà in \ref{def:sezione:part} sono verificate, in quanto ogni $x \in \R$ è in $\MM(A)$ oppure in $B$.

    Siano ora $b \in B$ e $m \in \MM(A)$ e mostriamo che necessariamente $b < m$. Siccome $b$ non è un maggiorante di $A$ dovrà esistere un $a \in A$ tale che $b < a$. D'altra parte $m$ è un maggiorante di $A$, dunque $a \leq m$. Da ciò segue che $b < m$, che verifica la proprietà \ref{def:sezione:sx}.

    Per l'\nameref{ax:dedekind} esiste ed è unico l'elemento separatore. Sia quindi $\xi$ l'elemento separatore della sezione: mostriamo che $\xi = \min \MM(A)$.

    Abbiamo già mostrato che $\xi \leq m$ per ogni $m \in \MM(A)$, quindi dobbiamo soltanto mostrare che $\xi \in \MM(A)$. 
    
    Supponiamo per assurdo che $\xi$ non sia in $\MM(A)$, ovvero che $\xi$ non sia un maggiorante di $A$: allora dovrà esistere un $a \in A$ tale che $a > \xi$. Ma allora varrebbe che \[
        \xi < \frac{\xi + a}{2} < a.
    \] Essendo $\dfrac{\xi + a}{2} < a$ allora anche esso appartiene a $B$, da cui segue che $\xi$ non può essere l'elemento separatore di $(B, \MM(A))$, il che è assurdo.

    Dunque $\xi \in \MM(A)$, ovvero $\xi = \min \MM(A) = \sup A$.
\end{proof}

Si può dimostrare un analogo risultato per l'estremo inferiore:
\begin{theorem}
    [Esistenza dell'estremo inferiore]
    Sia $A \subseteq \R$, $A \neq \varnothing$, $A$ limitato inferiormente.

    Allora l'insieme dei minoranti di $A$ ammette massimo.
\end{theorem}

\begin{proposition}
    [Caratterizzazione dell'estremo superiore] \label{prop:caratt_sup}
    Sia $A \subseteq \R$, $A \neq \varnothing$, $A$ limitato superiormente.

    Allora se $L = \sup A$ segue che \begin{enumerate}[label={(\roman*)}, ref={caratterizzazione del sup: (\roman*)}]
        \item \label{prop:caratt_sup:magg} per ogni $a \in A$ vale che $a \leq L$;
        \item \label{prop:caratt_sup:min_magg} per ogni $\eps > 0$ esiste un $a \in A$ tale che $L - \eps < a$.
    \end{enumerate}
\end{proposition}
\begin{proof}
    Infatti la (i) dice che $L$ è un maggiorante di $A$; la (ii) dice che $L$ è il minimo maggiorante di $A$, ovvero che qualsiasi numero minore di $L$ non è maggiorante di $A$.
\end{proof}

\begin{proposition}
    [Caratterizzazione dell'estremo inferiore] \label{prop:caratt_inf}
    Sia $A \subseteq \R$, $A \neq \varnothing$, $A$ limitato inferiormente.

    Allora se $L = \inf A$ segue che \begin{enumerate}[label={(\roman*)}, ref={caratterizzazione dell'inf: (\roman*)}]
        \item \label{prop:caratt_inf:magg} per ogni $a \in A$ vale che $a \geq L$;
        \item \label{prop:caratt_inf:min_magg} per ogni $\eps > 0$ esiste un $a \in A$ tale che $L - \eps > a$.
    \end{enumerate}
\end{proposition}

\subsection{Retta reale estesa}

\begin{definition}
    [Retta reale estesa]
    Si definisce retta reale estesa l'insieme \[
        \closure{\R} = \R \union \set{-\infty, +\infty}    
    \] tali che $-\infty < a < +\infty$ per ogni $a \in \R$.
\end{definition}

La retta reale estesa non è un campo ordinato poiché non possiamo estendere le operazioni di somma e prodotto in modo che siano consistenti. Possiamo tuttavia definirle parzialmente in questo modo:
\begin{enumerate}
    \item $a + (+\infty) = +\infty$ per ogni $a \in \R$;
    \item $a + (-\infty) = -\infty$ per ogni $a \in \R$;
    \item $a(+\infty) = +\infty$ per ogni $a > 0$;
    \item $a(+\infty) = -\infty$ per ogni $a < 0$;
    \item $a(-\infty) = -\infty$ per ogni $a > 0$;
    \item $a(-\infty) = +\infty$ per ogni $a < 0$.
\end{enumerate}

I casi che non sono automaticamente determinati vengono chiamati \emph{forme indeterminate} oppure \emph{di indecisione}, e sono \begin{align*}
    &+\infty - \infty & \pm\infty \cdot 0
\end{align*} a cui si aggiungono $\dfrac{0}{0}$ e $\dfrac{\pm\infty}{\pm\infty}$ nello studio dei limiti.
\section{Elementi di Topologia della retta}

\begin{definition}
    [Spazio metrico e funzione distanza] \label{def:spazio_metrico}
    Sia $X$ un insieme. 
    
    Allora una funzione $d : X \times X \to \R$ si dice \emph{distanza} o \emph{metrica} se rispetta le seguenti proprietà:
    \begin{enumerate}[label={(d\arabic*)}, ref={(d\arabic*)}]
        \item \label{def:distanza:pos} Per ogni $x, y \in X$ vale che $d(x, y) \geq 0$. In particolare, $d(x, y) = 0$ se e solo se $x = y$.
        \item \label{def:distanza:simm} Per ogni $x, y \in X$ vale che $d(x, y) = d(y, x)$.
        \item \label{def:distanza:dis_triang} Per ogni $x, y, z \in X$ vale che $d(x, y) \leq d(x, z) + d(z, y)$.
    \end{enumerate}

    La struttura $(X, d)$ si dice \emph{spazio metrico}.
\end{definition}

La proprietà \ref{def:distanza:dis_triang} è particolarmente importante e viene chiamata \emph{disuguaglianza triangolare}.

Se l'insieme $X$ è $\R^n$ possiamo definire diversi tipi di metriche $d$ che rendano la coppia $(\R^n, d)$ uno spazio metrico. Le più comuni sono:
\begin{itemize}
    \item la metrica $d_1$, anche detta \emph{metrica di Manhattan}, definita da \[
        d_1(x, y) = \sum_{i = 1}^n \abs*{x_i - y_i}.    
    \]
    \item la metrica $d_2$, che rappresenta la comune \emph{distanza euclidea}, definita da \[
        d_2(x, y) = \sqrt{\sum_{i = 1}^n \abs*{x_i - y_i}^2}.   
    \]
    \item la metrica $d_p$ con $p \in \R$, $p \geq 1$, definita da \[
        d_p(x, y) = \sqrt[p]{\sum_{i = 1}^n \abs*{x_i - y_i}^p}.   
    \]
    \item la metrica $d_\infty$, definita da \[
        d_\infty(x, y) = \max_{i = 1, \dots, n} {\abs*{x_i - y_i}}.    
    \]
\end{itemize}

\begin{remark}
    L'insieme $\R$ con la distanza $d(x, y) \deq \abs*{x - y}$ è uno spazio metrico.
\end{remark}

D'ora in avanti considereremo lo spazio metrico $(X, d)$ dove $X$ è un insieme e $d$ è una metrica qualunque su $X$.

\begin{definition}
    [Palle aperte e chiuse] \label{def:palla_aperta_chiusa}
    Sia $x_0 \in X$, $\eps \in \R$, $\eps > 0$.

    Allora si dice \emph{palla aperta centrata in $x_0$ di raggio $\eps$} l'insieme \[
        \oB{x_0} \deq \set{x \in X \suchthat d(x, x_0) < \eps}.   
    \] L'insieme di tutte le palle aperte centrate in $x_0$ si denota con $\BB(x_0)$.

    Invece si dice \emph{palla chiusa centrata in $x_0$ di raggio $\eps$} l'insieme \[
        \cB{x_0} \deq \set{x \in X \suchthat d(x, x_0) \leq \eps}.   
    \] L'insieme di tutte le palle chiuse centrate in $x_0$ si denota con $\closure{\BB}(x_0)$.
\end{definition}

\begin{definition}
    [Caratterizzazione dei punti di uno spazio metrico] \label{def:caratt_punti}
    Sia $x \in X$, $A \subseteq X$ non vuoto. Si dice che:
    \begin{enumerate}
        \item $x$ è \emph{interno ad $A$} se esiste $\eps > 0$ reale tale che \[
                \oB{x} \subseteq A, 
        \] ovvero se vi è una palla centrata in $x$ tutta contenuta in $A$.
        \item $x$ è \emph{aderente ad $A$} se per ogni $\eps > 0$ reale vale che \[
            \oB{x} \inters A \neq \varnothing,
        \] ovvero se per ogni palla centrata in $x$ c'è un punto che cade nell'insieme $A$.
        \item $x$ è \emph{sulla frontiera di $A$} se per ogni $\eps > 0$ reale vale che \[
            \oB{x} \inters A \neq \varnothing, \quad \oB{x} \inters (X \setminus A) \neq \varnothing, 
        \] ovvero se per ogni palla centrata in $x$ c'è un punto che cade nell'insieme $A$ e un punto che cade nel suo complementare $X \setminus A$.
        \item $x$ è \emph{isolato in $A$} se esiste $\eps > 0$ reale tale che \[
                \oB{x} \inters A = \set{x}, 
        \] ovvero se vi è una palla centrata in $x$ in cui non cadono punti di $A$ tranne $x$.
        \item $x$ è \emph{punto di accumulazione per $A$} se per ogni $\eps > 0$ reale vale che \[
            \oB{x} \inters (A \setminus \set{x}) \neq \varnothing,
        \] ovvero se per ogni palla centrata in $x$ c'è un punto diverso da $x$ che cade nell'insieme $A$.
    \end{enumerate}
\end{definition}

\begin{definition}
    [Caratterizzazione dei punti di uno spazio metrico - Insiemi] \label{def:caratt_punti_insiemi}
    Sia $A \subseteq X$ non vuoto. Si definiscono allora \begin{enumerate}
        \item la parte interna di $A$, definita da \[
            \interior{A} \deq \set{x \in X \suchthat x \text{ è punto interno di } A}.
        \]
        \item la chiusura di $A$, definita da \[
            \closure{A} \deq \set{x \in X \suchthat x \text{ è punto di aderenza per } A}
        \]
        \item la frontiera di $A$, definita da \[
            \partial{A} \deq \set{x \in X \suchthat x \text{ è punto di frontiera per } A}   
        \]
        \item l'insieme dei punti isolati di $A$, ovvero \[
            \operatorname{Isol}(A) \deq \set{x \in X \suchthat x \text{ è punto isolato in } A}.
        \]
        \item l'insieme derivato di $A$, definito da \[
            \derived{A} \deq \set{x \in X \suchthat x \text{ è punto di accumulazione per } A}.
        \]
    \end{enumerate}
\end{definition}

\begin{remark}
    Notiamo che tutti i punti di $A$ sono punti di aderenza per $A$ (in quanto l'intersezione tra la palla centrata nel punto e $A$ deve contenere il punto stesso, e quindi non può essere vuota); inoltre anche i punti di accumulazione sono punti di aderenza, per definizione di punto di accumulazione. Segue quindi che la chiusura di $A$ è data dall'unione di $A$ con il suo derivato. In formule \[
        \closure{A} = A \union \derived{A}.    
    \]
\end{remark}

\begin{definition}
    [Insieme aperto] \label{def:aperto}
    Sia $A \subseteq X$. Allora si dice che $A$ è un \emph{insieme aperto}, o semplicemente un \emph{aperto}, se per ogni $x \in A$ esiste un $\eps > 0$ reale tale che \[
        \oB{x} \subseteq A,    
    \] ovvero se tutti i punti di $A$ sono punti interni ad $A$.
\end{definition}

\begin{proposition}
    [L'unione di una famiglia qualsiasi di aperti è aperta]\label{prop:unione_aperti}
    Sia $\FF \subseteq \pset{X}$ una famiglia di insiemi aperti.

    Allora la loro unione $\displaystyle \bigunion_{A \in \FF} A$ è un aperto.
\end{proposition}
\begin{proof}
    Sia $x \in \bigunion \FF$; dunque $x \in A$ per qualche $A \in \FF$. 

    Siccome $A$ è aperto deve esistere una palla centrata in $x$ tutta contenuta in $A$; dunque a maggior ragione questa palla sarà contenuta in $\FF$, che risulterà quindi aperto.
\end{proof}

\begin{proposition}
    [L'intersezione di due aperti è un aperto]\label{prop:inters_aperti}
    Siano $A, B \subseteq X$ due aperti. Allora $A \inters B$ è aperto.
\end{proposition}
\begin{proof}
    Sia $x \in A \inters B$. Siccome $A$ e $B$ sono aperti, dovranno esistere $\eps_A, \eps_B > 0$ e reali tali che \[
        \oB[\eps_A]{x} \subseteq A, \quad \oB[\eps_B]{x} \subseteq B. 
    \]

    Sia ora $\eps = \min \set{\eps_A, \eps_B}$. Allora l'intorno $\oB{x}$ è contenuto sia in $A$ che in $B$, dunque deve essere in $A \inters B$, da cui viene la tesi.
\end{proof}
\begin{corollary}
    [L'intersezione di un numero finito di aperti è aperto] \label{cor:inters_aperti}
    Siano $A_1, \dots, A_n \subseteq X$ aperti. Allora la loro intersezione $\displaystyle \biginters_{i = 1}^n A_i$ è aperta. 
\end{corollary}
\begin{proof}
    Per induzione su $n$.
\end{proof}

\begin{definition}
    [Insieme chiuso] \label{def:chiuso}
    Sia $D \subseteq X$. Allora $D$ si dice \emph{chiuso} se e solo se $D$ contiene tutti i suoi punti di accumulazione.
\end{definition}

\begin{proposition}
    [Un insieme è chiuso se e solo se il suo complementare è aperto] \label{prop:chiuso_sse_compl_aperto}
    Sia $D \subseteq X$. Allora $D$ è chiuso se e solo se $X \setminus D$ è aperto.
\end{proposition}
\begin{proof}
    Definiamo $A \deq X \setminus D$ per brevità.
    \begin{description}
        \item[($\implies$)] Sia $x \in A$; siccome $D$ contiene tutti i suoi punti di accumulazione allora $x$ non può essere punto di accumulazione per $D$. Questo significa che deve esistere una palla centrata in $x$ per cui $\oB{x} \inters D$ è vuoto, ovvero $\oB{x}$ è tutto contenuto nel complementare di $D$, ovvero $A$.
        
        Dunque $x$ è un punto interno ad $A$. Per la generalità di $x$ segue che tutti i punti di $A$ sono interni, ovvero $A$ è aperto.
        \item[($\impliedby$)] Sia $x$ punto di accumulazione per $D$. 
        
        Siccome $A$ è aperto, se per assurdo $x \in A$ dovrebbe esistere una palla centrata in $x$ tale che $\oB{x}$ è tutta contenuta in $A$. Ma questa palla non può contenere punti di $D$, il che è assurdo in quanto $x$ è un punto di accumulazione per $D$ e quindi ogni sua palla ha un punto in comune con $D$.

        Segue quindi che $D$ è chiuso. \qedhere
    \end{description}
\end{proof}


\subsection{Insiemi compatti}

\begin{definition}[Diametro]
    Sia $A \subseteq X$. Allora il \emph{diametro} di $A$ è \[
        \diameter A \deq \sup \set{d(x, y) \suchthat x, y \in A}.
    \]
\end{definition}

\begin{definition}[Insieme limitato]
    Sia $A \subseteq X$. Allora $A$ si dice \emph{limitato} se il suo diametro è finito.
\end{definition}

\begin{remark}
    Un insieme ha diametro limitato se e solo se è contenuto in una palla di raggio finito, ovvero se esiste un punto $x \in X$ e un raggio $r \in \R$ tali che \[
        A \subseteq \oB[r]{x}.    
    \]
\end{remark}

\begin{definition}
    [Ricoprimento aperto di un insieme] \label{def:ricoprimento}
    Sia $\FF = \left\{U_i\right\}_{i \in I}$ una famiglia di sottoinsiemi aperti di $X$, $A$ un altro sottoinsieme di $X$. 
    
    Allora si dice che $\FF$ è un \emph{ricoprimento di $A$} se $A$ è contenuto nell'unione degli elementi $\FF$, ovvero se \[
        A \subseteq \bigunion_{i \in I} U_i.
    \] 
\end{definition}

La famiglia $\FF$ di sottoinsiemi di $X$ può essere qualunque: in particolare, può essere formata da infiniti insiemi.

\begin{definition}
    [Sottoricoprimento finito di un insieme] \label{def:sottoricoprimento_finito}
    Sia $A$ un sottoinsieme di $X$ e sia $\FF = \left\{U_i\right\}_{i \in I}$ un suo ricoprimento.     
    Sia inoltre $J \subseteq I$. 
    
    Allora si dice che $\left\{U_i\right\}_{i \in J}$ è un \emph{sottoricoprimento finito} di $A$ se
    \begin{enumerate}[label={(\roman*)}]
        \item $J$ è un insieme finito (ovvero con un numero finito di elementi);
        \item $\left\{U_i\right\}_{i \in J}$ è ancora un ricoprimento di $A$.
    \end{enumerate}
\end{definition}

\begin{definition}
    [Compattezza per ricoprimenti] \label{def:compatto_ricopr}
    Sia $A \subseteq X$. Si dice che $A$ è \emph{compatto per ricoprimenti} o semplicemente \emph{compatto} se ogni ricoprimento di $A$ ammette un sottoricoprimento finito.
\end{definition}

\begin{proposition}
    [Un compatto è chiuso e limitato]
    Sia $(X, d)$ uno spazio metrico e sia $A \subseteq X$ compatto.

    Allora vale che \begin{enumerate}[label={(\roman*)}]
        \item $A$ è chiuso;
        \item $A$ è limitato.
    \end{enumerate}
\end{proposition}
\begin{proof}
    Dimostriamo entrambe le proprietà degli insiemi compatti in uno spazio metrico.
    \paragraph{Chiusura.} Supponiamo per assurdo $A$ non sia chiuso. Per definizione allora dovrà esistere almeno un punto di accumulazione di $A$ che non è contenuto in $A$.

    Sia $x_0 \in \derived{A}$ tale che $x_0 \notin A$. Costruiamo un ricoprimento di $A$ fatta in questo modo: per ogni punto $x \in A$ associamo ad esso una palla $\oB[r_x]{x}$ dove \[
        r_x = \frac12 d(x, x_0).    
    \] Ovviamente questa è un ricoprimento aperto di $A$, dunque siccome $A$ è compatto possiamo estrarne un sottoricoprimento finito della forma \[
        \oB[r_{x_1}]{x_1} \union \dots \union \oB[r_{x_n}]{x_n}.   
    \] Sia $r = \min \set{r_{x_1}, \dots, r_{x_n}}$. Allora la palla $\oB[r]{x_0}$ non interseca nessuna delle palle del sottoricoprimento finito, poiché altrimenti $x_0$ sarebbe in $A$.

    Dato che $x_0$ è di accumulazione per $A$ segue che in $\oB[r]{x_0}$ devono esserci infiniti punti di $A$, il che contraddice la conclusione che il sottoricoprimento sia un ricoprimento di $A$. Dunque segue che $A$ è chiuso.
    \paragraph{Limitatezza} Supponiamo per assurdo $A$ non limitato. Allora per ogni $x \in X$, $r \in \R$ dovranno esistere dei punti di $A$ che non sono in $\oB[r]{x}$.

    Fisso $x \in X$. L'unione di tutte le palle di centro $x$ con raggio $n \in \N$ è un ricoprimento di $A$, ma da questo ricoprimento non si può estrarre un sottoricoprimento finito. Infatti se prendo un numero finito di palle centrate in $x$ e ne faccio l'unione ottengo \[
        \oB[n_1]{x} \union \dots \union \oB[n_k]{x}.
    \] Sia $\bar n = \max \set{n_1, \dots, n_k}$, allora il sottoricoprimento è uguale alla palla $\oB[\bar n]{x}$. Ma per ipotesi $A$ è illimitato, dunque non può essere contenuto in una palla di dimensione finita $\bar n$.

    Dunque concludiamo che $A$ è limitato.
\end{proof}

Nel caso specifico dello spazio euclideo $\R^n$ vale anche il viceversa, come ci viene garantito dal seguente teorema.

\begin{theorem}
    [Teorema di Heine-Borel] \label{th:heine-borel}
    Sia $A \subseteq \R^n$. Allora $A$ è compatto per ricoprimenti se e solo se è chiuso e limitato.
\end{theorem}

\subsection{Topologia reale in una dimensione}

Nel caso della retta valgono tutte le definizioni date sopra nel caso più generale di uno spazio metrico qualsiasi.

In particolare le palle aperte si dicono \emph{intorni aperti} e sono rappresentati da intervalli aperti: \[
    \oB{x_0} = \set{x \in \R \suchthat \abs*{x - x_0} < \eps} = (x_0 - \eps, x_0 + \eps).    
\] Studiando la retta reale estesa invece sorge la necessità di definire degli \emph{intorni di $+\infty$ e $-\infty$}.

Si dice \emph{intorno di $+\infty$ di raggio $M > 0$} una semiretta di $\R$ della forma \[
    \oB[M]{+\infty} = \set{x \in \R \suchthat x > M} = (M, +\infty).
\] Allo stesso modo si dice \emph{intorno di $-\infty$ di raggio $M > 0$} una semiretta di $\R$ della forma \[
    \oB[M]{-\infty} = \set{x \in \R \suchthat x < -M} = (-\infty, -M).
\] 



\chapter{Successioni}

\section{Limiti di successioni}

\begin{definition}
    [Predicati veri definitivamente e frequentemente]\label{def:definit_frequent}
    Sia $P$ un predicato sui numeri naturali, ovvero $P : \N \to \set{\TT, \FF}$.

    Allora si dice che:
    \begin{itemize}
        \item $P$ è vero \emph{definitivamente} se esiste un $n_0 \in \N$ tale che per ogni $n \geq n_0$ vale $P(n)$. In formule \begin{equation*}
            \exists n_0 \in \N. \quad \forall n \geq n_0. \quad P(n) = \TT. \tag{definitivamente}
        \end{equation*}
        \item $P$ è vero \emph{frequentemente} se è vero per infiniti valori di $n$, ovvero se per ogni $n_0 \in \N$ esiste un $n \geq n_0$ tale che vale $P(n)$. In formule \begin{equation*}
            \forall n_0 \in \N. \quad \exists n \geq n_0. \quad P(n) = \TT. \tag{frequentemente}
        \end{equation*}
    \end{itemize}
\end{definition}

Introduciamo ora il concetto di successione. 

\begin{definition}
    [Successione] \label{def:successione}
    Sia $X$ un insieme. Allora si dice \emph{successione a valori in $X$} una funzione \[
        a : \N \to X.    
    \]

    Non è necessario che il dominio sia $\N$, ma basta un qualsiasi sottoinsieme di $\N$ della forma \[
        A = \set{n \in \N \suchthat n \geq n_0}    
    \] per qualche $n_0 \in \N$. 

    Spesso si scrive $a_i$ per indicare $a(i)$; inoltre per indicare l'intera successione si usa la notazione $(a_n)_{n \in A}$ oppure semplicemente $(a_n)$ se i valori degli indici sono facilmente deducibili dal contesto.
\end{definition}

\begin{definition}
    [Limite di una successione] \label{def:lim_succ}
    Sia $\seqn*{a}$ una successione a valori reali, definita anche solo definitivamente. 
    \begin{enumerate}[label={(\arabic*)}, ref={(\arabic*)}]
        \item \label{def:lim_succ:conv} Si dice che la successione tende a $l \in \R$, e si scrive \[
            a_n \to l \quad \text{oppure} \quad \lim_{n \to \infty} a_n = l    
        \] se vale che per ogni $\eps > 0$ vale che $\abs*{a_n - l} < \eps$ definitivamente.

        In questo caso si dice inoltre che la successione \emph{converge} ad $l$.
        \item \label{def:lim_succ:div_+inf} Si dice che la successione tende a $+\infty$, e si scrive \[
            a_n \to +\infty \quad \text{oppure} \quad \lim_{n \to \infty} a_n = +\infty    
        \] se vale che per ogni $M \in \R$ vale che $a_n \geq M$ definitivamente.

        In questo caso si dice inoltre che la successione \emph{diverge positivamente}.
        \item \label{def:lim_succ:div_-inf} Si dice che la successione tende a $-\infty$, e si scrive \[
            a_n \to -\infty \quad \text{oppure} \quad \lim_{n \to \infty} a_n = -\infty    
        \] se vale che per ogni $M \in \R$ vale che $a_n \leq M$ definitivamente.

        In questo caso si dice inoltre che la successione \emph{diverge negativamente}.
        \item \label{def:lim_succ:no_lim} Si dice che la successione è \emph{indeterminata} oppure che \emph{non ha limite} se non rientra in nessuno dei casi precedenti.
    \end{enumerate}
\end{definition}

In particolare una successione convergente a $0$ si dice \emph{infinitesima}, mentre una successione divergente si dice \emph{infinita}.

Da adesso in poi assumeremo che le successioni siano a valori reali (a meno che non venga specificato diversamente) e che siano definite anche solo definitivamente.

\subsection{Primi teoremi sui limiti di successioni}

\begin{theorem}
    [Permanenza del segno] \label{th:perm_segno_succ}
    Sia $\seqn*{a}$ una successione. Supponiamo che valga una tra \begin{itemize}
        \item $a_n \to l \in \R$ per qualche $l > 0$,
        \item $a_n \to +\infty$.
    \end{itemize}

    Allora $a_n > 0$ definitivamente.
\end{theorem}
\begin{proof}
    Dimostriamo i due casi separatamente.
    \begin{description}
        \item[(Hp: $a_n \to l > 0$)]  Per definizione di limite \[
            \forall \eps > 0. \quad \abs*{a_n - l} \leq \eps \quad \text{definitivamente.}
        \]
        Sia $\eps = \frac{l}{2}$. Allora \[
            0 < \frac{l}{2} \leq a_n \leq \frac{3l}{2} \quad \text{definitivamente,}
        \] da cui segue che $a_n > 0$ definitivamente.
        \item[(Hp: $a_n \to +\infty$)] Per definizione di limite \[
            \forall M \in \R. \quad a_n \geq M \quad \text{definitivamente.}
        \]
        Sia $M$ positivo, ad esempio $M = 1$. Allora \[
            a_n \geq M = 1 > 0 \quad \text{definitivamente}
        \] da cui segue che $a_n > 0$ definitivamente.\qedhere
    \end{description}  
\end{proof}

Il teorema vale ovviamente anche per successioni che tendono ad un valore (finito o infinito) negativo: i valori assunti dalla successione saranno definitivamente negativi.

\begin{theorem}
    [Unicità del limite] \label{th:unic_lim_succ}
    Sia $\seqn*{a}$ una successione. Allora $\seqn*{a}$ può assumere uno e uno solo dei comportamenti descritti in \ref{def:lim_succ}.

    Inoltre se $\seqn*{a}$ converge segue che esiste un unico $l \in \R$ tale che $a_n \to l$.
\end{theorem}

Per dimostrare il teorema dimostriamo separatamente i seguenti lemmi.

\begin{lemma} \label{lem:unic_lim_conv}
    Sia $\seqn*{a}$ una successione. Allora se $a_n \to l \in \R$, tale limite è unico. Inoltre $\seqn*{a}$ non può divergere.
\end{lemma}
\begin{proof}
    Supponiamo che esista $m \in \R$ tale che $a_n \to m$ e mostriamo che necessariamente $m = l$. Per definizione di limite avremo che \begin{align}
        &\forall \eps > 0. \quad \exists n_1 \in \N. \quad \forall n \geq n_1. \quad \abs*{a_n - l} < \eps. \label{eq:unic_lim_conv:1}\\
        &\forall \eps > 0. \quad \exists n_2 \in \N. \quad \forall n \geq n_2. \quad \abs*{a_n - m} < \eps. \label{eq:unic_lim_conv:2}
    \end{align}

    Sia ora $n_0 \deq \min \set{n_1, n_2}$. Allora per ogni $n \geq n_0$ dovranno valere sia la \eqref{eq:unic_lim_conv:1} che la \eqref{eq:unic_lim_conv:2}, ovvero \[
        \abs*{a_n - l} < \eps, \quad  \abs*{a_n - m} < \eps.
    \]

    Allora dovrà valere che \begin{align*}
        \abs*{l - m} &= \abs*{l - a_n + a_n - m} \tag{per \ref{def:distanza:dis_triang}}\\
        &< \abs*{l - a_n} + \abs*{m - a_n}\\
        &< 2\eps.
    \end{align*}
    Dunque per l'arbitrarietà di $\eps$ segue che $l = m$.

    Dimostriamo ora che $\seqn*{a}$ non può divergere. Supponiamo per assurdo che $\seqn*{a}$ diverga positivamente, ovvero \begin{equation}
        \label{eq:unic_lim_conv:3} \forall M \in \R. \quad \exists n_3 \in \N. \quad \forall n \geq n_3. \quad a_n \geq M.
    \end{equation} Mostriamo che questa richiesta è incompatibile con la condizione \eqref{eq:unic_lim_conv:1}, che rappresenta l'ipotesi $a_n \to l \in \R$ e che possiamo scrivere anche nella forma $l - \eps < a_n < l + \eps$ (definitivamente).

    Fissiamo un $\eps > 0$ e poniamo $M = l + \eps$. Allora combinando le condizioni della \eqref{eq:unic_lim_conv:1} e della \eqref{eq:unic_lim_conv:3} otteniamo che per ogni $n > \max \set{n_1, n_3}$ vale che \[
        l + \eps = M \stackrel{\eqref{eq:unic_lim_conv:3}}{\leq} a_n \stackrel{\eqref{eq:unic_lim_conv:1}}{<} l + \eps   
    \] che è assurdo. Analogo ragionamento per dimostrare che $\seqn*{a}$ non può divergere negativamente.
\end{proof}

\begin{lemma}\label{lem:unic_lim_div+}
    Sia $\seqn*{a}$ una successione. Allora se $a_n \to +\infty$ segue che $\seqn*{a}$ non può né convergere né divergere negativamente.
\end{lemma}
\begin{proof}
    È evidente che se $a_n \to +\infty$ segue che $\seqn*{a}$ non può divergere negativamente. Infatti se per assurdo divergesse negativamente allora per ogni $M \in \R$ varrebbe che $a_n < M$ definitivamente, mentre per ipotesi $a_n > M$ definitivamente.

    Mostriamo ora che $\seqn*{a}$ non può convergere. Supponiamo per assurdo che $a_n \to l \in \R$; allora per definizione di successione convergente avremmo che \begin{equation}
        \forall \eps > 0. \quad \exists n_1 \in \N. \quad \forall n \geq n_1. \quad l - \eps < a_n < l + \eps. \label{eq:unic_lim_div+:1}\\
    \end{equation}
    Per l'ipotesi che $\seqn*{a}$ diverga positivamente sappiamo inoltre che \begin{equation}
        \forall M \in \R. \quad \exists n_2 \in \N. \quad \forall n \geq n_2. \quad a_n \geq M. \label{eq:unic_lim_div+:2}
    \end{equation}

    Fissiamo un $\eps > 0$ e poniamo $M = l + \eps$. Allora combinando le condizioni della \eqref{eq:unic_lim_div+:1} e della \eqref{eq:unic_lim_div+:2} otteniamo che per ogni $n > \max \set{n_1, n_2}$ vale che \[
        l + \eps = M \stackrel{\eqref{eq:unic_lim_div+:2}}{\leq} a_n \stackrel{\eqref{eq:unic_lim_div+:1}}{<} l + \eps   
    \] che è assurdo, dunque la tesi.
\end{proof}

\begin{lemma} \label{lem:unic_lim_div-}
    Sia $\seqn*{a}$ una successione. Allora se $a_n \to +\infty$ segue che $\seqn*{a}$ non può né convergere né divergere positivamente.
\end{lemma}
\begin{proof}
    È evidente che se $a_n \to -\infty$ segue che $\seqn*{a}$ non può divergere positivamente. Infatti se per assurdo divergesse positivamente allora per ogni $M \in \R$ varrebbe che $a_n > M$ definitivamente, mentre per ipotesi $a_n < M$ definitivamente.

    Mostriamo ora che $\seqn*{a}$ non può convergere. Supponiamo per assurdo che $a_n \to l \in \R$; allora per definizione di successione convergente avremmo che \begin{equation}
        \forall \eps > 0. \quad \exists n_1 \in \N. \quad \forall n \geq n_1. \quad l - \eps < a_n < l + \eps. \label{eq:unic_lim_div-:1}\\
    \end{equation}
    Per l'ipotesi che $\seqn*{a}$ diverga negativamente sappiamo inoltre che \begin{equation}
        \forall M \in \R. \quad \exists n_2 \in \N. \quad \forall n \leq n_2. \quad a_n \leq M. \label{eq:unic_lim_div-:2}
    \end{equation}

    Fissiamo un $\eps > 0$ e poniamo $M = l - \eps$. Allora combinando le condizioni della \eqref{eq:unic_lim_div-:1} e della \eqref{eq:unic_lim_div-:2} otteniamo che per ogni $n > \max \set{n_1, n_2}$ vale che \[
        l - \eps \stackrel{\eqref{eq:unic_lim_div-:1}}{<} a_n \stackrel{\eqref{eq:unic_lim_div-:2}}{\leq} M = l - \eps   
    \] che è assurdo, dunque la tesi.
\end{proof}

I tre lemmi ci consentono di dimostrare il \autoref{th:unic_lim_succ}:
\begin{proof}
    Per il \autoref{lem:unic_lim_conv} sappiamo che se una successione è convergente allora non può divergere né positivamente né negativamente. Inoltre il limite reale è unico.

    Per il \autoref{lem:unic_lim_div+} sappiamo che se una successione è divergente positivamente allora non può né convergere né divergere negativamente.

    Per il \autoref{lem:unic_lim_div-} sappiamo che se una successione è divergente negativamente allora non può né convergere né divergere positivamente.

    Infine se una successione è indeterminata allora per definizione non ha nessuno dei precedenti tre caratteri, dunque la tesi.
\end{proof}

\begin{theorem}
    [Confronto Asintotico] \label{th:confr_asint_succ}
    Siano $\seqn*{a}, \seqn*{b}$ due successioni tali che $a_n \leq b_n$ definitivamente.
    Allora valgono le seguenti affermazioni: \begin{enumerate}[label={(\roman*)}, ref={thetheorem: (\roman*)}]
        \item \label{th:confr_succ:a->+inf} se $a_n \to +\infty$ allora anche $b_n \to +\infty$;
        \item \label{th:confr_succ:b->-inf} se $b_n \to -\infty$ allora anche $a_n \to -\infty$.
    \end{enumerate}
\end{theorem}
\begin{proof}
    Dimostriamo i due casi separatamente.
    \begin{enumerate}[label={(\roman*)}]
        \item Supponiamo che $a_n \to +\infty$, ovvero che \[
            \forall M \in \R. \quad \exists n_1 \in \N. \quad \forall n \geq n_0. \quad a_n \geq M. 
        \] Inoltre per ipotesi dovrà esistere $n_2 \in \N$ tale che $a_n \leq b_n$ per ogni $n \geq n_2$.

        Allora per ogni $n \geq \max \set{n_1, n_2}$ varrà che \[
            b_n \geq a_n \geq M    
        \] ovvero $\seqn*{b}$ diverge positivamente.
        \item Supponiamo che $b_n \to -\infty$, ovvero che \[
            \forall M \in \R. \quad \exists n_1 \in \N. \quad \forall n \geq n_0. \quad b_n \leq M. 
        \] Inoltre per ipotesi dovrà esistere $n_2 \in \N$ tale che $a_n \leq b_n$ per ogni $n \geq n_2$.

        Allora per ogni $n \geq \max \set{n_1, n_2}$ varrà che \[
            a_n \leq b_n \leq M    
        \] ovvero $\seqn*{a}$ diverge negativamente. \qedhere
    \end{enumerate}
\end{proof}

\begin{theorem}
    [Teorema dei Carabinieri] \label{th:carab_succ}
    Siano $\seqn*{a}, \seqn*{b}, \seqn*{c}$ tre successioni tali che (per qualche $l \in \R$)\begin{enumerate}[label={(\roman*)}]
        \item $a_n \to l$,
        \item $c_n \to l$,
        \item $a_n \leq b_n \leq c_n$ definitivamente.
    \end{enumerate}
    Allora segue che $b_n \to l$ e dunque in particolare $\seqn*{b}$ è convergente.
\end{theorem}
\begin{proof}
    Per ipotesi sappiamo che \begin{align}
        \forall \eps > 0 \quad &\exists n_1 \in \N \quad \forall n \geq n_1. \quad {}&&l-\eps \leq a_n \leq l+\eps \label{eq:carab_succ:1}\\ 
        \forall \eps > 0 \quad &\exists n_2 \in \N \quad \forall n \geq n_2. \quad {}&&l-\eps \leq c_n \leq l+\eps \label{eq:carab_succ:2}\\
        &\exists n_3 \in \N \quad \forall n \geq n_3. \quad {}&&a_n \leq b_n \leq c_n. \label{eq:carab_succ:3}
    \end{align}

    Dunque per ogni $n \geq \max \set{n_1, n_2, n_3}$ dovrà valere che \[
        l-\eps \stackrel{\eqref{eq:carab_succ:1}}{\leq} a_n \stackrel{\eqref{eq:carab_succ:3}}{\leq} b_n \stackrel{\eqref{eq:carab_succ:3}}{\leq} c_n \stackrel{\eqref{eq:carab_succ:2}}{\leq} l - \eps
    \] da cui segue che $b_n \to l$.
\end{proof}

\subsection{Limitatezza}

Come per gli insiemi, possiamo definire il concetto di limitatezza per le successioni.

\begin{definition}
    [Successione limitata] \label{def:succ_limit}
    Sia $\seqn*{a}$ una successione. 
    Allora $\seqn*{a}$ si dice \begin{enumerate}[label={(\roman*)}]
        \item \emph{limitata superiormente} se esiste $u \in \R$ tale che \begin{equation}
            \label{def:succ_bound:sup} a_n \leq u \text{  per ogni } n \in \N.
        \end{equation}
        \item \emph{limitata inferiormente} se esiste $l \in \R$ tale che \begin{equation}
            \label{def:succ_bound:inf} a_n \geq l \text{  per ogni } n \in \N.
        \end{equation}
        \item \emph{limitata} se è sia limitata inferiormente che superiormente, ovvero se esistono $l, u \in \R$ tali che \begin{equation}
            \label{def:succ_bound:both} l \leq a_n \leq u \text{  per ogni } n \in \N.
        \end{equation}
    \end{enumerate}
\end{definition}

\begin{proposition}
    [Una successione convergente è limitata] \label{prop:succ_conv=>limit}
    Sia $\seqn*{a}$ una successione tale che $a_n \to a \in \R$. Allora $\seqn*{a}$ è limitata.
\end{proposition}
\begin{proof}
    Per definizione di successione convergente sappiamo che \[
        \forall \eps > 0. \quad \exists n_0 \in \N. \quad \forall n \geq n_0. \quad \abs*{a_n - a} < \eps.    
    \] Fissiamo $\eps = 1$; sia inoltre $M = \abs*{a_0} + \abs*{a_1} + \dots + \abs*{a_{n_0}} + \abs*{a} + 1$. Mostriamo che per ogni $n \in \N$ vale che $\abs*{a_n} < M$.

    Se $n < n_0$ allora la disequazione è ovvia, in quanto \[
        \abs*{a_n} < \abs*{a_0} + \dots + \abs*{a_n} + \dots + \abs*{a_{n_0}} + \abs*{a} + 1.    
    \]

    Invece se $n \geq n_0$ sappiamo che $\abs*{a_n - a} < 1$ (abbiamo fissato $\eps = 1$), da cui segue \begin{align*}
        \abs*{a_n} &= \abs*{a_n - a + a} \\
        &< \abs*{a_n - a} + \abs*{a} \\
        &< 1 + \abs*{a} \\
        &< M.
    \end{align*}

    Dunque $\abs*{a_n} < M$ per ogni $n \in \N$, ovvero $\seqn*{a}$ è limitata.
\end{proof}

\begin{proposition}
    [Comportamento del limite di una successione conv. limitata]
    \label{prop:comp_limite_succ_limit}
    Sia $\seqn*{a}$ una successione tale che $a_n \to a \in \R$. Allora \begin{itemize}
        \item se $\seqn*{a}$ è limitata superiormente da $U \in \R$ segue che $a \leq U$;
        \item se $\seqn*{a}$ è limitata inferiormente da $L \in \R$ segue che $a \geq L$;
        \item se $\seqn*{a}$ è limitata (ovvero $L \leq a_n \leq U$) segue che $L \leq a \leq U$.
    \end{itemize}
\end{proposition}
\begin{proof}
    Dimostriamo il primo caso, gli altri due sono analoghi.

    Supponiamo per assurdo che $a > U$. Per definizione di successione convergente \[
        \forall \eps > 0. \quad \exists n_0 \in \N. \quad \forall n \geq n_1. \quad a - \eps < a_n < a + \eps.   
    \] Scegliamo $\eps = a - U$ (che è positivo in quanto $a > U$), da cui segue che \[
        a_n > a - \eps = U    
    \] che è assurdo. Dunque deve valere che $a \leq U$.
\end{proof}

\subsection{Monotonia}

\begin{definition}
    [Successioni monotone] \label{def:succ_monotona}
    Sia $\seqn*{a}$ una successione. Allora $\seqn*{a}$ si dice \begin{enumerate}
        \item \emph{strettamente crescente} se per ogni $n \in \N$ vale che $a_{n+1} > a_n$;
        \item \emph{debolmente crescente} se per ogni $n \in \N$ vale che $a_{n+1} \geq a_n$;
        \item \emph{strettamente decrescente} se per ogni $n \in \N$ vale che $a_{n+1} < a_n$;
        \item \emph{debolmente decrescente} se per ogni $n \in \N$ vale che $a_{n+1} \leq a_n$.
    \end{enumerate}
\end{definition}

\begin{proposition}
    [Comportamento di una successione crescente] \label{prop:succ_cresc}
    Sia $\seqn*{a}$ una successione debolmente crescente. 
    
    Allora $\seqn*{a}$ converge oppure diverge positivamente. In entrambi i casi vale che \[
        \lim_{n \to +\infty} a_n = \sup \set{a_n \suchthat n \in \N}.    
    \]
\end{proposition}
\begin{proof}
    Dimostriamo i due casi separatamente. \begin{itemize}
        \item Supponiamo che $\sup \set{a_n \suchthat n \in \N} = +\infty$. 
        
        Sia $M \in \R$ qualsiasi. Dato che l'insieme $\set{a_n \suchthat n \in \N}$ è superiormente illimitato, dovrà esistere un $n_0 \in \N$ tale che $a_{n_0} \geq M$.

        Per ipotesi $\seqn*{a}$ è debolmente crescente, dunque per ogni $n \geq n_0$ vale che \[
            a_n \geq a_{n_0} \geq M
        \] ovvero $a_n \to +\infty$.
        \item Supponiamo che $\sup \set{a_n \suchthat n \in \N} = l \in \R$. 
        
        Per definizione di estremo superiore segue che $a_n \leq l$, dunque in particolare per qualsiasi $\eps > 0$ dovrà valere $a_n \leq l + \eps$.

        Per caratterizzazione dell'estremo superiore inoltre dovrà esistere $n_0 \in \N$ tale che $a_{n_0} \geq l - \eps$ (altrimenti $l - \eps$ sarebbe un maggiorante minore dell'estremo superiore). Dunque per ogni $n \geq n_0$ dovrà valere \[
            l - \eps \leq a_n \leq l + \eps, 
        \] ovvero $a_n \to l$. \qedhere
    \end{itemize}
\end{proof}

\begin{corollary} \label{cor:succ_cresc_limit}
    Sia $\seqn*{a}$ una successione debolmente crescente e limitata superiormente. 
    
    Allora $\seqn*{a}$ è convergente e $a_n \to \sup \set{a_n \suchthat n \in \N}$.
\end{corollary}
\begin{proof}
    Per la proposizione \ref{prop:succ_cresc} sappiamo che il limite della successione $\seqn*{a}$ è dato dal suo estremo superiore. Per ipotesi $\seqn*{a}$ è limitata superiormente, dunque $\sup \set{a_n \suchthat n \in \N} \in \R$, da cui segue che $\seqn*{a}$ è convergente.
\end{proof}

Enunciamo e dimostriamo ora le proposizioni analoghe per le successioni decrescenti.

\begin{proposition}
    [Comportamento di una funzione decrescente] \label{prop:succ_decresc}
    Sia $\seqn*{a}$ una successione debolmente decrescente. 
    
    Allora $\seqn*{a}$ converge oppure diverge negativamente. In entrambi i casi vale che \[
        \lim_{n \to +\infty} a_n = \inf \set{a_n \suchthat n \in \N}.    
    \]
\end{proposition}
\begin{proof}
    Dimostriamo i due casi separatamente. \begin{itemize}
        \item Supponiamo che $\inf \set{a_n \suchthat n \in \N} = -\infty$. 
        
        Sia $M \in \R$ qualsiasi. Dato che l'insieme $\set{a_n \suchthat n \in \N}$ è inferiormente illimitato, dovrà esistere un $n_0 \in \N$ tale che $a_{n_0} \leq M$.

        Per ipotesi $\seqn*{a}$ è debolmente crescente, dunque per ogni $n \geq n_0$ vale che \[
            a_n \leq a_{n_0} \leq M
        \] ovvero $a_n \to +\infty$.
        \item Supponiamo che $\inf \set{a_n \suchthat n \in \N} = l \in \R$. 
        
        Per definizione di estremo inferiore segue che $a_n \geq l$, dunque in particolare per qualsiasi $\eps > 0$ dovrà valere $a_n \geq l - \eps$.

        Per caratterizzazione dell'estremo inferiore inoltre dovrà esistere $n_0 \in \N$ tale che $a_{n_0} \leq l + \eps$ (altrimenti $l + \eps$ sarebbe un minorante maggiore dell'estremo inferiore). Dunque per ogni $n \geq n_0$ dovrà valere \[
            l - \eps \leq a_n \leq l + \eps, 
        \] ovvero $a_n \to l$. \qedhere
    \end{itemize}
\end{proof}

\begin{corollary}\label{cor:succ_decr_limit}
    Sia $\seqn*{a}$ una successione debolmente decrescente e limitata inferiormente. 
    
    Allora $\seqn*{a}$ è convergente e $a_n \to \inf \set{a_n \suchthat n \in \N}$.
\end{corollary}
\begin{proof}
    Per la proposizione \ref{prop:succ_decresc} sappiamo che il limite della successione $\seqn*{a}$ è dato dal suo estremo inferiore. Per ipotesi $\seqn*{a}$ è limitata inferiormente, dunque $\inf \set{a_n \suchthat n \in \N} \in \R$, da cui segue che $\seqn*{a}$ è convergente.
\end{proof}

\subsection{Numero di Nepero}

Consideriamo la successione $\seqn*[n][\N\setminus\set{0}]{e}$ definita da \[
    e_n \deq \left(1 + \frac{1}{n}\right)^n.    
\]

\begin{proposition} \label{prop:e}
    La successione $\seqn*{e}$ \begin{enumerate}
        \item verifica $2 \leq e_n \leq 3$ per ogni $n \geq 1$,
        \item è strettamente crescente.
    \end{enumerate}
\end{proposition}
\begin{proof}
    Dimostro inizialmente che $e_n \geq 2$ per ogni $n \geq 1$. 
\end{proof}
\section{Calcolo di limiti di successioni}

\subsection{Teoremi algebrici}

Presentiamo ora i vari teoremi sull'algebra dei limiti.

\begin{proposition}
    [Limite della somma di successioni convergenti]\label{prop:sum_lim_succ_conv}
    Siano $\seqn*{a}, \seqn*{b}$ due successioni convergenti, e siano $a, b \in \R$ rispettivamente i loro limiti. 
    
    Allora la successione $\left( a_n + b_n \right)$ converge e  \[
        a_n + b_n \to a + b.
    \]
\end{proposition}
\begin{proof}
    Per definizione di successione convergente \begin{align}
        &\forall \eps > 0. \quad \exists n_a \in \N. \quad \forall n \geq n_a. \quad \abs*{a_n - a} < \eps. \label{eq:sum_lim_succ_conv:1}\\
        &\forall \eps > 0. \quad \exists n_b \in \N. \quad \forall n \geq n_b. \quad \abs*{b_n - b} < \eps. \label{eq:sum_lim_succ_conv:2}
    \end{align}

    Fissiamo $\eps$; allora per $n \geq \max \set{n_a, n_b}$ dovrà valere che \begin{align*}
        &\abs*{a_n + b_n - (a+b)} \tag{per \ref{def:distanza:dis_triang}}\\
        <\ &\abs*{a_n - a} + \abs*{b_n - b}  \tag{per \eqref{eq:sum_lim_succ_conv:1} e \eqref{eq:sum_lim_succ_conv:2}}\\
        <\ &2\eps,
    \end{align*}
    dunque la successione $\left(a_n + b_n\right)$ è convergente e $a_n + b_n \to a + b$.
\end{proof}

\begin{remark}
    Ovviamente questo teorema vale anche per la differenza tra successioni convergenti: \[
        a_n - b_n \to a - b.    
    \]
\end{remark}

\begin{proposition}
    [Limite del prodotto di successioni convergenti]\label{prop:prod_lim_succ_conv}
    Siano $\seqn*{a}, \seqn*{b}$ due successioni convergenti, e siano $a, b \in \R$ rispettivamente i loro limiti. 
    
    Allora la successione $\left( a_n \cdot b_n \right)$ converge e \[
        a_n \cdot b_n \to ab.
    \]
\end{proposition}
\begin{proof}
    Siccome $\seqn*{a}$ è convergente, per la \autoref{prop:succ_conv=>limit} $\seqn*{a}$ è limitata, ovvero esiste $M > 0$ reale tale che \[
        \abs*{a_n} \leq M.    
    \] Inoltre per definizione di limite sappiamo che \begin{align}
        &\forall \eta > 0. \quad \exists n_a \in \N. \quad \forall n \geq n_a. \quad \abs*{a_n - a} < \eta. \label{eq:prod_lim_succ_conv:1}\\
        &\forall \eta > 0. \quad \exists n_b \in \N. \quad \forall n \geq n_b. \quad \abs*{b_n - b} < \eta. \label{eq:prod_lim_succ_conv:2}
    \end{align}

    Allora per ogni $n \geq \max \set{n_a, n_b}$ varrà che \begin{align*}
        \abs*{a_nb_n - ab} &= \abs*{a_nb_n - a_nb + a_nb - ab} \\
        &= \abs*{a_n(b_n - b) + b(a_n - a)} \tag{per \ref{def:distanza:dis_triang}}\\
        &< \abs*{a_n}\abs*{b_n - b} + \abs*{b}\abs*{a_n - a} \tag{$\seqn*{a}$ è limitata}\\
        &< M\abs*{b_n - b} + \abs*{b}\abs*{a_n - a} \tag{per \eqref{eq:prod_lim_succ_conv:1} e \eqref{eq:prod_lim_succ_conv:2}} \\
        &< M\eta + \abs*{b}\eta\\
        &= \eta(M + \abs*{b}).
        \intertext{Sia ora $\eps > 0$ qualunque. Fissiamo $\eta = \dfrac{\eps}{M + \abs*{b}}$, da cui segue che}
        \abs*{a_nb_n - ab} &< \frac{\eps}{M + \abs*{b}}(M + \abs*{b})\\
        &< \eps,
    \end{align*}
    ovvero la successione $\left( a_n \cdot b_n \right)$ converge e $a_nb_n \to ab$.
\end{proof}

\begin{proposition}
    [Limite del reciproco di una successione convergente] \label{prop:lim_recip_succ_conv}
    Sia $\seqn*{a}$ una successione convergente tale che $a_n \to a$ con $a \neq 0$. Supponiamo inoltre che $a_n \neq 0$ definitivamente.

    Allora la successione $\left(\dfrac{1}{a_n}\right)$ è convergente e \[
        \frac{1}{a_n} \to \frac1a.    
    \]
\end{proposition}
\begin{proof}
    Innanzitutto mostro che $\left(\dfrac{1}{a_n}\right)$ è limitata. Per definizione di successione convergente sappiamo che \begin{equation}
       \forall \eps > 0. \quad \exists n_a \in \N. \quad \forall n \geq n_a. \quad \abs*{a_n - a} < \eps. \label{eq:lim_recip_succ_conv:1}
    \end{equation}
    
    Sia $\eps = \frac{1}{2}\abs*{a}$. Allora segue che \begin{align*}
        \abs*{a} &= \abs*{a - a_n + a_n} \tag{per \ref{def:distanza:dis_triang}}\\
        &= \abs*{a - a_n} + \abs*{a_n} \tag{per \eqref{eq:lim_recip_succ_conv:1}}\\
        &< \frac{\abs*{a}}{2} + \abs*{a_n}\\
        \iff \abs*{a_n} &> \abs*{a} - \frac{\abs*{a}}{2} \\
        &= \frac{\abs*{a}}{2}.
    \end{align*}
    
    Per ipotesi $a_n \neq 0$ definitivamente, ovvero per ogni $n \geq n_0$ per qualche $n_0 \in \N$. Dunque per ogni $n \geq \max \set{n_0, n_a}$ otteniamo che \begin{equation}
        \abs*{\frac{1}{a_n}} < \frac{2}{\abs*{a}}. \label{eq:lim_recip_succ_conv:2}
    \end{equation}

    Allora (sempre per ogni $n \geq \max \set{n_0, n_a}$) varrà che \begin{align*}
        \abs*{\frac{1}{a_n} - \frac{1}{a}} &= \abs*{\frac{a - a_n}{a \cdot a_n}}\\
        &= \frac{\abs*{a_n - a}}{\abs{a}\abs{a_n}} \tag{per la \eqref{eq:lim_recip_succ_conv:2} e la \eqref{eq:lim_recip_succ_conv:1}}\\
        &< \frac{2\eps}{\abs*{a}},
    \end{align*}
    ovvero la successione $\left(\dfrac{1}{a_n}\right)$ converge e $\dfrac{1}{a_n} \to \dfrac{1}{a}$.
\end{proof}

\begin{proposition}
    [Limite del rapporto di successioni convergenti] \label{prop:rapp_lim_succ_conv}
    Siano $\seqn*{a}, \seqn*{b}$ due successione convergente rispettivamente ad $a, b \in \R$ con $b \neq 0$. Supponiamo inoltre che $b_n \neq 0$ definitivamente.

    Allora la successione $\left(\dfrac{a_n}{b_n}\right)$ è convergente e \[
        \frac{a_n}{b_n} \to \frac{a}{b}.    
    \]
\end{proposition}
\begin{proof}
    Possiamo scrivere la successione come $\left(a_n \cdot \frac{1}{b_n}\right)$: allora dato che la successione $\seqn*{b}$ verifica le ipotesi della \autoref{prop:lim_recip_succ_conv}, dunque $\frac{1}{b_n} \to \frac1b$. Dunque la successione originale verifica le ipotesi della \autoref{prop:prod_lim_succ_conv}, da cui segue che $\dfrac{a_n}{b_n} \to \dfrac{a}{b}$.
\end{proof}

Questo esclude tutti i casi in cui una delle due successioni (o entrambe) siano divergenti. Le prossime tre proposizioni si occupano di studiare questi casi.

\begin{proposition}
    [Algebra degli Infiniti - Successione divergente positivamente] \label{prop:alg_inf_pos}
    Siano $\seqn*{a}, \seqn*{b}$ due successioni, $a_n \to +\infty$.
    \begin{enumerate}[label={(\roman*)}, ref={\theproposition: (\roman*)}]
        \item Se $\seqn*{b}$ è limitata inferiormente, allora $a_n + b_n \to +\infty$.
        \item Se $\seqn*{b}$ converge a $b \in \R$, $b > 0$, allora $a_nb_n \to +\infty$.
        \item Se $\seqn*{b}$ converge a $b \in \R$, $b < 0$, allora $a_nb_n \to -\infty$.
    \end{enumerate}
\end{proposition}
\begin{proof}
    Siccome $\seqn*{a}$ diverge positivamente, per definizione abbiamo che: \begin{equation}
        \forall M \in \R. \quad \exists n_0 \in \N. \quad \forall n \geq n_0. \quad a_n \geq M.
    \end{equation}
    Dimostriamo separatamente le tre affermazioni.
    \begin{enumerate}[label={(\roman*)}]
        \item Siccome $\seqn*{b}$ è limitata inferiormente allora dovrà esistere $L \in \R$ tale che $b_n \geq L$ per ogni $n \in \N$. 

        Sia $N \in \R$ qualsiasi. Fisso $M = N - L$, da cui segue che definitivamente \begin{equation*}
            a_n + b_n \geq M + L = N - L + L = N,
        \end{equation*}
        dunque per l'arbitrarietà di $N$ segue che $a_n + b_n \to +\infty$.
        \item Siccome $b_n \to b > 0$ allora per definizione di limite \[
            \forall \eps > 0. \quad \exists n_b \in \N. \quad \forall n \geq n_b. \quad \abs*{b_n - b} < \eps.
        \] Fisso $\eps \deq \frac{b}{2}$, da cui segue che per ogni $n \geq n_b$ \[
            0 < \frac{b}{2} < b_n < \frac{3b}{2}.
        \]
        
        Sia $N \in \R$ qualsiasi; fisso $M \deq N\frac{2}{b}$. Allora per ogni $n \geq \max \set{n_0, n_b}$ vale che \begin{align*}
            a_n &\geq M \tag{sappiamo che $b_n > 0$}\\
            \implies a_nb_n &\geq Mb_n \tag{$b_n > \frac{b}{2}$}\\
            &> M\frac{b}{2} \\
            &= N \frac{2}{b}\frac{b}{2}\\
            &= N,
        \end{align*}
        dunque per arbitrarietà di $N$ segue che $a_nb_n \to +\infty$.
        \item Siccome $b_n \to b < 0$ allora per definizione di limite \[
            \forall \eps > 0. \quad \exists n_b \in \N. \quad \forall n \geq n_b. \quad \abs*{b_n - b} < \eps.
        \] Fisso $\eps \deq -\frac{b}{2}$, da cui segue che per ogni $n \geq n_b$ \[
            \frac{3b}{2} < b_n < \frac{b}{2} < 0.
        \]
        
        Sia $N \in \R$ qualsiasi; fisso $M \deq N\frac{2}{b}$. Allora per ogni $n \geq \max \set{n_0, n_b}$ vale che \begin{align*}
            a_n &\geq M \tag{sappiamo che $b_n < 0$}\\
            \implies a_nb_n &\leq Mb_n \tag{$b_n < \frac{b}{2}$}\\
            &< M\frac{b}{2} \\
            &= N \frac{2}{b}\frac{b}{2}\\
            &= N,
        \end{align*}
        dunque per arbitrarietà di $N$ segue che $a_nb_n \to -\infty$. \qedhere
    \end{enumerate}
\end{proof}

\begin{proposition}
    [Algebra degli Infiniti - Successione divergente negativamente] \label{prop:alg_inf_neg}
    Siano $\seqn*{a}, \seqn*{b}$ due successioni, $a_n \to -\infty$.
    \begin{enumerate}[label={(\roman*)}, ref={\theproposition: (\roman*)}]
        \item Se $\seqn*{b}$ è limitata superiormente, allora $a_n + b_n \to -\infty$.
        \item Se $\seqn*{b}$ converge a $b \in \R$, $b > 0$, allora $a_nb_n \to -\infty$.
        \item Se $\seqn*{b}$ converge a $b \in \R$, $b < 0$, allora $a_nb_n \to +\infty$.
    \end{enumerate}
\end{proposition}

\begin{proposition}
    [Algebra degli Infiniti - Reciproci] \label{prop:alg_inf_recip}
    Sia $\seqn*{a}$ una successione.
    \begin{enumerate}[label={(\roman*)}, ref={\theproposition: (\roman*)}]
        \item Se $\seqn*{a}$ diverge (positivamente o negativamente), allora $\frac{1}{a_n} \to 0$.
        \item Se $a_n \to 0$ e $a_n \neq 0$ definitivamente, allora $\frac{1}{\abs*{a_n}} \to +\infty$.
        
        In particolare \begin{itemize}
            \item se $a_n > 0$ definitivamente, allora $\frac{1}{a_n} \to +\infty$,
            \item se $a_n < 0$ definitivamente, allora $\frac{1}{a_n} \to -\infty$.
        \end{itemize}
    \end{enumerate}
\end{proposition}
\begin{proof}
    Dimostriamo i vari casi separatamente.
    \begin{enumerate}[label={(\roman*)}]
        \item Se $\seqn*{a}$ diverge allora il suo modulo dovrà divergere positivamente, ovvero \[
            \forall M \in \R. \quad \exists n_0 \in \N. \quad \forall n \geq n_0. \quad \abs*{a_n} > M.    
        \]

        Per il \thmref{th:perm_segno_succ} $a_n > 0$ definitivamente, dunque $a_n \neq 0$ definitivamente. Allora dovrà valere (definitivamente) \begin{align*}
            &\frac{1}{\abs*{a_n}} < \frac{1}{M} \\
            % \iff -\frac{1}{M} < &\frac{1}{a_n} < \frac{1}{M}.\\
            \intertext{Sia $\eps > 0$ qualsiasi. Fisso $M \deq \frac{1}{\eps}$, da cui segue}
            \iff &\abs*{\frac{1}{a_n}} < \eps,
        \end{align*}
        ovvero $\dfrac{1}{a_n} \to 0$ per l'arbitrarietà di $\eps$.
        \item Per definizione di successione convergente (a $0$)\[
            \forall \eps > 0. \quad \exists n_a \in \N. \quad \forall n \geq n_a. \quad \abs*{a_n} < \eps.    
        \] Siccome $a_n \neq 0$ definitivamente e $\eps > 0$ possiamo passare al reciproco: \[
            \frac{1}{\abs*{a_n}} > \frac{1}{\eps}.
        \] Sia $M \in \R$ qualsiasi; fisso allora $\eps \deq \frac{1}{\abs*{M}}$, da cui segue \begin{equation}
             \frac{1}{\abs*{a_n}} > \abs*{M}, \label{eq:alg_inf_recip:1}
        \end{equation} ovvero $\dfrac{1}{\abs*{a_n}} \to +\infty$.

        Consideriamo ora i due casi particolari. Possiamo scrivere la \eqref{eq:alg_inf_recip:1} equivalentemente come \begin{equation}
            \frac{1}{a_n} < -\abs*{M}, \quad \text{oppure} \quad \frac{1}{a_n} > \abs*{M}.
        \end{equation}
        \begin{itemize}
            \item Se $a_n > 0$ definitivamente allora anche il suo reciproco sarà definitivamente positivo, dunque non potrà essere minore di $-\abs*{M}$ che è negativo. Segue quindi che \[
                \frac{1}{a_n} > \abs*{M} \quad \text{definitivamente,}
            \] dunque $\dfrac{1}{a_n} \to +\infty$.
            \item Se $a_n < 0$ definitivamente allora anche il suo reciproco sarà definitivamente negativo, dunque non potrà essere maggiore di $\abs*{M}$ che è positivo. Segue quindi che \[
                \frac{1}{a_n} < -\abs*{M} \quad \text{definitivamente,}
            \] dunque $\dfrac{1}{a_n} \to \infty$. \qedhere
        \end{itemize}
    \end{enumerate}
\end{proof}

\begin{proposition}
    [Infinitesima per limitata è infinitesima] \label{prop:inf*lim=>inf}
    Sia $\seqn*{a}$ una successione infinitesima (ovvero $a_n \to 0$) e $\seqn*{b}$ una successione limitata.

    Allora la successione $(a_nb_n)$ è infinitesima.
\end{proposition}
\begin{proof}
    Siccome $\seqn*{a}$ è infinitesima deve valere che \begin{equation*}
        \forall \eta > 0. \quad \exists n_a \in \N. \quad \forall n \geq n_a. \quad \abs*{a_n} < \eta.
    \end{equation*} Inoltre $\seqn*{b_n}$ è limitata, dunque deve esistere un $L \in \R$ positivo tale che $\abs*{b_n} < L$.

    Moltiplicando la prima equazione per la seconda otteniamo che (per ogni $n \geq n_a$)
    \begin{align*}
        \abs*{a_nb_n} &< \eta \cdot L. \\
        \intertext{Sia $\eps > 0$ qualunque. Allora fisso $\eta \deq \frac{\eps}{L}$, da cui segue che}
        \abs*{a_nb_n} &< \frac{\eps}{L} \cdot L \\
        &= \eps,
    \end{align*}
    ovvero $a_nb_n \to 0$ per l'arbitrarietà di $\eps$.
\end{proof}

\begin{unnamed}
    [Teorema del Confronto a 2 per successioni] \label{th:confr_2_succ}
    Siano $\seqn*{a}$, $\seqn*{b}$ due successioni convergenti tali che $a_n \to a$, $b_n \to b$.Allora \begin{enumerate}[label={(\roman*)}, ref={\nameref*{th:confr_2_succ} (\ref*{th:confr_2_succ}: (\roman*))}]
        \item se $a < b$ allora vale che $a_n < b_n$ definitivamente.
        \item se $a_n \leq b_n$ definitivamente segue che $a \leq b$.
    \end{enumerate}
\end{unnamed}
\begin{proof}
    Dimostriamo i due punti separatamente. \begin{enumerate}[label={(\roman*)}]
        \item Consideriamo la successione $c_n \deq b_n - a_n$ per ogni $n \in \N$. Per \autoref{prop:sum_lim_succ_conv} segue che $c_n \to b - a > 0$, dunque per il \thmref{th:perm_segno_succ} vale che $c_n > 0$ definitivamente, da cui segue che $a_n < b_n$ definitivamente.
        \item Supponiamo per assurdo che $a > b$. Allora per il punto precedente dovrebbe essere $a_n > b_n$ definitivamente, il che è contrario all'ipotesi che $a_n \leq b_n$ definitivamente. Dunque $a \leq b$. \qedhere
    \end{enumerate}
\end{proof}

\subsection{Limiti di successioni elementari}

Presentiamo ora alcuni limiti di successioni elementari.

\paragraph{Successione costante} Consideriamo la successione definita da $a_n = k$ per qualche $k \in \R$. Mostriamo che $a_n \to k$.

Sia $\eps > 0$ generico. Siccome $a_n = k$ per ogni $n \in \N$ allora la successione è sempre compresa tra $k - \eps$ e $k + \eps$, qualunque sia il valore di $\eps$.

\paragraph{Successione lineare} Consideriamo la successione definita da $a_n = n$. Mostriamo che $a_n \to +\infty$.

Sia $M \in \R$ generico: allora sicuramente per ogni $n > M$ vale che $a_n = n > M$, dunque $a_n \to +\infty$.

\paragraph{Successione esponenziale} Consideriamo la successione definita da $a_n = b^n$ per qualche $b \in \R$ costante, $b > 0$, $b \neq 1$. Mostriamo che \[
    a_n \to \begin{cases}
        +\infty, &\text{se } a > 1\\
        0, &\text{se } 0 \leq a < 1.
    \end{cases}    
\]

\begin{description}
    \item[(Caso $b > 1$)] Siccome $b > 1$ per la disuguaglianza di Bernoulli vale che \[
        b^n = (1 + (b-1))^n \geq 1 + (b-1)n.    
    \] Il membro destro tende a $+\infty$ per il teorema algebrico, dunque per il \thmref{th:confr_asint_succ} segue che $b_n \to +\infty$.
    \item[(Caso $0 < b < 1$)] Sia $c \deq \frac{1}{b}$ con $c > 1$. Dunque \[
        b^n = \left( \frac{1}{c} \right)^n = \frac{1}{c^n} \to 0    
    \] per la \autoref{prop:alg_inf_recip}, in quanto $c^n \to +\infty$ per il caso precedente.
\end{description}

\paragraph{Radice $n$-esima} Consideriamo la successione definita da $a_n = \sqrt[n]{b}$ per qualche $b \in \R$ costante, $b > 0$. Mostriamo che $a_n \to 1$ qualunque sia $b$.

Consideriamo la disuguaglianza di Bernoulli con termine generico $x$ e applichiamo la radice $n$-esima ad entrambi i membri: \[
    1 + x \geq \sqrt[n]{1 + nx} 
\] che sostituendo a $x$ il valore $\dfrac{b - 1}{n}$ diventa \[
    1 + \frac{b-1}{n} \geq \sqrt[n]{b}.
\]

Distinguiamo ora due casi: \begin{description}
    \item[(Caso $b \geq 1$)] Dato che la radice $n$-esima è strettamente crescente, avremo che \[
        1 = \sqrt[n]{1} \leq \sqrt[n]{b} \leq 1 + \frac{b-1}{n}.
    \] Ma il membro di destra tende a $1$, dunque per il \thmref{th:carab_succ} vale che $\sqrt[n]{b} \to 1$.
    \item[(Caso $0 < b < 1$)] Sia $c \deq \frac{1}{b}$ con $c > 1$. Allora \[
        \sqrt[n]{b} = \sqrt[n]{\frac{1}{c}} = \frac{1}{\sqrt[n]{c}} \to 1
    \] per la \autoref{prop:lim_recip_succ_conv}, in quanto $\sqrt[n]{c} \to 1$ per il caso precedente.
\end{description}

\subsection{Criteri per le successioni}

\begin{proposition}
    [Criterio del rapporto] \label{prop:crit_rapporto_succ}
    Sia $\seqn*{a}$ una successione tale che $a_n > 0$ definitivamente. Sia inoltre \[
        l \deq \lim_{n \to +\infty} \frac{a_{n+1}}{a_n} \in \closure{\R}.    
    \]

    Allora \begin{enumerate}[label={(\roman*)}, ref={criterio del rapporto (\theproposition): (\roman*)}]
        \item se $0 \leq l < 1$ allora $a_n \to 0$,
        \item se $l > 1$ allora $a_n \to +\infty$.
    \end{enumerate}
\end{proposition}
\begin{proof}
    Dimostriamo separatamente i due casi.
    \begin{description}
        \item[($0 \leq l < 1$)] Sia $m \in \R$ tale che $l < m < 1$.
        
        Siccome $\dfrac{a_{n+1}}{a_n} \to l$ deve valere che (per ogni $\eps > 0$) $\dfrac{a_{n+1}}{a_n} \leq l + \eps$ definitivamente. In particolare poniamo $\eps = m - l$ (che è positivo in quanto $m > l$), ottenendo che esiste un $n_0 \in \N$ tale che \[
            \frac{a_{n+1}}{a_n} \leq m \text{    per ogni } n \geq n_0.
        \]
        Dunque partendo da $n_0$ abbiamo che \begin{align*}
            &\frac{a_{n_0+1}}{a_{n_0}} \leq m \tag{moltiplicando per $a_{n_0} > 0$}\\
            \iff &a_{n_0+1} \leq m \cdot a_{n_0}\\
            \iff &a_{n_0+2} \leq m \cdot a_{n_0 + 1} \leq m^2 \cdot a_{n_0}\\
            \intertext{Dunque per induzione si può mostrare che per ogni $k \in \N$}
            \iff &a_{n_0+k} \leq m^k \cdot a_{n_0}.
        \end{align*}

        Sia $k$ tale che $n = n_0 + k$, ovvero $k = n - n_0$. Allora \[
            0 \stackrel{\text{Hp.}}{<} a_n \leq m^n \cdot m^{-n_0}a_{n_0}. 
        \] Ma $m^n \cdot m^{-n_0}a_{n_0}$ è il prodotto di una successione infinitesima per una costante, dunque tende a $0$.

        Per il Teorema dei Carabinieri \ref{th:carab_succ} segue che $a_n \to 0$.
        \item[($l > 1$ reale)] Sia $m \in \R$ tale che $1 < m < l$.
        
        Siccome $\dfrac{a_{n+1}}{a_n} \to l$ deve valere che (per ogni $\eps > 0$) $\dfrac{a_{n+1}}{a_n} \geq l - \eps$ definitivamente. In particolare poniamo $\eps = l - m$ (che è positivo in quanto $m < l$), ottenendo che esiste un $n_0 \in \N$ tale che \[
            \frac{a_{n+1}}{a_n} \geq m \text{    per ogni } n \geq n_0.
        \]
        Dunque partendo da $n_0$ abbiamo che \begin{align*}
            &\frac{a_{n_0+1}}{a_{n_0}} \geq m \tag{moltiplicando per $a_{n_0} > 0$}\\
            \iff &a_{n_0+1} \geq m \cdot a_{n_0}\\
            \iff &a_{n_0+2} \geq m \cdot a_{n_0 + 1} \geq m^2 \cdot a_{n_0}\\
            \intertext{Dunque per induzione si può mostrare che per ogni $k \in \N$}
            \iff &a_{n_0+k} \geq m^k \cdot a_{n_0}.
        \end{align*}

        Sia $k$ tale che $n = n_0 + k$, ovvero $k = n - n_0$. Allora \[
            a_n \geq m^n \cdot m^{-n_0}a_{n_0}. 
        \] Ma $m^n \cdot m^{-n_0}a_{n_0}$ è il prodotto di una successione che divergente positivamente per una costante positiva, dunque tende a $+\infty$.

        Per il Teorema del Confronto Asintotico \ref{th:confr_asint_succ} segue che $a_n \to +\infty$.
        \item[($l = +\infty$)] Si dimostra analogamente al caso precedente scegliendo un $M > 1$ qualsiasi. \qedhere
    \end{description}
\end{proof}

\begin{proposition}
    [Criterio della radice] \label{prop:crit_radice_succ}
    Sia $\seqn*{a}$ una successione tale che $a_n > 0$ definitivamente. Sia inoltre \[
        l \deq \lim_{n \to +\infty} \sqrt[n]{a_{n}} \in \closure{\R}.    
    \]

    Allora \begin{enumerate}[label={(\roman*)}, ref={criterio della radice (\theproposition): (\roman*)}]
        \item se $0 \leq l < 1$ allora $a_n \to 0$,
        \item se $l > 1$ allora $a_n \to +\infty$.
    \end{enumerate}
\end{proposition}
\begin{proof}
    \begin{description}
        \item[($0 \leq l < 1$)] Sia $m \in \R$ tale che $l < m < 1$.
        
        Siccome $\sqrt[n]{a_n} \to l$ deve valere che (per ogni $\eps > 0$) $\sqrt[n]{a_n} \leq l + \eps$ definitivamente. In particolare poniamo $\eps = m - l$ (che è positivo in quanto $m > l$), ottenendo che esiste un $n_0 \in \N$ tale che \[
            \sqrt[n]{a_n} \leq m \text{    per ogni } n \geq n_0.
        \] Eleviamo entrambi i membri alla $n$, ottenendo che \[
            0 \stackrel{\text{Hp.}}{\leq} a_n \leq m^n.    
        \] Ma $m^n \to 0$ in quanto $m < 1$, da cui segue per il \thmref{th:carab_succ} che $a_n \to 0$.
        \item[($l > 1$ reale)] Sia $m \in \R$ tale che $1 < m < l$.
        
        Siccome $\sqrt[n]{a_n} \to l$ deve valere che (per ogni $\eps > 0$) $\sqrt[n]{a_n} \geq l - \eps$ definitivamente. In particolare poniamo $\eps = l - m$ (che è positivo in quanto $m < l$), ottenendo che esiste un $n_0 \in \N$ tale che \[
            \sqrt[n]{a_n} \geq m \text{    per ogni } n \geq n_0.
        \] Eleviamo entrambi i membri alla $n$, ottenendo che \[
            a_n \geq m^n.    
        \] Ma $m^n \to +\infty$ poiché $m > 1$, dunque per il \thmref{th:confr_asint_succ} vale che $a_n \to +\infty$.
        \item[($l = +\infty$)] Si dimostra analogamente al caso precedente scegliendo un $M > 1$ qualsiasi. \qedhere
    \end{description} 
\end{proof}

\begin{proposition}
    [Rapporto implica radice] \label{prop:rapp=>radice}
    Sia $\seqn*{a}$ una successione tale che $a_n > 0$ definitivamente. Allora se esiste $l \in \closure{\R}$ tale che \[
        l \deq \lim_{n \to +\infty} \frac{a_{n+1}}{a_n}    
    \] segue che la successione $(\sqrt[n]{a_n})$ è convergente e \[
        \lim_{n \to +\infty} \sqrt[n]{a_n} = l.
    \]
\end{proposition}
\begin{proof}
    Dato che $a_n > 0$ definitivamente segue che $l > 0$. 
    
    \paragraph{Caso (i)}Supponiamo inizialmente $l \in [0, +\infty)$.

    Sia $\eps > 0$ qualsiasi. Per ipotesi esiste $n_0 \in \N$ tale che \[
        l - \frac\eps{2} \leq \frac{a_{n+1}}{a_n} \leq l + \frac\eps{2}.  \quad \forall n \geq n_0  
    \] Moltiplicando tutto per $a_n > 0$ otteniamo \[
        a_n\left(l - \frac\eps{2}\right) \leq a_{n+1} \leq  a_n\left(l + \frac\eps{2}\right).
    \] Partendo da $n = n_0$ si può mostrare per induzione su $k$ che \[
        a_{n_0}\left(l - \frac\eps{2}\right)^k \leq a_{n_0 + k} \leq a_{n_0}\left(l + \frac\eps{2}\right)^k.
    \] Sia $k \deq n - n_0$; segue che \[
        a_{n_0}\left(l - \frac\eps{2}\right)^{n - n_0} \leq a_{n} \leq a_{n_0}\left(l + \frac\eps{2}\right)^{n - n_0}.
    \] Facendo la radice $n$-esima di tutti i membri \[
        \left(l - \frac{\eps}{2}\right)\sqrt[n]{a_{n_0}\left(l - \frac\eps{2}\right)^{n_0}} \leq \sqrt[n]{a_n} \leq \left(l + \frac{\eps}{2}\right)\sqrt[n]{a_{n_0}\left(l + \frac\eps{2}\right)^{n_0}}.
    \]

    Il membro destro tende a $l + \frac\eps{2}$ per $n \to +\infty$, dunque sarà definitivamente minore di $l + \eps$. Analogamente il membro sinistro sarà definitivamente maggiore di $l - \eps$. Dunque \[
        l - \eps \leq \sqrt[n]{a_n} \leq l + \eps \quad \text{definitivamente,}    
    \] da cui concludiamo che $\sqrt[n]{a_n} \to l$ per l'arbitrarietà di $\eps$.

    \paragraph{Caso (ii)} Supponiamo invece $l = +\infty$.

    Sia $M > 0$ qualsiasi; per ipotesi esiste $n_0 \in \N$ tale che \[
        \frac{a_{n+1}}{a_n} \geq M.  \quad \forall n \geq n_0  
    \] Moltiplicando entrambi i membri per $a_n > 0$ otteniamo \[
        a_{n+1} \geq Ma_n. 
    \] Per la stessa induzione del caso (i) possiamo mostrare che \[
        a_n \geq  M^n \cdot a_{n_0}M^{-n_0}.
    \] Il membro destro tende a $+\infty$, dunque per il \thmref{th:confr_asint_succ} vale che $a_n \to +\infty = l$.
\end{proof}
\section{Sottosuccessioni}

\begin{definition}
    [Successione di indici] \label{def:succ_indici}
    Sia $\seqn*{a}$ una successione a valori in $\N$ strettamente crescente. Allora $\seqn*{a}$ si dice \emph{successione di indici}.
\end{definition}

Ad esempio le successioni $a_n \deq 2n$ (numeri pari) e $b_n \deq 2n + 1$ sono successioni di indici.

\begin{definition}
    [Sottosuccessione estratta] \label{def:sottosucc}
    Siano $\seqn*{a}$, $\seqn*{b}$ due successioni.

    Allora si dice che la successione $\seqn*{b}$ è una \emph{sottosuccessione estratta} da $\seqn*{a}$ (o semplicemente \emph{sottosuccessione} di $\seqn*{a}$) se esiste una successione di indici $\seqn*[k]{n}$ tale che \[
        b_n = a_{n_k}.    
    \]
\end{definition}

\begin{proposition}
    [Le sottosucc. hanno lo stesso limite della successione originale] \label{prop:ssuc_conv_lim_succ}
    Sia $\seqn*{a}$ una successione e $\seqn*[n]{b}$ una sua sottosuccessione.

    Allora se $a_n \to l \in \closure{\R}$ segue che $b_{n} \to l$.
\end{proposition}
\begin{proof}
    Supponiamo $l \in \R$. Allora per definizione di limite avremo che \[
        \forall \eps > 0. \quad \exists n_0 \in \N. \quad \forall n \geq n_0. \quad \forall \abs*{a_n - l} < \eps.
    \] Siccome $\seqn*{b}$ è una sottosuccessione di $\seqn*{a}$ dovrà esistere una successione di indici $\seqn*{k}$ tale che $b_n = a_{k_n}$. Dato che $\seqn*{k}$ è crescente varrà che $k_n \geq n$; dunque se $n \geq n_0$ a maggior ragione $k_n \geq n_0$, per cui \[
        b_n = a_{k_n} \in (l - \eps, l+\eps),    
    \] ovvero $b_n \to l$.

    Analogo ragionamento nei casi $a_n \to +\infty$, $a_n \to -\infty$.
\end{proof}

\begin{corollary}
    Sia $\seqn*{a}$ una successione e $\seqn*{b}, \seqn*{c}$ due sottosuccessioni estratte. Se \[
        b_n \to \beta, \quad c_n \to \gamma     
    \] con $\beta \neq \gamma$ allora vale che $\seqn*{a}$ non ha limite.
\end{corollary}
\begin{proof}
    Supponiamo per assurdo che $a_n \to \alpha \in \closure{\R}$. Allora le sottosuccessioni estratte $\seqn*{b}, \seqn*{c}$ dovrebbero convergere ad $\alpha$ per la \autoref{prop:ssuc_conv_lim_succ}, il che è assurdo poiché convergono a due numeri diversi, dunque la tesi.
\end{proof}
\section{Limite superiore e inferiore di successioni}

\begin{definition}
    [Maggioranti e minoranti definitivi] \label{def:magg_min_def}
    Sia $\seqn*{a}$ una successione. 
    
    Allora si dice che $M \in \R$ è un \emph{maggiorante definitivo di $\seqn*{a}$} se esiste $n_0 \in \N$ tale che \begin{equation}
        \label{eq:magg_def} a_n \leq M \quad \text{per ogni } n \geq n_0.
    \end{equation}

    Allo stesso modo si dice che $N \in \R$ è un \emph{minorante definitivo di $\seqn*{a}$} se \begin{equation}
        \label{eq:min_def} a_n \geq N \quad \text{per ogni } n \geq n_0.
    \end{equation}

    L'insieme dei maggioranti definitivi si indica con $\defUppBound{a}$, mentre l'insieme dei minoranti definitivi si indica con $\defLowBound a$.
\end{definition}

Notiamo che se una successione è limitata superiormente, ovvero $\set {a_n \suchthat n \in \N}$ ammette un maggiorante $M$, allora tale maggiorante è anche un maggiorante definitivo: infatti $a_n \geq M$ per ogni $n$, dunque lo sarà anche definitivamente.
Non vale necessariamente il contrario: ad esempio la successione definita da $a_n \deq \dfrac{10}{n}$ ammette $1$ come maggiorante definitivo (in quanto $a_n \leq 1$ per $n \geq 10$) ma non come maggiorante (poiché la disuguaglianza non vale per $n < 10$).

\begin{definition} [Limite superiore e inferiore]
    Sia $\seqn*{a}$ una successione e siano $\defUppBound{a}$, $\defLowBound a$ gli insiemi dei maggioranti e minoranti defininitivi. Allora definisco \begin{equation}
        \label{def:lim_sup} \limsup_{n \to +\infty} a_n = \begin{cases}
            +\infty, &\text{se $\seqn*{a}$ è illimitata superiormente,}\\
            \inf \defUppBound a, &\text{altrimenti.}
        \end{cases}
    \end{equation}\begin{equation}
        \label{def:lim_inf} \liminf_{n \to +\infty} a_n = \begin{cases}
            -\infty, &\text{se $\seqn*{a}$ è illimitata inferiormente,}\\
            \sup \defLowBound a, &\text{altrimenti.}
        \end{cases}
    \end{equation}
\end{definition}

\begin{remark}
    Vale sempre che il limite inferiore è minore o uguale del limite superiore.
\end{remark}

Vediamo alcune caratterizzazioni alternative del limite superiore e inferiore.

\begin{proposition} [Caratterizzazione I di limite superiore e inferiore]
    \label{prop:caratt_1_limsupinf}
    Sia $\seqn*{a}$ una successione limitata.
    
    Allora vale che \begin{align}
        \limsup_{n \to +\infty} a_n &= \inf_{n \in \N} \sup \set{a_k \suchthat k \geq n}.\label{eq:caratt_1_limsup}\\
        \liminf_{n \to +\infty} a_n &= \sup_{n \in \N} \inf \set{a_k \suchthat k \geq n}.\label{eq:caratt_1_liminf}
    \end{align}
\end{proposition}
\begin{proof}
    Dimostriamo la \eqref{eq:caratt_1_liminf}; la dimostrazione della \eqref{eq:caratt_1_limsup} è analoga.

    Sia $\left( I^{(a)}_n \right)$ la successione definita da \[
        I^{(a)}_n \deq \inf \set{a_k \suchthat k \geq n}.
    \] Allora per ogni $n \in \N$ dovrà valere che $I^{(a)}_n$ è un minorante definitivo (infatti $a_k \geq I^{(a)}_n$ per ogni $k \geq n$, ovvero definitivamente).
    Da ciò segue che \[
        \liminf_{n\to +\infty} a_n \geq I^{(a)}_n \quad \forall n \in \N    
    \] Dunque, siccome $\liminf a_n$ è un maggiorante dell'insieme $\set{I^{(a)}_n\suchthat n \in \N}$ allora dovrà essere maggiore o uguale all'estremo superiore dell'insieme, ovvero \[
        \liminf_{n \to +\infty} a_n \geq \sup_{n \in \N} I^{(a_n)}.    
    \]

    Dimostriamo ora l'uguaglianza opposta.
    Sia $N$ un minorante definitivo di $\left( I^{(a)}_n \right)$. Allora per definizione dovrà esistere un intero positivo $n_0 \in \N$ tale che $a_k \geq N$ per ogni $k \geq n_0$. 
    
    Siccome $N$ è minore o uguale ad ogni $a_k$ dovrà essere minore o uguale all'estremo inferiore dell'insieme, ovvero \[
        N \leq \inf_{k \geq n_0} a_k = I^{(a)}_{n_0} \leq \sup_{n \in \N} I^{(a)}_n.
    \] Ma questa disuguaglianza vale per qualsiasi minorante definitivo $N$, dunque \[
        \liminf_{n \to +\infty} a_n \leq \sup_{n \in \N} I^{(a)}_n.    \qedhere
    \]
\end{proof}


\begin{remark}
    La successione $\left( I^{(a)}_n \right)$ definita nella dimostrazione precedente è debolmente crescente (stiamo calcolando l'estremo inferiore di un insieme via via più piccolo, ed in particolare sempre contenuto nei precedenti), dunque per il \autoref{cor:succ_cresc_limit} vale che \[
        \lim_{n \to +\infty} I^{(a)}_n = \sup \set{I^{(a)}_n \suchthat n \in \N} = \liminf_{n \to +\infty} a_n.    
    \] 
    Similmente definiendo la successione $\left( S^{(a)}_n \right)$ tale che \[
        S^{(a)}_n \deq \sup \set{a_k \suchthat k \geq n}.
    \] notiamo che essa è debolmente decrescente, dunque per la \autoref{cor:succ_decr_limit} ammette limite reale e quindi \[
        \lim_{n \to +\infty} S^{(a)}_n = \inf \set{S^{(a)}_n \suchthat n \in \N} = \limsup_{n \to +\infty} a_n.    
    \] 
\end{remark}
Dunque si può questa definizione alternativa di limite superiore e limite inferiore:
\begin{corollary}
    [Caratterizzazione II di limite superiore e inferiore] \label{cor:caratt_2_limsupinf}
    Sia $\seqn*{a}$ una successione. Allora vale che 
    \begin{align}
        \limsup_{n \to +\infty} a_n &= \begin{cases}
            +\infty, &\text{se $\seqn*{a}$ è illim. sup.,}\\
            \displaystyle \lim_{n \to +\infty} S^{(a)}_n , &\text{altrimenti.}
        \end{cases} \label{def:limsup_II}\\
        \liminf_{n \to +\infty} a_n &= \begin{cases}
            -\infty, &\text{se $\seqn*{a}$ è illim. inf.,}\\
            \displaystyle \lim_{n \to +\infty} I^{(a)}_n , &\text{altrimenti.} \label{def:liminf_II}
        \end{cases}
    \end{align}
    dove $\left(S^{(a)}_n\right), \left(I^{(a)}_n\right)$ sono due successioni tali che \[
        S^{(a)}_n = \sup \set{a_k \suchthat k \geq n}, \quad I^{(a)}_n = \inf \set{a_k \suchthat k \geq n}.   
    \]
\end{corollary}
    
Diamo una terza e ultima caratterizzazione dei limiti superiori ed inferiori.

\begin{proposition}
    [Caratterizzazione III del limite superiore]
    \label{prop:caratt_3_limsup}
    Sia $\seqn*{a}$ una successione. Allora $L \in \R$ è il limite superiore della successione $\seqn*{a}$ se e solo se
    \begin{enumerate}[label={(\roman*)}]
        \item per ogni $\eps > 0$ vale che $a_n < L + \eps$ definitivamente;
        \item per ogni $\eps > 0$ vale che $a_n > L - \eps$ frequentemente.
    \end{enumerate}

    Invece nel caso la successione non sia limitata abbiamo che \begin{itemize}
        \item $\limsup a_n = +\infty$ se e solo se per ogni $M \in \R$ vale che $a_n > M$ frequentemente;
        \item $\limsup a_n = -\infty$ se e solo se per ogni $M \in \R$ vale che $a_n < M$ definitivamente, ovvero se $a_n \to -\infty$.
    \end{itemize}
\end{proposition}
\begin{proof} 
    Dimostriamo separatamente i casi in cui il limite superiore è un numero reale, è $+\infty$ oppure è $-\infty$.
    
    \paragraph{Limite superiore reale}
    La (i) equivale ad affermare che ogni numero maggiore di $L$ è un maggiorante definitivo, ovvero che $L \geq \limsup a_n$

    La (ii) invece equivale ad affermare che ogni numero minore di $L$ non è un maggiorante definitivo, in quanto $L - \eps < a_n$ per infiniti valori di $n$ (ovvero frequentemente), dunque $L \leq \limsup a_n$.

    Concludiamo quindi che \[
        L = \limsup_{n \to +\infty} a_n. 
    \]

    \paragraph{Limite superiore $+\infty$} Questo equivale ad affermare che la successione $a_n$ non è limitata superiormente per definizione, ovvero per ogni $M \in \R$ deve valere che $a_n > M$ per almeno un valore di $n \in \N$. Tuttavia se ciò accadesse un numero finito di volte la successione avrebbe massimo, ma ciò è assurdo in quanto la successione non è limitata superiormente, dunque $a_n > M$ per infiniti valori di $n$ (ovvero frequentemente).

    \paragraph{Limite superiore $-\infty$}
    Per definizione di limite superiore sappiamo che $\limsup a_n = -\infty$ se e solo se  $\inf \defUppBound{a} = -\infty$, ovvero se e solo se ogni $M \in \R$ è un maggiorante definitivo. Ma questo significa che $a_n \leq M$ definitivamente, ovvero $a_n \to -\infty$.
\end{proof}

Analogamente la proposizione per i limiti inferiori:
\begin{proposition}
    [Caratterizzazione III del limite inferiore]
    \label{prop:caratt_3_liminf}
    Sia $\seqn*{a}$ una successione. Allora $L \in \R$ è il limite inferiore della successione $\seqn*{a}$ se e solo se
    \begin{enumerate}[label={(\roman*)}]
        \item per ogni $\eps > 0$ vale che $a_n > L + \eps$ definitivamente;
        \item per ogni $\eps > 0$ vale che $a_n < L - \eps$ frequentemente.
    \end{enumerate}

    Invece nel caso la successione non sia limitata abbiamo che \begin{itemize}
        \item $\liminf a_n = -\infty$ se e solo se per ogni $M \in \R$ vale che $a_n < M$ frequentemente;
        \item $\liminf a_n = +\infty$ se e solo se per ogni $M \in \R$ vale che $a_n > M$ definitivamente, ovvero se $a_n \to +\infty$.
    \end{itemize}
\end{proposition}

\begin{theorem}
    [Regolarità tramite limite inferiore e superiore]
    Sia $\seqn*{a}$ una successione. Allora $\seqn*{a}$ è regolare se e solo se esiste $l \in \closure{\R}$ tale che \[
        \limsup_{n \to +\infty} a_n = \liminf_{n \to +\infty} a_n = l.   
    \]

    In particolare varrà che \[
        \lim_{n \to +\infty} a_n = l.
    \]
\end{theorem}
\begin{proof}
    Mostriamo entrambi i versi dell'implicazione.
    \begin{description}
        \item[($\impliedby$)] Se $a_n \to +\infty$ allora la funzione è illimitata superiormente (dunque $\limsup a_n = +\infty$) e l'insieme dei minoranti definitivi è tutto $\R$, dunque $\liminf a_n = \sup \defLowBound{a} = +\infty$ da cui la tesi.
        
        Analogo ragionamento se $a_n \to -\infty$.

        Supponiamo quindi che $a_n \to l \in \R$. Allora per definizione di limite esisterà $\eps > 0$ tale che \[
            l - \eps < a_n < l + \eps \quad \text{definitivamente.}    
        \] Ma questo significa che $l + \eps$ è un maggiorante definitivo e $l - \eps$ è un minorante definitivo, dunque \[
            l - \eps < \liminf_{n \to +\infty} a_n \leq \limsup_{n \to +\infty} a_n < l + \eps,
        \] da cui segue che \[
            \liminf_{n \to +\infty} a_n = \limsup_{n \to +\infty} a_n = l.
        \]
        \item[($\implies$)] Se $l = +\infty$ allora per la \autoref{prop:caratt_3_liminf} segue che $a_n \to +\infty$. Analogamente se $l = -\infty$ per la \autoref{prop:caratt_3_limsup} segue che $a_n \to -\infty$.
        
        Supponiamo ora che $l \in \R$. Siccome $l$ è il limite superiore, allora per il \autoref{prop:caratt_3_limsup} segue che per ogni $\eps > 0$ \[
            a_n < l + \eps \quad \text{definitivamente.}
        \] Allo stesso modo, siccome $l$ è il limite inferiore, per la \autoref{prop:caratt_3_liminf} varrà che  per ogni $\eps > 0$ \[
            a_n > l - \eps \quad \text{definitivamente.}    
        \] Dunque \[
            l - \eps < a_n < l + \eps \quad \text{definitivamente,}    
        \] ovvero $a_n \to l$. \qedhere
    \end{description}
\end{proof}

\subsection{Teoremi in versione limite superiore/inferiore}

\begin{theorem}
    [Teorema del Confronto]
    \label{th:confr_succ_limsupinf}
    Siano $\seqn*{a}, \seqn*{b}$ due successioni tali che $a_n \leq b_n$ definitivamente. Allora valgono le seguenti:
    \begin{enumerate}[label={(\roman*)}, ref={\thetheorem: (\roman*)}]
        \item $\displaystyle \liminf_{n \to +\infty} a_n \leq \liminf_{n \to +\infty} b_n$ 
        \item $\displaystyle \limsup_{n \to +\infty} a_n \leq \limsup_{n \to +\infty} b_n$
    \end{enumerate}
\end{theorem}
\begin{proof}
    Supponiamo per semplicità che le successioni siano limitate. Per il \autoref{cor:caratt_2_limsupinf} possiamo considerare le successioni $\left(S^{(a)}_n\right)$ e $\left(S^{(b)}_n\right)$.

    Per ipotesi $a_n \leq b_n$, dunque dovrà valere definitivamente che $S^{(a)}_n \leq S^{(a)}_n$, ovvero facendo i limiti a $+\infty$ per il \autoref{th:confr_2_succ} segue che \[
        \limsup_{n \to +\infty} a_n \leq \limsup_{n \to +\infty} b_n.
    \]

    Analoga dimostrazione per il limite inferiore.
\end{proof}

\begin{theorem}
    [Teorema dei Due Carabinieri]
    \label{th:carab_limsupinf}
    Siano $\seqn*{a}, \seqn*{b}, \seqn*{c}$ tre successioni tali che $a_n \leq b_n \leq c_n$ definitivamente.
    Allora vale che \[
        \liminf_{n\to +\infty} a_n \leq \liminf_{n\to +\infty} b_n \leq \limsup_{n\to +\infty} b_n \leq \limsup_{n\to +\infty} c_n.
    \]
\end{theorem}
\begin{proof}
    Abbiamo già osservato che $\liminf b_n \leq \limsup b_n$. Per il \autoref{th:confr_succ_limsupinf} avremo allora \begin{align*}
        \liminf_{n \to +\infty} a_n &\leq \liminf_{n \to +\infty} b_n \tag{siccome $a_n \leq b_n$}\\
        \limsup_{n \to +\infty} b_n &\leq \limsup_{n \to +\infty} c_n. \tag{siccome $b_n \leq c_n$}
    \end{align*}
    da cui segue che \[
        \liminf_{n \to +\infty} a_n \leq \liminf_{n \to +\infty} b_n \leq \limsup_{n \to +\infty} b_n \leq \limsup_{n \to +\infty} c_n. \qedhere
    \]
\end{proof}


\chapter{Funzioni di variabile reale}

\section{Funzioni di variabile reale}

Studieremo nel seguito principalmente funzioni di variabile reale, ovvero funzioni definite da un sottoinsieme $A$ di $\R$ non vuoto in $\R$.

Studiamo le proprietà fondamentali delle funzioni di variabile reale.

\begin{definition} [Monotonia di una funzione]
    Sia $A \subseteq \R$ non vuoto, $f : A \to \R$.
    
    Allora $f$ si dice
    \begin{enumerate}[label=(\arabic*)]
        \item \emph{strettamente crescente} se per ogni $x, y \in A$, $y > x$ vale che $f(y) > f(x)$;
        \item \emph{debolmente crescente} se per ogni $x, y \in A$, $y > x$ vale che $f(y) \geq f(x)$;
        \item \emph{strettamente decrescente} se per ogni $x, y \in A$, $y > x$ vale che $f(y) < f(x)$;
        \item \emph{debolmente decrescente} se per ogni $x, y \in A$, $y > x$ vale che $f(y) \leq f(x)$;
    \end{enumerate}
    
    Se $f$ è strettamente crescente o decrescente si dice anche che è \emph{strettamente monotona}; invece se è debolmente crescente o decrescente si dice anche \emph{debolmente monotona}.
\end{definition}

\begin{definition}
    [Simmetrie di una funzione]
    Sia $A \subseteq \R$ non vuoto, $f : A \to \R$.

    Allora la funzione $f$ si dice \begin{itemize}
        \item \emph{pari} se per ogni $x \in A$ vale che $f(-x) = f(x)$,
        \item \emph{dispari} se per ogni $x \in A$ vale che $f(-x) = -f(x)$.
    \end{itemize}
\end{definition}

\begin{definition}
    [Periodicità]
    Sia $A \subseteq \R$ non vuoto, $f : A \to \R$.

    Allora la funzione $f$ si dice \emph{periodica} se esiste almeno un $T \in \R$, $T > 0$ tale che \[
        f(x + T) = f(x)
    \] per ogni $x \in A$. Ogni $T$ che soddisfa la precedente equazione si dice \emph{periodo di $f$}. 
    
    Se esiste un minimo $T^*$ che la soddisfa, $T^*$ si dice \emph{minimo periodo di $f$}.
\end{definition}

\section{Limiti di funzioni}

\begin{definition}
    [Limite di funzione]
    Sia $A \subseteq \R$ non vuoto, $f : A \to \R$, $x_0 \in \derived{A}$.

    Allora si dice che il \emph{limite di $f(x)$ per $x$ che tende a $x_0$ è $l \in \closure{\R}$}, e si scrive \[
        \lim_{x \to x_0} f(x) = l, \quad \text{oppure} \quad f(x) \to l \text{   (per $x \to x_0$)}     
    \] se vale che per ogni intorno $\UU$ di $l$ esiste un intorno $\VV$ di $x_0$ tale che per ogni $x$ in $\VV \inters A \setminus x_0$ vale che $f(x) \in \UU$. 
    In simboli \[
        \forall \UU \in \BB(l). \quad \exists \VV \in \BB(x_0). \quad \forall x \in \VV \inters A \setminus \set{x_0}. \quad f(x) \in \UU.    
    \]
\end{definition}

\begin{remark}
    L'insieme derivato di $A$ può anche includere i punti $\pm\infty$, dunque possiamo fare i limiti per $x$ che tende ad infinito se $\pm\infty \in \derived{A}$.
\end{remark}

La definizione topologica (ovvero per intorni) è equivalente alla definizione tramite $\eps$ e $\delta$. Ad esempio nel caso $x_0 \in \R$, $l \in \R$ abbiamo che il limite di $f(x)$ per $x \to x_0$ è $l$ se e solo se \[
    \forall \eps > 0. \quad \exists \delta > 0. \quad \forall x \in (x_0 - \delta, x_0 + \delta) \inters A \setminus \set{x_0}. \quad f(x) \in (l - \eps, l + \eps).
\] Tuttavia dato che gli intorni di $\pm\infty$ sono diversi dagli intorni dei punti di $\R$ dovremmo dare nove definizioni diverse di limite, mentre la definizione topologica è uguale in tutti i casi ed è quindi più comoda.

\end{document}