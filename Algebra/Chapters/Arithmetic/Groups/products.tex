\section{Prodotto diretto di gruppi}

\begin{definition}
    Siano $(G_1, *)$, $(G_2, \star)$ due gruppi. Consideriamo il loro prodotto cartesiano \[
        G_1 \times G_2 = \set*{(g_1, g_2) \given g_1 \in G_1, g_2 \in G_2}    
    \] e un'operazione $\cdot$ su $G_1 \times G_2$ tale che \begin{align*}
    \cdot : (G_1 \times G_2) \times (G_1 \times G_2) &\to (G_1 \times G_2)\\
    ((x, y), (z, w)) &\mapsto (x * z, y \star w).
    \end{align*}
    
    La struttura $(G_1 \times G_2, \cdot)$ si dice \emph{prodotto diretto dei gruppi $G_1$ e $G_2$}.
\end{definition}

\begin{proposition}
    [Il prodotto diretto di gruppi è un gruppo]
    Siano $(G_1, *)$, $(G_2, \star)$ due gruppi. Allora il prodotto diretto $(G_1 \times G_2, \cdot)$ è un gruppo. 
\end{proposition}
\begin{proof}
    Sappiamo già che $\cdot$ è un'operazione su $G_1 \times G_2$, quindi basta mostrare i tre assiomi di gruppo.
    \begin{description}
        \item[Associatività] Siano $(x, y), (z, w), (h, k) \in G_1 \times G_2$. Mostriamo che vale la proprietà associativa.
        \begin{align*}
            &(x, y) \cdot ((z, w) \cdot (h, k)) \tag{def. di $\cdot$}\\
            &=\ (x, y) \cdot (z * h, w \star k)\tag{def. di $\cdot$}\\
            &=\ (x * (z * h), y \star (w \star k))\tag{ass. di $*$ e $\star$}\\
            &=\ ((x * z) * h, (y \star w) \star k) \\
            &=\ (x * z, y \star w) \cdot (h, k) \\
            &=\ ((x, y) \cdot (z, w)) \cdot (h, k).
        \end{align*}  
        \item[Elemento neutro] Siano $e_1 \in G_1, e_2 \in G_2$ gli elementi neutri dei due gruppi. Mostro che $(e_1, e_2)$ è l'elemento neutro del prodotto diretto.
        
        Sia $(x, y) \in G_1 \times G_2$ qualsiasi. Allora \begin{align*}
            &(x, y) \cdot (e_1, e_2) = (x * e_1, y \star e_2) = (x, y)\\
            &(e_1, e_2)\cdot (x, y)  = (e_1 * x, e_2 \star y) = (x, y).
        \end{align*}
        \item[Invertibilità] Sia $(x, y) \in G_1 \times G_2$. Mostriamo che $(x, y)$ è invertibile e il suo inverso è $(x\inv, y\inv) \in G_1 \times G_2$, dove $x\inv$ è l'inverso di $x$ in $G_1$ e $y\inv$ è l'inverso di $y$ in $G_2$.
        \begin{align*}
            &(x, y) \cdot (x\inv, y\inv) = (x * x\inv, y \star y\inv) = (e_1, e_2)\\
            &(x\inv, y\inv)\cdot (x, y)  = (x\inv * x, y\inv \star y) = (e_1, e_2).
        \end{align*} 
    \end{description}
    Dunque il prodotto diretto $(G_1 \times G_2, \cdot)$ è un gruppo.
\end{proof}

\begin{proposition}
    [Il centro del prodotto diretto è il prodotto diretto dei centri]
    Siano $(G_1, *)$, $(G_2, \star)$ due gruppi e sia $(G_1 \times G_2, \cdot)$ il loro prodotto diretto.
    Allora vale che \[
        Z(G_1 \times G_2) = Z(G_1) \times Z(G_2).
    \]
\end{proposition}
\begin{proof}
    Per definizione di centro sappiamo che \begin{multline*}
        Z(G_1 \times G_2) = \left\{\;(x, y) \in G_1 \times G_2 \ColonGiven \right. \\
        \left. (g_1, g_2) \cdot (x, y) = (x, y) \cdot (g_1, g_2) \quad \forall(g_1,g_2) \in G_1 \times G_2\;\right\}.  
    \end{multline*}
    Sia $(x, y) \in Z(G_1 \times G_2)$. Allora per ogni $ (g_1,g_2) \in G_1 \times G_2$ vale che \begin{align*}
        &(g_1, g_2) \cdot (x, y) = (x, y) \cdot (g_1, g_2) \\
        \iff &(g_1 * x, g_2 \star y) = (x * g_1, y \star g_2)\\
        \iff &g_1 * x = x*g_1 \text{ e } g_2 \star y = y \star g_2\\
        \iff &x \in Z(G_1) \text{ e } y \in Z(G_2)\\
        \iff &(x, y) \in Z(G_1) \times Z(G_2).
    \end{align*}
    Seguendo la catena di equivalenze al contrario segue la tesi.
\end{proof}

\begin{proposition}
    [Ordine nel prodotto diretto]
    {ord_prod_diretto}
    Siano $(G_1, *)$, $(G_2, \star)$ due gruppi e sia $(G_1 \times G_2, \cdot)$ il loro prodotto diretto. Sia $(x, y) \in G_1 \times G_2$. Allora vale che \[
        \ord*(G_1 \times G_2){(x, y)} = \gcd*{\ord_{G_1}{x}, \ord_{G_2}{y}}.    
    \]
\end{proposition}
\begin{proof}
    Sia $n = \ord{x}$, $m = \ord y$ e $d = \ord{(x, y)}$. Mostriamo che $d = \gcd{n, m}$.
    \begin{description}
        \item[($d \divides \gcd{n, m}$)] Vale che \[
            (x, y)^{\gcd{n, m}} = \parens*{x^{\gcd{n, m}}, y^{\gcd{n, m}}}.   
        \] Siccome $\ord x = n \divides \gcd{n, m}$ e stessa cosa per $\ord y = m$, per la \Cref{prop:sgr_generato:ord_div_n} segue che \begin{equation*}
            \parens*{x^{\gcd{n, m}}, y^{\gcd{n, m}}} = (e_1, e_2)
        \end{equation*}
        da cui (per la \Cref{prop:sgr_generato:ord_div_n}) segue che $d \divides \gcd{n, m}$.
        \item[($\gcd{n, m} \divides d$)] Per definizione di potenza intera nel prodotto diretto sappiamo che $(x, y)^d = (x^d, y^d)$. Inoltre dato che $d$ è l'ordine di $(x, y)$ segue che $(x, y)^d = (e_1, e_2)$. Dunque \begin{align*}
            &x^d = e_1, \; y^d = e_2\\
            \iff &n \divides d, \; m \divides d \\
            \iff &\gcd{n, m} \divides d.
        \end{align*}
    \end{description}
    Dunque $d = \gcd{n, m}$, ovvero la tesi.
\end{proof}

\begin{theorem}
    [Teorema Cinese del Resto (III forma)] {cinese_III}
    Siano $n, m \in \Z$ entrambi non nulli. Allora vale che \[
        \Zmod{nm} \isomorph \Zmod{n}\times \Zmod{m} \iff \gcd{n, m} = 1.
    \]
\end{theorem}
\begin{proof}
    Sia $G = \Zmod n \times \Zmod m$. Siccome $\abs*{G} = nm$ in virtù del \Cref{th:iso_ciclico} per mostrare che $G \isomorph \Zmod{nm}$ è sufficiente mostrare che $G$ è ciclico.

    Un gruppo è ciclico se e solo se esiste $g \in G$ tale che $\ord{g} = \abs G$: infatti per ogni $g \in G$ vale che $\gen*{g} \sgr G$, dunque se i due insiemi hanno anche la stessa cardinalità devono essere uguali.

    Siano $\eqcl+ x \in \Zmod n, \eqcl+ y \in \Zmod m$ tali che $g = \parens[\big]{\eqcl+ x, \eqcl+ y}$. Per la \Cref{prop:ord_prod_diretto} vale che \[
        \ord{g} = \ord[\big]{(\eqcl+ x, \eqcl+ y)} = \lcm[\Big]{\ord[\big]{\eqcl+ x}, \ord[\big]{\eqcl+ y}}.    
    \]
    D'altro canto però $\ord[\big]{\eqcl+ x} = \dfrac{n}{\gcd{n, x}}$, $\ord[\big]{\eqcl+ y} = \dfrac{m}{\gcd{m, y}}$ (dove $x, y$ sono rappresentanti qualsiasi delle classi $\eqcl+ x$, $\eqcl+ y$ rispettivamente), dunque \[
        \ord g = \lcm*{\dfrac{n}{\gcd{n, x}}, \dfrac{m}{\gcd{m, y}}} \leq \lcm{n, m}. 
    \]

    Possiamo dunque distinguere i due casi: \begin{enumerate}
        \item se $\gcd{n, m} = d > 1$ allora per la PROPOSIZIONE DA INSERIRE per ogni $g \in G$ vale che \[
            \ord g \leq \lcm{n, m} = \frac{mn}{d} < mn    
        \] da cui segue che $G$ non può essere ciclico;
        \item se $\gcd{n, m} = 1$ allora per ogni $g \in G$ vale che \[
            \ord g \leq \lcm{n, m} = mn.    
        \] In particolare se consideriamo $g = \parens[\big]{\eqcl+ 1, \eqcl+ 1}$ si ha che \[
            \ord[big]{\eqcl+ 1, \eqcl+ 1} = \lcm*{\frac{n}{\gcd{n, 1}}, \frac{m}{\gcd{m, 1}}} = \lcm{m, n} = mn  
        \], dunque $G = \gen*{(\eqcl+ 1, \eqcl+ 1)}$, da cui segue che \[
            G \isomorph \Zmod{nm}    
        \] per il \Cref{th:iso_ciclico}. \qedhere
    \end{enumerate} 
\end{proof}

\begin{remark}
    Per il Teorema Cinese del Resto (II Forma) sappiamo che la funzione 
    \begin{align} \label{eq:iso_Znm->Zn_x_Zm}
        \begin{split}
            \phi : \Zmod{nm} &\to \Zmod n \times \Zmod m\\
            \eqcl*{a}[mn] &\mapsto \parens[\big]{\eqcl{a}[n], \eqcl{a}[m]}
        \end{split}
    \end{align} è bigettiva. Inoltre \begin{align*}
        \phi\parens[\big]{\eqcl*{a}[mn] + \eqcl*{b}[mn]} &= \phi\parens[\big]{\eqcl{a+b}[mn]}\\
        &= \parens[\big]{\eqcl{a+b}[n], \eqcl{a+b}[m]}\\
        &= \parens[\big]{\eqcl{a}[n] + \eqcl{b}[n], \eqcl{a}[m] + \eqcl{b}[m]}\\
        &= \parens[\big]{\eqcl{a}[n], \eqcl{a}[m]} + \parens[\big]{\eqcl{b}[n], \eqcl{b}[m]}\\
        &= \phi\parens[\big]{\eqcl*{a}[mn]} + \phi\parens[\big]{\eqcl*{b}[mn]},
    \end{align*} ovvero $\phi$ è un omomorfismo di gruppi. Dunque $\phi$ è un isomorfismo di gruppi e \[
        \Zmod{nm} \isomorph \Zmod{n}\times \Zmod{m}.  
    \]
\end{remark}

\begin{corollary}
    [Isomorfismo tra i gruppi degli invertibili] Siano $n, m \in \Z$ entrambi non nulli. Allora se $\gcd{n, m} = 1$ segue che \begin{equation}
        \units{\Zmod{nm}} \isomorph \units{\Zmod{n}} \times \units{\Zmod{m}}.
    \end{equation}
\end{corollary}
\begin{proof}
    Consideriamo la funzione 
    \begin{align*}
        \phi^* : \units{\Zmod{nm}} &\to \units{\Zmod{n}} \times \units{\Zmod{m}}\\
        \eqcl*{a}[mn] &\mapsto \eqcl{a}[n] \times \eqcl{a}[m].
    \end{align*}
    Essa è ben definita: infatti se $\eqcl*{a}[mn] \in \units{\Zmod{nm}}$ segue che $\gcd{a, mn} = 1$. Siccome per ipotesi $\gcd{m, n} = 1$ per la PROPOSIZIONE NON SCRITTA segue che $\gcd{m, n} = \gcd{a, m} = 1$, ovvero $\eqcl{a}[n] \in \units{\Zmod n}$ e $\eqcl{a}[m] \in \units{\Zmod m}$.

    Inoltre questa funzione è una restrizione della $\phi$ definita in \eqref{eq:iso_Znm->Zn_x_Zm}, dunque è iniettiva. Infine \[
        \abs*{\Zmod{nm}} = \eulerphi(nm) = \eulerphi(n)\eulerphi(m) = \abs*{\Zmod{n} \times \Zmod{m}}    
    \] siccome $\gcd{n, m} = 1$, dunque $\phi$ è anche surgettiva e quindi è bigettiva.

    Tramite passaggi analoghi a quelli visti nell'osservazione precedente si dimostra che $\phi^*$ è un omomorfismo, dunque essendo bigettiva è anche un isomorfismo di gruppi, da cui segue la tesi.
\end{proof}

\subsection{Prodotto interno di sottogruppi}

\begin{definition}
    Sia $(G, \cdot)$ un gruppo e siano $H,K \sgr G$. Allora si definisce il \emph{prodotto tra $H$ e $K$} come \begin{equation}
        HK \deq \set*{h\cdot k \given h \in H, k \in K}.
    \end{equation} Analogamente si definisce il \emph{prodotto tra $K$ e $H$} come \begin{equation}
        KH \deq \set*{k \cdot h \given k \in K, h \in H}.
    \end{equation}
\end{definition}

\begin{remark}
    Se il gruppo è in notazione additiva il prodotto di sottogruppi diventa somma di sottogruppi e si indica $H + K$ (o $K + H$).
\end{remark}

\begin{proposition}[Condizione per cui il prodotto tra sottogruppi è un sottogruppo] 
    {cond_prod_sgr_e'_sgr}
    Sia $(G, \cdot)$ un gruppo e siano $H,K \sgr G$.
    Allora l'insieme $HK$ è un sottogruppo di $G$ se e solo se $HK = KH$.
\end{proposition}
\begin{proof}
    Dimostriamo entrambi i versi dell'implicazione.
    \begin{description}
        \item[\boximplby] Siccome entrambi gli insiemi contengono $e_G$, per la \Cref{prop:cond_sgr} mi basta mostrare che $HK$ è chiuso rispetto all'operazione $\cdot$ e che contiene l'inverso di ogni suo elemento. 
        \begin{description}
            \item[Chiusura] Siano $h_1k_1, h_2k_2 \in HK$. Voglio mostrare che il loro prodotto $(h_1k_1) \cdot (h_2k_2)$ sia ancora una volta un elemento di $HK$. Per associatività, posso scriverlo come \[
                h_1 \cdot (k_1h_2) \cdot k_2.    
            \] Siccome $KH = HK$ esisteranno $h_3 \in H, k_3 \in K$ tali che $k_1h_2 = h_3k_3$. Da ciò segue che \[
                h_1 \cdot (k_1h_2) \cdot k_2 = h_1h_3k_3k_2 \in HK.
            \]
            \item[Invertibilità] Sia $hk \in HK$ e mostriamo che anche il suo inverso $(hk)\inv = k\inv h\inv$ è in $HK$. Siccome $k\inv h\inv \in KH$ e $KH = HK$, segue la tesi.
        \end{description}
        \item[\boximpl ] Dimostriamo che $HK = KH$ mostrando che $HK \subseteq KH$ e $KH \subseteq HK$.
        \begin{description}
            \item[($KH \subseteq HK$)] Banalmente $H \subseteq HK$ (infatti $H \ni h = he_G \in HK$) e $K \subseteq HK$. Ma allora per ogni $h, k \in HK$ segue che $k \cdot h \in HK$ (in quanto $HK \sgr G$) dunque $KH \subseteq HK$.
            \item[($HK \subseteq KH$)] Consideriamo la funzione \begin{align*}
                f : HK &\to KH\\
                x &\mapsto x\inv.
            \end{align*} Questa funzione è ben definita, in quanto se $x \in HK$, ovvero se $x = hk$ per qualche $h \in H, k \in K$ allora \[
                x\inv = (hk)\inv = k\inv h\inv \in KH    
            \] poiché $k\inv \in K$ e $h\inv \in H$. Inoltre questa funzione è ovviamente iniettiva, da cui segue che $HK \subseteq KH$.
        \end{description}
    \end{description}
    Dunque $HK$ è sottogruppo se e solo se $HK = KH$.
\end{proof}