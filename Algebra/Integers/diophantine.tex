\section{Equazioni diofantee}

\begin{definition}
    [Equazione diofantea] \label[def]{def:eq_diofantea} Si dice \emph{equazione diofantea} un'equazione della forma \begin{equation}
        ax + by = c
    \end{equation} dove $a, b, c \in \Z$, con $(x, y) \in \Z^2$.
\end{definition}

\begin{remark}
    Se $c = \mcd{a}{b}$ la soluzione della diofantea ci è data dall'algoritmo di Euclide e in particolare dall'identità di Bézout.
\end{remark}

\begin{proposition}[Condizione necessaria e sufficiente per le diofantee]
    \label{prop:cond_nec_suff_diofantee}
    L'equazione diofantea $ax + by = c$ ha soluzione se e solo se $\mcd{a}{b} \divides c$.
\end{proposition}
\begin{proof}
    Sia $d \deq \mcd{a}{b}$. Mostriamo entrambi i versi dell'implicazione.
    \begin{description}
        \item[($\implies$)] Sia $(\bar x, \bar y) \in \Z^2$ una soluzione della diofantea $ax + by = c$. Dato che $d \divides a$ e $d \divides b$ segue che esistono $h, k \in \Z$ tali che \[
            a = kd, \quad b = hd.    
        \] Ma ciò significa che \[
            c = a\bar x + b\bar y = d(k\bar x + h\bar y)    
        \] ovvero $d \divides c$.
        \item[($\impliedby$)] Supponiamo che $d \divides c$, ovvero $c = dk$ per qualche $k \in \Z$. Per l'identità di Bézout esistono $x_0, y_0 \in \Z$ tali che \[
            d = ax_0 + by_0.    
        \] Moltiplicando entrambi i membri per $k$ otteniamo che \[
            a(kx_0) + b(ky_0) = dk = c,    
        \] ovvero l'equazione $ax + by = c$ ha come soluzione la coppia $(kx_0, ky_0)$. \qedhere
    \end{description}
\end{proof}

\begin{corollary} \label{cor:mcd=1_sse_comb_lin_1}
    Siano $a, b \in \Z$ non entrambi nulli. Allora vale che $\mcd{a}{b} = 1$ se e solo se esistono $x_0, y_0 \in \Z$ tali che $ax_0 + by_0 = 1$.
\end{corollary}
\begin{proof}
    Dimostriamo entrambe le implicazioni.
    \begin{description}
        \item[($\implies$)] È l'identità di Bézout.
        \item[($\impliedby$)] $ax + by = c$ ha soluzione, dunque per la \autoref{prop:cond_nec_suff_diofantee} segue che $\mcd{a}{b} \divides 1$, ovvero $\mcd{a}{b} = 1$.
    \end{description}
\end{proof}

\begin{corollary}
    Siano $a, b \in \Z$ non entrambi nulli. Sia inoltre $d \deq \mcd{a}{b}$.

    Allora se $a_1, b_1 \in \Z$ sono tali che $a = da_1$ e $b = db_1$ segue che $\mcd{a_1}{b_1} = 1$.
\end{corollary}
\begin{proof}
    Per la \autoref{prop:cond_nec_suff_diofantee} l'equazione $ax + by = d$ ha soluzione, ovvero esistono $x_0, y_0 \in \Z$ tali che \[
        ax_0 + by_0 = d.
    \] Siccome $a = da_1$ e $b = db_1$ possiamo dividere entrambi i membri per $d$, ottenendo \[
        a_1x_0 + b_1y_0 = 1,    
    \] dunque per il \autoref{cor:mcd=1_sse_comb_lin_1} segue che $\mcd{a_1}{b_1} = 1$.
\end{proof}

Notiamo tuttavia che la soluzione di una diofantea non è in generale unica: dobbiamo quindi sfruttare le equazioni omogenee associate per trovare tutte le soluzioni.

\begin{proposition}
    [Struttura delle soluzioni di una diofantea non omogenea]
    Sia $ax + by = c$ un'equazione diofantea non omogenea e sia $ax + by = 0$ la sua omogenea associata.
    Sia inoltre $(\bar x, \bar y)$ una soluzione particolare della non omogenea. 
    
    Allora le soluzioni della non omogenea sono tutte e solo della forma \begin{equation}
        (\bar x + x_0, \bar y + y_0)
    \end{equation} al variare di $(x_0, y_0)$ tra le soluzioni dell'omogenea associata.
\end{proposition}
\begin{proof}
    Sia $(x_1, x_2) \in \Z^2$ un'altra soluzione della non omogenea. Mostriamo che la differenza $(\bar x - x_1, \bar y - y_1)$ è soluzione dell'omogenea associata.
    \begin{align*}
        a(\bar x - x_1) + b(\bar y - y_1) &= (a\bar x + b\bar y) - (ax_1 + by_1) \\
        &= c - c \\
        &= 0.
    \end{align*}

    Sia ora $(x_0, y_0)$ una soluzione generica dell'omogenea associata. Mostriamo che $(\bar x + x_0, \bar y + y_0)$ è un'altra soluzione della non omogenea.
    \begin{align*}
        a(\bar x + x_0) + b(\bar y + y_0) &= (a\bar x + b\bar y) + (ax_0 + by_0) \\
        &= c + 0 \\
        &= c. \qedhere
    \end{align*}
\end{proof}

Dunque per risolvere un'equazione non omogenea ci basta trovare una soluzione particolare e sommare ad essa le soluzioni dell'omogenea associata. Prima di spiegare come si trovino le soluzioni dell'omogenea associata enunciamo e dimostriamo un lemma che ci sarà utile anche in futuro.

\begin{lemma} \label{lem:divide_prod_coprimo_col_primo}
    Se $m \divides ab$ e $\mcd{m}{a} = 1$ segue che $m \divides b$.
\end{lemma}
\begin{proof}
    Per il \autoref{cor:mcd=1_sse_comb_lin_1} sappiamo che esistono $x_0, y_0 \in \Z$ tali che \[
        mx_0 + ay_0 = 1.    
    \] Moltiplicando entrambi i membri per $b$ otteniamo \[
        mbx_0 + aby_0 = b.    
    \] Siccome $m \divides ab$ esisterà un $k \in \Z$ tale che $ab = mk$, ovvero \[
        mbx_0 + mky_0 = m(bx_0 + ky_0) = b,     
    \] da cui segue che $m \divides b$.
\end{proof}

\begin{proposition}
    [Soluzioni di una diofantea omogenea]
    Sia $ax+by = 0$ un'equazione diofantea omogenea. Allora le sue soluzioni sono tutte e sole della forma \begin{equation}
        \left(-\frac{b}{\mcd{a}{b}}t, \frac{a}{\mcd{a}{b}}t \right)
    \end{equation} al variare di $t \in \Z$.
\end{proposition}