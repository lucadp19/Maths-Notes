\section{Presentazioni di gruppo}

Abbiamo visto studiando il gruppo diedrale $D_n$ che se vogliamo esprimere un gruppo in termini dei suoi generatori è necessario esplicitare anche quali condizioni devono essere rispettate dai generatori: se non lo facessimo, il gruppo non sarebbe necessariamente univoco. Per formalizzare il concetto di \emph{presentazione} abbiamo bisogno di alcune definizioni iniziali.

\begin{definition}
    {Gruppo libero su un insieme}{} Sia $X = \set*{x_1, x_2, \dots}$ un insieme di simboli e poniamo $X\inv \deq \set*{x_1\inv, x_2\inv, \dots}$ l'insieme dei loro inversi formali.

    Poniamo $\LL \deq X \union X\inv$; una \emph{parola} è un elemento di \[
        \bigunion_{n \geq 0} \LL^n;
    \] ovvero è sequenza finita (ma arbitrariamente lunga) di elementi di $\LL$.

    Una parola si dice \emph{ridotta} se non contiene consecutivamente i simboli $x_i$ e $x_i\inv$ (o viceversa).

    Un gruppo $G \supseteq X$ si dice \emph{libero su $X$} se $G$ è generato da $X$ e tutte le parole ridotte rappresentano elementi diversi di $G$.
\end{definition}

\begin{remark}
    Se $X = \set*{x}$ allora le parole ridotte sono delle seguenti forme:
    \begin{itemize}
        \item la parola è vuota;
        \item la parola è della forma $xxx\dots x$, che può essere rappresentata con $x^n$ (dove $n$ è la lunghezza della sequenza);
        \item la parola è della forma $x\inv x\inv x\inv \dots x\inv$, che può essere rappresentata con $x^{-n}$ (dove $n$ è la lunghezza della sequenza). 
    \end{itemize}

    Quindi $G$ è libero su $X$ se e solo se le parole sono tutte delle tre forme precedenti; dunque $G$ deve essere isomorfo a $\Z$: questo ci mostra che $\Z$ è un gruppo libero sull'insieme $X = \set*{1}$.
\end{remark}

Avevamo già osservato che se $H$ è un gruppo qualsiasi, allora esiste una bigezione tra gli elementi di $H$ e gli omomorfismi $\Z \to H$: questa bigezione è data da \begin{align*}
    \Hom{\Z, H} &\leftrightarrow H\\
    (n \mapsto h^n) &\mapsfrom h.
\end{align*}

Questa osservazione può essere estesa ai gruppi liberi con più generatori: se $G$ è libero su $X$ e $H$ è un gruppo qualunque allora esiste una bigezione tra gli omomorfismi $G \to H$ e le funzioni $X \to H$, dato da
\begin{align*}
    \Hom{G, H} &\biject \set*{f : X \to H}\\
    (x_{i_1}^{\pm 1}\cdots x_{i_k}^{\pm 1} \mapsto h_{i_1}^{\pm 1}\cdots h_{i_k}^{\pm 1}) &\mapsfrom \begin{pmatrix}
        x_1 \mapsto h_1 \\
        x_2 \mapsto h_2 \\
        \vdotswithin{\mapsto}
    \end{pmatrix}
\end{align*}
Le funzioni $X \to H$ ci dicono dove vengono mappati i generatori (ovvero gli elementi di $X$): questo determina univocamente un omomorfismo da $G$ in $H$ che mappa ogni parola in modo da rispettare la mappa $X \to H$. Nel caso il generatore sia uno solo (ovvero nel caso di $\Z$) esiste una sola funzione dal generatore in un dato elemento del gruppo $H$, dunque la bigezione è con gli elementi di $H$.

\paragraph{Costruzione della presentazione di un gruppo} Consideriamo ora un gruppo $H$ generato da $g_1, \dots, g_n$ (non libero). Per l'osservazione precedente deve esistere un omomorfismo dal gruppo libero su $n$ elementi (chiamiamolo $F(n)$) verso $H$:
\begin{equation}
        F(n) \xrightarrow{\phi} H
\end{equation} tale che $x_i \mapsto g_i$ per ogni $i = 1, \dots, n$.

Notiamo che $\phi$ è un omomorfismo surgettivo: l'immagine di $\phi$ contiene i generatori di $H$, dunque deve essere tutto $H$. Per il \nameref{cor:G/ker=Imm} vale quindi che \begin{equation}
    H = \Imm \phi \isomorph \quot{F(n)}{\ker \phi}.
\end{equation}

Una \emph{presentazione} di $H$ è quindi un'espressione del tipo \begin{equation}
    H = \gen{x_1, \dots, x_n \given w_1, \dots, w_m}
\end{equation} dove $x_1, \dots, x_n$ sono i generatori e $w_1, \dots, w_m$ sono delle parole contenenti gli $x_i$ e i loro inversi che generano $\ker \phi$.

\begin{corollary}{}{}
    Sia $H = \gen{x_1, \dots, x_n \given w_1, \dots, w_m}$ e sia $K$ un gruppo qualsiasi. Allora esiste una bigezione tra $\Hom{H, K}$ e l'insieme delle funzioni \[
        f : \set{x_1, \dots, x_n} \to K    
    \] tali che le immagini di $x_1, \dots, x_n$ rispettano le condizioni $w_1, \dots, w_m$.
\end{corollary}
\begin{proof}
    Abbiamo già mostrato che $\quot{F(n)}{\gen{w_1, \dots, w_m}} \isomorph H$; inoltre, siccome esiste sempre un omomorfismo dal gruppo libero su $n$ elementi ad un gruppo generato da $n$ elementi, dovrà esistere un omomorfismo \[
        g : F(n) \to \gen*{f(x_1), \dots, f(x_n)}.    
    \] Dall'ipotesi che $f(x_i)$ rispetta le condizioni date da $w_1, \dots, w_m$ segue che $w_1, \dots, w_m \in \ker g$, ovvero \[
        \gen{w_1, \dots, w_m} \subseteq \ker g.    
    \]
    Per il \nameref{th:first_iso} esisterà allora un unico omomorfismo $\phi$ tale che il seguente diagramma commuti:
    \begin{equation}
        \begin{tikzcd}
            F(n) \arrow[d, swap, "\pi"] \arrow[r, "g"] & \gen[\big]{f(x_1), \dots, f(x_n)} \subseteq K \\
            H \isomorph \frac{F(n)}{\gen{w_1, \dots, w_m}} \arrow[ur, swap, "\phi"] &
        \end{tikzcd}
    \end{equation}

    In particolare quindi per ogni scelta di $f$ esiste un unico omomorfismo da $H$ in $\gen[\big]{f(x_1), \dots, f(x_n)} \subseteq K$, da cui la tesi.
\end{proof}

Questo corollario ci consente di trovare gli omomorfismi tra gruppi molto semplicemente, a patto di conoscere una presentazione del gruppo di partenza. Infatti per descrivere un omomorfismo da $H$ in $K$ è sufficiente trovare una funzione $f$ dai generatori di $H$ in $K$ tale che le immagini dei generatori rispettino le condizioni date dalla presentazione di $H$.

\begin{exercise}
    Descrivere tutti gli omomorfismi di $S_3$ in sè.
\end{exercise}   
\begin{solution} 
    Una presentazione di $S_3$ è data da \[
        S_3 = \gen[\big]{\sigma, \tau \given \sigma^3 = 1, \tau^2 = 1, \tau\sigma\tau = \sigma\inv}.    
    \] Per il corollario precedente $\Hom(S_3, S_3)$ è in bigezione con le funzioni $f : \set{\sigma, \tau} \to S_3$ tali che \begin{itemize}
        \item $f(\sigma)^3 = 1$,
        \item $f(\tau)^2 = 1$,
        \item $f(\tau\sigma\tau) = f(\sigma\inv)$.
    \end{itemize}

    Per la prima condizione segue che $f(\sigma) \in \set{\id, \sigma, \sigma^2}$.
    \begin{enumerate}[label={(\roman*)}]
        \item Se $f(\sigma) = \id$ la terza relazione è banale: per ogni scelta di $f(\tau)$ che rispetta la seconda relazione si ha \[
            f(\tau\sigma\tau) = f(\tau)f(\sigma)f(\tau) = f(\tau)f(\tau) = f(\tau)^2 = \id = f(\sigma)\inv.
        \] Le scelte di $f(\tau)$ sono $4$: $\id$, $\tau$, $\tau\sigma$ e $\tau\sigma^2$.
        \item Se $f(\sigma) = \sigma$, la terza relazione è verificata per ogni scelta di $f(\tau)$ che rispetti la seconda condizione, tranne la scelta $f(\tau) = \id$. Ho quindi $3$ scelte per $f(\tau)$.
        \item Se $f(\sigma) = \sigma^2$ ho le stesse $3$ scelte per $f(\tau)$ del punto precedente.
    \end{enumerate}

    Vi sono quindi $10$ omomorfismi da $S_3$ in sé, tutti univocamente determinati dalle immagini di $\sigma$ e $\tau$. In particolare vi sono $6$ automorfismi, che corrispondono agli omomorfismi del secondo e terzo punto.
\end{solution}