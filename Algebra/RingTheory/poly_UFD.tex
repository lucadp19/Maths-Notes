\section{Polinomi come domini a fattorizzazione unica}

Lo scopo di questa sezione sarà dimostrare il seguente teorema.
\begin{theorem}
    Sia $A$ un dominio a fattorizzazione unica. Allora $A[X]$ è un dominio a fattorizzazione unica.
\end{theorem}

Da questo teorema segue per induzione anche il prossimo corollario.
\begin{corollary}
    Sia $A$ un dominio a fattorizzazione unica. Allora $A[X_1, \dots, X_n]$ è un dominio a fattorizzazione unica.
\end{corollary}

Dimostriamo innanzitutto che $A[X]$ è un dominio: dati $f, g \in A[X] \setminus \set{0}$ sappiamo che \[
    \deg fg = \deg f + \deg g \geq 0,    
\] dunque $fg$ non può essere il polinomio nullo in quanto esso non ha grado. Ricordiamo inoltre che \[
    \units{A[X]} = \units{A}.    
\]

Per mostrare che $A[X]$ è un UFD sfrutteremo il \autoref{th:caratt_UFD}: dimostreremo quindi che \begin{itemize}
    \item ogni irriducibile di $A[X]$ è primo;
    \item ogni catena discendente di divisibilità è stazionaria.
\end{itemize}

\subsection*{Ogni irriducibile di $A[X]$ è primo}

Per dimostrare che gli irriducibili di $A[X]$ sono anche primi dobbiamo espandere alcuni concetti introdotti nella prima parte.

\begin{definition}
    [Contenuto di un polinomio]
    Sia $A$ un UFD, $f \in A[X]$ tale che \[
        f(X) = \sum_{i=0}^n a_iX^i.    
    \] Si dice \strong{contenuto} di $f$ la quantità \[
        \content*{f} \deq \operatorname{mcd}\set*{a_0, \dots, a_n}.
    \]
\end{definition}

Osserviamo che, siccome $A$ è un UFD, $\content*{f}$ è univocamente definito a meno di moltiplicazione per un'unità.

\begin{definition}
    [Polinomio primitivo]
    Sia $A$ un UFD, $f \in A[X]$. $f$ si dice \strong{primitivo} se $\content*{f} \sim 1$.
\end{definition}

È facile mostrare che ogni polinomio può essere scritto come prodotto del suo contenuto e di un polinomio primitivo. 

Infatti sia $f \in A[X]$ e sia $d \deq \content*{f}$. Allora il polinomio \[
    f'(X) \deq \sum_{i=0}^n \frac{a_i}{d}X^i    
\] è un polinomio a coefficienti in $A$, in quanto $d \divides a_i$ per ogni $i$. Inoltre essendo $d$ il massimo comun divisore tra tutti gli $a_i$ segue che il contenuto di $f'$ è (associato a) $1$, ovvero $f'$ è primitivo.

\begin{theorem}
    [Lemma di Gauss]
    Sia $A$ un UFD e siano $f, g \in A[X]$. Vale che \[
        \content*{fg} = \content*{f}\content*{g}.    
    \]
\end{theorem}
\begin{proof}
    Dividiamo la dimostrazione in due casi.
    \begin{description}
        \item[Caso 1] Supponiamo $\content*{f} = \content*{g} = 1$ (ovvero entrambi primitivi) e mostriamo che $\content*{fg} = 1$.
        
        Se per assurdo $fg$ non fosse primitivo allora $\content*{fg}$ non sarebbe invertibile, da cui (per l'ipotesi che $A$ è un UFD) esisterebbe un elemento primo $p \in A$ tale che $p \divides \content*{fg}$.

        Consideriamo la proiezione canonica \begin{align*}
            \pi : A[X] \to \quot{A}{\ideal{p}}[X].
        \end{align*}
        Osserviamo che \begin{itemize}
            \item $\pi\parens[\big]{f(X)}$ non è l'elemento nullo di $\quot{A}{\ideal{p}}[X]$ in quanto $p \ndivides \content*{f} = 1$;
            \item $\pi\parens[\big]{g(X)}$ non è l'elemento nullo di $\quot{A}{\ideal{p}}[X]$ in quanto $p \ndivides \content*{g} = 1$;
            \item $\pi\parens[\big]{f(X)}$ è l'elemento nullo di $\quot{A}{\ideal{p}}[X]$.
        \end{itemize}
        Ma per la \autoref{prop:primo_sse_dom/max_sse_campo} siccome $\ideal{p}$ è un ideale primo segue che $\quot{A}{\ideal{p}}$ è un dominio, da cui anche $\quot{A}{\ideal{p}}[X]$ è un dominio, dunque abbiamo trovato un assurdo e $\content*{fg} = 1$.
        \item[Caso 2] Scriviamo \[
            f(X) = \content*{f}\cdot f'(X), \qquad g(X) = \content*{g}\cdot g'(X),
        \] dove $f', g'$ sono polinomi primitivi. Allora \[
            \content*{fg}(fg)' = fg = \content*{f}\content*{g}f'g'.    
        \] Osserviamo che i polinomi $(fg)'$ e $f'g'$ sono entrambi primitivi: il primo per costruzione, il secondo per il caso precedente.
        Uguagliamo quindi i contenuti di entrambi i membri: \begin{align*}
            &\content*{fg} \content[\big]{(fg)'} = \content*{f}\content*{g}\content*{f'g'}\\
            \iff {}&\content*{fg} \cdot 1 = \content*{f}\content*{g} \cdot 1\\
            \iff {}&\content*{fg} = \content*{f}\content*{g},
        \end{align*}
        cioè la tesi.
    \end{description}
\end{proof}

Per il resto della sezione considereremo $\K \deq Q(A)$.

\begin{corollary}
    Siano $f, g \in A[X]$, $f$ primitivo e tali che $f \divides g$ in $\K[X]$. Allora $f \divides g$ in $A[X]$.
\end{corollary}
\begin{proof}
    $f \divides g$ in $\K[X]$ significa che esiste $h \in \K[X]$ tali che $g = fh$. Sia ora $d \in A$ tale che $h_1(X) \deq d \cdot h(X)$ sia un polinomio a coefficienti in $A$ (basta prendere il massimo comune divisore dei denominatori). Allora $h_1(X)f(X) = d\cdot g(X)$ è a sua volta un polinomio a coefficienti in $A$: prendendo i contenuti si ottiene che \[
        d\content*{g} = \content*{h_1f} = \content*{h_1}\content*{f} = \content*{h_1},    
    \] ovvero $d \divides \content*{h_1}$. Ma questo significa che il polinomio $\dfrac{h_1(X)}{d} = h(X)$ è ancora a coefficienti in $A$, che è la tesi.
\end{proof}

\begin{corollary}
    \label{cor:rid_in_K[X]=>rid_in_A[X]}
    Sia $f \in A[X]$. Se $f$ è riducibile in $\K[X]$ (ovvero se esistono $g, h \in \K[X]$ di grado maggiore o uguale a $1$ tali che $f = gh$) allora esiste un $\delta \in \units{\K}$ tale che \begin{itemize}
        \item $g_1 \deq \delta\cdot g \in A[X]$,
        \item $h_1 \deq \delta\inv\cdot h \in A[X]$,
    \end{itemize}
    da cui $f = g_1h_1$ è riducibile in $A[X]$ e i fattori sono associati ai rispettivi fattori in $\K[X]$.
\end{corollary}
\begin{proof}
    Sia $d \in A$ tale che $g_1 \deq d\cdot g$ sia un polinomio a coefficienti in $A$. Sicuramente $d$ ammette inverso in $\K$, dunque \[
        f = (d \cdot g)(d\inv \cdot h) = g_1 (d\inv \cdot h) = \content*{g_1}(g_1)'(d\inv \cdot h).    
    \] Siccome $(g_1)'$ è un polinomio primitivo a coefficienti in $A$ e $(g_1)' \divides f$ in $\K[X]$, per il corollario precedente segue che $(g_1)' \divides f$ in $A[X]$, ovvero $h_1 \deq \content*{g_1} d\inv h$ è un polinomio in $A[X]$ e $\delta \deq d\inv\content*{g}$.
\end{proof}

Possiamo quindi finalmente caratterizzare gli irriducibili di $A[X]$.

\begin{theorem}[Caratterizzazione degli irriducibili di $A[X]$]
    Sia $A$ un UFD. Gli irriducibili di $A[X]$ sono tutti e soli i polinomi $f \in A[X]$ che soddisfano una delle seguenti proprietà:
    \begin{enumerate}
        \item $f$ è una costante irriducibile in $A$;
        \item $f$ ha grado maggiore o uguale di $1$, è primitivo ed irriducibile in $\K[X]$.
    \end{enumerate}
\end{theorem}
\begin{proof}
    Dimostriamo i due casi separatamente.
    \begin{description}
        \item[Caso 1.] Sia $f \in A[X]$ una costante irriducibile in $A$. 
        
        Sia $f = gh$ con $g, h \in A[X]$. Allora \[
            0 = \deg f = \deg g + \deg h,    
        \] da cui segue che $\deg g = \deg h = 0$, ovvero anche $g$ e $h$ sono costanti. Siccome gli invertibili di $A[X]$ sono gli invertibili di $A$ segue che $f$ è riducibile in $A[X]$ se e solo se $f$ è riducibile in $A$.
        \item[Caso 2.] Sia $f \in A[X]$ con $\deg f \geq 1$. Mostriamo entrambi i versi dell'implicazione.
        \begin{description}
            \item[($\implies$)] Supponiamo che $f$ sia irriducibile in $A[X]$. Scriviamo innanzitutto \[
                f(X) = \content*{f}\cdot f'(X),    
            \] da cui segue che $\content*{f}$ è un'unità di $A[X]$, ovvero è un'unità di $A$, da cui $f$ è primitivo.

            Scriviamo ora $f = gh$ in $\K[X]$. Per il \autoref{cor:rid_in_K[X]=>rid_in_A[X]}
        \end{description} 
    \end{description}
\end{proof}