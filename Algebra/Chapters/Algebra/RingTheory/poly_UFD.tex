\section{Polinomi come UFD}

Lo scopo di questa sezione sarà dimostrare il seguente teorema.
\begin{theorem}{}{}
    Sia $A$ un \UFD. Allora $A[X]$ è un \UFD.
\end{theorem}

Da questo teorema segue per induzione anche il prossimo corollario.
\begin{corollary}{}{}
    Sia $A$ un \UFD. Allora $A[X_1, \dots, X_n]$ è un \UFD.
\end{corollary}

Dimostriamo innanzitutto che $A[X]$ è un dominio: dati $f, g \in A[X] \setminus \set{0}$ sappiamo che \[
    \deg fg = \deg f + \deg g \geq 0,    
\] dunque $fg$ non può essere il polinomio nullo in quanto esso non ha grado. Ricordiamo inoltre che \[
    \units{A[X]} = \units{A}.    
\]

Per mostrare che $A[X]$ è un \UFD  sfrutteremo il \Cref{th:caratt_UFD}: dimostreremo quindi che \begin{itemize}
    \item ogni irriducibile di $A[X]$ è primo;
    \item ogni catena discendente di divisibilità è stazionaria.
\end{itemize}

\subsection*{Ogni irriducibile di $A[X]$ è primo}

Per dimostrare che gli irriducibili di $A[X]$ sono anche primi dobbiamo espandere alcuni concetti introdotti nella prima parte.

\begin{definition}
    {Contenuto di un polinomio}{}
    Sia $A$ un \UFD, $f \in A[X]$ tale che \[
        f(X) = \sum_{i=0}^n a_iX^i.    
    \] Si dice \strong{contenuto} di $f$ la quantità \[
        \content*{f} \deq \operatorname{mcd}\set*{a_0, \dots, a_n}.
    \]
\end{definition}

Osserviamo che, siccome $A$ è un \UFD, $\content*{f}$ è univocamente definito a meno di moltiplicazione per un'unità.

\begin{definition}
    {Polinomio primitivo}{}
    Sia $A$ un \UFD, $f \in A[X]$. $f$ si dice \strong{primitivo} se $\content*{f} \sim 1$.
\end{definition}

È facile mostrare che ogni polinomio può essere scritto come prodotto del suo contenuto e di un polinomio primitivo. 

Infatti sia $f \in A[X]$ e sia $d \deq \content*{f}$. Allora il polinomio \[
    f'(X) \deq \sum_{i=0}^n \frac{a_i}{d}X^i    
\] è un polinomio a coefficienti in $A$, in quanto $d \divides a_i$ per ogni $i$. Inoltre essendo $d$ il massimo comun divisore tra tutti gli $a_i$ segue che il contenuto di $f'$ è (associato a) $1$, ovvero $f'$ è primitivo.

\begin{theorem}
    {Lemma di Gauss}
    {gauss_generale}
    Sia $A$ un \UFD e siano $f, g \in A[X]$. Vale che \[
        \content*{fg} = \content*{f}\content*{g}.    
    \]
\end{theorem}
\begin{proof}
    Dividiamo la dimostrazione in due casi.
    \begin{description}
        \item[Caso 1] Supponiamo $\content*{f} = \content*{g} = 1$ (ovvero entrambi primitivi) e mostriamo che $\content*{fg} = 1$.
        
        Se per assurdo $fg$ non fosse primitivo allora $\content*{fg}$ non sarebbe invertibile, da cui (per l'ipotesi che $A$ è un \UFD) esisterebbe un elemento primo $p \in A$ tale che $p \divides \content*{fg}$.

        Consideriamo la proiezione canonica \begin{align*}
            \pi : A[X] \to \quot{A}{\ideal{p}}[X].
        \end{align*}
        Osserviamo che \begin{itemize}
            \item $\pi\parens[\big]{f(X)}$ non è l'elemento nullo di $\quot{A}{\ideal{p}}[X]$ in quanto $p \ndivides \content*{f} = 1$;
            \item $\pi\parens[\big]{g(X)}$ non è l'elemento nullo di $\quot{A}{\ideal{p}}[X]$ in quanto $p \ndivides \content*{g} = 1$;
            \item $\pi\parens[\big]{f(X)}$ è l'elemento nullo di $\quot{A}{\ideal{p}}[X]$.
        \end{itemize}
        Ma per la \Cref{prop:primo_sse_dom/max_sse_campo} siccome $\ideal{p}$ è un ideale primo segue che $\quot{A}{\ideal{p}}$ è un dominio, da cui anche $\quot{A}{\ideal{p}}[X]$ è un dominio, dunque abbiamo trovato un assurdo e $\content*{fg} = 1$.
        \item[Caso 2] Scriviamo \[
            f(X) = \content*{f}\cdot f'(X), \qquad g(X) = \content*{g}\cdot g'(X),
        \] dove $f', g'$ sono polinomi primitivi. Allora \[
            \content*{fg}(fg)' = fg = \content*{f}\content*{g}f'g'.    
        \] Osserviamo che i polinomi $(fg)'$ e $f'g'$ sono entrambi primitivi: il primo per costruzione, il secondo per il caso precedente.
        Uguagliamo quindi i contenuti di entrambi i membri: \begin{align*}
            &\content*{fg} \content[\big]{(fg)'} = \content*{f}\content*{g}\content*{f'g'}\\
            \iff {}&\content*{fg} \cdot 1 = \content*{f}\content*{g} \cdot 1\\
            \iff {}&\content*{fg} = \content*{f}\content*{g},
        \end{align*}
        cioè la tesi.
    \end{description}
\end{proof}

Per il resto della sezione considereremo $\K \deq Q(A)$.

\begin{corollary}{}{primitivo_div_in_K[X]=>div_in_A[X]}
    Siano $f, g \in A[X]$, $f$ primitivo e tali che $f \divides g$ in $\K[X]$. Allora $f \divides g$ in $A[X]$.
\end{corollary}
\begin{proof}
    $f \divides g$ in $\K[X]$ significa che esiste $h \in \K[X]$ tali che $g = fh$. Sia ora $d \in A$ tale che $h_1(X) \deq d \cdot h(X)$ sia un polinomio a coefficienti in $A$ (basta prendere il massimo comune divisore dei denominatori). Allora $h_1(X)f(X) = d\cdot g(X)$ è a sua volta un polinomio a coefficienti in $A$: prendendo i contenuti si ottiene che \[
        d\content*{g} = \content*{h_1f} = \content*{h_1}\content*{f} = \content*{h_1},    
    \] ovvero $d \divides \content*{h_1}$. Ma questo significa che il polinomio $\dfrac{h_1(X)}{d} = h(X)$ è ancora a coefficienti in $A$, che è la tesi.
\end{proof}

\begin{corollary}{}
    {rid_in_K[X]=>rid_in_A[X]}
    Sia $f \in A[X]$. Se $f$ è riducibile in $\K[X]$ (ovvero se esistono $g, h \in \K[X]$ di grado maggiore o uguale a $1$ tali che $f = gh$) allora esiste un $\delta \in \units{\K}$ tale che \begin{itemize}
        \item $g_1 \deq \delta\cdot g \in A[X]$,
        \item $h_1 \deq \delta\inv\cdot h \in A[X]$,
    \end{itemize}
    da cui $f = g_1h_1$ è riducibile in $A[X]$ e i fattori sono associati ai rispettivi fattori in $\K[X]$.
\end{corollary}
\begin{proof}
    Sia $d \in A$ tale che $g_1 \deq d\cdot g$ sia un polinomio a coefficienti in $A$. Sicuramente $d$ ammette inverso in $\K$, dunque \[
        f = (d \cdot g)(d\inv \cdot h) = g_1 (d\inv \cdot h) = \content*{g_1}(g_1)'(d\inv \cdot h).    
    \] Siccome $(g_1)'$ è un polinomio primitivo a coefficienti in $A$ e $(g_1)' \divides f$ in $\K[X]$, per il corollario precedente segue che $(g_1)' \divides f$ in $A[X]$, ovvero $h_1 \deq \content*{g_1} d\inv h$ è un polinomio in $A[X]$ e $\delta \deq d\inv\content*{g}$.
\end{proof}

Possiamo quindi finalmente caratterizzare gli irriducibili di $A[X]$.

\begin{theorem}{Caratterizzazione degli irriducibili dell'anello dei polinomi}
    {caratt_irr_A[X]}
    Sia $A$ un \UFD. Gli irriducibili di $A[X]$ sono tutti e soli i polinomi $f \in A[X]$ che soddisfano una delle seguenti proprietà:
    \begin{enumerate}
        \item $f$ è una costante irriducibile in $A$;
        \item $f$ ha grado maggiore o uguale di $1$, è primitivo ed irriducibile in $\K[X]$.
    \end{enumerate}
\end{theorem}
\begin{proof}
    Dimostriamo i due casi separatamente.
    \begin{description}
        \item[Caso 1.] Sia $f \in A[X]$ una costante irriducibile in $A$. 
        
        Sia $f = gh$ con $g, h \in A[X]$. Allora \[
            0 = \deg f = \deg g + \deg h,    
        \] da cui segue che $\deg g = \deg h = 0$, ovvero anche $g$ e $h$ sono costanti. Siccome gli invertibili di $A[X]$ sono gli invertibili di $A$ segue che $f$ è riducibile in $A[X]$ se e solo se $f$ è riducibile in $A$.
        \item[Caso 2.] Sia $f \in A[X]$ con $\deg f \geq 1$. Mostriamo entrambi i versi dell'implicazione.
        \begin{description}
            \item[\boximpl ] Supponiamo che $f$ sia irriducibile in $A[X]$. Scriviamo innanzitutto \[
                f(X) = \content*{f}\cdot f'(X),    
            \] da cui segue che $\content*{f}$ è un'unità di $A[X]$, ovvero è un'unità di $A$, da cui $f$ è primitivo.

            Scriviamo ora $f = gh$ in $\K[X]$. Per il \Cref{cor:rid_in_K[X]=>rid_in_A[X]} varrà quindi che $f = g_1h_1$ con $g_1, h_1 \in A[X]$ e $\deg g_1 = \deg g$, $\deg h_1 = \deg h$. Ma $f$ è invertibile in $A$, dunque uno tra $g_1$ e $h_1$ deve essere invertibile. Si ha quindi che \begin{align}
                &\deg g_1 = 0 \text{ oppure } \deg h_1 = 0 \\
                \iff &\deg g = 0 \text{ oppure } \deg h = 0\\
                \iff &g \in \units{\K[X]} \text{ oppure } h \in \units{\K[X]},
            \end{align}
            cioè $f$  è irriducibile in $\K[X]$.
            \item[\boximplby] Supponiamo $f$ primitivo e irriducibile in $\K[X]$.
            
            Sia $f = gh$ con $g, h \in A[X]$ (e quindi anche in $\K[X]$). Poiché $f$ è irriducibile in $\K[X]$ segue che uno tra $g$ e $h$ è invertibile in $\K[X]$, cioè è una costante. Supponiamo senza perdita di generalità che $g$ sia costante (ovvero $g(X) = g_0$) e consideriamo il contenuto di entrambi i membri: \[
                1 = \content*{f} = \content*{gh} = \content*{g}\content*{h} = g_0\cdot\content*{h}.
            \] Segue quindi che $g_0$ è invertibile in $A$, ovvero $g$ è invertibile in $A[X]$, da cui $f$ è irriducibile in $A[X]$. \qedhere
        \end{description} 
    \end{description}
\end{proof}

\begin{proposition}
    {Irriducibili e primi negli UFD}{}
    Sia $A$ un \UFD. Ogni irriducibile di $A[X]$ è anche primo.
\end{proposition}
\begin{proof}
    Sia $f \in A[X]$ irriducibile. Per definizione $f$ è un elemento primo se per ogni $g, h \in A[X]$ vale che \[
        f \divides gh \text{ (in $A[X]$) } \implies f \divides g \text{ oppure } f \divides h \text{ (in $A[X]$). }
    \] Dal \Cref{th:caratt_irr_A[X]} si ha che $f$ è irriducibile se e solo se è una costante irriducibile oppure è primitivo e irriducibile in $\K[X]$.
    
    Se $f$ è una costante irriducibile, allora $f$ è un elemento primo in $A$ poiché $A$ è un \UFD. Allora supponiamo che $f \divides gh$ per qualche $g, h \in A[X]$. Segue quindi che \[
        f = \content*{f} \divides \content*{gh} = \content*{g}\content*{h},
    \] ma per primalità di $f = \content*{f}$ in $A$ segue che \[
        f \divides \content*{g} \text{ oppure } f \divides \content*{h},
    \] ovvero \[
        f \divides g \text{ oppure } f \divides h,
    \] cioè $f$ è primo in $A[X]$.

    Supponiamo ora che $f$ sia un polinomio di grado maggiore o uguale ad $1$, primitivo e irriducibile in $\K[X]$. Siccome $\K[X]$ è un \ED segue che $f$ è primo in $\K[X]$, cioè se $f \divides gh$ in $A[X]$ allora $f$ divide uno tra $g$ ed $h$ in $\K[X]$. Ma essendo $f$ primitivo vale il \Cref{cor:primitivo_div_in_K[X]=>div_in_A[X]}, ovvero $f$ divide uno tra $g$ e $h$ in $A[X]$, cioè $f$ è primo in $A[X]$. 
\end{proof}

\subsection*{Ogni catena discendente di divisibilità è stazionaria}

\begin{theorem}{}{catena_disc_A[X]}
    Sia $\seqn*{f_n}_{n \in \N}$ una successione di elementi di $A[X]$ tali che \[
        \dots \divides f_3 \divides f_2 \divides f_1 \divides f_0.
    \] Allora esiste un $n_0$ tale che $f_i \sim f_{n_0}$ per ogni $i \geq n_0$.
\end{theorem}

Osserviamo innanzitutto che dal \nameref{th:lemma_gauss_generale} si ha che se $f \divides g$ allora $\content*{f} \divides \content*{g}$ e $f' \divides g'$. Infatti se $g = fh$ per qualche $h \in A[X]$ allora $\content*{g} = \content*{f}\content*{h}$ (cioè $\content*{f} \divides \content*{g}$) ma anche \[
    \content*{g}g' = \content*{f}\content*{h}f'h'
\] ovvero $f' \divides g'$.

\begin{proof}
    [Dimostrazione del \Cref{th:catena_disc_A[X]}]
    Associamo alla successione $\seqn*{f_n}$ le successioni $\seqn*{\content*{f_n}}$ e $\seqn*{f'_n}$. Per quanto osservato sopra si ha che queste due successioni rispettano la stessa catena discendente di divisibilità, cioè per ogni $i geq 0$ si ha \[
        \content*{f_{i+1}} \divides \content*{f_i} \text{ e } f'_{i+1} \divides f'_i.
    \] Mostriamo che queste due successioni sono stazionarie:
    \begin{itemize}
        \item quella dei contenuti lo è poiché è una catena discendente di divisibilità nell'\UFD $A$, dunque esiste un $m_0$ tale che $\content*{f_i} \sim \content*{f_{m_0}}$ per ogni $i \geq m_0$;
        \item consideriamo ora la successione dei polinomi primitivi. Associamo ad essa la successione dei gradi $\seqn*{\deg f'_n}$. Siccome per ogni $i \geq 0$ vale che $f'_{i+1} \divides f'_i$ segue che $\deg f'_{i+1} \leq \deg f'_i$. La successione $\seqn*{\deg f'_n}$ è pertanto una successione debolmente decrescente di numeri naturali, e pertanto deve stabilizzarsi.
        
        Sia quindi $d_0$ tale che $\deg f'_i = \deg f'_{d_0}$ per ogni $i \geq d_0$. Allora $f'_i$ e $f'_{d_0}$ hanno lo stesso grado e $f'_i \divides f'_{d_0}$, dunque i due polinomi differiscono per una costante. Tuttavia i due polinomi sono primitivi, dunque devono differire per un'unità, ovvero $f'_i \sim f'_{d_0}$ per ogni $i \geq d_0$, cioè la successione è stazionaria. 
    \end{itemize}

    Sia $n_0 \deq \max\set{m_0, d_0}.$ Da quanto detto sopra segue che, per ogni $i \geq n_0$, $\content*{f_i} \sim \content*{f_{n_0}}$ e $f'_i \sim f'_{n_0}$, ovvero \[
        f_i = \content*{f_i}f'_i \sim \content*{f_{n_0}}f'_{n_0} = f_{n_0},
    \] ovvero la successione di polinomi originale è stazionaria.
\end{proof}

Segue quindi che se $A$ è un \UFD allora $A[X]$ è un \UFD. 