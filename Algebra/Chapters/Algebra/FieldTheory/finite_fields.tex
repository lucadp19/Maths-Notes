\section{Campi finiti}

Finora abbiamo genericamente studiato estensioni di campi qualsiasi, pensando a $\Q$ nella maggior parte dei casi. Tuttavia, non tutti i campi sono infiniti perché esistono i campi $\F_p \deq \Zmod{p}$ al variare di $p$ primo. Mostreremo in questa sezione che questi campi non sono gli unici campi finiti e determineremo univocamente (a meno di isomorfismo) tutti i campi finiti esistenti.

\subsection{Caratteristica di un anello}

Sia $R$ un anello commutativo con identità qualsiasi e consideriamo la mappa \begin{align*}
    \eta: \Z &\to R \\
    n &\mapsto n \cdot 1_R \deq \underbrace{1_R + 1_R + \dots + 1_R}_{n \text{ volte}}.
\end{align*}

Tale mappa è ovviamente un omomorfismo di anelli, dunque possiamo considerare il suo nucleo \[
    \ker \eta = \set*{n \in \Z \given n \cdot 1_R = 0}.
\]

\begin{remark}
    $\eta$ è in realtà l'unico omomorfismo di anelli $\Z \to R$.
\end{remark}

\begin{definition}
    {Caratteristica di un anello}{}
    Sia $R$ un anello commutativo con identità e sia $\eta : \Z \to R$ l'unico omomorfismo di anelli da $\Z$ in $R$. 
    Si dice \strong{caratteristica} di $R$ la quantità \[
        \FieldChar R \deq \begin{cases}
            0 &\text{se } \ker \eta = \ideal[\big]{0}\\
            n &\text{se } \ker \eta = \ideal[\big]{n} = n\Z.
        \end{cases}
    \]
\end{definition}

Un anello qualsiasi può avere una caratteristica uguale ad un qualunque intero positivo; nel caso dei campi però vi sono delle limitazioni.

\begin{proposition}
    {Caratteristica di un campo}{}
    Sia $\K$ un campo. Allora $\FieldChar \K = 0$ oppure $\FieldChar \K = p$ con $p \in \Z$ primo.
\end{proposition}
\begin{proof}
    Supponiamo per assurdo che $\K$ abbia caratteristica $n$ con $n$ non primo e $n \neq 0$. 
    
    Se $\FieldChar \K = 1$ allora $\eta(1) = 1_{\K} = 0_{\K}$, ma ciò è assurdo in quanto $0_{\K} \neq 1_{\K}$. 
    
    Segue quindi che $n > 1$. Essendo $\Z$ un \UFD, dato che $n$ è non-primo $n$ è anche riducibile, ovvero esistono $1 < a, b < n$ tali che $n = ab$. Ma allora \begin{align*}
        &\eta(n)
        = \eta(a)\eta(b)
        = 0_{\K} \\
        \iff {}&\eta(a) = 0_{\K} \;\text{ oppure }\; \eta(b) = 0_{\K} \\
        \iff {}&
        a \in \ker \eta \;\text{ oppure }\; b \in \ker \eta.
    \end{align*} 
    Tuttavia ciò è assurdo poiché $a, b < n$ e dunque $a, b \notin \ker \eta$.
    
    Segue quindi che $\FieldChar \K$ è $0$ oppure un primo. 
\end{proof}

Nel caso $\FieldChar \K = 0$ allora il nucleo di $\Z \xrightarrow{\eta} \K$ è banale, da cui $\Z \embeds \K$. Segue quindi che il campo delle frazioni di $\Z$ è contenuto in $\K$ (poiché il campo delle frazioni di un anello è per definizione il più piccolo campo contenente tale anello), ovvero \[
    \Q \embeds \K.
\]  

Nel caso invece in cui $\FieldChar \K = p$ possiamo considerare il diagramma dato dal \Cref{th:first_iso}: \[
    \begin{tikzcd}
        \Z \arrow[dr, two heads, "\pi"] \arrow[rr, "\eta"] & & \K \\
        & \quot{\Z}{\ker \eta} = \Zmod{p} \arrow[ur, hook] 
    \end{tikzcd}
\] Segue quindi che \[
    \Zmod{p} = \F_p \embeds \K.
\]

Vale quindi il seguente risultato.
\begin{theorem}
    {Esistenza del campo base}{}
    Sia $\K$ un campo qualunque.
    \begin{enumerate}[(1)]
        \item Se $\FieldChar \K = 0$ allora $\Q \embeds \K$.
        \item Se $\FieldChar \K = p$ allora $\F_p \embeds \K$.  
    \end{enumerate}
\end{theorem}

\subsection{Campi finiti}

Vogliamo ora dimostrare che esistono campi finiti che non siano isomorfi ad $\F_p$ e studiare la loro forma.

Sia quindi $f \in \F_p[x]$ un polinomio irriducibile e consideriamo il campo \[
    \F \deq \quot{\F_p[x]}{\ideal[\big]{f}}.
\] Dato che $\F$ è ottenuto quozientando $\F_p[x]$ per un polinomio di grado $n \deq \deg f$, una sua $\F_p$-base sarà data dalle classi di resto delle potenze di $x$, ovvero da $\parens[\big]{\eqcl{1}, \eqcl{x}, \dots, \eqcl*{x^{n-1}}}$.

Ogni elemento di $\F$ può quindi essere espresso come $\F_p$-combinazione lineare di questi elementi, ovvero \[
    \F = \set*{\sum_{i=0}^{n-1} a_i\eqcl{x^i} \given a_i \in \F_p}.
\] Dato che ogni $a_i$ può essere scelto in $p$ modi (in quanto $\F_p$ ha $p$ elementi) segue che $\F$ ha $p^n$ elementi, e dunque è anch'esso un campo finito.

Dobbiamo ora risolvere tre problemi.
\begin{enumerate}[(1)]
    \item Posso costruire campi finiti con un numero di elementi che non sia una potenza di un primo?
    \item Dato $p$ primo e $n > 1$, esiste sempre un campo con $p^n$ elementi? 
    \item Esiste sempre un polinomio $f \in \F_p[x]$ irriducibile e di grado $n$? 
    \item Se esistono due polinomi diversi $f, g \in \F_p[x]$, in che relazione sono i campi ottenuti quozientando $\F_p[x]$ per l'ideale generato da $f$ e da $g$?   
\end{enumerate}

La prossima proposizione risponde alla prima domanda.
\begin{proposition}
    {}{}
    Se $\F$ è un campo finito, allora $\card{\F} = p^n$ per qualche primo $p$ e qualche $n \in \N$. 
\end{proposition}
\begin{proof}
    Osserviamo che $\FieldChar \F$ non può essere $0$, poiché in quel caso $\Q \embeds \F$ e dunque $\F$ sarebbe infinito.

    Sia quindi $p \deq \FieldChar \F$ primo: per quanto studiato in precedenza $\F_p \embeds \F$, e quindi il grado dell'estensione $\ext{\F / \F_p}$ deve essere finito in quanto $\F$ è finito.
    
    Sia dunque $n \deq \ExtDegree{\F:\F_p}$. Allora se $\parens*{v_1, \dots, v_n}$ è una $\F_p$-base di $\F$ si ha che \[
        \F = \set*{\sum_{i=1}^n a_iv_i \given a_i \in \F_p}
    \] e dunque, dato che abbiamo $p$ possibili scelte per ognuno degli $a_i$ e queste sono tutte distinte poiché $\parens[\big]{v_i}$ è una $\F_p$-base di $\F$, $\F$ ha $p^n$ elementi.
\end{proof}

In realtà vale una condizione molto più forte, che risponde alle domande (2) e (4).
\begin{theorem}
    {}{exist_unique_F_p^n}
    Per ogni $p$ primo, $n \in \N$ esiste un unico campo con $p^n$ elementi in una fissata chiusura algebrica di $\F_p$. 
\end{theorem}

Per dimostrare questo teorema abbiamo bisogno del concetto di derivata formale di un polinomio e del cosiddetto \strong{criterio della derivata}.

\begin{definition}
    {Derivata formale di un polinomio}{}
    Sia $R$ un anello e $f \in R[x]$. Se \[
        f(x) = \sum_{k=0}^n a_kx^k
    \] si dice \strong{derivata formale} di $f$ il polinomio \[
        Df(x) = f'(x) \deq \sum_{k=1}^{n} ka_kx^{k-1}.
    \]
\end{definition}

La derivata formale è esattamente la derivata nel senso analitico e pertanto rispetta tutte le proprietà solite, soltanto che nel contesto di anelli e campi non sempre è possibile definire un concetto di limite e quindi non possiamo usare la \emph{derivata analitica}.

\begin{proposition}
    {Criterio della derivata}{crit_deriv}
    Sia $\K$ un campo qualsiasi. Allora $f \in \K[x]$ ha radici multiple in $\closure{\K}$ se e solo se $\gcd{f, f'} \neq 1$, ovvero se e solo se $f$ ha radici in comune con la sua derivata.
\end{proposition}
\begin{proof}
    Sia $\alpha \in \closure{\K}$ una radice di $f$, ovvero \[
        f(x) = (x-\alpha)g(x) \qquad \text{in } \closure{\K}[x].
    \] Allora \[
        f'(x) = g(x) + (x-\alpha)g'(x),
    \] da cui, valutando i polinomi in $\alpha$, segue che \[
        f'(\alpha) = g(\alpha) + (\alpha-\alpha)g'(\alpha) = g(\alpha).
    \]

    Segue quindi che $\alpha$ è una radice in comune tra $f$ e $f'$, ovvero $f'(\alpha) = 0$, se e solo se $g(\alpha) = 0$, ovvero $g(x) = (x-\alpha)h(x)$, ovvero \[
        f(x) = (x-\alpha)^2h(x). \qedhere
    \]  
\end{proof}

\begin{corollary}
    {}{crit_deriv_irrid}
    Sia $f \in \K[x]$ irriducibile. Allora $f$ ha radici multiple in $\closure{\K}$ se e solo se $f' = 0$. 
\end{corollary}
\begin{proof}
    Per il \nameref{prop:crit_deriv} $f$ ha radici multiple se e solo se $\gcd{f, f'} \neq 1$. Siccome $f$ è irriducibile però segue che $\gcd{f, f'} \in \set*{1, f}$, dunque si hanno radici multiple se e solo se $\gcd{f, f'} = f$, ovvero se e solo se $f$ divide $f'$. Ma $f'$ ha grado minore di $f$, dunque ciò è possibile se e solo se $f' = 0$.     
\end{proof}

Possiamo quindi dimostrare il \Cref{th:exist_unique_F_p^n}.
\begin{proof}
    [Dimostrazione del \Cref{th:exist_unique_F_p^n}]
    Innanzitutto sicuramente se $\card{\F} = p^n$ allora $\F_p \embeds \F$. Inoltre $\ext{\F / \F_p}$ è sicuramente un'estensione finita (poiché $\F$ è finito) e dunque per la \Cref{prop:finite_ext=>alg_ext} si ha che $\ext{\F / \F_p}$ è algebrica. 
    
    Sia allora $\closure{\F_p}$ una chiusura algebrica di $\F_p$: vogliamo mostrare che esiste un $\F$ tale che \[
        \begin{tikzcd}[every arrow/.append style={hook}]
            \F_p \arrow[r]
            &\F \arrow[r]
            &\closure{\F}.
        \end{tikzcd}
    \] Se tale $\F$ esiste, il suo gruppo moltiplicativo $\units{\F}$ ha ordine $p^n - 1$. Questo significa che per ogni $\alpha \in \units{\F} = \F \setminus \set{0}$ si ha che \[
        \alpha^{p^n - 1} = 1,
    \] ovvero che $\alpha$ è radice del polinomio $x^{p^n - 1} - 1$.

    Segue quindi che ogni elemento di $\F$ è radice del polinomio \[
        \Psi(x) \deq x\parens*{x^{p^n - 1} - 1} = x^{p^n} - x,
    \] in quanto $0$ è radice poiché annulla il fattore $x$, mentre gli altri elementi sono invertibili e come abbiamo mostrato annullano $x^{p^n - 1} - 1$.
    
    Vale quindi che \[
        \F \subseteq \set*{\alpha \in \closure{\F_p} \given \alpha^{p^n} - \alpha = 0}.
    \] Osservo ora che il polinomio $\Psi$ ha esattamente $p^n$ radici distinte in $\closure{\F_p}$. In effetti
    \begin{itemize}
        \item ne ha al più $p^n$ poiché ha grado $p^n$;
        \item per il \nameref{prop:crit_deriv}, dato che $\F$ ha caratteristica $p$ si ha che \[
            D\Psi(x) = px^{p^n - 1} - 1 = -1
        \] e quindi $\gcd{\Psi, \Psi'} = 1$ e quindi tutte le radici di $\Psi$ sono distinte.
    \end{itemize}
    
    Segue quindi che se $\F$ è un campo, allora è l'unico possibile campo contenuto in $\closure{\F_p}$ di cardinalità $p^n$.
    
    Basta ora mostrare che $\F$ è effettivamente un campo.
    Sicuramente $0$ e $1$ sono elementi di $\F$ in quanto sono radici di $\Psi$. Mostriamo allora che se $\alpha, \beta$ sono radici di $\Psi$ segue che $\alpha \pm \beta$, $\alpha\beta$ e $\nicefrac{1}{\alpha}$ sono ancora radici di $\Psi$.
    
    Osserviamo che $\lambda$ è radice di $\Psi$ se e solo se $\lambda^{p^n} = \lambda$. Allora \begin{align*}
        &\alpha\pm\beta \in \F: &&(\alpha\pm\beta)^{p^n} = \alpha^{p^n}\pm\beta^{p^n} = \alpha\pm\beta, \\
        &\alpha\beta \in \F: &&(\alpha\beta)^{p^n} = \alpha^{p^n}\beta^{p^n} = \alpha\beta\\
        &\frac1{\alpha} \in \F: &&\parens*{\frac{1}{\alpha}}^{p^n} = \frac{1}{\alpha^{p^n}} = \frac1{\alpha}.
    \end{align*} 

    $\F$ è quindi un campo, da cui la tesi. 
\end{proof}

\begin{definition}
    {}{}
    Fissata una chiusura algebrica $\closure{\F_p}$ di $\F_p$, denotiamo con $\F_{p^n}$ l'unico sottocampo di $\closure{\F_p}$ con $p^n$ elementi.
\end{definition}

\begin{remark}
    Non abbiamo ancora dimostrato che per ogni $p$ primo, $n \in \N$ esiste un polinomio irriducibile di grado $n$ in $\F_p[x]$! 
\end{remark}