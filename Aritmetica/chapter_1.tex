\chapter{Interi e induzione}

\section{Numeri naturali}

\begin{definition}
    Si dice insieme dei numeri naturali l'insieme \[
        \N = \{0, 1, 2, 3, \dots\}.   
    \]
\end{definition}

\begin{definition}
    Si dice operazione su un insieme $X$ una funzione $X \times X \to X$.
\end{definition}

\begin{principle}[Principio del minimo intero]
    Sia $S \subseteq \N$ non vuoto. Allora $S$ ammette minimo, ovvero \[
        \forall S \subseteq \N, S \neq \varnothing \,\, \exists n \in S \tc \forall s \in S.\ s \geq m.  
    \]
\end{principle}

