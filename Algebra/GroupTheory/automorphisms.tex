\section{Automorfismi di un gruppo}

\begin{definition}
    [Automorfismo] Sia $G$ un gruppo. Si dice \emph{automorfismo di $G$} un isomorfismo da $G$ in $G$.
    Inoltre si indica con $\Aut{G}$ l'insieme di tutti gli automorfismi di $G$.
\end{definition}

\begin{proposition}
    [Gli automorfismi formano un gruppo] Sia $G$ un gruppo. Allora $(\Aut{G}, \circ)$ è un gruppo;
    in particolare $\Aut{G} \sgr \SS(G)$.
\end{proposition}
\begin{proof}
    Innanzitutto l'identità $\id : G \to G$ è un automorfismo di $G$, dunque $\id \in \Aut{G}$.

    Sia $\phi$ un automorfismo di $G$: essendo un isomorfismo, esso ammette un inverso $\phi\inv$. 
    Siccome $\phi\inv$ è ancora un isomorfismo da $G$ in $G$ segue che $\phi\inv$ è un automorfismo di $G$.

    Infine siano $\phi, \psi$ due automorfismi di $G$ : allora la composizione $\phi \circ \psi$ è ancora un automorfismo di $G$.
    Infatti la composizione è ancora un isomorfismo da $G$ in $G$, dunque è un automorfismo.

    Il fatto che $\Aut{G}$ è un sottogruppo di $\SS(G)$ segue banalmente dal fatto che 
    $\Aut{G}$ è contenuto nell'insieme delle bigezioni da $G$ in $G$ 
    insieme con il fatto che $\Aut{G}$ è un gruppo con la stessa operazione di $\SS{G}$.
\end{proof}

\begin{definition}
    Sia $G$ un gruppo. Per ogni $g \in G$ definiamo 
    \begin{align*}
        \phi_g : G &\to G\\
        g &\mapsto gxg\inv.
    \end{align*} Questa mappa viene chiamata \emph{coniugio di $x$ per $g$}.
\end{definition}

\begin{definition}
    [Insieme degli automorfismi interni]
    Sia $G$ un gruppo. Si dice \emph{insieme degli automorfismi interni} l'insieme \[
        \Inn{G} \deq \set*{\phi_g \given g \in G}.    
    \]
\end{definition}

\begin{lemma}
    [Proprietà degli automorfismi interni]
    Siano $g, h \in G$. Allora valgono le seguenti due affermazioni:
    \begin{align}
        \phi_g \circ \phi_h = \phi_{gh}.\\
        \parens*{\phi_g}\inv = \phi_{g\inv}.
    \end{align}
\end{lemma}

\begin{proposition}
    Sia $G$ un gruppo, $g \in G$. Allora il coniugio per $g$ è un automorfismo di $G$.
    Inoltre vale che \[
        \Inn{G} \normal \Aut{G}. 
    .\] 
\end{proposition}
\begin{proof}
    Mostriamo innanzitutto che $\phi_g$ è ben definita: per ogni $x \in G$ segue che $\phi_g(x) = gxg\inv \in G.$
    \begin{description}
        \item[Omomorfismo] Dati $x, y \in G$ mostriamo che $\phi_g(xy) = \phi_g(x)\phi_g(y)$.
            \begin{align*}
                \phi_g(xy) = &= g(xy)g\inv \\
                             &= gx(gg\inv)y \\
                             &= (gxg\inv)(gyg\inv)\\
                             &= \phi_g(x)\phi_g(y).
            \end{align*}
        \item[Iniettività]  Siano $x, y \in \G$: mostriamo che se $\phi_g(x) = \phi_g(y)$ allora $x = y$.
            \begin{align*}
                &\phi_g(x) = \phi_g(y)\\
                \iff &gxg\inv = gyg\inv\\
                \iff &x = y,
            \end{align*}
            dove l'ultimo passaggio è giustificato moltiplicando a sinistra per $g\inv$ e a destra per $g$.
        \item[Surgettività] Sia $y \in G$ qualunque; siccome $g\inv yg \in G$ e $\phi_g(g\inv y g) = gg\inv y gg\inv = y$, segue che $\phi_g$ è surgettiva.
    \end{description}
    Segue quindi che $\phi_g$ è un isomorfismo, dunque un automorfismo di $G$.

    Mostriamo ora che $\Inn{G} \normal \Aut{G}$.
    
    Innanzitutto l'insieme dei coniugi è un sottogruppo di $\Aut{G}$, in quanto
    \begin{itemize}
        \item $\id = \phi_e \in \Inn{G}$;
        \item Per ogni coppia di automorfismi interni $\phi_g, \phi_h \in \Inn{g}$ segue che $\phi_g \circ \phi_h \in \Inn{G}$. Infatti
    \end{itemize}
\end{proof}


