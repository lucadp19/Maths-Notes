\section{Estensioni di campi}

Iniziamo a studiare la teoria dei campi ricordando alcune considerazioni fatte nella prima parte.

Sia $\ext{\L / \K}$ un'estensione di campi, ovvero siano $\K$ e $\L$ due campi tali che $\K \subseteq \L$. Un elemento $\alpha \in \L$ si dice \strong{algebrico} su $\K$ se esiste un polinomio $f \in \K[X] \setminus \set{0}$ tale che $f(\alpha) = 0$; se $\alpha$ non è algebrico si dice \strong{trascendente}.

Sia quindi $\alpha \in \L$ un elemento algebrico su $\K$ e consideriamo ora l'omomorfismo di anelli \begin{align*}
    \phi_\alpha : \K[X] &\to \K[\alpha] \subseteq \L\\
    f(x) \mapsto f(\alpha).
\end{align*} Si ha che $\alpha$ è algebrico su $\K$ se e solo se $\ker \phi_\alpha \neq \set{0}$.

Infatti consideriamo il seguente diagramma (dato dal \nameref{th:first_iso}): \[
    \begin{tikzcd}
            \K[X] \arrow[dr, swap, "\pi"] \arrow[rr, two heads, "\phi_\alpha"] & & \K[\alpha] \\
            & \quot{\K[X]}{\ker \phi_\alpha} \arrow[ur, swap, two heads, hook, "\overline{\phi_\alpha}"] &
        \end{tikzcd}
\] Per il \nameref{th:first_iso} segue quindi che \[
    \K[\alpha] \isomorph \quot{\K[X]}{\ker \phi_\alpha}.
\] Tuttavia $\K[\alpha]$ è un sottoanello di un campo, dunque è un dominio, da cui $\ker \phi_\alpha$ deve essere un ideale primo. Essendo $\K[X]$ un \PID e siccome $\ker \phi_\alpha \neq \set{0}$ (poiché $\alpha$ è algebrico su $\K$) segue che $\ker \phi_\alpha$ è \strong{massimale}.

Segue quindi che $\K[\alpha]$ è un campo ed in particolare è isomorfo al campo $\K(\alpha)$, ovvero al campo generato da $\K$ e dall'elemento $\alpha$.

Se $\alpha$ non fosse algebrico allora avremmo che \[
    \K[\alpha] \isomorph \quot{\K[X]}{\ker \phi_\alpha} = \K[X],
\] dunque $\K[\alpha]$ non sarebbe isomorfo a $\K(\alpha)$, il quale sarebbe invece isomorfo al \strong{campo delle funzioni razionali} a coefficienti in $\K$, ovvero  \[
    \K(x) \deq \set*{\frac{f(X)}{g(X)} \given f, g \in \K[X]}.
\]  

Inoltre siccome $\K[X]$ è un \PID segue che $\ker \phi_\alpha = \ideal[\big]{\mu_\alpha}$ per qualche $\mu_\alpha \in \K[X]$. Tale polinomio è il \strong{polinomio minimo} di $\alpha$ su $\K$, ed è l'unico generatore monico di $\ker \phi_\alpha$. In effetti è unico poiché \begin{itemize}
    \item $\K[X]$ è un \ED, dunque i generatori di un suo ideale sono gli elementi di grado minimo;
    \item gli elementi di $\ideal[\big]{\mu_\alpha}$ sono i multipli di $\mu_\alpha$, ovvero sono della forma \begin{itemize}
        \item $\mu_\alpha \cdot q$ con $q \in \K[X]$ di grado maggiore o uguale a $1$, ma questo polinomio non è più di grado minimo;
        \item $k\cdot\mu_\alpha$ dove $k \in \K$, ma questo polinomio non è più monico.   
    \end{itemize} 
\end{itemize}
Infine abbiamo visto prima che $\ideal[\big]{\mu_\alpha}$ è massimale, da cui $\mu_\alpha$ è irriducibile in $\K[X]$. 

\subsection*{Estensioni finite}

Data un'estensione di campi $\ext{\L / \K}$ possiamo sempre considerare il campo $\L$ come un $\K$-spazio vettoriale. Definiamo quindi il grado dell'estensione come la dimensione di $\L$ come $\K$-spazio: in particolare diremo che un'estensione è finita se il grado è finito.

Sia ora $\alpha \in \L$: allora \[
    \ExtDegree{\K(\alpha) : \K} = \begin{cases}
        \deg \mu_\alpha, &\text{se $\alpha$ è algebrico},\\
        +\infty, &\text{altrimenti}. 
    \end{cases}
\] In effetti abbiamo visto che $\alpha$ è trascendente se e solo se $\phi_\alpha$ è iniettiva, cioè se e solo se $\K(\alpha) \isomorph \K(x)$ che ha dimensione infinita come $\K$-spazio vettoriale. Invece se $\alpha$ è algebrico segue che \[
    \K(\alpha) = \K[\alpha] \isomorph \quot{\K[X]}{\ideal[\big]{\mu_\alpha}}
\] e una $\K$-base di $\K(\alpha)$ è $\parens*{1, \alpha, \dots, \alpha^{n-1}}$.  