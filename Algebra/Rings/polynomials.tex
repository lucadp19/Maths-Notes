\section{Anello dei polinomi}

Introduciamo ora uno degli esempi più importanti di anello: l'anello dei polinomi

\begin{definition}
    [Polinomi a coefficienti in un anello]
    Sia $(A, +, \cdot)$ un anello commutativo con identità e consideriamo una successione $\seqn{a_i}$ di elementi di $A$ che sia definitivamente nulla, ovvero tale che esista un $n \in \N$ tale che \[
        a_m = 0 \quad \text{per ogni } m > n.    
    \]
    Si dice \strong{polinomio nell'indeterminata $X$} la scrittura formale \[
        p = p(X) = \sum_{i = 0}^{\infty} a_iX^i.    
    \] Gli $a_i$ si dicono \strong{coefficienti del polinomio}.

    L'insieme dei polinomi a coefficienti in $A$ si indica con $A[X]$.
\end{definition}

Dato che la successione che definisce il polinomio è definitivamente nulla, possiamo scrivere il polinomio come una sequenza finita di termini: basta prendere i termini fino al massimo indice per cui $a_i$ è diverso da $0$. Diamo però alcune definizioni preliminari.

Innanzitutto d'ora in avanti $(A, +, \cdot)$ è un anello commutativo con identità a meno di ulteriori specifiche.

\begin{definition}
    [Polinomio nullo]
    Si dice \strong{polinomio nullo in $A[X]$} il polinomio definito dalla successione costantemente nulla, e lo si indica come $p(X) = 0_{A[X]}$.
\end{definition}

\begin{definition}
    [Grado di un polinomio]
    Sia $p \in A[X]$, $p(X) \neq 0_{A[X]}$. Allora si dice grado di $p$ il numero \[
        \deg p = \max \set*{n \in \N \given a_n \neq 0}.    
    \] Il polinomio $0_{A[X]}$ non ha grado.
\end{definition}

Notiamo che i polinomi di grado $0$ sono tutti e solo della forma $p(X) = a_0$ per qualche $a_0 \in A$; ovvero sono tutte e sole le costanti dell'anello $A$. Possiamo quindi considerare l'anello $A$ come un sottoinsieme dell'insieme dei polinomi $A[X]$.

\begin{definition}
    [Uguaglianza tra polinomi]
    Siano $p, q \in A[X]$. Allora i polinomi $p$ e $q$ sono uguali se e solo se tutti i loro coefficienti sono uguali.
\end{definition}

Definiamo ora le operazioni di somma e prodotto tra polinomi.

\begin{definition}
    [Somma tra polinomi]
    Siano $p, q \in A[X]$.
    Allora definisco l'operazione di somma \begin{align*}
        + : A[X] \times A[X] &\to A[X]\\
        (p, q) &\mapsto p + q
    \end{align*} nel seguente modo: \begin{align*}
        p(X) = \sum_{i = 0}^\infty a_iX^i, \quad q(X) = \sum_{i = 0}^\infty b_iX^i\\
        \implies (p + q)(X) \deq \sum_{i = 0}^\infty (a_i + b_i)X^i.
    \end{align*}
\end{definition}

\begin{definition}
    [Prodotto tra polinomi]
    Siano $p, q \in A[X]$.
    Allora definisco l'operazione di prodotto tra polinomi \begin{align*}
        \cdot : A[X] \times A[X] &\to A[X]\\
        (p, q) &\mapsto p \cdot q
    \end{align*} nel seguente modo: \begin{align*}
        p(X) = \sum_{i = 0}^\infty a_iX^i, \quad q(X) = \sum_{j = 0}^\infty b_jX^j \\
        \implies (p \cdot q)(X) \deq \sum_{i = 0}^\infty \sum_{j = 0}^\infty a_ib_jX^{i+j}.
    \end{align*}
\end{definition}

\begin{theorem}
    [L'insieme dei polinomi è un anello]
    La struttura $(A[X], +, \cdot)$ è un anello commutativo con identità (dove l'identità è il polinomio $1_{A[X]}(X) = 1_A$).
\end{theorem}
\begin{proof}
    Basta verificare tutti gli assiomi degli anelli.
\end{proof}

\begin{proposition}
    [Grado della somma e del prodotto]
    \label{prop:deg_somma_prod_polinomi}
    Siano $p, q \in A[X] \setminus \set*{0_{A[X]}}$. Allora vale che \begin{enumerate}[label={(\roman*)}]
        \item $\deg (p + q) \leq \max \set*{\deg p, \deg q}$.
        \item se $A$ è un dominio, allora $\deg (pq) = \deg p + \deg q$.
    \end{enumerate}
\end{proposition}
\begin{proof}
    Siano i due polinomi \[
        p(X) = \sum_{i = 0}^\infty a_iX^i, \quad q(X) = \sum_{i = 0}^\infty b_iX^i.   
    \] e siano $n = \deg p$, $m = \deg q$.

    \paragraph{Grado della somma} Sia $k = \max {n, m}$. Allora per ogni $i > k$ varrà che $a_i = b_i = 0$, ovvero $a_i + b_i = 0$, da cui $\deg (p + q) \leq k$.

    \paragraph{Grado del prodotto} Il termine di grado massimo di $(pq)(X)$ deve essere quello in posizione $n + m$. 
    
    Mostriamo che per ogni $i > n$, $j > m$ vale che il coefficiente del termine di grado $i + j$ è uguale a $0$.
    Infatti per definizione di grado segue che $a_i, b_j = 0$ se $i > n$ o $j > m$, dunque il prodotto $a_i \cdot b_j$ sarà $0$, ovvero il coefficiente di grado $i + j$ sarà nullo. Da ciò segue che $\deg (pq) \leq n + m$.

    Inoltre essendo $A$ un dominio il termine $a_nb_m$ deve essere diverso da $0$, in quanto altrimenti uno tra $a_n$ e $b_m$ dovrebbe essere $0$, contro la definizione di grado.

    Dunque $\deg (pq) = \deg p + \deg q$.
\end{proof}

\begin{corollary}
    Se $A$ è un dominio, allora $A[X]$ è un dominio.
\end{corollary}
\begin{proof}
    Siano $p, q \in A[X] \setminus \set*{0_{A[X]}}$, con $\deg p = n \geq 0$, $\deg q = m \geq 0$. Allora per la \autoref{prop:deg_somma_prod_polinomi} vale che \[
        \deg (pq) = \deg p + \deg q = n + m \geq 0.    
    \] Dunque il polinomio $(pq)(X)$ non può essere il polinomio nullo (che non ha grado), da cui segue che in $A[X]$ non vi sono divisori dello zero.
\end{proof}

\begin{corollary}
    Se $A$ è un dominio, allora gli invertibili di $A[X]$ sono tutti e soli gli elementi invertibili di $A$, ovvero \[
        \units{A[X]} = \units{A}. 
    \]
\end{corollary}
\begin{proof}
    Sia $p \in \units{A[X]}$ e sia $q \in A[X]$ il suo inverso, ovvero tale che $(pq)(X) = 1_A$.

    Notiamo che $p, q \neq 0_{A[X]}$. Infatti se uno dei due fosse il polinomio nullo per la \autoref{prop:prop_anelli:per_0} il loro prodotto dovrebbe essere il polinomio nullo e non l'unità. Allora esistono $\deg p, \deg q \geq 0$ e vale che \[
        \deg (pq) = \deg p + \deg q \seteq \deg 1 = 0.    
    \]

    Dato che i gradi di $p$ e $q$ sono positivi o nulli, il grado del prodotto è $0$ se e solo se entrambi i polinomi $p$ e $q$ sono di grado zero, ovvero se e solo se sono elementi dell'anello $A$.

    Siano $\alpha, \beta \in A$ tali che $f(X) = \alpha$ e $q(X) = \beta$. Allora $(pq)(X) = \alpha \cdot \beta = 1$, ovvero $\alpha$ è invertibile, cioè $\alpha \in \units{A}$.
\end{proof}

Dopo aver caratterizzato gli elementi invertibili in $A[X]$ possiamo definire il concetto di \emph{elementi associati}.

\begin{definition}
    [Polinomi associati]
    Siano $f, g \in A[X]$. Allora $f$, $g$ si dicono \emph{associati} se esiste $\alpha \in \units{A[X]}$ (ovvero in $\units{A}$) tale che \[
        f(X) = \alpha g(X).    
    \]
\end{definition}

\begin{definition}
    [Funzione polinomiale]
    Sia $p \in A[X]$, $p(X) = \sum_{i = 0}^{\deg p} a_iX^i$. Allora possiamo associare al polinomio $p$ una funzione $A \to A$ tale che \begin{equation}
        A \ni \alpha \mapsto \sum_{i = 0}^{\deg p} a_i{\alpha}^i \in A.
    \end{equation} Tale funzione si dice \emph{funzione polinomiale associata a $p$} e si indica solitamente come il polinomio a cui è associata.
\end{definition}

\subsection{Polinomi a coefficienti in un campo}

In questa sezione studieremo l'anello $\K[X]$, dove $\K$ è un campo generico. Questo anello ha una relazione molto stretta con l'insieme $\Z$ dei numeri interi, soprattutto per quanto riguarda le proprietà di divisibilità.

\begin{theorem}
    [Esistenza e unicità della Divisione Euclidea]
    \label{th:divisione_euclidea_KX}
    Siano $f, g \in \K[X]$ con $f(X) \neq 0_{\K[X]}$. Allora esistono e sono unici due polinomi $q, r \in \K[X]$ tali che \[
        g(X) = q(X)f(X) + r(X),
    \] con $r(X) = 0_{\K[X]}$ oppure $0 \leq \deg r \leq \deg f$.
\end{theorem}
\begin{proof}[Dimostrazione dell'esistenza]
    Se $g(X) = 0_{\K[X]}$ allora posso scegliere $q(X) = 0_{\K[X]}$ e $r(X) = q(X) = 0_{\K[X]}$.
    Altrimenti procedo per induzione su $n \deq \deg g$.
    \begin{description}
        \item[Caso base] Supponiamo $\deg g = 0$, ovvero $g(X) = g_0$. Abbiamo due casi: \begin{itemize}
            \item se $\deg f = 0$, ovvero $f(X) = f_0 \in \K$, allora \[
                q(X) = g_0{f_0}\inv, \; r(X) = \bm 0;
            \]
            \item se $\deg f > \deg g$ allora \[
                q(X) = \bm 0, \; r(X) = g(X).    
            \]
        \end{itemize}
        \item[Passo induttivo] Sia $m \deq \deg f$. Come nel caso base, se $\deg f > \deg g$ basta scegliere $q$ uguale al polinomio nullo, $r(X) = g(X)$.
        
        Supponiamo invece che $\deg f \leq \deg g$. Possiamo scrivere i due polinomi come \[
            f(X) = \sum_{i = 0}^m a_iX^i, \; g(X) = \sum_{i = 0}^n b_iX^i.    
        \]
        Sia $g_1 \in \K[X]$ il seguente polinomio: \begin{align*}
            g_1[X] &\deq g(X) - \frac{b_n}{a_m}X^{n-m}f(X)\\  
            &= g(X) - b_nX^n + \dots 
        \end{align*}
        dove i puntini indicano termini di grado inferiore al termine di grado massimo (ovvero $n$).

        Il polinomio $g_1$ ha sicuramente grado inferiore al polinomio $g$, in quanto il termine di grado $n$ (ovvero $b_nX^n$) è stato eliso.

        Segue quindi per ipotesi induttiva che esistono $q_1, r_1 \in \K[X]$ tali che \[
            g_1(X) = q_1(X)f(X) + r_1(X)    
        \] con $r_1 = 0_{\K[X]}$ oppure $0 \leq r_1 \leq \deg f$.

        Dunque possiamo ricavare un'espressione per $g$ dalla definizione di $g_1$:
        \begin{align*}
            g(X) &= g_1(X) + \frac{b_n}{a_m}x^{n-m}f(X)\\
            &= q_1(X)f(X) + r_1(X) + \frac{b_n}{a_m}x^{n-m}f(X)\\
            &= (q_1(X) + \frac{b_n}{a_m}x^{n-m})f(X) + r_1(X).
        \end{align*}
        Dunque scegliendo $q(X) = q_1(X) + \frac{b_n}{a_m}x^{n-m}$ e $r(X) = r_1(X)$ otteniamo la divisione euclidea tra $f$ e $g$.
    \end{description}
\end{proof}
\begin{proof}[Dimostrazione dell'unicità]
    Siano $q_1, r_1, q_2, r_2 \in \K[X]$ tali che \begin{align*}
        g(X) = q_1(X)f(X) + r_1(X) = q_2(X)f(X) + r_2(X)
    \end{align*} con $r_1 = 0_{\K[X]}$ oppure $0 \leq \deg r_1 \leq \deg f$, $r_2 = 0_{\K[X]}$ oppure $0 \leq \deg r_2 \leq \deg f$.

    Riarrangiando i termini otteniamo \begin{equation}
        (q_1(X) - q_2(X))f(X) = r_2(X) - r_1(X).    \label{eq:unic_div_eucl_pol}
    \end{equation} Se $r_1 = r_2$ segue che $q_1 = q_2$ (per differenza), dunque supponiamo per assurdo $r_1 \neq r_2$. 
    
    Consideriamo i gradi dei polinomi contenuti nell'equazione \eqref{eq:unic_div_eucl_pol}: \begin{align*}
        \deg (r_2 - r_1) = \deg f + \deg (q_1 - q_2) \geq \deg f.
    \end{align*} Ma il grado della differenza $r_2 - r_1$ è minore o uguale al grado dei polinomi $r_1$ e $r_2$, dunque non può essere maggiore del grado di $f$. Abbiamo quindi trovato un assurdo, da cui segue che $r_1 = r_2$.
\end{proof}

\begin{theorem}
    [Teorema di Ruffini] \label{th:ruffini}
    Sia $f \in \K[X]$ un polinomio e sia $\alpha \in \K$. Allora \begin{equation}
        f(\alpha) = 0 \iff (X - \alpha) \divides f(X).
    \end{equation}
\end{theorem}
\begin{proof}
    Per il \hyperref[th:divisione_euclidea_KX]{Teorema di Divisione Euclidea} esisteranno $q, r \in \K[X]$ tali che \[
        f(X) = (X - \alpha)q(X) + r(X),   
    \] con $r = 0_{\K[X]}$ oppure $0 \leq \deg r < \deg (X - \alpha)$.Siccome $\deg (X - \alpha) = 1$ segue che $\deg r = 0$, ovvero $r(X) = r_0$ per qualche $r_0 \in \K$. Valutando $f$ in $\alpha$ otteniamo quindi \[
        f(\alpha) = (\alpha - \alpha)q(\alpha) + r(\alpha) = r_0.    
    \] Allora $f(\alpha) = 0$ se e solo se $r_0 = 0$, ovvero se e solo se $(X-\alpha) \divides f$, cioè la tesi.
\end{proof}

\begin{definition}
    [Massimo comun divisore tra polinomi]
    Siano $f, g \in \K[X]$ non entrambi nulli. Allora $d \in \K[X]$ è un \emph{massimo comun divisore} di $f$ e $g$ se \begin{enumerate}[label={(\roman*)}]
        \item $d \divides f$, $d \divides g$;
        \item se $h \divides f$, $h \divides g$ allora $h \divides d$.
    \end{enumerate}
\end{definition}

\begin{theorem}
    [Esistenza ed unicità del massimo comun divisore]
    Siano $f, g \in \K[X]$ non entrambi nulli. Allora \begin{itemize}
        \item esiste $d \in \K[X]$ tale che $d$ è un massimo comun divisore di $f$ e $g$;
        \item esistono $a, b \in \K[X]$ tali che $d(X) = a(X)f(X) + b(X)g(X)$;
        \item se $d^\prime \in \K[X]$ è un altro massimo comun divisore di $f$ e $g$, allora $d$ e $d^\prime$ sono polinomi associati, ovvero esiste un $\gamma \in \units{A}$ tale che $d(X) = \gamma d^\prime(X)$.
    \end{itemize}
\end{theorem}

Anche nell'anello dei polinomi possiamo definire il concetto di \emph{elemento primo} e \emph{elemento irriducibile}.

\begin{definition}
    [Polinomio irriducibile]
    Sia $f \in \K[X]$, $\deg f > 1$. Allora $f$ si dice \emph{irriducibile} in $\K[X]$ se \[
        f(X) = g(X)h(X) \implies g \in \units{\K[X]} \text{ oppure } h \in \units{\K}.
    \]
\end{definition}

\begin{definition}
    [Polinomio primo]
    Sia $f \in \K[X]$, $\deg f > 1$. Allora $f$ si dice \emph{primo} in $\K[X]$ se \[
        f(X) = g(X)h(X) \implies f \divides g \text{ oppure } f \divides h.
    \]
\end{definition}

Nel caso particolare in cui il polinomio sia a coefficienti in un campo vale la stessa uguaglianza tra elementi primi e elementi irriducibili che sussiste in $\Z$:

\begin{proposition}
    [Un polinomio è primo se e solo se è irriducibile]
    Sia $f \in \K[X]$, $\deg f > 1$. Allora $f$ è irriducibile se e solo se è primo.
\end{proposition}
\begin{proof}
    La dimostrazione è uguale alla dimostrazione della \autoref{prop:primo_sse_irr_in_Z}.
\end{proof}

\begin{theorem}[Teorema di fattorizzazione unica]
    \label{th:fatt_unica}
    Sia $f \in \K[X]$, $\deg f > 1$. Allora $f$ si fattorizza in modo unico come prodotto di polinomi irriducibili, a meno di fattori invertibili e dell'ordine dei fattori.
\end{theorem}

\begin{corollary}
    Sia $f \in \K[X]$, $f \neq 0_{\K[X]}$. Allora $f$ ha al massimo $\deg f$ radici in $\K$ (contate con la loro molteplicità).
\end{corollary}