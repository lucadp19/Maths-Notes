\documentclass[italian,oneside,headinclude,10pt]{scrbook}
    \usepackage[utf8]{inputenc}
    \usepackage[italian]{babel}
    \usepackage[T1]{fontenc}
    \usepackage{textcomp, microtype}
    \usepackage{amsmath, amsthm, amssymb, cases, mathtools, bm, nicefrac}
    % \usepackage{breqn}
    \usepackage{stmaryrd} % per \trianglelefteqslant
    \usepackage{array, multicol}
    \usepackage{centernot}
    \usepackage{faktor}

    \usepackage{xparse}

    \usepackage{enumitem}
    \usepackage{float}
    \usepackage{letltxmacro}

    \LetLtxMacro\amsproof\proof
    \LetLtxMacro\amsendproof\endproof

    
    % \usepackage[pdfspacing]{classicthesis}
    \usepackage[style=arsclassica, pdfspacing, eulermath]{classicthesis}
    % \usepackage{lmodern}
    
    \usepackage{thmtools}
    \usepackage[framemethod=TikZ]{mdframed}
    \usepackage{tikz-cd}
    \usepackage{hyperref} % penultimo package da caricare!
    \usepackage{cleveref} % ultimo package da caricare!


\restylefloat{table}

% \AtBeginDocument{
%     \LetLtxMacro\proof\amsproof
%     \LetLtxMacro\endproof\amsendproof
% }
\makeatletter %only needed in preamble
\renewcommand\large{\@setfontsize\large{11.5pt}{18}}
\makeatother

% \titleformat*{\chapter}{\LARGE\scshape}
% \titleformat*{\section}{\Large\scshape}

% % \renewenvironment{proof}[1][\proofname]{{\scshape #1. }}{\qed\medskip}
\renewcommand*{\chapterformat}{%
\mbox{\chapappifchapterprefix{\nobreakspace}%
\scalebox{2}{\color{gray}\thechapter\autodot}\enskip}}

\makeatletter
\renewenvironment{proof}[1][\proofname]
  {\par\pushQED{\qed}%
   \normalfont \topsep3\p@\@plus6\p@\relax
   \list{}{\leftmargin=2em
          \rightmargin=\leftmargin
          \settowidth{\itemindent}{\itshape#1}%
          \labelwidth=\itemindent
          % the following line is not needed with amsart, but might be with other classes
          \parsep=0pt \listparindent=\parindent 
  }
   \item[\hskip\labelsep
        %  \scshape
         \bfseries
         #1\@addpunct{.\hspace{1em}}]\ignorespaces}
  {\popQED\endlist\@endpefalse}
\makeatother

\declaretheoremstyle[
    spaceabove=2\topsep, spacebelow=2\topsep,
    headindent=0pt,
    headfont=\bfseries,
    notefont=\normalfont\normalsize\bfseries, notebraces={}{.},
    bodyfont=\itshape\normalsize,
    headformat={\llap{\smash{\parbox[t]{1.1in}{\centering \NAME\\ \NUMBER}}} \NOTE},
    headpunct={},
    postheadspace=10pt
]{thmstyle}
\declaretheorem[numberwithin=section, style=thmstyle]{principle}
\declaretheorem[name=Teorema, numberwithin=section, style=thmstyle]{theorem}
\declaretheorem[name=Assioma, sibling=theorem, style=thmstyle]{axiom}
\declaretheorem[name=Corollario, sibling=theorem, style=thmstyle]{corollary}
\declaretheorem[name=Proposizione, sibling=theorem, style=thmstyle]{proposition}
\declaretheorem[name=Lemma, sibling=theorem, style=thmstyle]{lemma}

\declaretheoremstyle[
    spaceabove=2\topsep, spacebelow=2\topsep,
    headindent=0pt,
    % headfont=\bfseries,
    % notefont=\bfseries, notebraces={ (}{)},
    headfont=\bfseries,
    notefont=\bfseries, notebraces={}{},
    bodyfont=\itshape\normalsize,
    headformat={\llap{\smash{\parbox[t]{1.1in}{\centering \NUMBER\\ \NAME}}} \NOTE},
    headpunct={},
    % qed={$\triangleright$},
    postheadspace=10pt
]{unnamedstyle}
\declaretheorem[name={\ignorespaces}, sibling=theorem, style=unnamedstyle]{unnamed}

\declaretheoremstyle[
    spaceabove=2\topsep, spacebelow=2\topsep,
    headindent=0pt,
    headfont=\bfseries,
    notefont=\normalfont\normalsize\bfseries, notebraces={}{.},
    bodyfont=\normalfont\normalsize,
    headformat={\llap{\smash{\parbox[t]{1.1in}{\centering \NAME\\ \NUMBER}}} \NOTE},
    headpunct={},
    % qed={$\triangleright$},
    postheadspace=10pt
]{defstyle}
\declaretheorem[name=Definizione, sibling=theorem, style=defstyle]{definition}

\declaretheoremstyle[
    headfont=\scshape,
    notefont=\normalfont, notebraces={ - }{.},
    bodyfont=\normalfont,
    postheadspace=1em
]{exmplstyle}
\declaretheorem[name=Esempio, sibling=theorem, style=exmplstyle]{example}
\declaretheorem[name=Esercizio, sibling=theorem, style=exmplstyle]{exercise}


\declaretheoremstyle[
    headfont=\scshape,
    notefont=\normalfont, notebraces={(}{)},
    bodyfont=\normalfont,
    numbered=no,
    postheadspace=1em
]{remarkstyle}
\declaretheorem[name=Osservazione, style=remarkstyle]{remark}
\declaretheorem[name=Soluzione, style=remarkstyle]{solution}
\declaretheorem[name=Intuizione, style=remarkstyle]{intuition}

\newcolumntype{z}{r<{{}}}
\newcolumntype{o}{@{}>{{}}c<{{}}@{}}

% Set related symbols
\newcommand{\set}[1]{\left\{\;#1\;\right\}}
\newcommand{\union}{\cup}
\newcommand{\inters}{\cap}
\newcommand{\bigunion}{\bigcup}
\newcommand{\biginters}{\bigcap}
\newcommand{\disjunion}{\sqcup}
\newcommand{\bigdisjunion}{\bigsqcup}
\newcommand{\suchthat}{\,:\,} % oppure con {:}
\DeclareMathOperator{\tc}{\text{ tale che }}

\renewcommand{\epsilon}{\varepsilon}
\renewcommand{\theta}{\vartheta}
\renewcommand{\rho}{\varrho}
\renewcommand{\phi}{\varphi}

\DeclarePairedDelimiter{\braces}{[}{]}
\DeclarePairedDelimiter{\abs}{\lvert}{\rvert}
\DeclarePairedDelimiter{\norm}{\lVert}{\rVert}
\DeclarePairedDelimiter{\ang}{\langle}{\rangle}
\DeclarePairedDelimiter{\cycl}{\langle}{\rangle}

\let\oldre\Re
\let\oldim\Im
\renewcommand{\Re}[1]{\operatorname{Re}(#1)}
\renewcommand{\Im}[1]{\operatorname{Im}(#1)}
\newcommand{\conj}[1]{\overline{#1}}

\newcommand{\deq}{:=}
\newcommand{\iseq}{\overset{?}{=}}
\newcommand{\seteq}{\overset{!}{=}}
\newcommand{\divides}{\mid} % divide esattamente
\newcommand{\ndivides}{\not\mid} % non divide esattamente
\newcommand{\congr}{\equiv} % congruo 
\newcommand{\ncongr}{\not\congr} % non congruo
% \newcommand{\isomorph}{\cong}
\newcommand{\isomorph}{\simeq}
\newcommand{\normal}{\trianglelefteqslant}
\newcommand{\grindex}[2]{\braces*{#1\;:\;#2}}

\newcommand{\Mod}[1]{\ \left(#1\right)}
\newcommand{\mcm}[2]{\left[#1, #2\right]}
\newcommand{\mcd}[2]{\left(#1, #2\right)}
\newcommand{\ord}[2][]{\operatorname{ord}_{#1}\!\left( #2 \right)}


% \renewcommand{\prime}{^\prime}
\newcommand{\inv}{^{-1}}
\newcommand{\Imm}[1]{\operatorname{Im} #1}
% \newcommand{\ang}[1]{\left\langle #1 \right\rangle}
\newcommand{\invertible}[1]{#1^{\times}}
\newcommand{\compl}[1]{#1^C}

\newcommand{\Mat}[2]{\operatornamewithlimits{Mat}_{#1 \times #1}\!(#2)}
\newcommand{\Hom}[2]{\operatorname{Hom}\left(#1, #2\right)}
\newcommand{\Aut}[1]{\operatorname{Aut}\left(#1\right)}
\DeclareMathOperator{\id}{id}

\NewDocumentCommand{\eqclass}{sm}{
    \IfBooleanTF{#1}{
        \left[C_{#2}\right]
    }{
        \overline{#2}
    }
}
\newcommand{\quot}[2]{{#1}/_{#2}}
% \newcommand{\quot}[2]{\faktor{#1}{#2}}
\newcommand{\N}{\mathbb{N}}
\newcommand{\Z}{\mathbb{Z}}
\newcommand{\Zmod}[1]{\Z/_{#1\Z}}
% \newcommand{\Zmod}[1]{\quot{\Z}{#1\Z}}
\newcommand{\Q}{\mathbb{Q}}
\newcommand{\R}{\mathbb{R}}
\newcommand{\C}{\mathbb{C}}
\newcommand{\K}{\mathbb{K}}


\newcommand{\HH}{\mathcal{H}}
\newcommand{\KK}{\mathcal{K}}
\renewcommand{\SS}{\mathcal{S}}
\newcommand{\PP}{\mathcal{P}}

\begin{document}

\author{Luca De Paulis}
\title{Aritmetica}
\maketitle

\tableofcontents

% \chapter{Interi e induzione}

\section{Numeri naturali}

\begin{definition}
    Si dice insieme dei numeri naturali l'insieme \[
        \N = \{0, 1, 2, 3, \dots\}.   
    \]
\end{definition}

\begin{definition}
    Si dice operazione su un insieme $X$ una funzione $X \times X \to X$.
\end{definition}

\begin{principle}[Principio del minimo intero]
    Sia $S \subseteq \N$ non vuoto. Allora $S$ ammette minimo, ovvero \[
        \forall S \subseteq \N, S \neq \varnothing \,\, \exists n \in S \tc \forall s \in S.\ s \geq m.  
    \]
\end{principle}


\chapter{I numeri interi}

\section{Relazioni}

\begin{definition}
    [Relazione su un insieme]
    Sia $X$ un insieme. Allora si dice \emph{relazione su $X$} un sottoinsieme $R \subseteq X \times X$. 
    
    Le coppia $(x, y) \in R$ \emph{soddisfano} $R$, e si scrive anche $xRy$. 
\end{definition}

\begin{definition}
    [Relazione di equivalenza]
    Sia $X$ un insieme e $\sim$ una relazione su $X$. Allora $\sim$ si dice \emph{relazione di equivalenza} se valgono i seguenti assiomi: \begin{enumerate}[label={(EQ\arabic*)}]
        \item La relazione $\sim$ è \emph{riflessiva}:
        
        per ogni $x \in X$ vale che $x \sim x$.
        \item La relazione $\sim$ è \emph{simmetrica}:
        
        per ogni $x, y \in X$, se $x \sim y$ allora necessariamente $y \sim x$.
        \item La relazione $\sim$ è \emph{transitiva}:
        
        per ogni $x, y, z \in X$, se $x \sim y$ e $y \sim z$ allora necessariamente $x \sim z$.
    \end{enumerate}
\end{definition}

Un esempio di relazione di equivalenza è la relazione di uguaglianza tra numeri: diciamo che due numeri $a, b$ sono uguali (e lo scriviamo $a = b$) se sono lo stesso numero. Questa relazione verifica molto semplicemente tutti gli assiomi delle relazioni di equivalenza, ma ci sono altre relazioni di equivalenza che non siano l'uguaglianza.

\begin{definition}
    [Classi di equivalenza]
    Sia $X$ un insieme e $\sim$ una relazione di equivalenza su $X$. Sia inoltre $a \in X$ qualsiasi. Allora si dice \emph{classe di equivalenza di $a$} l'insieme di tutti gli elementi di $X$ che sono in relazione con $a$, ovvero: \begin{equation}
        \eqcl*{a} = \set*{x \in X \given a \sim x}.
    \end{equation}
\end{definition}

La relazione di equivalenza divide quindi l'insieme in classi di equivalenza, ognuna delle quali racchiude tutti gli elementi "identificabili tra loro", nel senso che sono in relazione l'uno con l'altro.

Mostriamo ora che le classi di equivalenza formano una partizione dell'insieme.

\begin{lemma}
    [Le classi sono o disgiunte o coincidenti]
    Sia $X$ un insieme, $a, b \in X$ e $\sim$ una relazione di equivalenza su $X$.

    Allora \begin{enumerate}
        \item se $a \nsim b$ segue che $\eqcl*{a} \inters \eqcl*{b} = \varnothing$;
        \item se $a \sim b$ segue che $\eqcl*{a} = \eqcl*{b}$.
    \end{enumerate}
\end{lemma}
\begin{proof}
    Supponiamo $a \nsim b$ e supponiamo per assurdo esista $x \in \eqcl*{a} \inters \eqcl*{b}$, ovvero $x \sim a$ e $x \sim b$. Per simmetria la prima delle due relazioni può essere scritta come $a \sim x$, dunque per transitività segue che $a \sim b$. Ma questo è assurdo per ipotesi, dunque le due classi sono disgiunte.

    Ora supponiamo $a \sim b$. Supponiamo per assurdo esista qualche $y \in X$ che appartiene alla classe di $a$ ma non alla classe di $b$. Allora $y \sim a$, ma dato che $a \sim b$ per transitività segue che $y \sim b$, il che è assurdo. Dunque le due classi coincidono.
\end{proof}

\begin{theorem}
    [Le classi di equivalenza partizionano l'insieme] 
    Sia $X$ un insieme e $\sim$ una relazione di equivalenza su $X$.

    Allora l'insieme delle classi di equivalenza forma una partizione dell'insieme, ovvero classi distinte sono disgiunte e la loro unione è l'intero insieme: \[
        X = \bigunion_{a \in X} \eqcl*{a}.    
    \]
\end{theorem}

Possiamo considerare quindi l'insieme formato da tutte le classi di equivalenza indotte dalla relazione $\sim$ su $X$.

\begin{definition}
    [Insieme quoziente]
    Sia $X$ un insieme e $\sim$ una relazione di equivalenza su $X$. Allora si definisce \emph{insieme quoziente} l'insieme \begin{equation}
        X/_{\!\sim} \deq \set*{\eqcl*{a} \given a \in X}.
    \end{equation}
\end{definition}

Notiamo che anche se alcune classi coincidono, dato che l'insieme quoziente è un insieme esse compariranno una singola volta.

Diamo ora un altro tipo di relazione su insiemi.

\begin{definition}
    [Relazione di ordinamento]
    Sia $X$ un insieme e $\leq$ una relazione su $X$. Allora $\leq$ si dice \emph{relazione di ordinamento} se valgono i seguenti assiomi: \begin{enumerate}[label={(ORD\arabic*)}]
        \item La relazione $\leq$ è \emph{riflessiva}:
        
        per ogni $a \in \K$ vale che $a \leq a$.
        \item La relazione $\leq$ è \emph{antisimmetrica}:
        
        per ogni $a, b \in \K$, se $a \leq b$ e $b \leq a$ allora necessariamente $a = b$.
        \item La relazione $\leq$ è \emph{transitiva}:
        
        per ogni $a, b, c \in \K$, se $a \leq b$ e $b \leq c$ allora necessariamente $a \leq c$.
    \end{enumerate}

    In particolare l'ordinamento si dice \emph{totale} se vale anche che
    \begin{enumerate}[label={(O\arabic*)}, start=4]
        \item La relazione $\leq$ è \emph{totale}:
        
        per ogni $a, b \in \K$ vale che $a \leq b$ oppure $b \leq a$.
    \end{enumerate}
\end{definition}

Esempi tipici di relazioni di ordinamento sono l'ordinamento tra numeri $\leq$ e l'inclusione tra insiemi $\subseteq$ (che è un \emph{ordinamento parziale}).
\section{I numeri naturali}

In questa sezione studieremo il primo insieme numerico, l'insieme dei numeri naturali.

\begin{definition}
    [Numeri naturali]
    Si dice \emph{insieme dei numeri naturali} l'insieme $\N$ formato dal numero $0$ e da tutti i suoi successori, ovvero \begin{equation}
        \N = \set*{0, 1, 2, 3, \dots}.
    \end{equation}
\end{definition}

\begin{definition}
    [Operazione su un insieme] Sia $X$ un insieme. Allora si dice \emph{operazione su $X$} una funzione $X \times X \to X$.
\end{definition}

Esempi di operazioni sui numeri naturali sono la somma e il prodotto, mentre la sottrazione e la divisione non sono operazioni poiché non sono definite per qualsiasi coppia di naturali: la sottrazione $a - b$ è definita solo quando $a \geq b$, mentre la divisione è definita solo se il dividendo è un multiplo del divisore.

Per caratterizzare l'insieme dei numeri naturali, enunciamo il seguente assioma.

\begin{axiom}
    [Principio del Minimo Intero] \label{ax:min_intero} Ogni sottoinsieme non vuoto dei numeri naturali ammette minimo, ovvero se $S \subseteq \N$, $S \neq \emptyset$, allora esiste $m \in S$ tale che $a \geq m$ per ogni $a \in S$.
\end{axiom}

Dal Principio del Minimo Intero seguono altri principi; in particolare segue il Principio di Induzione in entrambe le sue varianti.

\begin{theorem}
    [Principio di Induzione (debole)] {induzione} Sia $n_0 \in \Z$, $n_0 \geq 0$ e sia $\PP$ un predicato definito per $n \geq n_0$. Se \begin{enumerate}
        \item vale $\PP(n_0)$,
        \item per ogni $n \geq n_0$ vale che $\PP(n) \implies \PP(n+1)$
    \end{enumerate}
    allora $\PP$ vale per ogni $n \geq n_0$.
\end{theorem}
\begin{proof}
    Dimostriamo che il Principio di Induzione segue dal \nameref{ax:min_intero}.

    Sia $S$ il seguente insieme: \[
        S \deq \set*{n \in \N \given n\geq n_0, \PP(n) \text{ è falsa}}.    
    \] Supponiamo per assurdo $S \neq \varnothing$. Allora per il \nameref{ax:min_intero} $S$ ammette minimo. 
    
    Sia $m \deq \min S$. Per definizione di $S$ dovrà essere $m \geq n_0$; inoltre per ipotesi $\PP(n_0)$ è vera, dunque $m > n_0$.

    Siccome $m = \min S$ allora $m - 1 \notin S$. Questo può accadere per tre motivi: \begin{itemize}
        \item $m - 1 \notin \Z$, il che è impossibile;
        \item $m - 1 < n_0$, che è impossibile in quanto $m > n_0$;
        \item vale $\PP(m-1)$.
    \end{itemize}

    Dunque $\PP(m-1)$ è vera. Per la seconda ipotesi siccome vale $\PP(m-1)$ (e $m - 1 \geq n_0$ dovrà valere $\PP(m)$, il che è assurdo in quanto $m \in S$.

    Dunque segue che $S$ è vuoto, che è la tesi.
\end{proof}

\begin{theorem}
    [Principio di Induzione (forte)] {induzione_forte} Sia $n_0 \in \N$ e sia $\PP$ un predicato definito per $n \geq n_0$. Se \begin{enumerate}
        \item vale $\PP(n_0)$,
        \item per ogni $n \geq n_0$ vale che $\PP(n_0), \PP(n_0 + 1), \dots, \PP(n) \implies \PP(n+1)$
    \end{enumerate}
    allora $\PP$ vale per ogni $n \geq n_0$.
\end{theorem}

\begin{remark}
    Il \nameref{ax:min_intero}, il \nameref{th:induzione} e il \nameref{th:induzione_forte} sono logicamente equivalenti, ovvero ognuno di essi è vero se e solo se sono veri gli altri due.
\end{remark}

\section{Numeri interi}

Costruiamo i numeri interi a partire dai naturali tramite una particolare relazione di equivalenza.

Sia $\sim$ una relazione sulle coppie di naturali (ovvero su $\N \times \N$) tale che \[
    (a, b) \sim (c, d) \iff a + d = b + c.
\]
Questa è una relazione di equivalenza, in quanto \begin{itemize}
    \item $\sim$ è riflessiva: infatti per ogni $(a, b) \in \N \times \N$ vale che $a + b = b + a$.
    \item $\sim$ è simmetrica: se vale che $(a, b) \sim (c, d)$ (ovvero $a + d = b + c$) allora varrà anche che $c + b = d + a$, ovvero $(c, d) \sim (a, b)$.
    \item $\sim$ è transitiva. Siano $(a, b), (c, d), (e, f) \in \N \times \N$ e supponiamo che $(a, b) \sim (c, d)$ e $(c, d) \sim (e, f)$. Allora \[
        a + d = b + c, \quad c + f = d + e.    
    \] Sommando le due equazioni membro a membro otteniamo \begin{align*}
        &a + c + f + d = b + d + e + c \\
        \iff &a + f = b + e  
    \end{align*} ovvero $(a, b) \sim (e, f)$.
\end{itemize}

Notiamo che se $a \geq b$ la coppia $(a, b)$ è equivalente alla coppia $(a-b, 0)$, mentre se $a < b$ la stessa coppia è equivalente a $(0, b-a)$.

L'insieme quoziente $(\N \times \N)/_{\sim}$ è l'insieme dei numeri interi: basta identificare tutte le coppie equivalenti ad $(a, 0)$ con il numero intero $+a$, mentre tutte le coppie equivalenti a $(0, a)$ vengono identificate con il numero intero $-b$.

\begin{definition}
    [Numeri interi]
    Si dice insieme dei numeri interi l'insieme \[
        \Z = \set*{\dots, -2, -1, 0, 1, 2, \dots}.    
    \]
\end{definition}

Nei numeri interi possiamo definire una funzione che prende ogni numero e lo trasforma nel numero naturale corrispondente, ovvero privato del segno:
\begin{definition}
    [Valore assoluto]
    Si dice valore assoluto la funzione $\abs*{\cdot} : \Z \to \N$ tale che \[
        \abs*{a} = \begin{cases}
            a, &\text{se } a \geq 0\\
            -a, &\text{se } a < 0.
        \end{cases}    
    \]
\end{definition}

\begin{theorem}
    [Esistenza e unicità della divisione euclidea] \label[th]{th:esist_unic_div_euclidea}
    Siano $a, b \in \Z$ con $b$ non nullo. Allora esistono e sono unici $q, r \in \Z$ tali che \[
        a = bq + r, \quad \text{con } 0 \leq r < \abs*{b}.    
    \]

    La scrittura $bq + r$ si dice \emph{divisione euclidea di $a$ per $b$}.
\end{theorem}
\begin{proof}
    Dimostriamo prima l'esistenza di $q, r$ e poi la loro unicità.
    \begin{description}
        \item[Esistenza] Supponiamo che $b > 0$, la dimostrazione è analoga nel caso $b < 0$. 
        
        Sia \[
            X = \set*{a - kb \in \Z \given a-kb \geq 0, k \in \Z};
        \] siccome $a-kb \geq 0$ per ogni $k$ varrà che $X \subseteq \N$; inoltre ponendo $k = -\abs*{a}$ otteniamo $a+ \abs*{a}b \geq 0$, dunque l'insieme $X$ non è vuoto.

        Per il \nameref{ax:min_intero} segue che esiste $r \in X$ tale che $r = \min X$. Sia inoltre $q \in \Z$ tale che $r = a - bq$ (ovvero $a = bq + r$).

        Mostriamo che $r < \abs*{b}$. Suppniamo per assurdo $r \geq \abs*{b} = b$: allora segue che \[
            0 \leq r - b = a - qb - b = a - (q+1)b.   
        \] Siccome $q+1 \in \Z$ e $a - (q+1)b \geq 0$ segue che $r^\prime = a-(q+1)b \in X$; ma ciò è impossibile in quanto $r^\prime < r$ e abbiamo supposto che $r$ fosse il minimo di $X$.

        Dunque segue che $0 \leq r < \abs*{b}$.
        \item[Unicità] Siano $q_1, q_2, r_1, r_2 \in \Z$ tali che \[
            a = q_1b + r_1 = q_2b + r_2    
        \] con $0 \leq r_1, r_2 < \abs*{b}$. Possiamo supporre senza perdita di generalità che $r_1 \leq r_2$.
        Allora vale che \[
            r_2 - r_1 = b(q_1 - q_2)  
        \] e pertanto \begin{align*}
            \abs*{b}\abs*{q_1 - q_2} = \abs*{b(q_1 - q_2)} = r_2 - r_1 \leq r_2 \leq \abs*{b}.
        \end{align*} Se fosse $q_2 - q_1 \neq 0$ allora $\abs*{b} > \abs*{b}\abs*{q_1 - q_2}$, il che è assurdo.
        Dunque segue che $q_1 = q_2$ e $r_1 = r_2$. \qedhere
    \end{description}
\end{proof}
\section{Divisibilità}

Consideriamo la relazione di divisibilità tra numeri interi:
\begin{definition}
    [Divisibilità]
    Siano $a, b \in \Z$. Allora si dice che \emph{$a$ divide $b$} (e si indica con $a \divides b$) se \[
        a = kb
    \] per qualche $k \in \Z$.
\end{definition}

\begin{proposition}
    [Divisibilità come relazione d'ordine]
    La relazione di divisibilità tra numeri interi è una relazione di ordine parziale su $\N \setminus \set{0}$.
\end{proposition}
\begin{proof}
    Per definizione di relazione d'ordine dobbiamo mostrare tre cose:
    \begin{description}
        \item[Riflessività] Sia $a \in \N$ non nullo. Allora $a \divides a$ poiché $a = 1\cdot a$.
        \item[Simmetria] Siano $a, b \in \N$ non nulli e supponiamo che $a \divides b$ e $b \divides a$. Allora per definizione di divisibilità segue che \[
            a = kb, \quad b = ha    
        \]  per qualche $k, h \in \Z$. Sostituendo la seconda equazione nella prima otteniamo $a = kha$, ovvero $kh = 1$. Ma dato che $a, b \in \N$ segue che $k = h = 1$, dunque $a = b$.
        \item[Transitività] Siano $a, b, c \in \N$ non nulli tali che $a \divides b$ e $b \divides c$. Allora per definizione vale che \[
            a = kb, \quad b = hc    
        \] per qualche $k, h \in \Z$.

        Sostituendo la seconda nella prima ottengo quindi $a = khc$, ovvero $a \divides c$ in quanto $kh \in \Z$.
    \end{description}
\end{proof}

La relazione di divisibilità può essere pensata anche come un ordinamento su $\Z \setminus {0}$, ma l'antisimmetria è "a meno del segno", ovvero \[
    a \divides b, b \divides a \implies a = b \text{ oppure } a = -b.
\] In questi casi scriveremo più semplicemente $a = \pm b$ per indicare che $a$ può essere $b$ oppure il suo opposto.

\begin{definition}
    [Massimo comun divisore]
    Siano $a, b \in \Z$ non nulli. Si dice \emph{massimo comun divisore di $a,b$} il numero $d \in \Z$ tale che \begin{enumerate}[label={(\roman*)}]
        \item $d \divides a$ e $d \divides b$;
        \item se $c \divides a$ e $c \divides b$ allora $c \divides d$.
    \end{enumerate}

    Tale $d$ si indica anche con $\operatorname{mcd}\!(a, b)$, oppure con $\gcd(a, b)$ oppure anche con $\mcd{a}{b}$.
\end{definition}

\begin{theorem}
    [Esistenza ed unicità del massimo comun divisore]
    Siano $a, b \in \Z$ non nulli. Allora esiste ed è unico (a meno del segno) $d \in \Z$ tale che $d = \mcd{a}{b}$.
\end{theorem}
\begin{proof}
    Mostriamo sia l'esistenza che l'unicità del massimo comun divisore.
    \begin{description}
        \item[Esistenza] Sia $X$ il sottoinsieme di $\Z$ tale che \[
            X \deq \set{ax + by \suchthat x, y \in \Z}    
        \] e sia $Y \deq X \inters \N \setminus \set{0}$. Notiamo che $Y \subseteq \N$ e $Y \neq \varnothing$, in quanto se $a > 0$ allora posso scegliere $x = 1 - b$, $y = a$ da cui segue che \[
            ax + by = a(1-b)+ab = a - ab + ab = a > 0 \; \implies \; a \in X,
        \] mentre se fosse $a < 0$ potrei scegliere $x = - 1 - b$ e $y = a$, da cui \[
            ax + by = a(-1-b)+ab = -a - ab + ab = -a > 0 \; \implies \; -a \in X.
        \]

        Da ciò segue che per il Principio del Minimo l'insieme $Y$ ammette minimo. Sia $d \deq \min Y$. Mostro ora che $d = \mcd{a}{b}$.

        Dato che $d \in Y$ allora dovranno esistere $x_0, y_0 \in \Z$ tali che $d = ax_0 + by_0$.
        \begin{enumerate}[label={(\roman*)}]
            \item Dimostro che $d \divides a$; per simmetria è sufficiente dimostrare questa parte.
            
            Per la divisione euclidea scrivo \begin{equation}
                a = qd + r \quad \text{per qualche } 0 \leq r < \abs{a}.
            \end{equation} Allora \begin{equation*}
                0 \leq r = a - qd = a - q(ax_0 + by_0) = a(1 - qx_0) - b(qy_0)
            \end{equation*}
        \end{enumerate}
    \end{description}
\end{proof}
\section{Equazioni diofantee}

\begin{definition}
    [Equazione diofantea] \label[def]{def:eq_diofantea} Si dice \emph{equazione diofantea} un'equazione della forma \begin{equation}
        ax + by = c
    \end{equation} dove $a, b, c \in \Z$, con $(x, y) \in \Z^2$.
\end{definition}

\begin{remark}
    Se $c = \mcd{a}{b}$ la soluzione della diofantea ci è data dall'algoritmo di Euclide e in particolare dall'identità di Bézout.
\end{remark}

\begin{proposition}[Condizione necessaria e sufficiente per le diofantee]
    \label{prop:cond_nec_suff_diofantee}
    L'equazione diofantea $ax + by = c$ ha soluzione se e solo se $\mcd{a}{b} \divides c$.
\end{proposition}
\begin{proof}
    Sia $d \deq \mcd{a}{b}$. Mostriamo entrambi i versi dell'implicazione.
    \begin{description}
        \item[($\implies$)] Sia $(\bar x, \bar y) \in \Z^2$ una soluzione della diofantea $ax + by = c$. Dato che $d \divides a$ e $d \divides b$ segue che esistono $h, k \in \Z$ tali che \[
            a = kd, \quad b = hd.    
        \] Ma ciò significa che \[
            c = a\bar x + b\bar y = d(k\bar x + h\bar y)    
        \] ovvero $d \divides c$.
        \item[($\impliedby$)] Supponiamo che $d \divides c$, ovvero $c = dk$ per qualche $k \in \Z$. Per l'identità di Bézout esistono $x_0, y_0 \in \Z$ tali che \[
            d = ax_0 + by_0.    
        \] Moltiplicando entrambi i membri per $k$ otteniamo che \[
            a(kx_0) + b(ky_0) = dk = c,    
        \] ovvero l'equazione $ax + by = c$ ha come soluzione la coppia $(kx_0, ky_0)$. \qedhere
    \end{description}
\end{proof}

\begin{corollary} \label{cor:mcd=1_sse_comb_lin_1}
    Siano $a, b \in \Z$ non entrambi nulli. Allora vale che $\mcd{a}{b} = 1$ se e solo se esistono $x_0, y_0 \in \Z$ tali che $ax_0 + by_0 = 1$.
\end{corollary}
\begin{proof}
    Dimostriamo entrambe le implicazioni.
    \begin{description}
        \item[($\implies$)] È l'identità di Bézout.
        \item[($\impliedby$)] $ax + by = c$ ha soluzione, dunque per la \autoref{prop:cond_nec_suff_diofantee} segue che $\mcd{a}{b} \divides 1$, ovvero $\mcd{a}{b} = 1$.
    \end{description}
\end{proof}

\begin{corollary}
    Siano $a, b \in \Z$ non entrambi nulli. Sia inoltre $d \deq \mcd{a}{b}$.

    Allora se $a_1, b_1 \in \Z$ sono tali che $a = da_1$ e $b = db_1$ segue che $\mcd{a_1}{b_1} = 1$.
\end{corollary}
\begin{proof}
    Per la \autoref{prop:cond_nec_suff_diofantee} l'equazione $ax + by = d$ ha soluzione, ovvero esistono $x_0, y_0 \in \Z$ tali che \[
        ax_0 + by_0 = d.
    \] Siccome $a = da_1$ e $b = db_1$ possiamo dividere entrambi i membri per $d$, ottenendo \[
        a_1x_0 + b_1y_0 = 1,    
    \] dunque per il \autoref{cor:mcd=1_sse_comb_lin_1} segue che $\mcd{a_1}{b_1} = 1$.
\end{proof}

Notiamo tuttavia che la soluzione di una diofantea non è in generale unica: dobbiamo quindi sfruttare le equazioni omogenee associate per trovare tutte le soluzioni.

\begin{proposition}
    [Struttura delle soluzioni di una diofantea non omogenea]
    Sia $ax + by = c$ un'equazione diofantea non omogenea e sia $ax + by = 0$ la sua omogenea associata.
    Sia inoltre $(\bar x, \bar y)$ una soluzione particolare della non omogenea. 
    
    Allora le soluzioni della non omogenea sono tutte e solo della forma \begin{equation}
        (\bar x + x_0, \bar y + y_0)
    \end{equation} al variare di $(x_0, y_0)$ tra le soluzioni dell'omogenea associata.
\end{proposition}
\begin{proof}
    Sia $(x_1, x_2) \in \Z^2$ un'altra soluzione della non omogenea. Mostriamo che la differenza $(\bar x - x_1, \bar y - y_1)$ è soluzione dell'omogenea associata.
    \begin{align*}
        a(\bar x - x_1) + b(\bar y - y_1) &= (a\bar x + b\bar y) - (ax_1 + by_1) \\
        &= c - c \\
        &= 0.
    \end{align*}

    Sia ora $(x_0, y_0)$ una soluzione generica dell'omogenea associata. Mostriamo che $(\bar x + x_0, \bar y + y_0)$ è un'altra soluzione della non omogenea.
    \begin{align*}
        a(\bar x + x_0) + b(\bar y + y_0) &= (a\bar x + b\bar y) + (ax_0 + by_0) \\
        &= c + 0 \\
        &= c. \qedhere
    \end{align*}
\end{proof}

Dunque per risolvere un'equazione non omogenea ci basta trovare una soluzione particolare e sommare ad essa le soluzioni dell'omogenea associata. Prima di spiegare come si trovino le soluzioni dell'omogenea associata enunciamo e dimostriamo un lemma che ci sarà utile anche in futuro.

\begin{lemma} \label{lem:divide_prod_coprimo_col_primo}
    Se $m \divides ab$ e $\mcd{m}{a} = 1$ segue che $m \divides b$.
\end{lemma}
\begin{proof}
    Per il \autoref{cor:mcd=1_sse_comb_lin_1} sappiamo che esistono $x_0, y_0 \in \Z$ tali che \[
        mx_0 + ay_0 = 1.    
    \] Moltiplicando entrambi i membri per $b$ otteniamo \[
        mbx_0 + aby_0 = b.    
    \] Siccome $m \divides ab$ esisterà un $k \in \Z$ tale che $ab = mk$, ovvero \[
        mbx_0 + mky_0 = m(bx_0 + ky_0) = b,     
    \] da cui segue che $m \divides b$.
\end{proof}

\begin{proposition}
    [Soluzioni di una diofantea omogenea]
    Sia $ax+by = 0$ un'equazione diofantea omogenea. Allora le sue soluzioni sono tutte e sole della forma \begin{equation}
        \left(-\frac{b}{\mcd{a}{b}}t, \frac{a}{\mcd{a}{b}}t \right)
    \end{equation} al variare di $t \in \Z$.
\end{proposition}
\section{Numeri primi}

\begin{definition}
    [Irreducibile]
    Sia $p \in \Z$, $p > 1$. Tale $p$ si dice irriducibile se non esistono $x, y \in \Z$ entrambi diversi da $\pm 1$ tali che $p = xy$, ovvero se \begin{equation}
        p = xy \implies x = \pm 1 \text{ oppure } y = \pm 1.
    \end{equation}
\end{definition}

\begin{definition}
    [Primo]
    Sia $p \in \Z$, $p > 1$. Tale $p$ si dice primo se per ogni $a, b \in \Z$ vale che \begin{equation}
        p \divides ab \implies p \divides a \text{ oppure } p \divides b.
    \end{equation}
\end{definition}

Nell'insieme dei numeri interi le due classi di elementi sono in realtà la stessa, come ci assicura la prossima proposizione:
\begin{proposition}
    [Un intero è primo se e solo se è irriducibile] 
    \label[prop]{primo_sse_irr_in_Z} Sia $p \in \Z$. Allora vale che \[
        p \text{ è primo} \iff p \text{ è irriducibile.}    
    \]
\end{proposition}
\begin{proof}
    Dimostriamo entrambi i versi dell'implicazione.
    \begin{description}
        \item[($\implies$)] Siano $x, y \in \Z$ tali che $p = xy$. Allora $p \divides xy$, dunque essendo $p$ primo per definizione segue che $p \divides x$ oppure $p \divides y$.
        
        Supponiamo senza perdita di generalità $p \divides x$, ovvero $x = pz$ per qualche $z \in \Z$. Allora \begin{align*}
            p &= xy \\
            &= pzy \\
            \implies zy &= 1 \\
            \implies y &= \pm 1
        \end{align*} ovvero $p$ è irriducibile.
        \item[($\impliedby$)] Supponiamo che per qualche $a, b \in \Z$ valga che $p \divides ab$. Se $p \divides a$ segue che $p$ è primo (per definizione), dunque supponiamo $p \ndivides a$ e mostriamo che $p \divides b$.
        
        Siccome $p$ è irriducibile ha come divisori soltanto $\pm 1$ e $\pm p$, dunque $\gcd{p}{a} = 1$. Inoltre per ipotesi $p \divides ab$, dunque per il \autoref{lem:divide_prod_coprimo_col_primo} segue che $p \divides b$, dunque $p$ è primo.
    \end{description}
\end{proof}

\begin{lemma}
    Siano $a, b, m \in \Z$ tali che \begin{itemize}
        \item $a \divides m$,
        \item $b \divides m$,
        \item $\gcd{a}{b} = 1$.
    \end{itemize}
    Allora $ab \divides m$.
\end{lemma}

% CAPITOLO - CONGRUENZE
\chapter{Congruenze tra interi}

\section{Definizione di congruenza}

\begin{definition}
    [Congruenza modulo $n$]
    Siano $a, b, n \in \Z$ con $n \geq 2$. Allora si dice che $a$ è \emph{congruo a $b$ modulo $n$}, e si scrive \[
        a \congr b \pmod{n},  \quad \text{oppure }  a \congr b \Mod{n}
    \] se vale che $n \divides a - b$. 
\end{definition}

\begin{proposition}
    [La congruenza modulo $n$ è un'equivalenza]
    Sia $n \in \Z$, $n \geq 2$. Allora la relazione di congruenza modulo $n$ è una relazione di equivalenza su $\Z$.
\end{proposition}
\begin{proof}
    Dimostriamo che valgono le proprietà delle relazioni di equivalenza.
    \begin{description}
        \item[Riflessività] Sia $a \in \Z$. Allora $a \congr a \Mod{n}$ in quanto $n \divides a-a = 0$.
        \item[Simmetria] Siano $a, b \in \Z$ tali che $a \congr b \Mod{n}$, ovvero $n \divides a - b$. Ma allora $n \divides b-a$, dunque $b \congr a \Mod{n}$.
        \item[Transitività] Siano $a, b, c \in \Z$ tali che \[
            a \congr b \Mod n, \quad b \congr c \Mod n.    
        \] Per definizione allora $n \divides a - b$ e $n \divides b-c$, ovvero esistono $k, h \in \Z$ tali che $a - b = nk$ e $b - c = nh$.
        Allora vale che \begin{align*}
            a - c &= a - b + b - c \\
            &= nk + nh \\
            &= n(k + h),
        \end{align*} ovvero $n \divides a - c$, da cui segue che $a \congr c \Mod n$. \qedhere
    \end{description}
\end{proof}

Le classi di equivalenza rispetto alla relazione di congruenza vengono dette \emph{classi di congruenza modulo $n$}, e si indicano con \[
    [a]_n \deq \set{b \in \Z \suchthat a \congr b \Mod n}.    
\] Quando il modulo è deducibile dal contesto possiamo usare la scrittura abbreviata $\eqclass a$.

\begin{proposition}[Caratterizzazione della relazione di congruenza] \label{prop:caratt_congr}
    Siano $a, b, n \in \Z$, $n \geq 2$. Allora sono fatti equivalenti:
    \begin{enumerate}[label={(\roman*)}]
        \item $a \congr b \Mod{n}$,
        \item esiste un $k_0 \in \Z$ tale che $a = b + nk_0$,
        \item la progressione aritmetica di ragione $n$ che passa per $a$ passa anche per il punto $b$, ovvero \[
            \left(nk + a\right)_{k \in \Z} = \left(nk + b\right)_{k \in \Z},
        \]
        \item $a$ e $b$ hanno lo stesso resto nella divisione euclidea per $n$.
    \end{enumerate}
\end{proposition}
\begin{proof}
    Dimostriamo la catena di implicazioni $(i) \implies (ii) \implies (iii) \implies (iv) \implies (i)$.
    \begin{description}
        \item[($(i) \implies (ii)$)] Siccome $n \divides a-b$ allora per qualche $k_0 \in \Z$ vale che $a-b = nk_0$, ovvero $a = b + nk_0$.
        \item[($(ii) \implies (iii)$)] Per la $(ii)$ vale che $a = b+nk_0$, dunque \begin{align*}
            \left(nk + a\right)_{k \in \Z} &= \left(nk + nk_0 + b\right)_{k \in \Z} \\
            &= \left(n(k + k_0) + b\right)_{k \in \Z} \tag{pongo $h \deq k + k_0$}\\
            &= \left(nh + b\right)_{h \in \Z}.
        \end{align*}
        \item[($(iii) \implies (iv)$)] Per la divisione euclidea esistono $q, r \in \Z$ tali che $a = qn + r$ con $0 \leq r < n$.
        
        Dunque $r \in \left(nk + a\right)_{k \in \Z} = \left(nk + b\right)_{k \in \Z}$, ovvero esiste $k_0 \in \Z$ tale che $r = b + k_0n$, ovvero $b = (-k_0)n + r$. 

        Questa espressione è la divisione euclidea di $b$ per $n$ (infatti $0 \leq r < n$), dunque essendo essa unica (per il \autoref{th:esist_unic_div_euclidea}) segue che $r$ è il resto della divisione euclidea di $b$ per $n$.
        \item[($(iv) \implies (i)$)] Per ipotesi \[
            a = q_1n + r, \quad b = q_2n + r.    
        \] Ma allora \begin{align*}
            a - b &= q_1n + r - q_2n - r\\
            &= (q_1 + q_2)n,
        \end{align*} ovvero $n \divides a - b$, cioè $a \congr b \Mod n$. \qedhere
    \end{description}
\end{proof}

Sappiamo dalla sezione sulle relazioni di equivalenza che le classi di equivalenza da essa indotte sono a due a due disgiunte. Se scegliamo un rappresentante per ogni classe e lo includiamo nell'insieme $R$ dei rappresentanti, otteniamo che \[
    \Z = \bigsqcup_{a \in \R} [a]_n. 
\]

L'insieme dei rappresentati più naturale per la relazione di congruenza modulo $n$ è l'insieme $\set{0, 1, \dots, n-1}$. Essi rappresentano tutte le classi di congruenza possibili e rappresentano tutte classi disgiunte, come ci viene garantito dal prossimo corollario.

\begin{corollary}
    I numeri $0, 1, \dots, n-1$ sono un insieme di rappresentati delle classi di congruenza modulo $n$, ovvero per ogni $m \in \Z$ esiste un unico $r \in \set{0, \dots, n-1}$ tale che $m \congr r \Mod{n}$.
\end{corollary}
\begin{proof}
    Per la \autoref{prop:caratt_congr} sappiamo che $a \congr b \Mod n$ se e solo se $a$ e $b$ hanno lo stesso resto nella divisione euclidea per $n$.

    Dunque l'insieme dei possibili resti forma sicuramente un insieme di rappresentanti (ogni numero è congruo al suo resto); inoltre due resti distinti non possono essere nella stessa classe di congruenza, altrimenti dovrebbero essere uguali.
\end{proof}

\begin{definition}
    [Insieme $\Zmod{n}$]
    Sia $n \in \Z$, $n \geq 2$. Si indica con $\Zmod n$ l'insieme di tutte le classi di congruenza modulo $n$, ovvero l'insieme quoziente ottenuto da $\Z$ attraverso la relazione di congruenza modulo $n$: \begin{equation}
        \Zmod n \deq \set{[0]_n, [1]_n, \dots, [n-1]_n} = \set{[a]_n \suchthat a \in \Z}.
    \end{equation}
\end{definition}

\begin{proposition}\label{prop:caratt_congr}
    Valgono le seguenti proprietà per le congruenze.
    \begin{enumerate}[label={(\arabic*)}]
        \item Se $a \congr b \Mod n$ e $c \congr d \mod n$ allora vale che \[
            a+c \congr b + d \Mod{n}, \quad ac \cong bd \Mod n.   
        \]
        \item Se $a \congr b \Mod n$ e $a \congr b \Mod m$ allora $a \congr b \Mod{\mcm{n}{m}}$.
        \item Se $a \congr b \Mod{n}$ allora $\mcd{a}{n} = \mcd{b}{n}$.
        \item Se $a \congr b \Mod n$ e $d \divides n$ allora $a \congr b \Mod d$.
        \item Se $ra \congr rb \Mod n$ allora $a \congr b \Mod{\nicefrac{n}{\mcd{n}{r}}}$.
        \item Se $a \congr b \Mod n$ allora $ka \congr kb \Mod n$ per ogni $k \in \Z$.
    \end{enumerate}
\end{proposition}
\begin{proof}
    Dimostriamo singolarmente le varie proprietà.
    \begin{enumerate}[label={(\arabic*)}]
        \item Per ipotesi $a - b \divides n$, $c - d \divides n$, ovvero esistono $k, h \in \Z$ tali che \[
            a - b = nk, \quad c - d = nh,    
        \] da cui segue che 
        \begin{alignat*}{1}
            (a + c) - (b + d) &= (a - b) + (c - d) \\
            &= nk + nh = n(k+h) \\
            \implies a+c &\congr b+d \Mod n.\\
            ac &= (b + nk)(d +nh)\\
             &= bd + n(kd + hb + nkh) \\
            \implies &ac \congr bd \Mod n.
        \end{alignat*}
        \item Per ipotesi $n \divides a-b$, $m\divides a - b$, dunque per definizione di minimo comune multiplo \[
            \mcm{a}{b} \divides a - b    
        \] ovvero $a \congr b \Mod{\mcm{a}{b}}$.
        \item Per la \autoref{prop:caratt_congr} l'ipotesi equivale a dire che $a, b$ hanno lo stesso resto $r$ nella divisione euclidea per $n$, ovvero \[
            a = qn + r, \quad b = q^\prime n + r    
        \] per qualche $q, q^\prime \in \Z$.

        Allora segue che \begin{align*}
            \mcd{a}{n} = \mcd{qn + r}{n} \tag{per il \autoref{lem:mcd_diff}}\\
            = \mcd{r}{n} \tag{per il \autoref{lem:mcd_diff}}\\
            = \mcd{q^\prime n + r}{n}\\ 
            = \mcd{b}{n}.
        \end{align*}
        \item Per ipotesi $n \divides a - b$, ma $d \divides n$ dunque per transitività $d \divides a - b$, ovvero $a \congr b \Mod d$.
        \item Per ipotesi $n \divides r(a - b)$, ovvero $r(a-b) = kn$ per qualche $k \in \Z$. Dividendo entrambi i membri per $\mcd{n}{r}$ (che ovviamente divide sia il membro sinistro che il membro destro) otteniamo \[
            \frac{r}{\mcd{n}{r}}(a-b) = \frac{n}{\mcd{n}{r}}k,   
        \] ovvero \[
            \frac{n}{\mcd{n}{r}} \divides \frac{r}{\mcd{n}{r}}(a-b).
        \] Inoltre per LEMMA DA INSERIRE sappiamo che \[
            \mcd{\frac{r}{\mcd{n}{r}}}{\frac{n}{\mcd{n}{r}}} = 1,    
        \] dunque per LEMMA DA INSERIRE segue che \[
            \frac{n}{\mcd{n}{r}} \divides a - b,    
        \] da cui segue la tesi.
        \item Per la \autoref{prop:caratt_congr} l'ipotesi equivale a dire $a = b + nh$ per qualche $h \in \Z$. Moltiplicando tutto per $k$ segue che $ka = kb + n(kh)$, ovvero $ka \congr kb \Mod n$.
    \end{enumerate}
\end{proof}

\begin{example}
    Ogni numero è congruo alla somma delle sue cifre modulo $3$.

    Infatti se $n = a_k \dots a_1a_0$ in notazione posizionale, ovvero \[
        n = a_k\cdot 10^k + \dots + a_1\cdot 10 + a_0,   
    \] e sapendo che $10 \congr 1 \Mod 3$, segue che \begin{align*}
        n &\congr a_k\cdot 10^k + \dots + a_1\cdot 10 + a_0 \Mod{3}\\
        &\congr a_k \cdot 1^k + \dots + a_1 \cdot 1^1 + a_0\\
        &\congr a_k + \dots + a_1 + a_0.
    \end{align*}
\end{example}



% CAPITOLO - GRUPPI
\chapter{Gruppi}

\section{Introduzione ai gruppi}

\begin{definition}
    [Gruppo] \label{def:gruppo}
    Sia $G \neq \varnothing$ un insieme e sia $*$ un'operazione su $G$, ovvero \begin{align} \begin{split}
        * : G  \times  G &\to       G  \\
            (a, b)       &\mapsto  a*b.
    \end{split} \end{align}

    Allora la struttura $(G, *)$ si dice \emph{gruppo} se valgono i seguenti assiomi: \begin{enumerate}[label={(G\arabic*)}]
        \item \label{def:gruppo:ass} L'operazione $*$ è \emph{associativa}:
        
        per ogni $a, b, c \in G$ vale che $a * (b * c) = (a * b) * c$.
        \item Esiste un elemento $e_G \in G$ che fa da \emph{elemento neutro} rispetto all'operazione $*$:
        
        per ogni $a \in G$ vale che $a * e_G = e_G * a = a$.
        \item Ogni elemento di $G$ è \emph{invertibile} rispetto all'operazione $*$:
        
        per ogni $a \in G$ esiste $a\inv \in G$ tale che $a * a\inv = a\inv * a = e_G$.
        
        Tale $a\inv$ si dice \emph{inverso di $a$}.
    \end{enumerate}
\end{definition}

\begin{definition}
    [Gruppo abeliano] \label{def:gruppo abeliano}
    Sia  $(G, *)$ un gruppo.
    Allora $(G, *)$ si dice \emph{gruppo abeliano} se vale inoltre \begin{enumerate}[label={(G\arabic*)}, start=4]
        \item l'operazione $*$ è commutativa, ovvero \begin{align*}
            \forall a, b \in G \quad a * b = b * a.
        \end{align*}
    \end{enumerate}
\end{definition}

L'elemento neutro di $G$ si può rappresentare come $e_G$, $\operatorname{id}_G$, $1_G$ o semplicemente $e$ nel caso sia evidente il gruppo a cui appartiene.

Possiamo rappresentare un gruppo in \emph{notazione moltiplicativa}, come abbiamo fatto finora, oppure in \emph{notazione additiva}, spesso usata quando si studiano gruppi abeliani. 

In notazione additiva, ovvero considerando un gruppo $(G, +)$ gli assiomi diventano \begin{enumerate}[label={(G\arabic*)}]
    \item l'operazione $+$ è associativa, ovvero \begin{align*}
        \forall a, b, c \in G. \quad a + (b + c) = (a + b) + c
    \end{align*}
    \item esiste un elemento $e_G \in G$ che fa da elemento neutro rispetto all'operazione $+$: \begin{align*}
        \forall a \in G. \quad a + e_G = e_G + a = a 
    \end{align*}
    \item ogni elemento di $G$ è invertibile rispetto all'operazione $+$: \begin{align*}
        \forall a \in G \;\;
        \exists (-a) \in G.
        \quad a + (-a) = (-a) + a = e_G.
    \end{align*}
    Per semplicità spesso si scrive $a - b$ per intendere $a + (-b)$.
    \item l'operazione $+$ è commutativa, ovvero \begin{align*}
        \forall a, b \in G \quad a + b = b + a.
    \end{align*}
\end{enumerate}

Facciamo alcuni esempi di gruppi.
\begin{example}
    Sono gruppi abeliani $(\Z, +)$ e le sue estensioni $(\Q, +)$, $(\R, +)$, $(\C, +)$, come è ovvio verificare.
\end{example}
\begin{example}
    $(\Zmod{n}, +)$ è un gruppo, definendo l'operazione di somma rispetto alle classi di resto.
\end{example}
\begin{example}
    è un gruppo la struttura $(\mu_n, \cdot)$ dove \[
        \mu_n \deq \set{x \in \C \suchthat x^n = 1}.    
    \]
\end{example}
\begin{proof}
    Infatti \begin{enumerate}[label={(G\arabic*)}, start=0]
        % \setcounter{enumi}{-1}
        \item $\cdot$ è un'operazione su $\mu_n$. Infatti se $x, y \in \mu_n$, ovvero \[
            x^n = y^n = 1    
        \] allora segue anche che \[
            (xy)^n = x^ny^n = 1    
        \] da cui $xy \in \mu_n$;
        \item $\cdot$ è associativa in $\C$, dunque lo è in $\mu_n \subseteq \C$;
        \item $1 \in \C$ è l'elemento neutro di $\cdot$ e $1 \in \mu_n$ in quanto $1^n = 1$;
        \item ogni elemento di $\mu_n$ ammette inverso. Infatti sia $x \in \mu_n$, dunque $x \neq 0$ (altrimenti $x^n = 0 \neq 1$) e sia $x\inv \in \C$ il suo inverso. Allora \[
            (x\inv)^n = (x^n)\inv = 1\inv = 1    
        \] ovvero $x\inv \in \mu_n$;
        \item inoltre $\cdot$ è commutativa in $\C$, dunque lo è anche in $\mu_n$.
    \end{enumerate}
    Da ciò segue che $\mu_n$ è un gruppo abeliano.
\end{proof}
\begin{example}
    $(\invert{\Z}, \cdot)$ dove \[
        \invert{\Z} \deq \set{n \in \Z \suchthat n \text{ è invertibile rispetto a } \cdot} = \set{\pm 1}
    \] è un gruppo abeliano;
\end{example}
\begin{example}
    $(\invert{\Zmod{n}}, \cdot)$ dove \[
        \invert{\Zmod{n}} \deq \set{\eqcl n \in \Zmod{n} \suchthat \eqcl n \text{ è invertibile rispetto a } \cdot}
    \] è un gruppo abeliano.
\end{example}
\begin{proof}
    Infatti \begin{enumerate}[label={(G\arabic*)}, start=0]
    % \setcounter{enumi}{-1}
        \item $\cdot$ è un'operazione su $\Zmod{n}$. Infatti se $\eqcl x, \eqcl y \in \Zmod{n}$ allora segue anche che $\eqcl{xy}$ è invertibile in $\Zmod{n}$ e il suo inverso è $\eqcl{x\inv} \cdot \eqcl{y\inv}$, da cui $\eqcl{xy} \in \Zmod n$;
        \item $\cdot$ è associativa in $\Zmod{n}$, dunque lo è in $\invert{\Zmod{n}} \subseteq \Zmod{n}$;
        \item $1 \in \Zmod{n}$ è l'elemento neutro di $\cdot$ e $1 \in \invert{\Zmod{n}}$ in quanto $1$ è invertibile e il suo inverso è $1$;
        \item ogni elemento di $\invert{\Zmod{n}}$ ammette inverso per definizione;
        \item inoltre $\cdot$ è commutativa in $\Zmod{n}$, dunque lo è in $\invert{\Zmod{n}} \subseteq \Zmod{n}$.
    \end{enumerate}
    Da ciò segue che $\Zmod{n}$ è un gruppo abeliano.
\end{proof}
\begin{example}
    Se $X$ è un insieme e $\SS(X)$ è l'insieme \[
        \SS(X) \deq \set{f : X \to X \suchthat f \text{ è bigettiva}}    
    \] allora $(\SS(X), \circ)$ è un gruppo (dove $\circ$ è l'operazione di composizione tra funzioni).
\end{example}
\begin{proof}
    Infatti \begin{enumerate}[label={(G\arabic*)}, start=0]
        % \setcounter{enumi}{-1}
        \item se $f, g \in \SS(X)$ allora $f \circ g : X \to X$ è bigettiva, dunque $f \circ g \in \SS(X)$;
        \item l'operazione di composizione di funzioni è associativa;
        \item la funzione \begin{align*}
            \operatorname{id} : X &\to X\\
            x &\mapsto x
        \end{align*} è bigettiva ed è l'elemento neutro rispetto alla composizione;
        \item Se $f \in \SS(X)$ allora $f$ è invertibile ed esisterà $f\inv : X \to X$ tale che $f \circ f\inv = \operatorname{id}$. Ma allora $f\inv$ è invertibile e la sua inversa è $f$, dunque $f\inv$ è bigettiva e quindi $f\inv \in \SS(X)$.
    \end{enumerate}
    Dunque $\SS(X)$ è un gruppo (non necessariamente abeliano).
\end{proof}

Esempi di strutture che non rispettano le proprietà di un gruppo sono invece:
\begin{itemize}
    \item $(\N, +)$ poichè nessun numero ha inverso ($-n \notin \N$);
    \item $(\Z, \cdot)$, $(\Q, \cdot)$, $(\R, \cdot)$ e $(\C, \cdot)$ non sono gruppi in quanto $0$ non ha inverso moltiplicativo;
    \item l'insieme \[
        \set{x \in \C \suchthat x^n = 2}    
    \] in quanto il prodotto due elementi di questo insieme non appartiene più all'insieme.
\end{itemize}

Definiamo ora alcune proprietà comuni a tutti i gruppi.

\begin{proposition}
    [Proprietà algebriche dei gruppi] \label{prop:prop_grp}
    Sia $(G, \cdot)$ un gruppo. Allora valgono le seguenti affermazioni:
    \begin{enumerate}[label={(\roman*)}, ref={\theproposition: (\roman*)}]
        \item \label{prop:prop_grp:e_unico} l'elemento neutro di $G$ è unico;\
        \item \label{prop:prop_grp:inv_unico} $\forall g \in G$ l'inverso di $g$ è unico;
        \item \label{prop:prop_grp:inv_inv} $\forall g \in G \;\; (g\inv)\inv = g$;
        \item \label{prop:prop_grp:inv_prod} $\forall h, g \in G \;\; (hg\inv)\inv = g\inv h\inv$; 
        \item \label{prop:prop_grp:canc} Valgono le \emph{leggi di cancellazione}: $\forall a, b, c \in G$ vale che \begin{align}
            ab = ac \iff b = c &\tag{sx} \label{prop:prop_grp:canc:sx}\\
            ba = ca \iff b = c &\tag{dx} \label{prop:prop_grp:canc:dx}
        \end{align}
    \end{enumerate}
\end{proposition}
\begin{proof}
    \begin{enumerate}[label={(\roman*)}]
        \item Siano $e_1, e_2 \in G$ entrambi elementi neutri. Allora \[
            e_1 = e_1 \cdot e_2 = e_2    
        \] dove il primo uguale viene dal fatto che $e_2$ è elemento neutro, mentre il secondo viene dal fatto che $e_1$ lo è.
        \item Siano $x, y \in G$ entrambi inversi di qualche $g \in G$. Allora per definizione di inverso \[
            xg = gx = e = gy = yg.    
        \]

        Ma allora segue che \begin{align*}
            &x \tag*{(el. neutro)}\\
            =\ &x \cdot e \tag*{($e = gy$)}\\
            =\ &x(gy) \tag*{(per (G1))}\\
            =\ &(xg)y \tag*{($xg = e$)}\\
            =\ &e \cdot y \tag*{(el. neutro)}\\
            =\ &g
        \end{align*} ovvero $x = y = g\inv$.
        \item Sappiamo che $gg\inv = g\inv g = e$. Sia $x$ l'inverso di $g\inv$, ovvero \[
            g\inv x = xg\inv = e.    
        \]
        
        Dunque $g$ è un inverso di $g\inv$, ma per \ref{prop:prop_grp:inv_unico} l'inverso è unico e quindi $(g\inv)\inv = g$.
        \item Sia $(hg)\inv$ l'inverso di $hg$. Allora per (G3) sappiamo che \begin{align*}
            &(hg)(hg)\inv = e \tag*{(moltiplico a sx per $h\inv$)} \\
            \iff\ &h\inv hg(hg)\inv = h\inv \tag*{(per (G3))}\\
            \iff\ &g(hg)\inv = h\inv \tag*{(moltiplico a sx per $g\inv$)} \\
            \iff\ &g\inv g(hg)\inv = g\inv h\inv \tag*{(per (G3))}\\
            \iff\ &(hg)\inv = g\inv h\inv.
        \end{align*}
        \item Legge di cancellazione sinistra: \begin{align*}
            &ab = ac \tag*{(moltiplico a sx per $a\inv$)} \\
            \iff\ &a\inv ab = a\inv ac \tag*{(per (G3))} \\
            \iff\ &b = c.
        \end{align*}

        Legge di cancellazione destra: \begin{align*}
            &ba = ca \tag*{(moltiplico a dx per $a\inv$)} \\
            \iff\ & ba a\inv = ca a\inv \tag*{(per (G3))} \\
            \iff\ &b = c. \tag*{\qedhere}
        \end{align*}
    \end{enumerate}
\end{proof}
\section{Sottogruppi}

\begin{definition}[Sottogruppo]\label{def:sottogruppo}
    Sia $(G, *)$ un gruppo e sia $H \subseteq G$, $H \neq \emptyset$.    
    Allora $H$ insieme ad un'operazione $*_H$ si dice \emph{sottogruppo} di $(G, *)$ se $(H, *_H)$ è un gruppo.

    Si scrive $H \sgr G$ se l'operazione $*_H$ è l'operazione $*$, ovvero l'operazione del sottogruppo è indotta da $G$.
\end{definition}

\begin{proposition}[Condizione necessaria e sufficiente per i sottogruppi]\label{prop:cond_sgr}
    Sia $(G, *)$ un gruppo e sia $H \subseteq G$, $H \neq \emptyset$.
    Allora $H \sgr G$ se e solo se \begin{enumerate}[label={(\roman*)}, ref={\theproposition: (\roman*)}]
        \item \label{prop:cond_sgr:op} $*$ è un'operazione su $H$, ovvero per ogni $a, b \in H$ vale che $a*b \in H$;
        \item \label{prop:cond_sgr:inv} ogni elemento di $H$ è invertibile (in $H$), ovvero per ogni $h \in H$ vale che $h\inv \in H$.
    \end{enumerate}
\end{proposition}
\begin{proof}
    Dimostriamo entrambi i versi dell'implicazione.
    \begin{description}
        \item[($\implies$)] Ovvio in quanto se $H \sgr G$ allora $H$ è un gruppo.
        \item[($\impliedby$)] Sappiamo che $*$ è associativa poichè lo è in $G$; dobbiamo quindi mostrare solamente che $e_G \in H$.
        
        Per ipotesi $H \neq \emptyset$, dunque esiste un $h \in H$. Siccome $H$ è chiuso per inversi (ipotesi \hyperref[prop:cond_sgr:inv]{(ii)}) dovrà esistere anche $h\inv \in H$, mentre dal fatto che $H$ è chiuso per prodotti (ipotesi \hyperref[prop:cond_sgr:op]{(i)}) deve valere che $h * h\inv \in H$. 

        Tuttavia $h * h\inv = e_G$, dunque $e_G \in H$ e quindi $H$ è un sottogruppo indotto da $G$. \qedhere
    \end{description}
\end{proof}

Un sottogruppo particolarmente importante di qualsiasi gruppo è il \emph{centro del gruppo}:

\begin{definition}
    [Centro di un gruppo] \label{def:centro}
    Sia $(G, *)$ un gruppo. Allora si definisce \emph{centro di $G$} l'insieme \[
        \Zentr_{G} \deq \set*{x \in G \given g*x = x*g \;\;\forall g \in G}.    
    \]
\end{definition}

Intuitivamente, il centro di un gruppo è l'insieme di tutti gli elementi per cui $*$ diventa commutativa.

Mostriamo che il centro di un gruppo è un sottogruppo tramite la prossima proposizione.

\begin{proposition}
    [Proprietà del centro di un gruppo]
    \label{prop:centro}
    Sia $(G, *)$ un gruppo e sia $\Zentr{G}$ il suo centro.
    Allora vale che \begin{enumerate}[label={(\roman*)}, ref={\theproposition: (\roman*)}]
        \item $\Zentr{G} \sgr G$;
        \item $\Zentr{G} = G$ se e solo se $G$ è abeliano.
    \end{enumerate}
\end{proposition}
\begin{proof} Mostriamo le due affermazioni separatamente.
    \paragraph{$\Zentr{G}$ è un sottogruppo} 
    Notiamo innanzitutto che $\Zentr{G} \neq \varnothing$ poichè $e_G \in \Zentr{G}$. Per la proposizione \ref{prop:cond_sgr} ci basta mostrare che $*$ è un'operazione su $\Zentr{G}$ e che ogni elemento di $\Zentr{G}$ è invertibile.
    \begin{enumerate}
        [label={(\arabic*)}]
        \item Siano $x, y \in \Zentr{G}$ e mostriamo che $x*y \in \Zentr{G}$, ovvero che per ogni $g \in G$ vale che $g*(x*y) = (x*y)*g$. 
        \begin{align*}
            &g*(x*y) \tag*{(per (G1))}\\
            =\ &(g*x)*y \tag*{(dato che $x \in \Zentr{G}$)}\\
            =\ &(x*g)*y \tag*{(per (G1))}\\
            =\ &x*(g*y) \tag*{(dato che $x \in \Zentr{G}$)} \\
            =\ &x*(y*g) \tag*{(per (G1))}\\
            =\ &(x*y)*g.
        \end{align*}
        \item Sia $x \in \Zentr{G}$, mostriamo che $x\inv \in \Zentr{G}$.
        
        Per ipotesi \begin{align*}
            &g*x = x*g \tag*{(moltiplico a sinistra per $x\inv$)} \\
            \iff\ &x\inv * g*x = x\inv * x*g\tag*{(dato che $x\inv * x = e$)} \\
            \iff\ &x\inv * g*x = g \tag*{(moltiplico a destra per $x\inv$)} \\
            \iff\ &x\inv * g*x*x\inv = g*x\inv \tag*{(dato che $x\inv * x = e$)} \\
            \iff\ &x\inv * g = g*x\inv
        \end{align*}
        da cui $x\inv \in \Zentr{G}$.
    \end{enumerate}

    Per la proposizione \ref{prop:cond_sgr} segue che $\Zentr{G} \sgr G$.

    \paragraph{$\Zentr{G} = G$ se e solo se $G$ abeliano} Dimostriamo entrambi i versi dell'implicazione.
    \begin{description}
        \item[($\implies$)] Ovvia: $\Zentr{G}$ è un gruppo abeliano, dunque se $G = \Zentr{G}$ allora $G$ è abeliano.
        \item[($\impliedby$)]  Ovvia: $\Zentr{G}$ è l'insieme di tutti gli elementi di $G$ per cui $*$ commuta, ma se $G$ è abeliano questi sono tutti gli elementi di $G$, ovvero $\Zentr{G} = G$.  \qedhere
    \end{description}
\end{proof}

Un altro esempio è dato dai sottogruppi di $(\Z, +)$.

\begin{definition}
    [Insieme dei multipli interi]
    Sia $n \in \Z$. Allora chiamo $n\Z$ l'insieme dei multipli interi di $n$ \[
         n\Z \deq \set*{nk \given k \in \Z}.
    \]
\end{definition}

È semplice verificare che $(n\Z, +)$ è un gruppo per ogni $n \in \Z$. In particolare vale la seguente proposizione.

\begin{proposition}
    [$n\Z$ è sottogruppo di $\Z$] \label{prop:nZ_sgr_Z}
    Per ogni $n \in \Z$ vale che $(n\Z, +) \sgr (\Z, +)$.
\end{proposition}
\begin{proof}
    Innanzitutto notiamo che $n\Z \neq \varnothing$ in quanto $n \cdot 0 = 0 \in n\Z$.     
    Mostriamo ora che $n\Z \sgr \Z$.
    \begin{enumerate}[label={(\arabic*)}]
        \item Siano $x, y \in n\Z$ e mostriamo che $x+y \in \Z$. 
        
        Per definizione di $n\Z$ esisteranno $k, h \in \Z$ tali che $x = nk$, $y = nh$.
        
        Allora $x + y = nk + nh = n(k + h) \in n\Z$ in quanto $k + h \in \Z$.
        \item Sia $x \in n\Z$, mostriamo che $-x \in n\Z$.
        
        Per definizione di $n\Z$ esisterà $k \in \Z$ tale che $x = nk$.

        Allora affermo che $-x = n(-k) \in n\Z$. Infatti \[
            x + (-x) = nk + n(-k) = n(k - k) = 0    
        \] che è l'elemento neutro di $\Z$.
    \end{enumerate}

    Dunque per la \autoref{prop:cond_sgr} segue che $n\Z \sgr \Z$, ovvero la tesi.
\end{proof}

\begin{corollary}\label{cor:nZ_mZ}
    Siano $n, m \in \Z$. Allora valgono i due fatti seguenti:
    \begin{enumerate}[label={(\roman*)}, ref={\thecorollary: (\roman*)}]
        \item \label{cor:nZ_mZ:subset} $n\Z \subseteq m\Z \iff m \divides n$;
        \item \label{cor:nZ_mZ:eq} $n\Z = m\Z \iff n = \pm m$.
    \end{enumerate}
\end{corollary}
\begin{proof} Dimostriamo le due affermazioni separatamente.
    \begin{enumerate}[label={(\roman*)}]
        \item Dimostriamo entrambi i versi dell'implicazione.

        \begin{description}
            \item[($\implies$)] Supponiamo $n\Z \subseteq m\Z$, ovvero che per ogni $x \in n\Z$ allora $x \in m\Z$.
            
            Sia $k \in \Z$ tale che $\gcd{k}{m} = 1$ e sia $x = nk$.
            
            Per definizione di $n\Z$ segue che $x \in n\Z$, dunque $x \in m\Z$. Allora dovrà esistere $h \in \Z$ tale che \begin{align*}
                &x = mh \\
                \iff\ &nk = mh \\
                \implies\ &m \divides nk
                \intertext{Ma abbiamo scelto $k$ tale che $\gcd{k}{m} = 1$, dunque}
                \implies\ &m \divides n.
            \end{align*}
            \item[($\impliedby$)] Supponiamo che $m \divides n$, ovvero $n = mh$ per qualche $h \in \Z$. Allora \[
                n\Z = (mh)\Z \subseteq m\Z    
            \] in quanto i multipli di $mh$ sono necessariamente anche multipli di $m$.
        \end{description}
        \item Se $n\Z = m\Z$ allora vale che $n\Z \subseteq m\Z$ e $m\Z \subseteq n\Z$, dunque per il punto precedente $m \divides n$ e $n \divides m$, ovvero $n$ e $m$ sono uguali a meno del segno. \qedhere
    \end{enumerate}
\end{proof}

\begin{proposition}
    [Intersezione di sottogruppi è un sottogruppo]
    Sia $(G, \cdot)$ un gruppo e siano $H, K \sgr G$.
    Allora $H \inters K \sgr G$.
\end{proposition}
\begin{proof}
    Innanzitutto dato che $e_G \in H$, $e_G \in K$ segue che $e_G \in H \inters K$, che quindi non può essere vuoto.

    Per la proposizione \ref{prop:cond_sgr} è sufficiente dimostrare che $H \inters K$ è chiuso rispetto all'operazione $\cdot$ e che ogni elemento è invertibile.

    \begin{enumerate}[label={(\roman*)}]
        \item Siano $x, y \in H \inters K$; mostriamo che $xy \in H \inters K$.
        
        Per definizione di intersezione sappiamo che $x, y \in H$ e $x, y \in K$. Dato che $H$ è un gruppo varrà che $xy \in H$; per lo stesso motivo $xy \in K$; dunque $xy \in H \inters K$.

        \item Sia $x \in H \inters K$; mostriamo che $x\inv \in H \inters K$.
        
        Per definizione di intersezione sappiamo che $x \in H$ e $x \in K$. Dato che $H$ è un gruppo varrà che $x\inv \in H$; per lo stesso motivo $x\inv \in K$; dunque $x\inv \in H \inters K$.
    \end{enumerate}

    Dunque per la \autoref{prop:cond_sgr} segue che $H \inters K \sgr G$.
\end{proof}
\section{Generatori e gruppi ciclici}

Innanzitutto diamo una definizione generale di potenze:
\begin{definition}[Potenze intere] \label{def:potenze_intere}
    Sia $(G, \cdot)$ un gruppo e sia $g \in G$ qualsiasi. 
    
    Allora definiamo $g^k$ per $k \in \Z$ nel seguente modo: \[
        g^k \deq \begin{cases}
            e_G &\text{se } k = 0\\
            g \cdot g^{k-1} &\text{se } k > 0\\
            (g\inv)^k &\text{se } k < 0.
        \end{cases}    
    \]
\end{definition}

Se il gruppo è definito in notazione additiva, le potenze diventano prodotti per numeri interi. 

Piu' formalmente, se $(G, +)$ è un gruppo e $g \in G$ qualsiasi, allora definiamo $ng$ per $n \in \Z$ nel seguente modo: \[
    ng \deq \begin{cases}
        e_G &\text{se } n = 0\\
        g + (n-1)g &\text{se } n > 0\\
        (-n)(-g) &\text{se } n < 0.
    \end{cases}   
\]

Le potenze intere soddisfano alcune proprietà interessanti, verificabili facilmente per induzione, tra cui \begin{enumerate}[label={(P\arabic*)}, ref={(P\arabic*)}]
    \item per ogni $n, m \in \Z$ vale che $g^mg^n = g^{n+m}$,
    \item per ogni $n, m \in \Z$ vale che ${(g^n)}^m = g^{nm}$.
\end{enumerate}

\begin{definition}
    [Sottogruppo generato] \label{def:sgr_generato}
    Sia $(G, \cdot)$ un gruppo e sia $g \in G$.

    Allora si dice \emph{sottogruppo generato da $g$} l'insieme \[
        \cycl{g} \deq \set{g^k \suchthat k \in \Z}.    
    \]
\end{definition}

\begin{proposition}
    [Il sottogruppo generato è un sottogruppo abeliano] \label{prop:sgr_generato_è_sgr}
    Sia $(G, \cdot)$ un gruppo e sia $g \in G$ qualsiasi.

    Allora $\cycl{g} \leq G$. Inoltre $\cycl{g}$ è abeliano.
\end{proposition}
\begin{proof}
    Innanzitutto notiamo che $\cycl{g} \neq \varnothing$ in quanto $g \in \cycl{g}$. Mostriamo che $\cycl{g}$ è un sottogruppo indotto da $G$.

    \begin{enumerate}[label={(\roman*)}]
        \item Se $g^n, g^m \in \cycl{g}$ allora $g^ng^m = g^{n+m} \in \cycl{g}$ in quanto $n+m \in \Z$;
        \item Sia $g^n \in \cycl{g}$. Per definizione di potenza, $g^{-n}$ è l'inverso di $g^n$ e $g^{-n} \in \cycl{g}$ in quanto $-n \in \Z$. 
    \end{enumerate}

    Dunque per la proposizione \ref{prop:cond_sgr} segue che $\cycl{g} \leq G$.

    Inoltre notiamo che \[
        g^ng^m = g^{n+m} = g^{m+n} = g^mg^n    
    \] dunque $\cycl{g}$ è abeliano.
\end{proof}

Notiamo che, al contrario di quanto succede con i numeri interi, può succedere che $g^h = g^k$ per qualche $h \neq k$.

Supponiamo senza perdita di generalità $k > h$. In tal caso \begin{align*}
    g^{k-h} &= e_G\\
    \implies g^{k-h+1} &= g^{k-h} \cdot g\\
    &= e_G \cdot g\\
    &= g.
\end{align*}
Dunque il sottogruppo generato da $g$ non è infinito, ovvero \[
    \abs*{\cycl*{g}} < +\infty.    
\]

Questo ci consente di parlare di ordine di un elemento di un gruppo:

\begin{definition}
    [Ordine di un elemento di un gruppo] \label{def:ord_grp}
    Sia $(G, \cdot)$ un gruppo e sia $x \in G$.

    Allora si dice ordine di $x$ in $G$ il numero \[
        \ord[G]{x} \deq \min\set{k > 0 \suchthat x^k = _Ge}.    
    \]

    Se l'insieme $\set{k > 0 \suchthat x^k = e_G}$ è vuoto, allora per definizione \[
        \ord[G]{x} \deq +\infty.
    \]
\end{definition}

Quando il gruppo di cui stiamo parlando sarà evidente scriveremo semplicemente $\ord{x}$.

\begin{proposition}
    [Scrittura esplicita del sottogruppo generato] \label{prop:sgr_generato}
    Sia $(G, \cdot)$ un gruppo e sia $x \in G$ tale che $\ord[G]{x} = d < +\infty$. 

    Allora valgono i seguenti due fatti:
    \begin{enumerate}[label={(\roman*)}, ref={\theproposition: (\roman*)}]
        \item \label{prop:sgr_generato:expr} Il sottogruppo generato $\cycl{x}$ è \[
            \cycl{x} = \set{e, x, x^2, \dots, x^{d-1}}.    
        \] Dunque in particolare $\abs*{\cycl* x} = d$.
        \item \label{prop:sgr_generato:ord_div_n} $x^n = e \iff d \divides n$.
    \end{enumerate}
\end{proposition}
\begin{proof}
    Dimostriamo le due affermazioni separatamente. 
    
    \paragraph{Parte 1.} Sicuramente vale che \[
        \set{e, x, \dots, x^{d-1}} \subseteq \cycl{x}.    
    \] Dimostriamo che vale l'uguaglianza.

    Sia $k \in \Z$ qualsiasi. Allora $x^k \in \cycl x$. 
    
    Dimostriamo che necessariamente $x^k \in \set{e, x, \dots, x^{d-1}}$.

    Per la divisione euclidea esisteranno $q, r \in \Z$ tali che \begin{align*}
        k = qd + r &&\text{con } 0 \leq r < d.
    \end{align*} Allora sostituendo $k = qd+r$ otteniamo \begin{align*}
        x^k &= x^{qd + r}\\
        &= x^{qd}x^r\\
        &= e^qx^r\\
        &= x^r.
    \end{align*}
    
    Per ipotesi $0 \leq r < d$, dunque $x^r \in \set{e, x, \dots, x^{d-1}}$. Dato che $x^r = x^k$ concludiamo che \[
        x^k \in \set{e, x, \dots, x^{d-1}}    
    \] e quindi \[
        \cycl x = \set{e, x, \dots, x^{d-1}}.    
    \]

    Ci rimane da mostrare che $\abs*{\cycl* x} = d$, ovvero che tutti gli elementi di $\cycl x$ sono distinti. 

    Supponiamo per assurdo che esistano $a, b \in \Z$ con $0 \leq a < b < d$ (senza perdita di generalità) tali che $x^a = x^b$.

    Da questo segue che $x^{b-a} = e$, ma questo è assurdo poichè $b-a < d$ e per definizione l'ordine è il minimo numero positivo per cui $x^d = e$.

    Di conseguenza tutti gli elementi di $\cycl x$ sono distinti, ovvero $\abs*{\cycl* x} = d$.

    \paragraph{Parte 2.} Dimostriamo entrambi i versi dell'implicazione.
    \begin{description}
        \item[($\implies$)] Sia $n \in \Z$ tale che $x^n = e$.
        
        Per divisione euclidea esistono $q, r \in \Z$ tali che \begin{align*}
            n = qd + r &&\text{con } 0 \leq r < d.
        \end{align*}

        Dunque $x^n = x^{qd+r} = x^r = e$. Ma questo è possibile solo se $r = 0$, altrimenti andremmo contro la minimalità dell'ordine.
        
        Dunque $x = qd$, ovvero $d \divides n$.
        \item[($\impliedby$)] Ovvia: se $n = kd$ per qualche $k \in \Z$ allora \[
            x^n = x^{kd} = (x^d)^k = e^k = e.    
        \]
    \end{description}
\end{proof}

\begin{definition}
    [Gruppo ciclico]\label{def:grp_ciclico}
    Sia $(G, \cdot)$ un gruppo. 
    
    Allora $G$ si dice \emph{ciclico} se esiste un $g \in G$ tale che \[
        G = \cycl{g}.    
    \]

    L'elemento $g$ viene detto \emph{generatore} del gruppo $G$.
\end{definition}

Ad esempio $\Z$ è un gruppo ciclico, in quanto $\Z = \cycl{1}$, come lo è $n\Z = \cycl n$. Questi due gruppi sono anche infiniti, in quanto contengono un numero infinito di elementi.

Un esempio di gruppo ciclico finito è $\Zmod{n} = \cycl{[1]_n}$, che è finito in quanto $\ord[\Zmod{n}]{[1]_n} = n$.

\begin{theorem}
    [Ogni sottogruppo di un gruppo ciclico è ciclico] \label{th:sottogr_ciclico}
    Sia $(G, \cdot)$ un gruppo ciclico, ovvero $G = \cycl{g}$ per qualche $g \in G$. Sia inoltre $H \leq G$ un sottogruppo.

    Allora $H$ è ciclico, ovvero esiste $h \in \Z$ tale che $H = \cycl{g^h}$.
\end{theorem}
\begin{proof}
    Innanzitutto notiamo che $e_G \in H$. 
    
    Se $H = \set{e_G}$ allora $H$ è ciclico, e $H = \cycl{e_G}$.

    Assumiamo $\set e_G \subset H$. Allora esiste $k \in \Z$, $k \neq 0$ tale che $g^k \in H$. 
    Dato che per (G3) se $g^k \in H$ allora $g^{-k} \in H$ possiamo supporre senza perdita di generalità $k > 0$.

    Consideriamo l'insieme $S$ tale che \[
        S \deq \set{h > 0 \suchthat g^h \in H} \subseteq \N.    
    \] Avendo assunto $k \in S$ sappiamo che $S \neq \varnothing$, dunque per il principio del minimo $S$ ammette minimo.

    Sia $h_0 = \min S$. Mostro che $H = \cycl*{g^{h_0}}$.
    \begin{description}
        \item[($\supseteq$)] Per ipotesi $g^{h_0} \in H$. 
        
        Dato che $H$ è un sottogruppo di $G$ tutte le potenze intere di $g^{h_0}$ dovranno appartenere ad $H$, ovvero $\cycl*{g^{h_0}} \subseteq H$.
        \item[($\subseteq$)] Sia $n \in \N$ tale che $g^n \in H$. Dimostriamo che $g^n \in \cycl*{g^{h_0}}$.
        
        Per divisione euclidea esistono $q, r \in \Z$ tali che \begin{align*}
            n = qh_0 + r &&\text{con } 0 \leq r < h_0.
        \end{align*}
        Dunque \begin{align*}
            g^n &= g^{qh_0 + r}\\
                &= g^{qh_0}g^r.
        \end{align*}
        Moltiplicando entrambi i membri per $g^{-qh_0}$ otteniamo \[
            \iff g^ng^{-qh_0} = g^r.
        \]

        Ma $g^n \in H$ e $g^{-qh_0} \in H$ (in quanto è una potenza intera di $g^{h_0}$), dunque anche il loro prodotto $g^r \in H$.

        Se $r > 0$ allora esisterebbe una potenza di $g$ con esponente positivo minore di $h_0$ contenuto in $H$, che è assurdo in quanto abbiamo assunto che $h_0$ sia il minimo dell'insieme $S$.
        
        Segue che $r = 0$, ovvero $n = qh_0$, ovvero che $g^n \in \cycl*{g^{h_0}}$, ovvero $H \subseteq \cycl*{g^{h_0}}$.
    \end{description}

    Concludiamo quindi che $H = \cycl*{g^{h_0}}$, ovvero $H$ è ciclico.
\end{proof}

Consideriamo i sottogruppi di $\Z$. Tramite la proposizione \ref{prop:nZ_sgr_Z} abbiamo dimostrato che per ogni $n \in \Z$ allora $n\Z \leq \Z$. La prossima proposizione mostra che i sottogruppi della forma $n\Z = \cycl{n}$ sono gli unici possibili.

\begin{proposition}
    [Caratterizzazione dei sottogruppi di $\Z$]\label{prop:sgr_Z}
    I sottogruppi di $\Z$ sono tutti e solo della forma $n\Z$ al variare di $n \in \N$.
\end{proposition}
\begin{proof}
    Nella proposizione \ref{prop:nZ_sgr_Z} abbiamo mostrato che $n\Z \leq \Z$ per ogni $n \in \Z$. Ora mostriamo che è sufficiente considerare $n \in \N$ e che questi sono gli unici sottogruppi possibili.

    Dato che $\Z$ è ciclico (poiché $\Z = \cycl{1}$) per il teorema \ref{th:sottogr_ciclico} ogni suo sottogruppo dovrà essere ciclico, ovvero dovrà essere della forma $\cycl{n}$ per qualche $n \in \N$.

    Per la proposizione \ref{cor:nZ_mZ:eq} sappiamo che $n\Z = (-n)\Z$, dunque possiamo considerare (senza perdita di generalità) $n$ positivo o nullo, ovvero $n \in \N$.

    Ma $\cycl{n} = n\Z$, dunque i sottogruppi di $\Z$ sono tutti e solo della forma $n\Z$ al variare di $n \in \N$.
\end{proof}


\subsection{Il gruppo ciclico $\Zmod{n}$}

In questa sezione analizzeremo il gruppo ciclico $(\Zmod{n}, +)$, anche definito da \[
    \Zmod{n} = \cycl{[1]_n} = \cycl{\eqclass 1}.
\]

L'ordine di $\eqclass 1$ in $\Zmod{n}$ è $n$. Infatti \begin{align*}
    &x\cdot \eqclass 1 = \eqclass 0\\
    \iff &x \congr 0 \Mod n \\
    \iff &x = nk
\end{align*}
con $k \in \Z$. 
La minima soluzione positiva a quest'equazione è per $k = 1$, dunque $x = n$. 
Per la proposizione \ref{prop:sgr_generato:expr} sappiamo quindi che \begin{equation} \label{eq:card_Z/nZ}
    \abs{\Zmod{n}} = \abs{\eqclass 1} = \ord[\Zmod{n}]{\eqclass 1} = n.
\end{equation}

\begin{proposition}
    [Ordine degli elementi di $\Zmod n$] \label{prop:ord_in_Z/nZ}
    Sia $\eqclass a \in \Zmod n$ qualsiasi. Allora vale che \[
        \ord{\eqclass a} = \frac{n}{\mcd{a}{n}}    
    \] dove $a \in \Z$ è un rappresentante della classe $\eqclass a$.
\end{proposition}
\begin{proof}
    Per definizione di ordine \[
        \ord{\eqclass a} = \min \set{k > 0 \suchthat k \eqclass a = \eqclass 0}.    
    \]

    Si tratta quindi di trovare la minima soluzione positiva di $ax \congr 0 \Mod{n}$. Divido entrambi i membri e il modulo per $a$, ottenendo \[
        x \congr 0 \Mod{\frac{n}{\mcd{n}{a}}} \implies x = \frac{n}{\mcd{n}{a}}t    
    \] al variare di $t \in \Z$. 

    Dato che siamo interessati alla minima soluzione positiva, questa è ottenuta per $t = 1$, da cui segue che \[
        \ord{\eqclass a} = \frac{n}{\mcd{n}{a}}. \qedhere    
    \]
\end{proof}

\begin{corollary}
    [Conseguenze della proposizione \ref{prop:ord_in_Z/nZ}]
    \label{cor:cons_ord_in_Z/nZ}
    Consideriamo il gruppo $(\Zmod n, +)$. Valgono le seguenti affermazioni:
    \begin{enumerate}[label={(\roman*)}, ref={\theproposition: (\roman*)}]
        \item \label{cor:cons_ord_in_Z/nZ:ord_a_div_n} $\forall \eqclass{a} \in \Zmod{n}. \quad \ord{\eqclass a} \divides n$.
        \item \label{cor:cons_ord_in_Z/nZ:phi(n)_gen} $\Zmod{n}$ ha $\phi(n)$ generatori.
        \item \label{cor:cons_ord_in_Z/nZ:phi(d)_el} Sia $d \in \Z$ tale che $d \divides n$. Allora in $\Zmod{n}$ ci sono esattamente $\phi(d)$ elementi di ordine $d$.
    \end{enumerate}
\end{corollary}

\begin{proof}
    \begin{enumerate}[label={(\roman*)}]
        \item Ovvia in quanto (per la proposizione \ref{prop:ord_in_Z/nZ}) $\ord{\eqclass a} = \dfrac{n}{\mcd{n}{a}} \divides n$.
        \item Sia $\eqclass x \in \Zmod{n}$. Sappiamo che $\eqclass x$ è un generatore di $\Zmod{n}$ se \[
            \cycl*{\eqclass x} = \Zmod n    
        \] ovvero se la cardinalità di $\cycl*{\eqclass x}$ è $n$.

        Per la proposizione \ref{prop:ord_in_Z/nZ} $\ord{\eqclass x} = \dfrac{n}{\mcd{n}{x}}$, dunque $\eqclass x$ è un generatore se e solo se $\mcd{n}{x} = 1$, ovvero se $x$ è coprimo con $n$.

        Ma ci sono $\phi(n)$ numeri coprimi con $n$, dunque ci sono $\phi(n)$ generatori di $\Zmod n$.
        \item Sia $\eqclass a \in \Zmod n$ tale che \[
            \ord{\eqclass a} = \frac{n}{\mcd{n}{a}} = d.    
        \]

        Allora $\mcd{n}{a} = \dfrac{n}{d}$, da cui segue che $\dfrac{n}{d} \divides a$.

        Sia $b \in \Z$ tale che $a = \dfrac{n}{d}b$. Dato che $\mcd{n}{a} = \dfrac{n}{d}$ segue che \begin{align*}
            &\mcd{n}{\frac{n}{d}b} = \frac{n}{d}\\[3pt]
            \iff &\mcd{\frac{n}{d}d}{\frac{n}{d}b} = \frac{n}{d}\\[3pt]
            \iff &\frac{n}{d}\mcd{d}{b} = \frac{n}{d}\\[3pt]
            \iff &\mcd{d}{b} = 1
        \end{align*}
        ovvero se e solo se $d$ e $b$ sono coprimi.

        Dunque segue che ho $\phi(d)$ scelte per $b$, ovvero ho $\phi(d)$ elementi di ordine $d$.
    \end{enumerate}
\end{proof}

Questo corollario ci consente di enunciare una proprietà della funzione $\phi(\cdot)$.

\begin{corollary}
    [Espressione per $n$ in termini di $\phi(n)$]
    Sia $n \in \Z$. Allora vale che \[
        n = \sum_{d \divides n} \phi(d).    
    \]
\end{corollary}
\begin{proof}
    Sia $X_d$ l'insieme \[
        X_d \deq \set{\eqclass a \in \Zmod n \suchthat \ord{\eqclass a} = d}.    
    \]
    Se $d \ndivides n$ per la proposizione \ref{cor:cons_ord_in_Z/nZ:ord_a_div_n} segue che $X_d = \varnothing$.

    Dunque abbiamo che \[
        \Zmod n = \bigsqcup_{d \divides n} X_d.  
    \] Sfruttando la proposizione \ref{cor:cons_ord_in_Z/nZ:phi(d)_el} sappiamo che $\abs{X_d} = \phi(d)$, dunque passando alle cardinalità segue che \[
        \abs{\Zmod n} = n = \sum_{d \divides n} X_d.    
    \]
\end{proof}

Studiamo ora i sottogruppi di $\Zmod{n}$.

\begin{proposition}
    [Caratterizzazione dei sottogruppi di $\Zmod{n}$] \label{prop:sgr_Z/nZ}
    Studiamo il gruppo $(\Zmod{n}, +)$. Valgono i due seguenti fatti:
    \begin{enumerate}[label={(\roman*)}, ref={\theproposition: (\roman*)}]
        \item \label{prop:sgr_Z/nZ:ciclico_ord_d} Sia $H \leq \Zmod{n}$. Allora $H$ è ciclico e $\abs{H} = d$ per qualche $d \divides n$.
        \item \label{prop:sgr_Z/nZ:unosolo_ord_d} Sia $d \in \Z, d \divides n$. Allora $\Zmod{n}$ ammette uno e un solo sottogruppo di ordine $d$.
    \end{enumerate}
\end{proposition}

\begin{proof}
    \begin{enumerate} [label={(\roman*)}]
        \item Sia $H \leq \Zmod n$; per il teorema \ref{th:sottogr_ciclico} sappiamo che $H$ deve essere ciclico, ovvero $H = \cycl*{\eqclass h}$ per qualche $\eqclass h \in \Zmod n$.
        
        Sia $d = \ord{\eqclass h}$. Allora per il corollario \ref{cor:cons_ord_in_Z/nZ:ord_a_div_n} segue che \[
            \abs{H} = \ord{\eqclass h} = d \divides n.   
        \]
        \item Sia $H_d$ l'insieme \[
            H_d = \set{\eqclass 0,\; \frac{\eqclass n}{d},\; 2\frac{\eqclass n}{d}, \dots,\; (d-1)\frac{\eqclass n}{d}}.    
        \] Mostriamo innanzitutto che $H_d = \cycl*{\dfrac{\eqclass{n}}{d}}$.
        
        Infatti ovviamente $H_d \subseteq \cycl*{\dfrac{\eqclass n}{d}}$. Per mostrare che sono uguali basta notare che \[
            \abs*{\cycl*{\frac{\eqclass n}{d}}} = \ord{\frac{\eqclass n}{d}} = \frac{n}{\mcd{\frac{n}{d}}{n}} = \frac{n}{\mcd{\frac{n}{d}}{\frac{n}{d}d}} = \frac{n}{\frac{n}{d}\mcd{1}{d}} = d
        \] dunque i due insiemi sono finiti, hanno la stessa cardinalità e il primo è incluso nel secondo, da cui segue che sono uguali.

        Sia ora $H \leq \Zmod n$ tale che $\abs{H} = d$. Per il teorema \ref{th:sottogr_ciclico} segue che $H = \cycl*{\eqclass x}$ per qualche $\eqclass x \in \Zmod n$ tale che $\ord{\eqclass x} = d$.

        Seguendo la dimostrazione di \ref{cor:cons_ord_in_Z/nZ:phi(d)_el} possiamo scrivere $\eqclass x = \dfrac{\eqclass n}{d}b$ con $b \in Z$ tale che $\mcd{b}{d} = 1$.

        Ma $H_d = \cycl*{\dfrac{\eqclass n}{d}}$ contiene tutti i multipli di $\dfrac{\eqclass n}{d}$, dunque deve contenere anche $\eqclass x$.

        Dunque dato che $\eqclass x \in H_d$ segue che $H = \cycl*{\eqclass x} \subseteq H_d$. Ma gli insiemi $H$ e $H_d$ hanno la stessa cardinalità, dunque $H = H_d$, ovvero vi è un solo sottogruppo di ordine $d$. \qedhere
    \end{enumerate}
\end{proof}
\section{Omomorfismi di gruppi}

\begin{definition}[Omomorfismo tra gruppi] \label{def:omo_gruppi}
    Siano $(G_1, *)$, $(G_2, \star)$ due gruppi. Allora la funzione \[
        f : G_1 \to G_2    
    \] si dice \emph{omomorfismo di gruppi} se per ogni $x, y \in G_1$ vale che \begin{equation}
        f(x * y) = f(x) \star f(y).
    \end{equation}
    L'insieme di tutti gli omomorfismi da $G_1$ a $G_2$ si indica con $\Hom{G_1}{G_2}$.
\end{definition}

\begin{example}
    Ad esempio la funzione \begin{align*}
    \pi_n : \Z &\to \Zmod{n}\\
    a &\mapsto [a]_n
\end{align*} è un omomorfismo tra i gruppi $\Z$ e $\Zmod{n}$. Infatti vale che \[
    \pi_n(a + b) = \eqclass{a + b} = \eqclass a + \eqclass b = \pi_n(a) + \pi_n(b).
\] Questo particolare omomorfismo si dice \emph{riduzione modulo $n$}.
\end{example}

\begin{example}
    Un altro esempio è la funzione \begin{align*}
    f : (\R, +) &\to (\R^+, {\cdot})\\
    x &\mapsto e^x.
\end{align*} Infatti vale che \[
    f(x + y) = e^{x + y} = e^xe^y = f(x)f(y).    
\]
\end{example}

\begin{proposition}
    [Composizione di omomorfismi] \label{prop:comp_omo}
    Siano $(G_1, *)$, $(G_2, \star)$, $(G_3, \cdot)$ tre gruppi e siano $\phi : G_1 \to G_2$ e $\psi : G_2 \to G_3$ omomorfismi.

    Allora la funzione $\psi \circ \phi : G_1 \to G_3$ è un omomorfismo tra i gruppi $G_1$ e $G_3$.
\end{proposition}
\begin{proof}
    Siano $h, k \in G_1$ e dimostriamo che \[
        (\psi \circ \phi)(h * k) = (\psi \circ \phi)(h) \cdot (\psi \circ \phi)(k).
    \] Infatti vale che
    \begin{align*}
        (\psi \circ \phi)(h * k) &= \psi(\phi(h * k)) \tag{$\phi$ omo.}\\
        &= \psi(\phi(h) \star \phi(k)) \tag{$\psi$ omo.}\\
        &= \psi(\phi(h)) \cdot \psi(\phi(k)) \\
        &= (\psi \circ \phi)(h) \cdot (\psi \circ \phi)(k)
    \end{align*}
    che è la tesi.
\end{proof}

Dato che un omomorfismo è una funzione, possiamo definire i soliti concetti di immagine e controimmagine.

\begin{definition}
    [Immagine e controimm. di un omomorf. attraverso un insieme] \label{def:omo_imm_controimm}
    Siano $(G_1, *)$, $(G_2, \star)$ due gruppi e sia $f : G_1 \to G_2$ un omomorfismo.

    Siano $H \leq G_1$, $K \leq G_2$. Allora definiamo l'insieme \[
        f(H) \deq \set{f(h) \in G_2 \suchthat h \in H} \subseteq G_2    
    \] detto \emph{immagine di $f$ attraverso $H$}, e l'insieme \[
        f\inv(K) \deq \set{g \in G_1 \suchthat f(g) \in K} \subseteq G_1
    \] detto \emph{controimmagine di $f$ attraverso $K$}.

    Definiamo inoltre l'\emph{immagine dell'omomorfismo $f$} come \[
        \Imm f \deq f(G_1) = \set{f(g) \in G_2 \suchthat g \in G_1}.     
    \]
\end{definition}

Per gli omomorfismi definiamo inoltre un concetto nuovo, il \emph{nucleo} o \emph{kernel} dell'omomorfismo.

\begin{definition}
    [Kernel di un omomorfismo] \label{def:kernel_omo}
    Siano $(G_1, *)$, $(G_2, \star)$ due gruppi e sia $f : G_1 \to G_2$ un omomorfismo.

    Allora si dice \emph{kernel} o \emph{nucleo} dell'omomorfismo $f$ l'insieme \[
        \ker f \deq \set{g \in G_1 \suchthat f(g) = e_2} \subseteq G_1.    
    \]
\end{definition}

Osserviamo che possiamo anche esprimere il nucleo di un omomorfismo in termini della controimmagine del sottogruppo banale $\set{e_2}$: \[
    \ker f = f\inv(\set{e_2}).
\]

\begin{proposition}
    [Proprietà degli omomorfismi] \label{prop:prop_omo}
    Siano $(G_1, \cdot)$, $(G_2, \star)$ due gruppi e sia $f : G_1 \to G_2$ un omomorfismo.

    Allora valgono le seguenti affermazioni.
    \begin{enumerate}[label={(\roman*)}, ref={\theproposition: (\roman*)}]
        \item \label{prop:prop_omo:e_va_in_e'}$f(e_1) = e_2$;
        \item \label{prop:prop_omo:inv_passa_dentro}$f(x\inv) = f(x)\inv$;
        \item \label{prop:prop_omo:f(H)_sgr_cod}$\forall H \leq G_1. \quad f(H) \leq G_2$;
        \item \label{prop:prop_omo:f\inv(K)_sgr_dom} $\forall K \leq G_2. \quad f\inv(K) \leq G_1$;
        \item \label{prop:prop_omo:imm_ker_sgr}$f(G_1) \leq G_2$ e $\ker f \leq G_1$;
        \item \label{prop:prop_omo:cond_iniett}$f$ è iniettivo se e solo se $\ker f = \set{e_1}$.
    \end{enumerate}
\end{proposition}
\begin{proof}
    \begin{enumerate}[label={(\roman*)}]
        \item $f(e_1) \stackrel{\text{(el. neutro)}}{=} f(e_1 \cdot e_1) \stackrel{\text{(omo.)}}{=} f(e_1) \star f(e_1)$.
        
        Applicando la legge di cancellazione \ref{prop:prop_grp:canc} otteniamo \[
            e_2 = f(e_1).    
        \]
        \item Sfruttando il punto \ref{prop:prop_omo:e_va_in_e'} sappiamo che \begin{align*}
            e_2 = f(e_1) = f(x \cdot x\inv) = f(x)\star f(x\inv) \\
            e_2 = f(e_1) = f(x\inv \cdot x) = f(x\inv)\star f(x).
        \end{align*} Dalla prima segue che $f(x\inv)$ è inverso a destra di $f(x)$, dalla seconda che $f(x\inv)$ è inverso a sinistra di $f(x)$.

        Dunque concludiamo che $f(x\inv)$ è inverso di $f(x)$, ovvero \[
            f(x)\inv = f(x\inv).    
        \]
        \item Sia $H \leq G_1$. Dato che $H \neq \varnothing$ esisterà un $h \in H$, dunque $f(H)$ non puo' essere vuoto in quanto dovrà contenere $f(h)$ (sicuramente $e_2 \in f(H)$).
        
        Dunque per la proposizione \ref{prop:cond_sgr} basta mostrare che $f(H)$ è chiuso rispetto al prodotto e che l'inverso di ogni elemento di $f(H)$ è ancora in $f(H)$.

        \begin{enumerate}[label={(\arabic*)}]
            \item Mostriamo che se $x, y \in f(H)$ allora $x\star y \in f(H)$.
            
            Per definizione di $f(H)$ dovranno esistere $h_x, h_y \in H$ tali che $x = f(h_x)$ e $y = f(h_y)$. Allora \begin{align*}
                x\star y &= f(h_x)\star f(h_y) \tag*{\text{($f$ è omo)}}\\
                &= f(h_x \cdot h_y) \tag*{$H$ è sottogr. di $G_1$}\\
                &\in f(H).
            \end{align*}
            \item Mostriamo che se $x \in f(H)$ allora $x\inv \in f(H)$.
            
            Per definizione di $f(H)$ dovrà esistere $h \in H$ tale che $x = f(h)$. Dato che $H \leq G_1$ allora $h\inv \in H$.

            Dunque $f(h\inv) \in f(H)$, ma per il punto \ref{prop:prop_omo:inv_passa_dentro} sappiamo che \[
                f(h\inv) = f(h)\inv = x\inv \in f(H).   
            \]
        \end{enumerate}
        Dunque $f(H) \leq G_2$.
        \item Sia $K \leq G_2$. Dato che $e_2 \in K$, sicuramente $f\inv(K) \neq \varnothing$, in quanto $e_1 = f\inv(e_2) \in f\inv(K)$.
        
        Dunque per la proposizione \ref{prop:cond_sgr} basta mostrare che $f\inv(K)$ è chiuso rispetto al prodotto e che l'inverso di ogni elemento di $f\inv(K)$ è ancora in $f\inv(K)$.

        \begin{enumerate}[label={(\arabic*)}]
            \item Mostriamo che se $x, y \in f\inv(K)$ allora $x* y \in f\inv(K)$.
            
            Per definizione di $f\inv(K)$ sappiamo che \begin{align*}
                x \in f\inv(K) &\iff f(x) \in K \\
                y \in f\inv(K) &\iff f(y) \in K.
            \end{align*}
            
            Dato che $K \leq G_2$ allora segue che \[
                f(x) \star f(y) = f(x * y) \in K     
            \] ovvero $x * y \in f\inv(K)$.
            \item Mostriamo che se $x \in f\inv(K)$ allora $x\inv \in f\inv(K)$.
            
            Per definizione di $f\inv(K)$ sappiamo che \[
                x \in f\inv(K) \iff f(x) \in K.    
            \] 
            
            Dato che $K \leq G_2$ segue che $f(x)\inv \in K$, ma per il punto \ref{prop:prop_omo:inv_passa_dentro} sappiamo che $f(x)\inv = f(x\inv)$, dunque \[
                f(x\inv) \in K \implies x\inv \in f\inv(K).  
            \]
        \end{enumerate}
        Dunque $f\inv(K) \leq G_1$.
        \item Dato che $G_1 \leq G_1$ per il punto \ref{prop:prop_omo:f(H)_sgr_cod} segue che $\Imm f = f(G_1) \leq G_2$.
        
        Per definizione $\ker f = f\inv(\set{e_2})$; inoltre $\set{e_1} \leq G_2$, dunque per il punto \ref{prop:prop_omo:f\inv(K)_sgr_dom} segue che $\ker f \leq G_1$.
        \item Dimostriamo entrambi i versi dell'implicazione.
        \begin{description}
            \item[($\implies$)] Supponiamo che $f$ sia iniettivo. Allora $\abs*{f\inv(\set{e_2})} = 1$.
            
            Tuttavia sicuramente $e_1 \in f\inv(\set{e_2}) = \ker f$ (in quanto $f(e_1) = e_2$), dunque dovrà necessariamente essere $\ker f = \set {e_1}$.
            \item[($\impliedby$)] Supponiamo che $\ker f = \set{e_1}$.
            
            Siano $x, y \in G_1$ tali che $f(x) = f(y)$. Moltiplicando entrambi i membri (ad esempio a destra) per $f(y)\inv \in G_2$ otteniamo \begin{align*}
                &f(x)\star f(y)\inv = f(y)\star f(y)\inv \tag{per la \ref{prop:prop_omo:inv_passa_dentro}}\\
                \iff &f(x)\star f(y\inv) = e_2 \tag{$f$ è omomorf.}\\
                \iff &f(x* y\inv) = e_2 \tag{def. di $\ker f$}\\
                \iff &x* y\inv \in \ker f \tag{ipotesi: $\ker f = \set{e_1}$}\\
                \iff &x*y\inv = e_1 \tag{moltiplico a dx per $y$}\\
                \iff &x = y.
            \end{align*}

            Dunque $f(x) = f(y)$ implica che $x = y$, ovvero $f$ è iniettivo.
        \end{description}
    \end{enumerate}
\end{proof}

\begin{proposition}
    [Omomorfismi e ordine]\label{prop:omo_ord}
    Siano $(G_1, *)$, $(G_2, \star)$ due gruppi e sia $f : G_1 \to G_2$ omomorfismo.

    Allora valgono le seguenti due affermazioni \begin{enumerate}[label={(\roman*)}, ref={\theproposition: (\roman*)}]
        \item \label{prop:omo_ord:ord_f_div_ord_x} per ogni $x \in G$ vale che $\ord[G_2]{f(x)} \divides \ord[G_1]{x}$;
        \item \label{prop:omo_ord:inj_sse_ord_f=ord_x} $f$ è iniettivo se e solo se $\ord[G_2]{f(x)} = \ord[G_1]{x}$.
    \end{enumerate}
\end{proposition}
\begin{proof}
    Innanzitutto diciamo che se $\ord x = +\infty$ allora $\ord{f(x)} \divides \ord{x}$ qualunque sia $\ord{f(x)}$ (anche se è $+\infty$).

    \begin{enumerate}[label={(\roman*)}]
        \item Sia $x \in G_1$. Se $\ord x = +\infty$ allora abbiamo finito, dunque supponiamo $\ord x = n$ per qualche $n \in \Z$, $n > 0$.
        
        Per definizione di ordine questo significa che $x^n = e_1$.
        Allora \begin{align*}
            f(x)^n &= f(x) \star \cdots \star f(x) \tag{$f$ è omo.}\\
            &= f(x^n) \\
            &= f(e_1) \tag{prop. \ref{prop:prop_omo:e_va_in_e'}}\\
            &= e_2.
        \end{align*}

        Dunque $f(x)^n = e_2$, quindi per la proposizione \ref{prop:sgr_generato:ord_div_n} segue che \[
            \ord{f(x)} \divides n = \ord x.    
        \]
        \item Dimostriamo entrambi i versi dell'implicazione.
        \begin{description}
            \item[($\implies$)] Supponiamo $f$ iniettiva. \begin{itemize}
                \item Se $\ord{f(x)} = +\infty$ allora per il punto \ref{prop:omo_ord:ord_f_div_ord_x} sappiamo che $+\infty \divides \ord x$, dunque $\ord x = +\infty = \ord{f(x)}$.
                \item Se $\ord{f(x)} = m < +\infty$ allora \[
                    f(x)^m = e_2 \iff f(x) \star \dots \star f(x) = e_2 \iff f(x^m) = e_2,    
                \] ovvero $x^m \in \ker f$.

                Ma $f$ è iniettiva, dunque per \ref{prop:prop_omo:cond_iniett} $\ker f = \set{e_1}$, da cui segue che $x^m = e_1$. 
                Dunque per la proposizione \ref{prop:sgr_generato:ord_div_n} segue che \[
                    \ord x \divides m = \ord{f(x)}.    
                \]

                Inoltre per il punto \ref{prop:omo_ord:ord_f_div_ord_x} sappiamo che $\ord{f(x)} \divides \ord x$, dunque $\ord{f(x)} = \ord x$.
            \end{itemize} 
            \item[($\impliedby$)] Sia $x \in \ker f$, ovvero $f(x) = e_2$. Allora \[
                1 = \ord[G_2]{e_2} = \ord{f(x)} \stackrel{\text{hp.}}{=} \ord[G_1]{x}.
            \] 
            
            Ma $\ord x = 1$ se e solo se $x = e_1$, ovvero $\ker f = \set{e_1}$, dunque per la proposizione \ref{prop:prop_omo:cond_iniett} $f$ è iniettiva.
        \end{description}
    \end{enumerate}
\end{proof}

\subsection{Isomorfismi}

Gli omomorfismi bigettivi sono particolarmente importanti e vanno sotto il nome di \emph{isomorfismi}.

\begin{definition}
    [Isomorfismo] \label{def:isomorfismo}
    Siano $(G_1, *)$, $(G_2, \star)$ due gruppi e sia $\phi : G_1 \to G_2$ un omomorfismo.

    Allora se $\phi$ è bigettivo si dice che $\phi$ è un \emph{isomorfismo}. Inoltre i gruppi $G_1$ e $G_2$ si dicono \emph{isomorfi} e si scrive $G_1 \isomorph G_2$.
\end{definition}

\begin{corollary}[Transitività della relazione di isomorfismo]\label{prop:trans_isomorf}
    Siano $(G_1, *)$, $(G_2, \star)$, $(G_3, \cdot)$ tre gruppi tali che $G_1 \isomorph G_2$ e $G_2 \isomorph G_3$.

    Allora $G_1 \isomorph G_3$.
\end{corollary}
\begin{proof}
    Dato che $G_1 \isomorph G_2$ e $G_2 \isomorph G_3$ dovranno esistere due isomorfismi $\phi : G_1 \to G_2$ e $\psi : G_2 \to G_3$.

    Per la proposizione \ref{prop:comp_omo} la funzione $\psi \circ \phi$ è ancora un isomorfismo; inoltre la composizione di funzioni bigettive è ancora bigettiva, da cui segue che $\psi \circ \phi$ è un isomorfismo tra $G_1$ e $G_3$ e quindi $G_1 \isomorph G_3$.
\end{proof}

Due gruppi isomorfi sono sostanzialmente lo stesso gruppo, a meno di "cambiamenti di forma". In particolare gli isomorfismi inducono naturalmente una bigezione sui sottogruppi dei due gruppi isomorfi, come ci dice la seguente proposizione.

\begin{proposition}
    [Bigezione tra i sottogruppi di gruppi isomorfi] \label{prop:big_sottogrp_isom}
    Siano $(G_1, *)$, $(G_2, \star)$ due gruppi e sia $\phi : G_1 \to G_2$ un isomorfismo.

    Siano inoltre $\HH$ e $\KK$ tali che \begin{align*}
        \HH = \set{H \suchthat H \leq G_1}, \quad \KK = \set{K \suchthat K \leq G_2}.
    \end{align*}

    Allora la funzione \begin{align*}
        f : \HH &\to \KK \\
        H &\mapsto \phi(H)
    \end{align*} è bigettiva.
\end{proposition}
\begin{proof}
    Siccome $H \leq G_1$ e $\phi$ è un omomorfismo, allora $f(H) = \phi(H) \leq G_2$ (ovvero $f(H) \in \KK$) per la proposizione \ref{prop:prop_omo:f(H)_sgr_cod}; dunque $f$ è ben definita.

    Definiamo ora una seconda funzione \begin{align*}
        g : \KK &\to \HH\\
        K &\mapsto \phi\inv(K).
    \end{align*} Anch'essa ben definita per la proposizione \ref{prop:prop_omo:f\inv(K)_sgr_dom}.

    Consideriamo ora le funzioni $g \circ f$ e $f \circ g$. Per la bigettività di $\phi$ vale che \begin{align*}
        &(g \circ f)(H) = \phi\inv(\phi(H)) = H &\forall H \in \HH\\
        &(f \circ g)(K) = \phi(\phi\inv(K)) = K &\forall K \in \KK
    \end{align*} ovvero la funzione $f$ è bigettiva e definisce quindi una bigezione tra l'insieme dei sottogruppi di $G_1$ e l'insieme dei sottogruppi di $G_2$.
\end{proof}

\begin{theorem}
    [Isomorfismi di gruppi ciclici]\label{th:iso_ciclico}
    Sia $(G, \cdot)$ un gruppo ciclico. Allora \begin{enumerate}[label={(\roman*)}, ref={\thetheorem: (\roman*)}]
        \item se $\abs G = +\infty$ segue che $G \isomorph \Z$;
        \item se $\abs G = n < +\infty$ segue che $G \isomorph \Zmod n$.
    \end{enumerate}
\end{theorem}
\begin{proof}
    Per ipotesi $G = \cycl*{g} = \set{g^k \suchthat k \in \Z}$ per qualche $g \in G$.
    \begin{enumerate}[label={(\roman*)}]
        \item Se $\abs G = +\infty$ allora $\abs*{\cycl*{g}} = +\infty$, ovvero per ogni $k, h \in \Z$ con $k \neq h$ segue che $g^k \neq g^h$. Sia allora \begin{align*}
            \phi : \Z &\to G\\
            k &\mapsto g^k.
        \end{align*}

        Per definizione di $G = \cycl{g}$ questa funzione è surgettiva. Dato che $G$ ha ordine infinito segue che questa funzione è iniettiva. Mostriamo che è un omomorfismo. \[
            \phi(k + h) = g^{k + h} = g^kg^h = \phi(k)\phi(h).    
        \]

        Dunque $\phi$ è un isomorfismo e $G \isomorph \Z$.
        \item Dato che $\abs{G} = n$ per la proposizione \ref{prop:sgr_generato} sappiamo che $\ord g = n$, ovvero che $g^n = e_G$. Sia allora \begin{align*}
            \phi : \Zmod n &\to G\\
            \eqclass a &\mapsto g^a
        \end{align*} dove $a$ è un generico rappresentante della classe $\eqclass a \in \Zmod n$. \begin{itemize}
            \item Mostriamo che $\phi$ è ben definita. Siano $a, b \in \eqclass a$ e mostriamo che $\phi(\eqclass a) = \phi(\eqclass b)$, ovvero che $g^a = g^b$. 
            
            Per ipotesi $a \congr b \Mod{n}$, ovvero $a = b+nk$ per qualche $k \in \Z$. Dunque \[
                g^a = g^{b + nk} = g^b(g^n)^k = g^b    
            \] poiché $g^n = e_G$.
            \item Mostriamo che $\phi$ è un omomorfismo. \[
                \phi(\eqclass a + \eqclass b) = g^{a + b} = g^ag^b = \phi(\eqclass a)\phi(\eqclass b).
            \] \item Mostriamo che $\phi$ è surgettiva. \[
                \Im \phi = \phi(\Zmod n) = \set{g^0, g^1, \dots, g^n} = \cycl g = G.    
            \]
        \end{itemize}
        Ma $\abs*{\Zmod n} = \abs*{G}$, dunque per cardinalità $\phi$ è anche iniettiva e dunque è bigettiva. 
        Quindi $\phi$ è un isomorfismo e $G \isomorph \Zmod n$.
    \end{enumerate}
\end{proof}

\begin{corollary}
    [Sottogruppi del gruppo ciclico] \label{cor:sgr_gruppo_ciclico}
    Sia $(G, \cdot)$ un gruppo ciclico.
    \begin{enumerate}[label={(\roman*)}]
        \item Se $G$ è infinito e $H \leq G$ allora segue che $H = \cycl*{g^n}$ per qualche $g \in G$, $n \in \Z$.
        \item Se $G$ ha ordine $n$ finito, allora $G$ ammette uno e un solo sottogruppo per ogni divisore di $n$.
        Inoltre se $H \leq G$ allora $H$ è ciclico.
    \end{enumerate}
\end{corollary}
\begin{proof}
    Ricordiamo che \begin{enumerate}
        \item i sottogruppi di $\Z$ sono tutti e soli della forma $n\Z$ al variare di $n \in \N$ per la \autoref{prop:sgr_Z},
        \item i sottogruppi di $\Zmod{n}$ hanno tutti cardinalità che divide $n$ per la \autoref{prop:sgr_Z/nZ:ciclico_ord_d}. Inoltre, per ogni $d$ che divide $n$ vi è uno e un solo sottogruppo di $\Zmod{n}$ di cardinalità $d$, per la \autoref{prop:sgr_Z/nZ:unosolo_ord_d}.
        \item per la \autoref{prop:big_sottogrp_isom} sappiamo che se $f : G_1 \to G_2$ è un isomorfismo, allora \[
            \set{K \suchthat K \leq G_2} = \set{f(H) \suchthat H \leq G_1}. 
        \]
    \end{enumerate}

    Mostriamo le due affermazioni separatamente.
    \begin{enumerate}[label={(\roman*)}]
        \item Se $G$ è ciclico ed infinito allora per il \autoref{th:iso_ciclico} segue che esiste un isomorfismo \begin{align*}
            \phi : \Z &\to G \\
            k &\mapsto g^k.
        \end{align*}

        Per la bigezione tra i sottogruppi di $\Z$ e $G$ allora ogni sottogruppo di $G$ dovrà essere scritto come immagine di qualche sottogruppo di $\Z$, ma come abbiamo osservato sopra i sottogruppi di $\Z$ sono tutti e solo della forma $n\Z$ per qualche $n \in \N$.
        
        Dunque i sottogruppi di $G$ sono \[
            \set{K \suchthat K \leq G} = \set{\phi(n\Z) = \cycl*{g^n} \suchthat n \in \N}.    
        \]
        \item Se $G$ è ciclico ed è finito, allora $G = \cycl*{g}$ per qualche $g \in G$, e inoltre $\abs*{G} = \ord{g} = n$ per qualche $n$ finito.
        
        Allora per il \autoref{th:iso_ciclico} esiste un isomorfismo \begin{align*}
            \psi : \Zmod n &\to G\\
            \eqclass a &\mapsto g^a.
        \end{align*}

        Per l'osservazione 2) sopra i sottogruppi di $\Zmod n$ sono tutti e solo della forma $\displaystyle \cycl*{\eqclass{d}}$, dunque per l'osservazione 3) segue che \[
            \set{K \suchthat K \leq G} = \set{
                \psi(\cycl*{\eqclass{d}}) 
            = \cycl*{g^{d}} \suchthat d \divides n}. \qedhere   
        \] 
    \end{enumerate}
\end{proof}

\begin{definition}[Automorfismo]
    Sia $(G, \cdot)$ un gruppo e sia $\phi : G \to G$ un isomorfismo. Allora $\phi$ viene detto \emph{automorfismo} e l'insieme di tutti gli automorfismi di un gruppo $G$ si denota con $\Aut{G}$.
\end{definition}

\begin{proposition}[Gruppo degli automorfismi]
    Sia $(G, \cdot)$ un gruppo. Allora la struttura $(\Aut{G}, \circ)$ (dove $\circ$ è la composizione di funzioni) è un gruppo.
\end{proposition}
\begin{proof}
    Mostriamo che valgono gli assiomi di gruppo.
    \begin{description}
        \item[Chiusura] La composizione di funzioni è un'operazione su $\Aut{G}$ in quanto la composizione di due omomorfismi è un omomorfismo (per la \autoref{prop:comp_omo}) e la composizione di due funzioni bigettive è ancora bigettiva, dunque la composizione di due automorfismi è ancora un automorfismo.
        \item[Associatività] La composizione di funzioni è associativa.
        \item[Elemento neutro] L'elemento neutro di $\Aut{G}$ è \begin{align*}
            \id_G : G &\to G \\
            g &\mapsto g.
        \end{align*}  Infatti $\id_G$ è un automorfismo di $G$ e inoltre per ogni $f \in \Aut{G}$ vale che\[
            \id_G \circ f = f = f \circ \id_G.
        \]
        \item[Invertibilità] Le funzioni in $\Aut{G}$ sono bigettive, dunque invertibili, e le loro inverse sono ancora automorfismi.
    \end{description}
    Dunque $(\Aut{G}, \circ)$ è un gruppo.
\end{proof}

\subsection{Omomorfismi di gruppi ciclici}

Studiamo ora gli insiemi $\Hom{G_1}{G_2}$ dove $G_1$ e $G_2$ sono gruppi ciclici. Per il \autoref{th:iso_ciclico} è sufficiente studiare gli omomorfismi tra i gruppi $\Z$ e $\Zmod{n}$ (con $n \in \N$ qualunque).

\paragraph{Omomorfismi con dominio $\Z$} Consideriamo l'insieme $\Hom{\Z}{G}$ dove $(G, \cdot)$ è un gruppo ciclico qualunque (quindi può essere isomorfo a $\Z$ oppure a $\Zmod n$ per qualche $n \in \N$). 

Sia $g \deq f(1)$. Allora possiamo mostrare per induzione che $f(n) = g^n$ per ogni $n \geq 0$. Per i negativi siccome $f$ è un omomorfismo vale che \[
    f(-n) = f(n)\inv = (g^n)\inv = g^{-n},    
\] da cui segue che gli omomorfismi $\Z \to G$ sono tutti della forma \[
    f(k) = g^k \quad \forall k \in \Z
\] e sono tutti identificati univocamente dal valore di $f(1)$.

Viceversa, per ogni $g \in G$ esiste un omomorfismo \begin{align*}
    \phi_g : \Z &\to G \\
    k &\mapsto g^k.
\end{align*} Questa funzione è un omomorfismo poiché \[
    \phi_g(k_1 + k_2) = g^{k_1 + k_2} = g^{k_1}g^{k_2} = \phi_g(k_1)\phi_g(k_2).
\]

Vi è dunque una bigezione tra $\Hom{\Z}{G}$ e $G$, data dalle due mappe \begin{align*}
    \Hom{\Z}{G} &\leftrightarrow G\\
    f &\mapsto f(1)\\
    \phi_g &\mapsfrom g.
\end{align*}

\subsection{Prodotto diretto di gruppi}

\begin{definition}
    Siano $(G_1, *)$, $(G_2, \star)$ due gruppi. Consideriamo il loro prodotto cartesiano \[
        G_1 \times G_2 = \set{(g_1, g_2) \suchthat g_1 \in G_1, g_2 \in G_2}    
    \] e un'operazione $\cdot$ su $G_1 \times G_2$ tale che \begin{align*}
    \cdot : (G_1 \times G_2) \times (G_1 \times G_2) &\to (G_1 \times G_2)\\
    ((x, y), (z, w)) &\mapsto (x * z, y \star w).
    \end{align*}
    
    La struttura $(G_1 \times G_2, \cdot)$ si dice \emph{prodotto diretto dei gruppi $G_1$ e $G_2$}.
\end{definition}

\begin{proposition}
    [Il prodotto diretto di gruppi è un gruppo]
    Siano $(G_1, *)$, $(G_2, \star)$ due gruppi. Allora il prodotto diretto $(G_1 \times G_2, \cdot)$ è un gruppo. 
\end{proposition}
\begin{proof}
    Sappiamo già che $\cdot$ è un'operazione su $G_1 \times G_2$, quindi basta mostrare i tre assiomi di gruppo.
    \begin{description}
        \item[Associatività] Siano $(x, y), (z, w), (h, k) \in G_1 \times G_2$. Mostriamo che vale la proprietà associativa.
        \begin{align*}
            &(x, y) \cdot ((z, w) \cdot (h, k)) \tag{def. di $\cdot$}\\
            &=\ (x, y) \cdot (z * h, w \star k)\tag{def. di $\cdot$}\\
            &=\ (x * (z * h), y \star (w \star k))\tag{ass. di $*$ e $\star$}\\
            &=\ ((x * z) * h, (y \star w) \star k) \\
            &=\ (x * z, y \star w) \cdot (h, k) \\
            &=\ ((x, y) \cdot (z, w)) \cdot (h, k).
        \end{align*}  
        \item[Elemento neutro] Siano $e_1 \in G_1, e_2 \in G_2$ gli elementi neutri dei due gruppi. Mostro che $(e_1, e_2)$ è l'elemento neutro del prodotto diretto.
        
        Sia $(x, y) \in G_1 \times G_2$ qualsiasi. Allora \begin{align*}
            &(x, y) \cdot (e_1, e_2) = (x * e_1, y \star e_2) = (x, y)\\
            &(e_1, e_2)\cdot (x, y)  = (e_1 * x, e_2 \star y) = (x, y).
        \end{align*}
        \item[Invertibilità] Sia $(x, y) \in G_1 \times G_2$. Mostriamo che $(x, y)$ è invertibile e il suo inverso è $(x\inv, y\inv) \in G_1 \times G_2$, dove $x\inv$ è l'inverso di $x$ in $G_1$ e $y\inv$ è l'inverso di $y$ in $G_2$.
        \begin{align*}
            &(x, y) \cdot (x\inv, y\inv) = (x * x\inv, y \star y\inv) = (e_1, e_2)\\
            &(x\inv, y\inv)\cdot (x, y)  = (x\inv * x, y\inv \star y) = (e_1, e_2).
        \end{align*} 
    \end{description}
    Dunque il prodotto diretto $(G_1 \times G_2, \cdot)$ è un gruppo.
\end{proof}

\begin{proposition}
    [Il centro del prodotto diretto è il prodotto diretto dei centri]
    Siano $(G_1, *)$, $(G_2, \star)$ due gruppi e sia $(G_1 \times G_2, \cdot)$ il loro prodotto diretto.
    Allora vale che \[
        Z(G_1 \times G_2) = Z(G_1) \times Z(G_2).
    \]
\end{proposition}
\begin{proof}
    Per definizione di centro sappiamo che \begin{multline*}
        Z(G_1 \times G_2) = \left\{\;(x, y) \in G_1 \times G_2 \suchthat \right. \\
        \left. (g_1, g_2) \cdot (x, y) = (x, y) \cdot (g_1, g_2) \quad \forall(g_1,g_2) \in G_1 \times G_2\;\right\}.  
    \end{multline*}
    Sia $(x, y) \in Z(G_1 \times G_2)$. Allora per ogni $ (g_1,g_2) \in G_1 \times G_2$ vale che \begin{align*}
        &(g_1, g_2) \cdot (x, y) = (x, y) \cdot (g_1, g_2) \\
        \iff &(g_1 * x, g_2 \star y) = (x * g_1, y \star g_2)\\
        \iff &g_1 * x = x*g_1 \text{ e } g_2 \star y = y \star g_2\\
        \iff &x \in Z(G_1) \text{ e } y \in Z(G_2)\\
        \iff &(x, y) \in Z(G_1) \times Z(G_2).
    \end{align*}
    Seguendo la catena di equivalenze al contrario segue la tesi.
\end{proof}

\begin{proposition}
    [Ordine nel prodotto diretto]
    \label{prop:ord_prod_diretto}
    Siano $(G_1, *)$, $(G_2, \star)$ due gruppi e sia $(G_1 \times G_2, \cdot)$ il loro prodotto diretto. Sia $(x, y) \in G_1 \times G_2$. Allora vale che \[
        \ord[G_1 \times G_2]{(x, y)} = \mcm{\ord[G_1]{x}}{\ord[G_2]{y}}.    
    \]
\end{proposition}
\begin{proof}
    Sia $n = \ord{x}$, $m = \ord y$ e $d = \ord{(x, y)}$. Mostriamo che $d = \mcm{n}{m}$.
    \begin{description}
        \item[$d \divides \mcm{n}{m}$] Vale che \[
            (x, y)^{\mcm{n}{m}} = (x^{\mcm{n}{m}}, y^{\mcm{n}{m}}).   
        \] Siccome $\ord x = n \divides \mcm{n}{m}$ e stessa cosa per $\ord y = m$, per la Proposizione \ref{prop:sgr_generato:ord_div_n} segue che \begin{equation*}
            (x^{\mcm{n}{m}}, y^{\mcm{n}{m}}) = (e_1, e_2)
        \end{equation*}
        da cui (per la Proposizione \ref{prop:sgr_generato:ord_div_n}) segue che $d \divides \mcm{n}{m}$.
        \item[$\mcm{n}{m} \divides d$] Per definizione di potenza intera nel prodotto diretto sappiamo che $(x, y)^d = (x^d, y^d)$. Inoltre dato che $d$ è l'ordine di $(x, y)$ segue che $(x, y)^d = (e_1, e_2)$. Dunque \begin{align*}
            &x^d = e_1, \; y^d = e_2\\
            \iff &n \divides d, \; m \divides d \\
            \iff &\mcm{n}{m} \divides d.
        \end{align*}
    \end{description}
    Dunque $d = \mcm{n}{m}$, ovvero la tesi.
\end{proof}

\begin{theorem}
    [Teorema Cinese del Resto (III forma)] \label{th:cinese_III}
    Siano $n, m \in \Z$ entrambi non nulli. Allora vale che \[
        \Zmod{nm} \isomorph \Zmod{n}\times \Zmod{m} \iff \mcd{n}{m} = 1.
    \]
\end{theorem}
\begin{proof}
    Sia $G = \Zmod n \times \Zmod m$. Siccome $\abs*{G} = nm$ in virtù del \autoref{th:iso_ciclico} per mostrare che $G \isomorph \Zmod{nm}$ è sufficiente mostrare che $G$ è ciclico.

    Un gruppo è ciclico se e solo se esiste $g \in G$ tale che $\ord{g} = \abs G$: infatti per ogni $g \in G$ vale che $\cycl*{g} \leq G$, dunque se i due insiemi hanno anche la stessa cardinalità devono essere uguali.

    Siano $\eqclass x \in \Zmod n, \eqclass y \in \Zmod m$ tali che $g = (\eqclass x, \eqclass y)$. Per la \autoref{prop:ord_prod_diretto} vale che \[
        \ord{g} = \ord{(\eqclass x, \eqclass y)} = \mcm{\ord{\eqclass x}}{\ord{\eqclass y}}.    
    \]
    D'altro canto però $\ord{\eqclass x} = \dfrac{n}{\mcd{n}{x}}$, $\ord{\eqclass y} = \dfrac{m}{\mcd{m}{y}}$ (dove $x, y$ sono rappresentanti qualsiasi delle classi $\eqclass x$, $\eqclass y$ rispettivamente), dunque \[
        \ord g = \mcm{\dfrac{n}{\mcd{n}{x}}}{\dfrac{m}{\mcd{m}{y}}} \leq \mcm{n}{m}. 
    \]

    Possiamo dunque distinguere i due casi: \begin{enumerate}
        \item se $\mcd{n}{m} = d > 1$ allora per la PROPOSIZIONE DA INSERIRE per ogni $g \in G$ vale che \[
            \ord g \leq \mcm{n}{m} = \frac{mn}{d} < mn    
        \] da cui segue che $G$ non può essere ciclico;
        \item se $\mcd{n}{m} = 1$ allora per ogni $g \in G$ vale che \[
            \ord g \leq \mcm{n}{m} = mn.    
        \] In particolare se consideriamo $g = (\eqclass 1, \eqclass 1)$ si ha che \[
            \ord(\eqclass 1, \eqclass 1) = \mcm{\frac{n}{\mcd{n}{1}}}{\frac{m}{\mcd{m}{1}}} = \mcm{m}{n} = mn  
        \], dunque $G = \cycl*{(\eqclass 1, \eqclass 1)}$, da cui segue che \[
            G \isomorph \Zmod{nm}    
        \] per il \autoref{th:iso_ciclico}. \qedhere
    \end{enumerate} 
\end{proof}

\begin{remark}
    Per il Teorema Cinese del Resto (II Forma) sappiamo che la funzione 
    \begin{align} \label{eq:iso_Znm->Zn_x_Zm}
        \begin{split}
            \phi : \Zmod{nm} &\to \Zmod n \times \Zmod m\\
            [a]_{nm} &\mapsto ([a]_n, [a]_m)
        \end{split}
    \end{align} è bigettiva. Inoltre \begin{align*}
        \phi([a]_{nm} + [b]_{nm}) &= \phi([a+b]_{nm})\\
        &= ([a+b]_n, [a+b]_m)\\
        &= ([a]_n + [b]_n, [a]_m + [b]_m)\\
        &= ([a]_n, [a]_m) + ([b]_n, [b]_m)\\
        &= \phi([a]_{nm}) + \phi([b]_{nm}),
    \end{align*} ovvero $\phi$ è un omomorfismo di gruppi. Dunque $\phi$ è un isomorfismo di gruppi e \[
        \Zmod{nm} \isomorph \Zmod{n}\times \Zmod{m}.  
    \]
\end{remark}

\begin{corollary}
    [Isomorfismo tra i gruppi degli invertibili] Siano $n, m \in \Z$ entrambi non nulli. Allora se $\mcd{n}{m} = 1$ segue che \begin{equation}
        \invertible{\Zmod{nm}} \isomorph \invertible{\Zmod{n}} \times \invertible{\Zmod{m}}.
    \end{equation}
\end{corollary}
\begin{proof}
    Consideriamo la funzione \begin{align*}
        \phi^* : \invertible{\Zmod{nm}} &\to \invertible{\Zmod{n}} \times \invertible{\Zmod{m}}\\
        [a]_{nm} &\mapsto [a]_n \times [a]_m.
    \end{align*}
    Essa è ben definita: infatti se $[a]_{nm} \in \invertible{\Zmod{nm}}$ significa che $\mcd{a}{mn} = 1$. Siccome per ipotesi $\mcd{m}{n} = 1$ per la PROPOSIZIONE NON SCRITTA segue che $\mcd{m}{n} = \mcd{a}{m} = 1$, ovvero $[a]_n \in \invertible{\Zmod n}$ e $[a]_m \in \invertible{\Zmod m}$.

    Inoltre questa funzione è una restrizione della $\phi$ definita in \eqref{eq:iso_Znm->Zn_x_Zm}, dunque è iniettiva. Inoltre \[
        \abs*{\Zmod{nm}} = \phi(nm) = \phi(n)\phi(m) = \abs*{\Zmod{n} \times \Zmod{m}}    
    \] siccome $\mcd{n}{m} = 1$, dunque $\phi$ è anche surgettiva e quindi è bigettiva.

    Tramite passaggi analoghi a quelli visti nell'osservazione precedente si dimostra che $\phi^*$ è un omomorfismo, dunque essendo bigettiva è anche un isomorfismo di gruppi, da cui segue la tesi.
\end{proof}

\subsection{Prodotto di sottogruppi}

\begin{definition}
    Sia $(G, \cdot)$ un gruppo e siano $H,K \leq G$. Allora si definisce il \emph{prodotto tra $H$ e $K$} come \begin{equation}
        HK \deq \set{h\cdot k \suchthat h \in H, k \in K}.
    \end{equation} Analogamente si definisce il \emph{prodotto tra $K$ e $H$} come \begin{equation}
        KH \deq \set{k \cdot h \suchthat k \in K, h \in H}.
    \end{equation}
\end{definition}

\begin{remark}
    Se il gruppo è in notazione additiva il prodotto di sottogruppi diventa somma di sottogruppi e si indica $H + K$ (o $K + H$).
\end{remark}

\begin{proposition}\label{prop:cond_prod_sgr_e'_sgr}
    [Condizione per cui il prodotto tra sottogruppi è un sottogruppo] Sia $(G, \cdot)$ un gruppo e siano $H,K \leq G$.

    Allora l'insieme $HK$ è un sottogruppo di $G$ se e solo se $HK = KH$.
\end{proposition}
\begin{proof}
    Dimostriamo entrambi i versi dell'implicazione.
    \begin{description}
        \item[($\impliedby$)] Siccome entrambi gli insiemi contengono $e_G$, per la \autoref{prop:cond_sgr} mi basta mostrare che $HK$ è chiuso rispetto all'operazione $\cdot$ e che contiene l'inverso di ogni suo elemento. 
        \begin{description}
            \item[Chiusura] Siano $h_1k_1, h_2k_2 \in HK$. Voglio mostrare che $(h_1k_1) \cdot (h_2k_2) \in HK$. Per associatività, posso scriverlo come \[
                h_1 \cdot (k_1h_2) \cdot k_2.    
            \] Siccome $KH = HK$ esisteranno $h_3 \in H, k_3 \in K$ tali che $k_1h_2 = h_3k_3$. Da ciò segue che \[
                h_1 \cdot (k_1h_2) \cdot k_2 = h_1h_3k_3k_2 \in HK.
            \]
            \item[Invertibilità] Sia $hk \in HK$ e mostriamo che anche il suo inverso $(hk)\inv = k\inv h\inv$ è in $HK$. Siccome $k\inv h\inv \in KH$ e $KH = HK$, segue la tesi.
        \end{description}
        \item[($\implies$)] Dimostriamo che $HK = KH$ mostrando che $HK \subseteq KH$ e $KH \subseteq HK$.
        \begin{description}
            \item[($KH \subseteq HK$)] Banalmente $H \subseteq HK$ (infatti $H \ni h = he_G \in HK$) e $K \subseteq HK$. Ma allora per ogni $h, k \in HK$ segue che $k \cdot h \in HK$ (in quanto $HK \leq G$) dunque $KH \subseteq HK$.
            \item[($HK \subseteq KH$)] Consideriamo la funzione \begin{align*}
                f : HK &\to KH\\
                x &\mapsto x\inv.
            \end{align*} Questa funzione è ben definita, in quanto se $x \in HK$, ovvero se $x = hk$ per qualche $h \in H, k \in K$ allora \[
                x\inv = (hk)\inv = k\inv h\inv \in KH    
            \] poiché $k\inv \in K$ e $h\inv \in H$. Inoltre questa funzione è ovviamente iniettiva, da cui segue che $HK \subseteq KH$.
        \end{description}
    \end{description}
    Dunque $HK$ è sottogruppo se e solo se $HK = KH$.
\end{proof}
\section{Classi laterali e gruppo quoziente}

Sia $(G, \cdot)$ un gruppo e sia $H \leq G$. Consideriamo la seguente relazione sugli elementi di $G$: diciamo che $x \sim_L y$ se e solo se $y\inv x \in H$.

Questa relazione è una relazione di equivalenza, infatti \begin{itemize}
    \item $\sim_L$ è riflessiva: $x\inv x = e_G \in H$, dunque $x \sim_L x$.
    \item $\sim_L$ è simmetrica: se $x \sim_L y$, ovvero $y\inv x \in H$, allora il suo inverso $(y\inv x)\inv = x\inv (y\inv)\inv = x\inv y \in H$, dunque $y \sim_L x$.
    \item $\sim_L$ è transitiva: supponiamo che $x \sim_L y$ e $y \sim_L z$ e mostriamo che $x \sim_L z$. Dalla prima sappiamo che $y\inv x \in H$, mentre dalla seconda segue che $z\inv y \in H$. Dato che $H$ è un sottogruppo, il prodotto di suoi elementi è ancora in $H$, dunque \[
        z\inv y \cdot y\inv x = z\inv x \in H    
    \] da cui segue che $x \sim_L z$.
\end{itemize}

Questa relazione di equivalenza forma delle classi di equivalenza che partizionano $G$: in particolare la classe di $x \in G$ sarà della forma \begin{align*}
    \eqclass*{x}_L &= \set{g \in G \suchthat g \sim_L x}\\
    &= \set{g \in G \suchthat x\inv g \in H}\\
    &= \set{g \in G \suchthat x\inv g = h \text{ per qualche } h \in H}\\
    &= \set{g \in G \suchthat g = xh \text{ per qualche } h \in H}.
\end{align*}

Notiamo che gli elementi della classe di $x$ sono quindi tutti e soli gli elementi del sottogruppo $h$ moltiplicati a sinistra per $x$.
Diamo dunque la seguente definizione.
\begin{definition}
    [Classe laterale sinistra]
    Sia $(G, \cdot)$ un gruppo e $H \leq G$ un suo sottogruppo. Sia inoltre $x \in G$.
    
    Allora si dice \emph{classe laterale sinistra di $H$ rispetto a $x$} l'insieme \[
        xH \deq \set{xh \suchthat h \in H}. 
    \]
\end{definition}

\begin{remark}
    Nel caso di gruppi additivi le classi laterali si scrivono in notazione additiva, ovvero nella forma $x + H$ per $x \in G$, $H \leq G$.
\end{remark}

\begin{example}
    Ad esempio le classi laterali di $n\Z \leq \Z$ sono della forma \[
        a + n\Z \deq \set{a + nk \suchthat k \in \Z}.
    \] La classe $a + n\Z$ denota tutti i numeri congrui ad $a$ modulo $n$.
\end{example}

Allo stesso modo possiamo definire un'altra relazione di equivalenza $\sim_R$ tale che \[
    x \sim_R y \iff xy\inv \in H.    
\] Le classi di equivalenza di questa relazione sono della forma \[
    \eqclass*{x}_R = \set{g \in G \suchthat g = hx \text{ per qualche } h \in H}.    
\] Possiamo dunque definire anche le classi laterali destre nel seguente modo.
\begin{definition}
    [Classe laterale destra]
    Sia $(G, \cdot)$ un gruppo e $H \leq G$ un suo sottogruppo. Sia inoltre $x \in G$.
    
    Allora si dice \emph{classe laterale destra di $H$ rispetto a $x$} l'insieme \[
        Hx \deq \set{hx \suchthat h \in H}. 
    \]
\end{definition}

\begin{remark}
    Siccome le classi laterali sinistre (o destre) rappresentano le classi di equivalenza rispetto alla relazione $\sim_L$ (risp. $\sim_R$) possiamo definire un insieme di rappresentanti $R$ per cui \begin{equation}
        G = \bigdisjunion_{a \in R} aH. \quad \text{(risp. $Ha$)}
    \end{equation}
\end{remark}

\begin{theorem}
    [Teorema di Lagrange] \label{th:lagrange}
    Sia $(G, \cdot)$ un gruppo finito e sia $H \leq G$ qualsiasi. Allora vale che \[
        \abs*{H} \divides \abs*{G}.    
    \]
\end{theorem}

In breve, il Teorema di Lagrange afferma che per ogni gruppo finito l'ordine di un suo qualsiasi sottogruppo divide l'ordine del gruppo. Prima di dimostrarlo, dimostriamo un lemma che ci tornerà utile.

\begin{lemma}\label{lem:ord_classilat=ord_sgr}
    Sia $(G, \cdot)$ un gruppo e sia $H$ un suo sottogruppo. Allora per qualsiasi $g \in G$ vale che \[
        \abs*{gH} = \abs*{H} = \abs*{Hg}.
    \]
\end{lemma}
\begin{proof}
    Per dimostrare che $\abs*{gH} = \abs*{H}$ consideriamo la mappa \begin{align*}
        \phi : H &\to gH \\   
        h &\mapsto gh 
    \end{align*} e facciamo vedere che è bigettiva.
    \begin{description}
        \item[Iniettività] Supponiamo che per qualche $h, k \in H$ valga che $\phi(h) = \phi(k)$, ovvero $gh = gk$. Siccome $gh, gk \in G$ vale la \hyperref[prop:prop_grp:canc:sx]{legge di cancellazione sinistra}, dunque segue che $h = k$, ovvero $\phi$ è iniettiva.
        \item[Surgettività] Segue naturalmente dalla definizione di $gH$. 
    \end{description}
    Dunque $\phi$ è bigettiva e quindi gli insiemi $gH$ e $H$ hanno la stessa cardinalità. Analogamente si mostra che la funzione \begin{align*}
        \psi : H &\to Hh \\   
        h &\mapsto hg 
    \end{align*} è bigettiva, dunque segue la tesi.
\end{proof}

Dimostriamo ora il Teorema di Lagrange
\begin{proof}[Dimostrazione del \autoref{th:lagrange}]
    Per l'osservazione precendente sappiamo che se $R$ è un insieme di rappresentanti della relazione di equivalenza $\sim_L$ allora \[
        G = \bigdisjunion_{a \in R} aH,
    \] dunque passando alle cardinalità \begin{align*}
        \abs*{G} &= \sum_{a \in R} \abs*{aH}.  \\
        \intertext{Per il \autoref{lem:ord_classilat=ord_sgr} segue quindi che}  
        &= \sum_{a \in R} \abs*{H}\\
        &= \abs*{R}\cdot \abs*{H}. 
    \end{align*}  Dunque $\abs*{H} \divides \abs*{G}$, dunque la tesi.
\end{proof}

\begin{remark}
    Osserviamo che in generale le classi laterali di un sottogruppo del gruppo $G$ non sono sottogruppi di $G$: dato che partizionano il gruppo una sola di esse contiene l'elemento neutro del gruppo.
\end{remark}

\begin{proposition}\label{prop:cond_laterale_è_sgrp}
    Sia $(G, \cdot)$ un gruppo, sia $H \leq G$ e sia $g \in G$ qualsiasi. Allora i seguenti fatti sono equivalenti:
    \begin{enumerate}[label={(\roman*)}]
        \item $gH \leq G$,
        \item $g \in H$,
        \item $H = gH$.
    \end{enumerate}
\end{proposition}
\begin{proof}
    Dimostriamo la catena di implicazioni $(i) \implies (ii) \implies (iii) \implies (i)$.
    \begin{description}
        \item[($(i) \implies (ii)$)] Supponiamo che $gH \leq G$. Allora $e_G \in gH$, ovvero esiste $h \in H$ tale che $gh = e_G$. Ma tale $h$ è $g\inv$, dunque se $g\inv \in H$ segue che $g \in H$.
        \item[($(ii) \implies (iii)$)] Supponiamo che $g \in H$. 
        \begin{description}
            \item[($gH \subseteq H$)] Supponiamo $gh \in gH$ per qualche $h \in H$. Ma essendo $g \in H$ per ipotesi il prodotto $gh$ sarà un elemento di $H$, dunque $gH \subseteq H$.
            \item[($H \subseteq gH$)] Sia $h \in H$. Siccome $g \in H$ e $H$ è un gruppo segue che $g\inv \in H$, dunque $g\inv h \in H$. Ma questo significa che $g\cdot(g\inv h) = h \in gH$, dunque $H \subseteq gH$.
        \end{description}
        Concludiamo che $gH = H$.
        \item[($(iii) \implies (i)$)] Siccome $gH = H$ e $H \leq G$ allora $gH \leq G$. \qedhere 
    \end{description}
\end{proof}

Siccome ogni elemento di una classe è un possibile rappresentante della classe stessa, la proposizione precedente ci dice che l'unica classe laterale (sinisra) di $H$ che è un sottogruppo di $G$ è quella che contiene l'identità, ovvero la classe $e_GH = H$.

\begin{corollary}[Corollario al Teorema di Lagrange] \label{cor:lagrange}
    Sia $(G, \cdot)$ un gruppo finito. Allora valgono i seguenti fatti:
    \begin{enumerate}[label={(\roman*)}, ref={\thecorollary: (\roman*)}]
        \item \label{cor:ord_el_divide_ord_gruppo} per ogni $g \in G$ vale che $\ord[G]{g} \divides \abs*{G}$,
        \item \label{cor:x_alla_ordG=e_G} per ogni $x \in G$ vale che $x^{\abs*{G}} = e_G$.
    \end{enumerate}
\end{corollary}
\begin{proof}
    \begin{enumerate}[label={(\roman*)}]
        \item Siccome $\cycl*{g} \leq G$, per il \nameref{th:lagrange} vale che \[
            \ord[G]{g} = \abs*{\cycl*{g}} \divides \abs*{G}.    
        \]
        \item Sia $n \deq \abs*{G}$ e $k \deq \ord[G]{g}$. Per il punto precedente vale che $k \divides n$, ovvero che esiste $m \in \Z$ tale che \[
            n = km.    
        \] Dunque segue che \begin{align*}
            g^{\abs*{G}} &= g^n \\
            &= (g^k)^m \tag{per def. di \hyperref[def:ord_grp]{ordine}}\\
            &= e^m \\
            &= e. \tag*{\qedhere}
        \end{align*}
    \end{enumerate}
\end{proof}

\begin{corollary}[I gruppi di ordine primo sono ciclici]
    Sia $(G, \cdot)$ un gruppo tale che $\abs*{G} = p$ per qualche $p \in \Z$, $p$ primo. Allora $G$ è ciclico ed in particolare \[
        G \isomorph \Zmod{p}.    
    \]
\end{corollary}
\begin{proof}
    Sia $x \in G$, $x \neq e_G$. Allora $\cycl*{x} \neq \set{e_G}$, da cui segue che \[
        1 \neq \ord[G]{x} \divides p = \abs*{G}.
    \] Dunque per definizione di numero primo $\ord[G]{x} = p$, ma siccome l'ordine del sottogruppo $\cycl*{x}$ è uguale all'ordine di $G$ segue che $G = \cycl*{x}$.
    
    Dunque $G$ è ciclico e per il \autoref{th:iso_ciclico} è isomorfo a $\Zmod{p}$.
\end{proof}

Il teorema di Lagrange ci consente inoltre di dimostrare molto semplicemente il Teorema di Eulero-Fermat.
\begin{proof}
    Segue dal \autoref{cor:lagrange} (in particolare dal \hyperref[cor:x_alla_ordG=e_G]{punto (ii)}) considerando come gruppo $(\invertible{\Zmod{n}}, \cdot)$: infatti per definizione $\phi(n) = \abs*{\invertible{\Zmod{n}}}$, da cui la tesi.
\end{proof}

\subsection{Sottogruppi normali e gruppo quoziente}

\begin{definition}
    [Sottogruppo normale] \label{def:sgr_normale}
    Sia $(G, \cdot)$ un gruppo e sia $H \leq G$. Allora si dice che $H$ è un \emph{sottogruppo normale} di $G$ se per ogni $g \in G$ vale che \begin{equation} \label{eq:def_normale}
        gH = Hg.
    \end{equation} 
    
    Se $H$ è normale si scrive $H \normal G$.
\end{definition}

\begin{remark}
    Se $G$ è abeliano allora tutti i suoi sottogruppi sono normali.
\end{remark}
\begin{remark}
    Se un sottogruppo $H$ è normale non significa che per ogni $h \in H$ vale che $gh = hg$, ma soltanto che per ogni $h \in H$ esiste un $h^\prime \in H$ tale che \[
        gh = h^\prime g.    
    \]
\end{remark}

\begin{proposition} \label{prop:normale_sse_chiuso_per_coniugio}
    Sia $(G, \cdot)$ un gruppo e $H \leq G$.
    Allora $H$ è normale se e solo se è chiuso per coniugio, ovvero se e solo se per ogni $g \in G$ vale che \[
        gHg\inv \subseteq H.    
    \]
\end{proposition}
\begin{proof}
    Mostriamo entrambi i versi dell'implicazione.
    \begin{description}
        \item[($\implies$)] Supponiamo che $H \normal G$, ovvero che per ogni $g \in G$ vale che \[
            gH = Hg,
        \] ovvero per ogni $h \in H$ esiste un $h^\prime \in H$ tale che \[
            gh = h^\prime g.    
        \] Moltiplicando a destra per $g\inv$ si ottiene che \[
            ghg\inv = h^\prime \in H,   
        \] da cui $gHg\inv \subseteq H$.
        % \item[($\impliedby$)] Supponiamo che $ghg\inv \in H$, ovvero esista $h^\prime$ tale che $ghg\inv = h^\prime$, il che è equivalente ad affermare $gh = h^\prime g \in Hg$. Questo significa che $gH \subseteq Hg$. Mostriamo ora che vale anche l'inclusione contraria.
        
        % Dato che la relazione deve valere per quasliasi $g$, dovrà valere anche per $g\inv \in G$
    \end{description}
\end{proof}

\begin{definition}
    [Indice di un sottogruppo]
    Sia $(G, \cdot)$ un gruppo e sia $H \leq G$. Allora si dice \emph{indice di $H$ in $G$} il numero di classi laterali sinistre di $H$, e si indica con \begin{equation}
        \grindex{G}{H}.
    \end{equation}
\end{definition}

\begin{proposition}
    Sia $(G, \cdot)$ un gruppo, $H \leq G$. Allora se $\grindex{G}{H} = 2$ segue che $H \normal G$.
\end{proposition}

\begin{proposition}
    [Nucleo di omomorfismi e normalità]
    \label{prop:rel_kernel_sgr_normali}
    Siano $(G, \cdot)$, $(G^\prime, *)$ due gruppi e sia $f : G \to G^\prime$ un omomorfismo. 
    
    Valgono le seguenti affermazioni.
    \begin{enumerate}[label={(\roman*)}]
        \item $\ker f \normal G$,
        \item per ogni $x, y \in G$ vale che $f(x) = f(y)$ se e solo se $x\ker f = y\ker f$, ovvero se $x, y$ appartengono alla stessa classe laterale del nucleo,
        \item se $z \in \Imm f$ (ovvero $f(x) = z$ per qualche $x \in G$) allora $f\inv(\set{z}) = x\ker f$.
    \end{enumerate}
\end{proposition}
\begin{proof}
    \begin{enumerate}[label={(\roman*)}]
        \item Per la \autoref{prop:normale_sse_chiuso_per_coniugio} la tesi è equivalente a dimostrare che \[
            g(\ker f) g\inv \subseteq \ker f   
        \] per ogni $g \in G$.
        
        Sia $x \in \ker f$ qualsiasi: mostriamo che $gxg\inv \in \ker f$. Per definizione di kernel, questo significa mostrare che $f(gxg\inv) = e_G$, ovvero (siccome $f$ è un omomorfismo) \[
            f(g) * f(x) * f(g\inv) = e_G.    
        \] Per ipotesi $x \in \ker f$, dunque $f(x) = e_G$; inoltre per la \hyperref[prop:prop_omo:inv_passa_dentro]{Proposizione \ref*{prop:prop_omo:inv_passa_dentro}} sappiamo che $f(g\inv) = f(g)\inv$.
        
        Dunque segue che \begin{align*}
            f(g) * f(x) * f(g\inv) &= f(g) * e_G * f(g)\inv\\
            &= f(g) * f(g)\inv \\
            &= e_G
        \end{align*} che è la tesi.
        \item Supponiamo $f(x) = f(y)$. Moltiplicando a destra per $f(y)\inv$ segue che \begin{align*}
            &f(x) * f(y)\inv = e_G \\
            \iff &f(x) * f(y\inv) = e_G \\
            \iff &f(xy\inv) = e_g\\
            \iff &xy\inv \in \ker f\\
            \iff &x \sim_L y.
        \end{align*}
        
        Dunque le classi di equivalenza di $x$ e $y$ sono uguali, ovvero \[
            x\ker f = y\ker f.    
        \]
        \item Per definizione di controimmagine: \begin{align*}
            f\inv(z) &= \set{g \in G \suchthat f(g) = z} \tag{hp: $f(x) = z$} \\
            &= \set{g \in G \suchthat f(g) = f(x)} \tag{per il punto (ii)}\\
            &= x\ker f. \tag*{\qedhere}
        \end{align*}
    \end{enumerate}
\end{proof}

Consideriamo ora l'insieme di tutte le possibili classi laterali sinistre di un sottogruppo $H \leq G$ e chiamiamo questo insieme $\quot{G}{H}$: \begin{equation}
    \quot{G}{H} \deq \set{gH \suchthat g \in G}.
\end{equation}

Se $H \normal G$ possiamo definire un'operazione su $\quot{G}{H}$: \begin{align} \label{eq:op_gruppo_quoziente}
    \begin{split}
        \cdot : \quot{G}{H} \times \quot{G}{H} &\to \quot{G}{H} \\
        (aH, bH) &\mapsto abH.
    \end{split}
\end{align}

La struttura $(\quot{G}{H}, \cdot)$ si definisce \emph{gruppo quoziente}.

\begin{proposition}
    Sia $(G, \cdot)$ un gruppo e sia $N \normal G$. Allora la struttura $(\quot{G}{N}, \star)$ (dove l'operazione è definita come in \eqref{eq:op_gruppo_quoziente}) è un gruppo.
\end{proposition}
\begin{proof}
    Mostriamo innanzitutto che l'operazione $*$ è ben definita.
    Supponiamo che $xN = x^\prime N$ e $yN = y^\prime N$ e mostriamo che $xyN = x^\prime y^\prime N$.

    Siano $n_1, n_2$ tali che \[
        x^\prime = xn_1, \quad y^\prime = yn_2.    
    \] Allora vale che \begin{align*}
        x^\prime y^\prime &= xn_1 yn_2.\\
        \intertext{Siccome $N \normal G$ segue che $Ny = yN$, ovvero che esiste un $n_3 \in N$ tale che $n_1y = yn_3$. Dunque}
        &= xy n_3n_2 \tag{$N$ è chiuso rispetto a $\cdot$}\\
        &\in xyN.
    \end{align*}
    Per simmetria dunque $xyN = x^\prime y^\prime N$.

    Mostriamo ora che valgono gli assiomi di gruppo.
    \begin{description}
        \item[Associatività] Siano $xN, yN, zN \in \quot{G}{N}$. Mostriamo che vale la proprietà associativa.
        \begin{align*}
            xN \star (yN \star zN) &= xN \star yzN \\
            &= x(yz)N \tag{ass. in $G$}\\
            &= (xy)zN \\
            &= xyN \star zN \\
            &= (xN \star yN) \star zN.
        \end{align*}
        \item[Elemento neutro] L'elemento neutro del gruppo è $e_GN$. Infatti per qualsiasi $xN \in \quot{G}{N}$ \begin{align*}
            &e_GN \star xN = e_GxN = xN.\\
            &xN \star e_GN = xe_GN = xN.
        \end{align*}
        \item[Invertibilità] Sia $xN \in \quot{G}{N}$. Mostriamo che il suo inverso rispetto a $\star$ è $x\inv N$.
        \begin{align*}
            &xN \star x\inv N = xx\inv N = e_GN.\\
            &x\inv N \star xN = x\inv xN = e_GN.
        \end{align*} 
    \end{description}
    Dunque $(\quot{G}{N}, \star)$ è un gruppo.
\end{proof}

\begin{example}
    Se consideriamo il gruppo $\Z$ e il suo sottogruppo normale $n\Z$ il gruppo quoziente $\Zmod{n}$ è esattamente il gruppo delle classi resto modulo $n$.
\end{example}

\begin{proposition}\label{prop:kernel_proiezione}
    Sia $(G, \cdot)$ un gruppo e sia $N \normal G$. Allora la mappa \begin{align}
        \begin{split}
            \pi_N : G &\to \quot{G}{N}\\
            x &\mapsto xN
        \end{split}
    \end{align}
    è un omomorfismo di gruppi e $\ker \pi_N = N$.
\end{proposition}
\begin{proof}
    Mostriamo innanzitutto che $\pi_N$ è un omomorfismo. \begin{align*}
        \pi_N(xy) &= xyN \\
        &= xN \cdot yN\\
        &= \pi_N(x) \cdot \pi_N(y).
    \end{align*}
    Inoltre per definizione \begin{align*}
        \ker \pi_N &= \set{x \in G \suchthat \pi_N(x) = xN = N} \\
        &= \set{x \in G \suchthat x \in N}\\
        &= N,  
    \end{align*} dove il secondo segno di uguaglianza viene dalla \autoref{prop:cond_laterale_è_sgrp} (in particolare per l'equivalenza tra i punti (ii) e (iii)).
\end{proof}

\begin{corollary}
    I sottogruppi normali di $G$ sono tutti e solo i nuclei degli omomorfismi definiti su $G$.
\end{corollary}
\begin{proof}
    Infatti se $N \normal G$ allora per la \autoref{prop:kernel_proiezione} segue che $N = \ker \pi_N$; invece dato un omomorfismo di gruppi $\phi : G \to G^\prime$ vale che $\ker \phi$ è normale per la \autoref{prop:rel_kernel_sgr_normali}.
\end{proof}
\section{Teoremi di Omomorfismo}

\subsection{Primo Teorema degli Omomorfismi}
\begin{theorem}
    [Primo Teorema degli Omomorfismi] \label{th:first_iso}
    Siano $(G, \cdot)$, $(G^\prime, *)$ due gruppi e sia $f : G \to G^\prime$ un omomorfismo di gruppi. Sia inoltre $N \normal G$, $N \subseteq \ker f$.

    Allora esiste un unico omomorfismo $\phi : \quot{G}{N} \to G^\prime$ per cui il seguente diagramma commuta:
    \begin{equation}
        \begin{tikzcd}
            G \arrow[d, swap, "\pi_N"] \arrow[r, "f"] & G^\prime \\
            \quot{G}{N} \arrow[ur, swap, "\phi"] &
        \end{tikzcd}
    \end{equation}
    Inoltre vale che \begin{align*}
        \Imm{f} = \Imm{\phi}, \quad \ker \phi = \quot{\ker f}{N}.
    \end{align*}
\end{theorem}
\begin{proof}
    Notiamo che se $\phi$ esiste allora è necessariamente unica. Infatti se $\phi$ rende il diagramma commutativo significa che $f = \phi \circ \pi_N$, da cui segue che per ogni $x \in G$ \begin{align*}
        f(x) &= (\phi \circ \pi_N)(x) \\
        &= \phi(\pi_N(x))\\
        &= \phi(xN).
    \end{align*}
    Questa equazione assegna a $\phi$ un valore per ogni elemento del dominio $\quot{G}{N}$, da cui segue l'unicità.

    Mostriamo dunque che la funzione \begin{align*}
        \phi : \quot{G}{N} &\to G^\prime \\
        gH &\mapsto f(g)
    \end{align*} è ben definita ed è un omomorfismo di gruppi. Inoltre verifichiamo le due proprietà dell'immagine e del nucleo.
    \begin{description}
        \item[Buona definizione] Siano $x, y$ tali che $xN = yN$. Dato che esse rappresentano classi di equivalenza, ciò significa che $x \in yN$. 
        
        Sia dunque $n \in N$ tale che $x = yn$. Allora vale che \begin{align*}
            f(x) &= f(yn) \tag{$f$ è omo.}\\
            &= f(y) * f(n) \tag{$N \subseteq \ker f$}\\
            &= f(y) * e^\prime \\
            &= f(y).
        \end{align*} Dunque segue che \begin{align*}
            \phi(xN) = f(x) = f(y) = \phi(yN),
        \end{align*} ovvero $\phi$ è ben definita.
        \item[Omomorfismo] Siano $xN, yN \in \quot{G}{N}$ e mostriamo che \[
            \phi(xN \cdot yN) = \phi(xN) * \phi(yN).    
        \] Infatti vale che \begin{align*}
            \phi(xN \cdot yN) &= \phi(xyN)\\
            &= f(xy) \tag{$f$ è omo.}\\
            &= f(x) * f(y)\\
            &= \phi(xN) * \phi(yN).
        \end{align*}
        \item[Proprietà delle immagini] Per definizione \begin{align*}
            \Imm \phi &= \set*{\phi(xN) \given xN \in \quot{G}{N}}\\
            &= \set*{f(x) \given xN \in \quot{G}{N}}.
            \intertext{Tuttavia, come abbiamo verificato nella parte relativa alla buona definizione di $\phi$, se $xN = yN$ allora $f(x) = f(y)$, dunque vale che}
            \Imm \phi &= \set*{f(x) \given x \in G}\\
            &= \Imm f.
        \end{align*}
        \item[Proprietà dei nuclei] Per definizione \begin{align*}
            \ker \phi &= \set*{xN \in \quot{G}{N} \given \phi(xN) = e^\prime}\\
            &= \set*{xN \in \quot{G}{N} \given f(x) = e^\prime}\\
            &= \set*{xN \in \quot{G}{N} \given x \in \ker f}\\
            &= \quot{\ker f}{N}. \tag*{\qedhere}
        \end{align*}
    \end{description}
\end{proof}

Nel caso particolare in cui $N = \ker f$ abbiamo che $\phi$ è iniettiva, come ci assicura il seguente corollario.

\begin{corollary}
    Siano $(G, \cdot)$, $(G^\prime, *)$ due gruppi e sia $f : G \to G^\prime$ un omomorfismo di gruppi. 
    
    Allora esiste un unico omomorfismo $\phi$ tale che il seguente diagramma commuta:
    \begin{equation}
        \begin{tikzcd}
            G \arrow[d, swap, "\pi_{\ker f}"] \arrow[r, "f"] & G^\prime \\
            \quot{G}{\ker f} \arrow[ur, hook, swap, "\phi"] &
        \end{tikzcd}
    \end{equation}

    In particolare $\phi$ è iniettivo, dunque ogni omomorfismo è fattorizzabile come composizione di un omomorfismo surgettivo e uno iniettivo.
\end{corollary}
\begin{proof}
    Siccome $\ker f \subseteq \ker f$ e $\ker f \normal G$ possiamo applicare il \nameref{th:first_iso}, da cui segue che esiste un unico omomorfismo $\phi$ tale che \[
        f = \phi \circ \pi_{\ker f}.    
    \]

    \paragraph{Iniettività di $\phi$} Per definizione di $\phi$ vale che $\phi(x\ker f) = e_{G^\prime}$ se e solo se $f(x) = e_{G^\prime}$, ovvero se e solo se $x \in \ker f$. Dunque il nucleo di $\phi$ è $\ker f$, che è l'elemento neutro del gruppo quoziente $\quot{G}{\ker f}$, da cui segue che $\phi$ è iniettiva. 

    Essendo inoltre $\pi_{\ker f}$ surgettivo segue la tesi.
\end{proof}

La fattorizzazione definita dal precedente corollario può essere resa ancora più precisa specificando un oggetto intermedio, l'immagine di $f$: l'omomorfismo $f$ viene quindi scomposto nella composizione di un omomorfismo surgettivo (la proiezione canonica modulo il kernel, ovvero $\pi_{\ker f}$), un isomorfismo e infine un omomorfismo iniettivo (l'inclusione canonica $\iota : \Imm f \to G^\prime$, $\iota(g) = g$). 

L'isomorfismo è proprio l'omomorfismo $\phi$ del \nameref{th:first_iso}: infatti per l'osservazione precedente $\phi$ è iniettivo; inoltre restringendo il codominio a $\Imm f$ e sapendo che $\Imm \phi = \Imm f$ segue che $\phi$ è anche surgettivo, rendendolo un isomorfismo.

Il seguente diagramma dunque commuta:
\begin{equation}
    \begin{tikzcd}
        G \arrow[r, swap, two heads, "\pi_{\ker f}"] \arrow[rrr, bend left, "f"] 
        & \quot{G}{\ker f} \arrow[r, hook, two heads, swap, "\phi"] 
        & \Imm f \arrow[r, hook, swap, "\iota"] 
        & G^\prime
    \end{tikzcd}
\end{equation}

Vale dunque il seguente corollario.
\begin{corollary}[Corollario al Primo Teorema degli Omomorfismi]
    \label{cor:G/ker=Imm}
    Siano $(G, \cdot)$, $(G^\prime, *)$ due gruppi e sia $f : G \to G^\prime$ un omomorfismo di gruppi. Allora \begin{equation}
        \quot{G}{\ker f} \isomorph \Imm f.
    \end{equation}
\end{corollary}

\subsection{Secondo Teorema degli Omomorfismi}
\begin{theorem}
    [Secondo Teorema degli Omomorfismi] \label{th:second_iso}
    Sia $(G, \cdot)$ un gruppo e siano $H, K \normal G$, con $H \subseteq K$. Allora \begin{equation}
        \frac{\quot{G}{H}}{\quot{K}{H}} \isomorph \quot{G}{K}.
    \end{equation}
\end{theorem}
\begin{proof}
    Consideriamo le proiezioni canoniche $\pi_H$ e $\pi_K$. Siccome $H \subseteq K = \ker \pi_K$ possiamo applicare il \nameref{th:first_iso} all'omomorfismo $\pi_K$ e al sottogruppo normale $H \normal G$ (tramite la proiezione $\pi_H$). Dunque esiste un unico omomorfismo \begin{align*}
        \phi : \quot{G}{H} &\to \quot{G}{K}\\
        gH &\mapsto gK
    \end{align*} che fa commutare il seguente diagramma:
    \begin{equation*}
        \begin{tikzcd}
            G \arrow[d, swap, "\pi_{H}"] \arrow[r, "\pi_K"] & \quot{G}{K} \\
            \quot{G}{H} \arrow[ur, two heads, swap, "\phi"] &
        \end{tikzcd}
    \end{equation*}
    Tale funzione è anche surgettiva, in quanto per il \nameref{th:first_iso} sappiamo che $\Imm \phi = \Imm \pi_K$, e $\pi_K$ è surgettiva. Inoltre \[
        \ker \phi = \quot{\ker \pi_K}{H} = \quot{K}{H}.    
    \]

    Consideriamo ora i gruppi $\quot{G}{H}$ e $\quot{G}{K}$ e il sottogruppo $\nicefrac{\quot{G}{H}}{\ker \phi}$, che corrisponde a $\frac{\quot{G}{H}}{\quot{K}{H}}$. Per il \nameref{th:first_iso} esiste un unico omomorfismo \begin{align*}
        \tilde{\phi} : \frac{\quot{G}{H}}{\quot{K}{H}} &\to \quot{G}{K}
    \end{align*} che fa commutare il seguente diagramma:
    \begin{equation*}
        \begin{tikzcd}
            \quot{G}{H} \arrow[r, two heads, "\phi"] \arrow[d, swap, "\pi_{\quot{K}{H}}"] & \quot{G}{K} \\
            \frac{\quot{G}{H}}{\quot{K}{H}} \arrow[ur, hook, two heads, swap, "\widetilde{\phi}"] &
        \end{tikzcd}
    \end{equation*}

    $\widetilde{\phi}$ è un isomorfismo di gruppi: infatti essendo $\phi$ surgettivo anche $\widetilde{\phi}$ lo è; inoltre la proiezione $\pi_{\quot{K}{H}}$ porta il gruppo $\quot{G}{H}$ nel quoziente modulo $\ker \phi = \quot{K}{H}$, dunque l'omomorfismo $\widetilde{\phi}$ è iniettivo ed è dunque un isomorfismo di gruppi.

    Segue quindi che \[
        \frac{\quot{G}{H}}{\quot{K}{H}} \isomorph \quot{G}{K}. \qedhere 
    \]
\end{proof}

\subsection{Terzo Teorema degli Omomorfismi}
\begin{theorem}
    [Terzo Teorema degli Omomorfismi] Sia $(G, \cdot)$ un gruppo e siano $H \sgr G$,$N \normal G$. 
    Valgono le seguenti affermazioni: \begin{itemize}
        \item $N$ è un sottogruppo normale di $HN$,
        \item $H \inters N$ è un sottogruppo normale di $H$,
        \item inoltre \begin{equation}
            \frac{H}{H \inters N} \isomorph \frac{HN}{N}.
        \end{equation}
    \end{itemize}
\end{theorem}
\begin{proof}
    Dimostriamo innanzitutto le due condizioni di normalità.
    \begin{description}
        \item[($N \normal HN$)] Mostriamo innanzitutto che $HN$ è un sottogruppo di $G$. Per la \autoref{prop:cond_prod_sgr_e'_sgr}, è sufficiente mostrare che $HN = NH$. 
        Siccome $N$ è normale in $G$ segue che $gN = Ng$ per ogni $g \in G$. Dato che $H \subseteq G$ segue che $hN = Nh$ per ogni $h \in H$, ovvero $HN = NH$.
        Dunque $HN$ è un sottogruppo di $G$.

        Notiamo inoltre che $N \subseteq HN$ (basta scegliere tutti gli elementi della forma $e_Gn$ al variare di $n \in N$), dunque essendo $N$ normale in $G$ segue che $N$ è normale in ogni sottogruppo di $G$ che lo contiene; in particolare $N \normal HN$.
        \item[($H \inters N \normal H$)] Sia $n \in H \inters N$ e sia $g \in H$.
        
        Ovviamente $gng\inv \in H$, in quanto $n$ ed $g$ sono entrambi elementi di $H$. Inoltre essendo $N$ un sottogruppo normale di $G$ segue che $gng\inv \in N$ per ogni $g \in G$, dunque a maggior ragione per ogni $g \in H \subseteq G$. 
        
        Dunque $gng\inv \in H \inters N$, da cui segue che $H \inters N$ è normale in $H$.
    \end{description}
    Consideriamo ora l'applicazione 
    \begin{align*}
        \begin{split}
            f : H &\to \quot{HN}{N} \\
            h &\mapsto hN.
        \end{split}
    \end{align*}  

    Quest'applicazione è una restrizione all'insieme $H \subseteq HN$ della proiezione canonica \[
        \pi_{N} : HN \to \quot{HN}{N};    
    \] questo ci garantisce che $f$ è ben definita e che è un omomorfismo di gruppi.
    
    Inoltre $f$ è surgettiva: basta mostrare che \begin{gather*}
        \Imm f = \quot{HN}{N}\\
        \intertext{il che equivale a}
        \set*{hN \in \quot{HN}{N} \given h \in H} = \set*{yN \in \quot{HN}{N} \given y \in HN}.
    \end{gather*}
    L'inclusione $\Imm f \subseteq \quot{HN}{N}$ è data dalla definizione; l'inclusione contraria viene dal fatto che se $yN \in \quot{HN}{N}$, ovvero $y = hn$ per qualche $hn \in HN$, allora $yN = hnN \in \set*{hN \given h \in H}$ in quanto $nN = N$.

    Inoltre \begin{align*}
        \ker f &= \set*{h \in H \given f(h) = N} \\
        &= \set*{h \in H \given hN = N} \\
        &= \set*{h \in H \given h \in N} \\
        &= H \inters N.
    \end{align*}

    Dunque per il \hyperref[cor:G/ker=Imm]{Corollario al Primo Teorema degli Omomorfismi} segue che \[
        \frac{H}{H \inters N} \isomorph \Imm f = \frac{HN}{N}. \qedhere    
    \]
\end{proof}

Prima di studiare il Teorema di Corrispondenza, introduciamo un lemma che ci sarà utile:
\begin{lemma}\label{lem:controimm_normal}
    Siano $(G, \cdot)$, $(G^\prime, \cdot)$ due gruppi e sia $f : G \to G^\prime$ un omomorfismo. Se $K \normal G^\prime$, allora $f\inv(K) \normal G$.

    Inoltre se $f$ è surgettivo e $H \normal G$ segue che \[
        f(H) \normal G^\prime = f(G).    
    \]
\end{lemma}

\begin{theorem}[Teorema di Corrispondenza tra Sottogruppi]
    \label{th:sgrp_corr}
    Sia $(G, \cdot)$ un gruppo e $N \normal G$. Sia $\GG$ l'insieme dei sottogruppi di $G$ che contengono $N$ e $\NN$ l'insieme dei sottogruppi di $\quot{G}{N}$.

    Allora esiste una corrispondenza biunivoca tra $\GG$ e $\NN$ che preserva l'indice di sottogruppo e i sottogruppi normali, ovvero esiste una funzione \begin{align*}
        \begin{split}
            \psi : \GG &\to \NN\\
            A &\mapsto \quot{A}{N}
        \end{split}
    \end{align*}    
    tale che \begin{itemize}
        \item $\GrpIndex{G:A} = \GrpIndex{\quot{G}{N} : \quot{A}{N}}$,
        \item se $A \normal G$ allora $\quot{A}{N} \normal \quot{G}{N}$.
    \end{itemize}
\end{theorem}
Prima di iniziare la dimostrazione, osserviamo che siccome la proiezione canonica è un omomorfismo, vale che \[
    \pi(H) \sgr \quot{G}{N}, \qquad \pi\inv(K) \sgr G    
\] per ogni $H \sgr G$, $K \sgr \quot{G}{N}$.
\begin{proof}
    Siano $\alpha$ e $\beta$ le mappe date da:
    \begin{align*}
        X &\biject Y\\
        H &\xmapsto{\alpha} \quot{H}{N} = \pi_N(H)\\
        \pi_N\inv(K) &\xmapsfrom{\beta} K.
    \end{align*}

    \paragraph{Buona definizione} $\alpha$ è ben definita poiché l'immagine di un sottogruppo attraverso la proiezione canonica è un sottogruppo: \[
        \alpha(H) = \pi_N(H) = \quot{H}{N} \sgr \quot{G}{N}.    
    \] Mostriamo quindi che $\beta$ è ben definita: sia $K \sgr \quot{G}{N}$ e mostriamo che $\beta(K) = \pi_N\inv(K)$ è un sottogruppo di $G$ che contiene $N$. Siccome $\quot{G}{N}$ è il quoziente modulo $N$ la sua identità è $N = eN$; per definizione di sottogruppo ogni elemento di $\NN$ dovrà contenere l'identità del gruppo, ovvero $N$. Segue quindi che \[
        N = \pi_N\inv(N) \subseteq \pi_N\inv(K),
    \] da cui $\pi_N\inv(K) \in \GG$.

    \paragraph{Le due funzioni sono una l'inversa dell'altra} Mostriamo che $\alpha \circ \beta = \id$. Sia $K \in \NN$: allora \[
        (\alpha \circ \beta)(K) = \alpha\parens[\big]{\pi\inv(K)} = \pi\parens[\big]{\pi\inv(K)} = K,
    \] dove il penultimo passaggio viene dal fatto che $\pi$ è surgettiva, e quindi invertibile da destra.

    Mostriamo ora che $\beta \circ \alpha = \id$. Sia $H \in \GG$: allora
    \begin{align*}
           (\beta \circ \alpha)(H) 
        &= \beta(\pi(H))\\
        &= \beta\parens[\big]{\quot{H}{N}} \\
        &= \pi_N\inv\parens[\big]{\quot{H}{N}}\\
        &= \set*{x \in G \given \pi_N(x) \in \quot{H}{N}}\\
        &= \set*{x \in G \given xN \in \quot{H}{N}}\\
        &= \set*{x \in G \given x \in H}\\
        &= H.
    \end{align*}

    \paragraph{La bigezione preserva i sottogruppi normali} Sia $H \in \GG$; mostriamo che \[
        H \normal G \iff \quot{H}{N} \normal \quot{G}{N}.    
    \] \begin{description}
        \item[($\implies$)] Segue dal \nameref{th:second_iso}. Infatti siccome $N, H \normal G$ e $N \subseteq H$ segue che \[
            \frac{\quot{G}{N}}{\quot{H}{N}} \isomorph \quot{G}{H}.    
        \] Ma questo significa che $\dfrac{\quot{G}{N}}{\quot{H}{N}}$ è un gruppo, da cui segue che \[
            \quot{H}{N} \normal \quot{G}{N}.
        \]
        \item[($\impliedby$)] Segue dal \autoref{lem:controimm_normal}.
    \end{description}

    \paragraph{La bigezione conserva l'indice di sottogruppo} Sia $H \in \GG$: mostriamo che \[
        \GrpIndex{G : H} = \GrpIndex{\quot{G}{N} : \quot{H}{N}}.    
    \] Siano $x, y \in G$ qualsiasi. Mostriamo che le classi laterali $xH$ e $yH$ sono uguali se e solo se \[
        (xN)\quot{H}{N} = (yN)\quot{H}{N}.    
    \] Per definizione \[
        (xN)\quot{H}{N} = \set*{xNhN \given h \in H} = \set*{xhN \given h \in H};    
    \] allo stesso modo \[
        (yN)\quot{H}{N} = \set*{yhN \given h \in H}.    
    \] FINIRE
\end{proof}


\chapter{Anelli e campi}

\section{Anelli}

\begin{restatable}[Anello]{definition}{defring}
    \label{def:ring}
    Sia $A$ un insieme e siano $+$ (\emph{somma}), $\cdot$ (\emph{prodotto}) due operazioni su $A$, ovvero \begin{align*}
        + : A \times A &\to A, & \cdot : A \times A &\to A. \\
        (a, b) &\mapsto a+b,      &             (a, b) &\mapsto a\cdot b.
    \end{align*} La struttura $(A, +, \cdot)$ si dice \strong{anello} se valgono i seguenti assiomi:
    \begin{itemize}
        \item[(S)] $(A, +)$ è un gruppo abeliano.
        \item[(P1)] Vale la \emph{proprietà associativa del prodotto}:            
        per ogni $a, b, c \in A$ vale che \[
            (a \cdot b) \cdot c = a \cdot (b \cdot c).
        \]
        \item[(D)] \label{def:anello:distr} Vale la \emph{proprietà distributiva del prodotto rispetto alla somma} sia a destra che a sinistra:         
        per ogni $a, b, c \in A$ vale che \[
            a(b + c) = ab + ac, \qquad (a + b)c = ac + bc.
        \]
    \end{itemize}
    L'anello $(A, +, \cdot)$ si dice \strong{anello commutativo} se vale inoltre l'assioma di commutatività:
    \begin{itemize}
        \item[(P2)] \label{def:anello_prod:com} Vale la \emph{proprietà commutativa del prodotto}:
        per ogni $a, b \in A$ vale che \[
            a \cdot b = b \cdot a.
        \]
    \end{itemize}
    L'anello $(A, +, \cdot)$ si dice \strong{anello con unità} se vale inoltre il seguente assioma:
    \begin{itemize}
        \item[(P3)] \label{def:anello_prod:unit} Esiste un elemento $1 \in A$ che è \emph{elemento neutro} per il prodotto:
        per ogni $a \in A$ vale che \[
            a \cdot 1 = 1 \cdot a = a.
        \]
        Tale elemento si dice \emph{unità dell'anello}.
    \end{itemize}
\end{restatable}

\begin{example}
    Le strutture $(\Z, +, \cdot)$, $(\Q, +, \cdot)$, $(\R, +, \cdot)$, $(\C, +, \cdot)$ sono tutti esempi di anelli commutativi con unità.
\end{example}
\begin{example}
    L'insieme delle matrici quadrate $\Mat{n, \R}$ (con $n \geq 2$) è un esempio di anello non commutativo con unità.
\end{example}
\begin{example}
    L'insieme dei numeri pari insieme alle operazioni di somma e prodotto, ovvero $(2\Z, +, \cdot)$, è un anello commutativo ma non ha l'identità.
\end{example}

\subsubsection{Elementi particolari di un anello}

\begin{definition} [Elementi invertibili]
    \label{def:units}
    Sia $(A, +, \cdot)$ un anello con identità. Un elemento $a \in A$ si dice \strong{invertibile} se esiste $y \in A$ tale che $xy = yx = 1$. L'insieme degli invertibili di $A$ si indica con $\units{A}$.
\end{definition}

\begin{definition} [Divisori di zero]
    \label{def:zero_divisor}
    Sia $(A, +, \cdot)$ un anello. Un elemento $a \in A$ si dice \strong{divisore di zero} se esiste $b \in A$, $b \neq 0$ tale che \[
        ab = 0.
    \] Indicheremo con $\ZeroDiv$ l'insieme dei divisori dello zero di $A$.
\end{definition}

\begin{definition} [Nilpotenti]
    Sia $(A, +, \cdot)$ un anello. Un elemento $a \in A$ si dice \strong{nilpotente} se esiste $n \in \N$ tale che \[
        a^n \deq \overbrace{a \cdot a \cdots a}^{n\text{ volte}} = 0.
    \] Indicheremo con $\Nilpotents$ l'insieme dei divisori dello zero di $A$.
\end{definition}

Mostriamo ora alcune proprietà degli anelli relative agli elementi invertibili e ai divisori di zero.

\begin{proposition} [Proprietà degli anelli]
    \label{prop:prop_anelli}
    Sia $(A, +, \cdot)$ un anello. Allora valgono le seguenti affermazioni:
    \begin{enumerate}[label={(\roman*)}, ref={\theproposition: (\roman*)}]
        \item \label{prop:prop_anelli:per_0} Per ogni $a \in A$ vale che $a \cdot 0 = 0 \cdot a = 0$. In particolare $0$ è sempre un divisore di zero.
        \item \label{prop:prop_anelli:gruppo_inv} $(\units{A}, \cdot)$ è un gruppo (abeliano se $A$ è commutativo).
        \item \label{prop:prop_anelli:div_zero_inv} Nessun $a \in A$ è contemporaneamente divisore dello zero e invertibile, ovvero $\ZeroDiv \inters \units{A} = \varnothing$.
    \end{enumerate}
\end{proposition}
\begin{proof}
    Dimostriamo separatamente le varie affermazioni.
    \begin{enumerate}[label={(\roman*)}]
        \item Per gli assiomi degli anelli \[
            a \cdot 0 = a \cdot (0 + 0) = a \cdot 0 + a \cdot 0.
        \]        
        Siccome $(A, +)$ è un gruppo, valgono le \hyperref[prop:prop_grp:canc]{leggi di cancellazione}, dunque segue che \[
            0 = a \cdot 0.    
        \]
        \item Mostriamo che $(\units{A}, \cdot)$ è un gruppo.
        \begin{enumerate}[label={(G\arabic*)}]
            \item Mostriamo che il prodotto di due elementi invertibili di $A$ è ancora in $\units{A}$, ovvero è ancora invertibile.
            
            Siano $x,y \in \units{A}$ (ovvero essi sono invertibili e i loro inversi sono rispettivamente $x\inv$ e $y\inv$); mostro che il loro prodotto $xy \in A$ è invertibile e il suo inverso è $y\inv x\inv$.
            \begin{align*}
                &(xy) \cdot (y\inv x\inv) \\
                =\ &x (y y\inv) x\inv  \\
                =\ &x \cdot x\inv \\
                =\ &1.
            \end{align*}
            Passaggi analoghi mostrano che $(y\inv x\inv) \cdot xy = 1$, ovvero $y\inv x\inv$ è l'inverso di $xy$ e quindi $xy \in \units{A}$.
            \item Vale la proprietà associativa del prodotto in quanto vale in $A$.
            \item L'elemento neutro del prodotto è $1$ ed è in $\units{A}$ in quanto $1 \cdot 1 = 1$ (ovvero $1$ è l'inverso di se stesso).
            \item Se l'anello è commutativo, allora $\cdot$ è commutativa su ogni suo sottoinsieme, dunque in particolare lo sarà anche su $\units{A}$.
        \end{enumerate}
        Da ciò segue che $(\units{A}, \cdot)$ è un gruppo.
        \item Supponiamo per assurdo esista $x \in A$ che è invertibile e divisore dello zero.
        Dato che è un divisore dello zero segue che esiste $z \in A \setminus \set{0}$ tale che $xz = 0$; siccome $x$ è invertibile dovrà esistere $y \in A$ tale che $xy = 1$. 
        Ma allora \begin{align*}
            z &= z\cdot 1\\
            &= z \cdot (xy)\\
            &= (zx) \cdot y \\
            &= 0 \cdot y\\
            &= 0.
        \end{align*} Tuttavia ciò è assurdo, in quanto abbiamo supposto $z \neq 0$, dunque non può esistere un divisore dello zero invertibile. \qedhere
    \end{enumerate}
\end{proof}

\subsubsection{Tipi di anelli}

\begin{definition}
    [Dominio di integrità]
    \label{def:dominio}
    Sia $(A, +, \cdot)$ un anello commutativo con identità. Esso si dice \emph{dominio di integrità} (o semplicemente \emph{dominio}) se l'unico divisore dello zero è $0$.
\end{definition}

In un dominio di integrità valgono alcune proprietà particolari.

\begin{proposition}
    [Legge di annullamento del prodotto]
    \label{prop:ann_prod_dominio}
    Sia $(A, +, \cdot)$ un dominio. Allora vale la legge di annullamento del prodotto, ovvero per ogni $a, b \in A$ vale che \[
        ab = 0 \implies a = 0 \text{ oppure } b = 0.    
    \]
\end{proposition}
\begin{proof}
    Se $a = 0$ la tesi è verificata. Supponiamo allora $a \neq 0$ e dimostriamo che deve essere $b = 0$.

    Dato che $a \neq 0$ segue che $a$ non è un divisore dello zero (poiché $A$ è un dominio), dunque se $ab = 0$ l'unica possibilità è $b = 0$.
\end{proof}

Dall'annullamento del prodotto seguono le leggi di cancellazione del prodotto:
\begin{corollary}
    [Leggi di cancellazione per il prodotto]
    \label{cor:canc_prod_dominio}
    Sia $(A, +, \cdot)$ un dominio di integrità e siano $a, b, x \in A$ con $x \neq 0$. Allora \[
        ax = bx \implies a = b.    
    \]
\end{corollary}
\begin{proof}
    Aggiungiamo ad entrambi i membri l'opposto di $bx$: \begin{align*}
        &ax - bx = bx - bx\\
        \iff &ax - bx = 0 \tag{per \ref{def:anello:distr}}\\
        \iff &(a - b)x = 0 \tag{per \ref{prop:ann_prod_dominio}}\\
        \iff &a-b = 0 \text{ oppure } x = 0.
    \end{align*} Ma per ipotesi $x \neq 0$, dunque deve seguire che $a - b = 0$, ovvero $a = b$.
\end{proof}

\begin{definition}
    [Corpi e campi]
    Sia $(\K, +, \cdot)$ un anello con identità.  $\K$ si dice \strong{corpo} se $\units{\K} = \K \setminus \set*{0}$. Se $\K$ è commutativo si dice \strong{campo}.
\end{definition}

\begin{remark}
    Un campo è una struttura $(\K, +, \cdot)$ tale che: 
    \begin{enumerate}[label={(S)}]
        \item La struttura $(\K, +)$ è un gruppo abeliano.
    \end{enumerate}
    \begin{enumerate}[label={(P)}]
        \item La struttura $(\K\setminus \set*{0}, \cdot)$ è un gruppo abeliano.
    \end{enumerate}
    \begin{enumerate}[label=(D)]
        \item \label{def:campo:distr} Vale la \emph{proprietà distributiva del prodotto rispetto alla somma}.
    \end{enumerate}
\end{remark}

\begin{proposition}
    [Ogni campo è un dominio] Sia $(\K, +, \cdot)$ un campo. Allora $\K$ è anche un dominio di integrità.
\end{proposition}
\begin{proof}
    Per la \autoref{prop:prop_anelli} i divisori dello zero non possono essere invertibili, dunque $\ZeroDiv \subseteq \K \setminus \units{\K}$. Ma per definizione di campo $\units{\K} = \K \setminus \set*{0}$, dunque l'unico possibile divisore dello zero è $0$, ovvero $\K$ è un dominio.
\end{proof}

\begin{proposition}
    [Ogni dominio finito è un campo]
    Sia $(A, +, \cdot)$ un dominio di integrità con un numero finito di elementi. Allora $A$ è un campo.
\end{proposition}
\begin{proof}
    Sia $x \in A \setminus \set*{0}$: mostriamo che $x$ è invertibile.
    Sia \begin{align*}
        \phi_x : A &\to A\\
        a &\mapsto ax.
    \end{align*} Mostriamo che $\phi_x$ è bigettiva.
    
    \begin{description}
        \item[Iniettività] Supponiamo che per qualche $a, b \in A$ valga che $\phi_x(a) = \phi_x(b)$ e mostriamo che segue che $a = b$.

        Per definizione di $\phi_x$ l'ipotesi equivale ad affermare che $ax = bx$, ma siccome $x \neq 0$ e $A$ è un dominio possiamo applicare la \hyperref[cor:canc_prod_dominio]{legge di cancellazione per il prodotto}, da cui segue che $a = b$, ovvero $\phi_x$ è iniettiva.
        \item[Surgettività] Segue dal fatto che dominio e codominio hanno la stessa cardinalità.
    \end{description}

    Dunque $\phi_x$ è bigettiva. Dato che $1 \in A = \phi_x(A)$ segue che esiste un $y \in A$ tale che \[
        xy = 1 = yx, 
    \] ovvero $x$ è invertibile e $A$ è un campo.
\end{proof}

\begin{definition}
    [Omomorfismo di anelli] \label{def:omo_anelli}
    Siano $(A, +, \cdot)$, $(B, \oplus, \odot)$ anelli con unità. Allora la funzione $\phi : A \to B$ si dice omomorfismo di anelli se \begin{enumerate}[label={(\roman*)}]
        \item $\phi(1_A) = 1_B$.
        \item Per ogni $a, b \in A$ vale che $\phi(a + b) = \phi(a) \oplus \phi(b)$.
        \item Per ogni $a, b \in A$ vale che $\phi(a \cdot b) = \phi(a) \odot \phi(b)$.
    \end{enumerate}
\end{definition}
\section{Anello dei polinomi}

\begin{definition}
    [Polinomi a coefficienti in un anello]
    Sia $(A, +, \cdot)$ un anello commutativo con identità e consideriamo una successione $\left(a_i\right)$ di elementi di $A$ che sia definitivamente nulla, ovvero tale che esista un $n \in \N$ tale che \[
        a_m = 0 \quad \text{per ogni } m > n.    
    \]
    Allora si dice \emph{polinomio nell'indeterminata $X$} la scrittura formale \[
        p = p(X) = \sum_{i = 0}^{\infty} a_iX^i.    
    \] Gli $a_i$ si dicono \emph{coefficienti del polinomio}.

    L'insieme dei polinomi a coefficienti in $A$ si indica con $A[X]$.
\end{definition}

Dato che la successione che definisce il polinomio è definitivamente nulla, possiamo scrivere il polinomio come una sequenza finita di termini: basta prendere i termini fino al massimo indice per cui $a_i$ è diverso da $0$. Diamo però alcune definizioni preliminari.

Innanzitutto d'ora in avanti $(A, +, \cdot)$ è un anello commutativo con identità a meno di ulteriori specifiche.

\begin{definition}
    [Polinomio nullo]
    Si dice \emph{polinomio nullo in $A[X]$} il polinomio definito dalla successione costantemente nulla, e lo si indica come $p(X) = 0_{A[X]}$.
\end{definition}

\begin{definition}
    [Grado di un polinomio]
    Sia $p \in A[X]$, $p(X) \neq 0_{A[X]}$. Allora si dice grado di $p$ il numero \[
        \deg p = \max \set{n \in \N \suchthat a_n \neq 0}.    
    \] Il polinomio $0_{A[X]}$ non ha grado.
\end{definition}

Notiamo che i polinomi di grado $0$ sono tutti e solo della forma $p(X) = a_0$ per qualche $a_0 \in A$; ovvero sono tutte e sole le costanti dell'anello $A$. Possiamo quindi considerare l'anello $A$ come un sottoinsieme dell'insieme dei polinomi $A[X]$.

\begin{definition}
    [Uguaglianza tra polinomi]
    Siano $p, q \in A[X]$. Allora i polinomi $p$ e $q$ sono uguali se e solo se tutti i loro coefficienti sono uguali.
\end{definition}

Definiamo ora le operazioni di somma e prodotto tra polinomi.

\begin{definition}
    [Somma tra polinomi]
    Siano $p, q \in A[X]$.
    Allora definisco l'operazione di somma \begin{align*}
        + : A[X] \times A[X] &\to A[X]\\
        (p, q) &\mapsto p + q
    \end{align*} nel seguente modo: \begin{align*}
        p(X) = \sum_{i = 0}^\infty a_iX^i, \quad q(X) = \sum_{i = 0}^\infty b_iX^i\\
        \implies (p + q)(X) \deq \sum_{i = 0}^\infty (a_i + b_i)X^i.
    \end{align*}
\end{definition}

\begin{definition}
    [Prodotto tra polinomi]
    Siano $p, q \in A[X]$.
    Allora definisco l'operazione di prodotto tra polinomi \begin{align*}
        \cdot : A[X] \times A[X] &\to A[X]\\
        (p, q) &\mapsto p \cdot q
    \end{align*} nel seguente modo: \begin{align*}
        p(X) = \sum_{i = 0}^\infty a_iX^i, \quad q(X) = \sum_{j = 0}^\infty b_jX^j \\
        \implies (p \cdot q)(X) \deq \sum_{i = 0}^\infty \sum_{j = 0}^\infty a_ib_jX^{i+j}.
    \end{align*}
\end{definition}

\begin{theorem}
    [L'insieme dei polinomi è un anello]
    La struttura $(A[X], +, \cdot)$ è un anello commutativo con identità (dove l'identità è il polinomio $1_{A[X]}(X) = 1_A$).
\end{theorem}
\begin{proof}
    Basta verificare tutti gli assiomi degli anelli.
\end{proof}

\begin{proposition}
    [Grado della somma e del prodotto]
    \label{prop:deg_somma_prod_polinomi}
    Siano $p, q \in A[X] \setminus \set{0_{A[X]}}$. Allora vale che \begin{enumerate}[label={(\roman*)}]
        \item $\deg (p + q) \leq \max \set{\deg p, \deg q}$.
        \item se $A$ è un dominio, allora $\deg (pq) = \deg p + \deg q$.
    \end{enumerate}
\end{proposition}
\begin{proof}
    Siano i due polinomi \[
        p(X) = \sum_{i = 0}^\infty a_iX^i, \quad q(X) = \sum_{i = 0}^\infty b_iX^i.   
    \] e siano $n = \deg p$, $m = \deg q$.

    \paragraph{Grado della somma} Sia $k = \max {n, m}$. Allora per ogni $i > k$ varrà che $a_i = b_i = 0$, ovvero $a_i + b_i = 0$, da cui $\deg (p + q) \leq k$.

    \paragraph{Grado del prodotto} Il termine di grado massimo di $(pq)(X)$ deve essere quello in posizione $n + m$. 
    
    Mostriamo che per ogni $i > n$, $j > m$ vale che il coefficiente del termine di grado $i + j$ è uguale a $0$.
    Infatti per definizione di grado segue che $a_i, b_j = 0$ se $i > n$ o $j > m$, dunque il prodotto $a_i \cdot b_j$ sarà $0$, ovvero il coefficiente di grado $i + j$ sarà nullo. Da ciò segue che $\deg (pq) \leq n + m$.

    Inoltre essendo $A$ un dominio il termine $a_nb_m$ deve essere diverso da $0$, in quanto altrimenti uno tra $a_n$ e $b_m$ dovrebbe essere $0$, contro la definizione di grado.

    Dunque $\deg (pq) = \deg p + \deg q$.
\end{proof}

\begin{corollary}
    Se $A$ è un dominio, allora $A[X]$ è un dominio.
\end{corollary}
\begin{proof}
    Siano $p, q \in A[X] \setminus \set{0_{A[X]}}$, con $\deg p = n \geq 0$, $\deg q = m \geq 0$. Allora per la \autoref{prop:deg_somma_prod_polinomi} vale che \[
        \deg (pq) = \deg p + \deg q = n + m \geq 0.    
    \] Dunque il polinomio $(pq)(X)$ non può essere il polinomio nullo (che non ha grado), da cui segue che in $A[X]$ non vi sono divisori dello zero.
\end{proof}

\begin{corollary}
    Se $A$ è un dominio, allora gli invertibili di $A[X]$ sono tutti e soli gli elementi invertibili di $A$, ovvero \[
        \invertible{A[X]} = \invertible{A}. 
    \]
\end{corollary}
\begin{proof}
    Sia $p \in \invertible{A[X]}$ e sia $q \in A[X]$ il suo inverso, ovvero tale che $(pq)(X) = 1_A$.

    Notiamo che $p, q \neq 0_{A[X]}$. Infatti se uno dei due fosse il polinomio nullo per la \autoref{prop:prop_anelli:per_0} il loro prodotto dovrebbe essere il polinomio nullo e non l'unità. Allora esistono $\deg p, \deg q \geq 0$ e vale che \[
        \deg (pq) = \deg p + \deg q \seteq \deg 1 = 0.    
    \]

    Dato che i gradi di $p$ e $q$ sono positivi o nulli, il grado del prodotto è $0$ se e solo se entrambi i polinomi $p$ e $q$ sono di grado zero, ovvero se e solo se sono elementi dell'anello $A$.

    Siano $a, b \in A$ tali che $f(X) = a$e $q(X) = b$. Allora $(pq)(X) = a\cdot b = 1$, ovvero $a$ è invertibile, cioè $a \in \invertible{A}$.
\end{proof}

\subsection{Polinomi a coefficienti in un campo}

In questa sezione studieremo l'anello $\K[X]$, dove $\K$ è un campo generico. Questo anello ha una relazione molto stretta con l'insieme $\Z$ dei numeri interi, soprattutto per quanto riguarda le proprietà di divisibilità.

\begin{theorem}
    [Esistenza e unicità della Divisione Euclidea]
    Siano $f, g \in \K[X]$ con $f(X) \neq 0_{\K[X]}$. Allora esistono e sono unici due polinomi $q, r \in \K[X]$ tali che \[
        g(X) = q(X)f(X) + r(X),
    \] con $r(X) = 0_{\K[X]}$ oppure $0 \leq \deg r \leq \deg f$.
\end{theorem}
\begin{proof}[Dimostrazione dell'esistenza]
    Se $g(X) = 0_{\K[X]}$ allora posso scegliere $q(X) = 0_{\K[X]}$ e $r(X) = q(X) = 0_{\K[X]}$.
    Altrimenti procedo per induzione su $n \deq \deg g$.
    \begin{description}
        \item[Caso base] Supponiamo $\deg g = 0$, ovvero $g(X) = g_0$. Abbiamo due casi: \begin{itemize}
            \item se $\deg f = 0$, ovvero $f(X) = f_0 \in \K$, allora \[
                q(X) = g_0{f_0}\inv, \; r(X) = \bm 0;
            \]
            \item se $\deg f > \deg g$ allora \[
                q(X) = \bm 0, \; r(X) = g(X).    
            \]
        \end{itemize}
        \item[Passo induttivo] Sia $m \deq \deg f$. Come nel caso base, se $\deg f > \deg g$ basta scegliere $q$ uguale al polinomio nullo, $r(X) = g(X)$.
        
        Supponiamo invece che $\deg f \leq \deg g$. Possiamo scrivere i due polinomi come \[
            f(X) = \sum_{i = 0}^m a_iX^i, \; g(X) = \sum_{i = 0}^n b_iX^i.    
        \]
        Sia $g_1 \in \K[X]$ il seguente polinomio: \begin{align*}
            g_1[X] &\deq g(X) - \frac{b_n}{a_m}X^{n-m}f(X)\\  
            &= g(X) - b_nX^n + \dots 
        \end{align*}
        dove i puntini indicano termini di grado inferiore al termine di grado massimo (ovvero $n$).

        Il polinomio $g_1$ ha sicuramente grado inferiore al polinomio $g$, in quanto il termine di grado $n$ (ovvero $b_nX^n$) è stato eliso.

        Segue quindi per ipotesi induttiva che esistono $q_1, r_1 \in \K[X]$ tali che \[
            g_1(X) = q_1(X)f(X) + r_1(X)    
        \] con $r_1 = 0_{\K[X]}$ oppure $0 \leq r_1 \leq \deg f$.

        Dunque possiamo ricavare un'espressione per $g$ dalla definizione di $g_1$:
        \begin{align*}
            g(X) &= g_1(X) + \frac{b_n}{a_m}x^{n-m}f(X)\\
            &= q_1(X)f(X) + r_1(X) + \frac{b_n}{a_m}x^{n-m}f(X)\\
            &= (q_1(X) + \frac{b_n}{a_m}x^{n-m})f(X) + r_1(X).
        \end{align*}
        Dunque scegliendo $q(X) = q_1(X) + \frac{b_n}{a_m}x^{n-m}$ e $r(X) = r_1(X)$ ottengo la divisione euclidea tra $f$ e $g$.
    \end{description}
\end{proof}
\begin{proof}[Dimostraizone dell'unicità]
    Siano $q_1, r_1, q_2, r_2 \in \K[X]$ tali che \begin{align*}
        g(X) = q_1(X)f(X) + r_1(X) = q_2(X)f(X) + r_2(X)
    \end{align*} con $r_1 = 0_{\K[X]}$ oppure $0 \leq \deg r_1 \leq \deg f$, $r_2 = 0_{\K[X]}$ oppure $0 \leq \deg r_2 \leq \deg f$.

    Riarrangiando i termini ottengo \[
        (q_1(X) - q_2(X))f(X) = r_2(X) - r_1(X).    
    \] Se $r_1 = r_2$ segue che $q_1 = q_2$ (per differenza), dunque supponiamo per assurdo $r_1 \neq r_2$.
\end{proof}

\end{document}