\section{Fattorizzazione di polinomi}

\subsection{Fattorizzazione sui complessi}

\begin{theorem}[Teorema Fondamentale dell'Algebra] \label{th:fondamentale_algebra}
    Sia $f \in \C[X]$ con $\deg f \geq 1$. Allora $f$ ha almeno una radice in $\C$.
\end{theorem}

\begin{corollary}
    [Gli irriducibili sui complessi sono lineari] Sia $f \in \C[X]$. Allora $f$ è irriducibile se e solo $\deg f = 1$.
\end{corollary}
\begin{proof}
    L'implicazione da destra verso sinistra è valida in ogni campo, dunque dimostriamo l'altra: sia $f \in \C[X]$ con $\deg f = n > 1$. Allora per il \nameref{th:fondamentale_algebra} esiste $\alpha \in \C$ tale che $f(\alpha) = 0$. Per il \nameref{th:ruffini} allora $X-\alpha \divides f(X)$, dunque $f(X) = (X-\alpha)g(X)$ per qualche $g \in \C[X]$. Da questa equazione segue che $\deg g = \deg f - 1 > 0$, dunque $f$ è riducibile, da cui segue la tesi.
\end{proof}

\begin{corollary}
    Sia $f(X) \in \C[X]$ di grado $\deg f \geq 1$. Allora vale che $f$ ha esattamente $\deg f$ radici complesse, ovvero $f$ è fattorizzabile in esattamente $n$ fattori lineari, contati con la loro molteplicità.
\end{corollary}
\begin{proof}
    In $\C[X]$ vale il \nameref{th:fatt_unica}; inoltre gli irriducibili di $\C[X]$ sono tutti e soli i polinomi di primo grado (per il corollario precedente): da ciò segue la tesi.
\end{proof}

\subsection{Fattorizzazione sugli interi e sui razionali}

\begin{definition}
    [Contenuto di un polinomio]
    Sia $f \in \Z[X]$ tale che $f(X) \deq \sum_{i=0}^n a_0X^i$. Si dice \emph{contenuto di $f$} il valore \[
        c(f) \deq \operatorname{mcd}\parens*{a_0, a_1, \dots, a_n}. 
    \]
\end{definition}

\begin{definition}
    [Polinomio primitivo]
    Sia $f \in \Z[X]$. Allora $f$ si dice \emph{primitivo} se $c(f) = 1$, ovvero se i suoi coefficienti non hanno fattori in comune.
\end{definition}

\begin{remark}
    Ogni polinomio a coefficienti interi può essere scritto come il prodotto del suo contenuto e di polinomio primitivo: \[
        f(X) = c(f) \cdot f_1(X),    
    \] dove $f_1 \in \Z[X]$ è primitivo.
\end{remark}

Il seguente Lemma ci permette di studiare la fattorizzazione su $\Q$ e su $\Z$ allo stesso modo.
\begin{theorem}
    [Lemma di Gauss] Sia $f \in \Z[X]$ primitivo. Allora $f$ è irriducibile in $\Z[X]$ se e solo se è irriducibile in $\Q[X]$.
\end{theorem}

\begin{proposition}
    [Radici razionali di un polinomio a coefficienti interi]
    Sia $f(X) \in \Z[X]$ un polinomio a coefficienti interi tale che \[
        f(X) = a_0 + a_1x + a_2x^2 + \dots + a_nx^n.    
    \] Sia $\frac{c}{d} \in \Q$ ridotta ai minimi termini (ovvero $\mcd{c}{d} = 1$). 
    
    Allora se $\frac{c}{d}$ è una radice di $f$ segue che $c \divides a_0$ e $d \divides a_n$.
\end{proposition}
\begin{proof}
    Per definizione di radice di un polinomio \[
        f\left(\frac{c}{d}\right) = a_0 + a_1\frac{c}{d} + \dots + a_{n-1}\left(\frac{c}{d}\right)^{n-1} + a_n\left(\frac{c}{d}\right)^n = 0.
    \]
    Moltiplicando entrambi i membri per $d^n$ otteniamo \[
        \iff a_0d^n + a_1cd^{n-1} + \dots + a_{n-1}c^{n-1}d + a_nc^n = 0.
    \]
    Se vale l'uguaglianza, allora i due membri saranno anche congrui modulo $d$: \[
        a_0d^n + a_1cd^{n-1} + \dots + a_{n-1}c^{n-1}d + a_nc^n \congr 0 \Mod{d}.
    \]
    \begin{align*}
        \iff &a_nc^n \congr 0 \Mod{d} \\
        \intertext{Dato che $\mcd{c}{d} = 1$, allora $c^n$ è invertibile modulo $d$)}
        \iff &a_n \congr 0 \Mod{d}\\
        \iff &d \divides a_n.
    \end{align*}

    Consideriamo ora la congruenza modulo $c$: \[
        a_0d^n + a_1cd^{n-1} + \dots + a_{n-1}c^{n-1}d + a_nc^n \congr 0 \Mod{c}.
    \]
    \begin{align*}
        \iff &a_0d^n \congr 0 \Mod{c}\\
        \iff &a_0 \congr 0 \Mod{c} \\
        \iff &c \divides a_0. \qedhere
    \end{align*}
\end{proof}

Un altro metodo per scomporre i polinomi a coefficienti interi è quello di sfruttare le congruenze. Sia $p \in \Z$ primo; chiamiamo \emph{riduzione modulo $p$} la seguente funzione:
\begin{align*}
    \pi_p : \Z[X] &\to \Zmod{p}[X]\\
    \sum_{i=0}^n a_iX^i &\mapsto \sum_{i=0}^n \eqcl{a_i}X^i.
\end{align*}
Si può verificare molto semplicemente che questa funzione è un omomorfismo di anelli; inoltre se $p \ndivides a_n$ segue che $\deg f = \deg \pi_p f$.

\begin{proposition}
    [Criterio di riduzione]
    Sia $p \in \Z$ primo, $f(X) = \sum_{i=0}^n a_iX^i \in \Z[X]$ primitivo. Se \begin{itemize}
        \item $p \ndivides a_n$;
        \item $\pi_p f$ è irriducibile in $\Zmod{p}$
    \end{itemize} allora $f$ è irriducibile in $\Z[X]$ e dunque in $\Q[X]$.
\end{proposition}
\begin{proof}
    Per dimostrare la proposizione è sufficiente mostrare la contronominale: se $f$ è riducibile in $\Z[X]$ allora deve esserlo anche in $\Zmod{p}[X]$ per qualunque $p$ primo.

    Siano $a, b \in \Z[X]$ di grado positivo tali che $f(X) = a(X)b(X)$: allora \[
        \pi_p\parens*{f(X)} = \pi_p\parens*{a(X)b(X)} = \pi_p\parens*{a(X)}\pi_p\parens*{b(X)},
    \] dunque la riduzione modulo $p$ del polinomio $f$ è riducibile se e solo se $\pi_p\parens*{a(X)}$ e $\pi_p\parens*{b(X)}$ sono entrambi di grado positivo.

    Per la \autoref{prop:deg_somma_prod_polinomi} sappiamo che $\deg f = \deg a + \deg b$. Inoltre siccome $p \ndivides a_n$ segue che $\deg f = \deg \pi_p(f)$. Combinando i due risultati e sapendo che il grado della riduzione modulo $p$ è minore o uguale al grado del polinomio originale: \begin{align*}
        \deg a + \deg b &= \deg f \\
        &= \deg \pi_p(f) \\
        &= \deg \pi_p(a) + \deg \pi_p(b)\\
        &\leq \deg a + \deg \pi_p(b) \\
        &\leq \deg a + \deg b.
    \end{align*} Dunque tutte le disuguaglianze sono uguaglianze e $\deg a = \deg \pi_p(a)$, $\deg b = \deg \pi_p(b)$. In particolare i grado delle riduzioni di $a$ e di $b$ sono positivi, da cui segue che $\pi_p(f)$ è riducibile.
\end{proof}

\begin{proposition}
    [Criterio di Eisenstein]
    Sia $f(X) = \sum_{i=0}^n a_iX^i \in \Z[X]$. Se esiste un primo $p \in \Z$ tale che 
    \begin{itemize}
        \item $p \divides a_i$ per ogni $i = 0, \dots, n-1$;
        \item $p \ndivides a_n$;
        \item $p^2 \ndivides a_0$
    \end{itemize}
    allora $f$ è irriducibile in $\Z[x]$.
\end{proposition}
\begin{proof}
    Supponiamo per assurdo che $f$ sia riducibile in $\Z[X]$, ovvero che esistano due polinomi $g, h \in \Z[X]$ di grado positivo e tali che $f(X) = g(X)h(X)$. Sia $n \deq \deg f$, $m \deq \deg g \geq 1$; da ciò segue che $\deg h = n - m \leq 1$.

    Siccome $f$ è primitivo e $p \ndivides a_n$ segue che \[
        \pi_p\parens*{f(X)} = \pi_p\parens*{g(X)}\pi_p\parens*{h(X)}
    \] e i gradi di $\pi_p(g)$, $\pi_p(h)$ sono uguali ai gradi di $g$ e di $h$, rispettivamente.

    Dal fatto che $p$ divide tutti i coefficienti di $f$ tranne $a_n$ segue che \[
        \pi_p(f(X)) = \eqcl{a_n}X^n.
    \] Siccome in $\Zmod{p}$ vale il \nameref{th:fatt_unica} (poiché $\Zmod{p}$ è un campo) gli unici fattori di $\eqcl{a_n}X^n$ sono della forma $X^k$ per qualche costante. 
    
    Da ciò segue che $\pi_p\parens*{g(X)} = \eqcl{b_n}X^m$, $\pi_p\parens*{h(X)} = \eqcl{c_n}X^m$, dove $\eqcl{b}\eqcl{c} = \eqcl{a_n}$. In particolare questo significa che i termini noti di $g$ e di $h$ (rispettivamente $b_0$ e $c_0$) devono essere divisibili per $p$, il che implica \[
        p^2 \divides b_0c_0 = a_0;
    \] ma ciò è assurdo, dunque $f$ è irriducibile.
\end{proof}