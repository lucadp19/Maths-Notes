\section{Classi laterali e gruppo quoziente}

Sia $(G, \cdot)$ un gruppo e sia $H \sgr G$. Consideriamo la seguente relazione sugli elementi di $G$: diciamo che $x \sim_L y$ se e solo se $y\inv x \in H$.

Questa relazione è una relazione di equivalenza, infatti \begin{itemize}
    \item $\sim_L$ è riflessiva: $x\inv x = e_G \in H$, dunque $x \sim_L x$.
    \item $\sim_L$ è simmetrica: se $x \sim_L y$, ovvero $y\inv x \in H$, allora il suo inverso $(y\inv x)\inv = x\inv (y\inv)\inv = x\inv y \in H$, dunque $y \sim_L x$.
    \item $\sim_L$ è transitiva: supponiamo che $x \sim_L y$ e $y \sim_L z$ e mostriamo che $x \sim_L z$. Dalla prima sappiamo che $y\inv x \in H$, mentre dalla seconda segue che $z\inv y \in H$. Dato che $H$ è un sottogruppo, il prodotto di suoi elementi è ancora in $H$, dunque \[
        z\inv y \cdot y\inv x = z\inv x \in H    
    \] da cui segue che $x \sim_L z$.
\end{itemize}

Questa relazione di equivalenza forma delle classi di equivalenza che partizionano $G$: in particolare la classe di $x \in G$ sarà della forma \begin{align*}
    \eqcl*{x}_L &= \set{g \in G \suchthat g \sim_L x}\\
    &= \set{g \in G \suchthat x\inv g \in H}\\
    &= \set{g \in G \suchthat x\inv g = h \text{ per qualche } h \in H}\\
    &= \set{g \in G \suchthat g = xh \text{ per qualche } h \in H}.
\end{align*}

Notiamo che gli elementi della classe di $x$ sono quindi tutti e soli gli elementi del sottogruppo $h$ moltiplicati a sinistra per $x$.
Diamo dunque la seguente definizione.
\begin{definition}
    [Classe laterale sinistra]
    Sia $(G, \cdot)$ un gruppo e $H \sgr G$ un suo sottogruppo. Sia inoltre $x \in G$.
    
    Allora si dice \emph{classe laterale sinistra di $H$ rispetto a $x$} l'insieme \[
        xH \deq \set{xh \suchthat h \in H}. 
    \]
\end{definition}

\begin{remark}
    Nel caso di gruppi additivi le classi laterali si scrivono in notazione additiva, ovvero nella forma $x + H$ per $x \in G$, $H \sgr G$.
\end{remark}

\begin{example}
    Ad esempio le classi laterali di $n\Z \sgr \Z$ sono della forma \[
        a + n\Z \deq \set{a + nk \suchthat k \in \Z}.
    \] La classe $a + n\Z$ denota tutti i numeri congrui ad $a$ modulo $n$.
\end{example}

Allo stesso modo possiamo definire un'altra relazione di equivalenza $\sim_R$ tale che \[
    x \sim_R y \iff xy\inv \in H.    
\] Le classi di equivalenza di questa relazione sono della forma \[
    \eqcl*{x}_R = \set{g \in G \suchthat g = hx \text{ per qualche } h \in H}.    
\] Possiamo dunque definire anche le classi laterali destre nel seguente modo.
\begin{definition}
    [Classe laterale destra]
    Sia $(G, \cdot)$ un gruppo e $H \sgr G$ un suo sottogruppo. Sia inoltre $x \in G$.
    
    Allora si dice \emph{classe laterale destra di $H$ rispetto a $x$} l'insieme \[
        Hx \deq \set{hx \suchthat h \in H}. 
    \]
\end{definition}

\begin{remark}
    Siccome le classi laterali sinistre (o destre) rappresentano le classi di equivalenza rispetto alla relazione $\sim_L$ (risp. $\sim_R$) possiamo definire un insieme di rappresentanti $R$ per cui \begin{equation}
        G = \bigdisjunion_{a \in R} aH. \quad \text{(risp. $Ha$)}
    \end{equation}
\end{remark}

\begin{theorem}
    [Teorema di Lagrange] \label{th:lagrange}
    Sia $(G, \cdot)$ un gruppo finito e sia $H \sgr G$ qualsiasi. Allora vale che \[
        \abs*{H} \divides \abs*{G}.    
    \]
\end{theorem}

In breve, il Teorema di Lagrange afferma che per ogni gruppo finito l'ordine di un suo qualsiasi sottogruppo divide l'ordine del gruppo. Prima di dimostrarlo, dimostriamo un lemma che ci tornerà utile.

\begin{lemma}\label{lem:ord_classilat=ord_sgr}
    Sia $(G, \cdot)$ un gruppo e sia $H$ un suo sottogruppo. Allora per qualsiasi $g \in G$ vale che \[
        \abs*{gH} = \abs*{H} = \abs*{Hg}.
    \]
\end{lemma}
\begin{proof}
    Per dimostrare che $\abs*{gH} = \abs*{H}$ consideriamo la mappa \begin{align*}
        \phi : H &\to gH \\   
        h &\mapsto gh 
    \end{align*} e facciamo vedere che è bigettiva.
    \begin{description}
        \item[Iniettività] Supponiamo che per qualche $h, k \in H$ valga che $\phi(h) = \phi(k)$, ovvero $gh = gk$. Siccome $gh, gk \in G$ vale la \hyperref[prop:prop_grp:canc:sx]{legge di cancellazione sinistra}, dunque segue che $h = k$, ovvero $\phi$ è iniettiva.
        \item[Surgettività] Segue naturalmente dalla definizione di $gH$. 
    \end{description}
    Dunque $\phi$ è bigettiva e quindi gli insiemi $gH$ e $H$ hanno la stessa cardinalità. Analogamente si mostra che la funzione \begin{align*}
        \psi : H &\to Hh \\   
        h &\mapsto hg 
    \end{align*} è bigettiva, dunque segue la tesi.
\end{proof}

Dimostriamo ora il Teorema di Lagrange
\begin{proof}[Dimostrazione del \autoref{th:lagrange}]
    Per l'osservazione precendente sappiamo che se $R$ è un insieme di rappresentanti della relazione di equivalenza $\sim_L$ allora \[
        G = \bigdisjunion_{a \in R} aH,
    \] dunque passando alle cardinalità \begin{align*}
        \abs*{G} &= \sum_{a \in R} \abs*{aH}.  \\
        \intertext{Per il \autoref{lem:ord_classilat=ord_sgr} segue quindi che}  
        &= \sum_{a \in R} \abs*{H}\\
        &= \abs*{R}\cdot \abs*{H}. 
    \end{align*}  Dunque $\abs*{H} \divides \abs*{G}$, dunque la tesi.
\end{proof}

\begin{remark}
    Osserviamo che in generale le classi laterali di un sottogruppo del gruppo $G$ non sono sottogruppi di $G$: dato che partizionano il gruppo una sola di esse contiene l'elemento neutro del gruppo.
\end{remark}

\begin{proposition}\label{prop:cond_laterale_è_sgrp}
    Sia $(G, \cdot)$ un gruppo, sia $H \sgr G$ e sia $g \in G$ qualsiasi. Allora i seguenti fatti sono equivalenti:
    \begin{enumerate}[label={(\roman*)}]
        \item $gH \sgr G$,
        \item $g \in H$,
        \item $H = gH$.
    \end{enumerate}
\end{proposition}
\begin{proof}
    Dimostriamo la catena di implicazioni $(i) \implies (ii) \implies (iii) \implies (i)$.
    \begin{description}
        \item[($(i) \implies (ii)$)] Supponiamo che $gH \sgr G$. Allora $e_G \in gH$, ovvero esiste $h \in H$ tale che $gh = e_G$. Ma tale $h$ è $g\inv$, dunque se $g\inv \in H$ segue che $g \in H$.
        \item[($(ii) \implies (iii)$)] Supponiamo che $g \in H$. 
        \begin{description}
            \item[($gH \subseteq H$)] Supponiamo $gh \in gH$ per qualche $h \in H$. Ma essendo $g \in H$ per ipotesi il prodotto $gh$ sarà un elemento di $H$, dunque $gH \subseteq H$.
            \item[($H \subseteq gH$)] Sia $h \in H$. Siccome $g \in H$ e $H$ è un gruppo segue che $g\inv \in H$, dunque $g\inv h \in H$. Ma questo significa che $g\cdot(g\inv h) = h \in gH$, dunque $H \subseteq gH$.
        \end{description}
        Concludiamo che $gH = H$.
        \item[($(iii) \implies (i)$)] Siccome $gH = H$ e $H \sgr G$ allora $gH \sgr G$. \qedhere 
    \end{description}
\end{proof}

Siccome ogni elemento di una classe è un possibile rappresentante della classe stessa, la proposizione precedente ci dice che l'unica classe laterale (sinisra) di $H$ che è un sottogruppo di $G$ è quella che contiene l'identità, ovvero la classe $e_GH = H$.

\begin{corollary}[Corollario al Teorema di Lagrange] \label{cor:lagrange}
    Sia $(G, \cdot)$ un gruppo finito. Allora valgono i seguenti fatti:
    \begin{enumerate}[label={(\roman*)}, ref={\thecorollary: (\roman*)}]
        \item \label{cor:ord_el_divide_ord_gruppo} per ogni $g \in G$ vale che $\ord[G]{g} \divides \abs*{G}$,
        \item \label{cor:x_alla_ordG=e_G} per ogni $x \in G$ vale che $x^{\abs*{G}} = e_G$.
    \end{enumerate}
\end{corollary}
\begin{proof}
    \begin{enumerate}[label={(\roman*)}]
        \item Siccome $\gen*{g} \sgr G$, per il \nameref{th:lagrange} vale che \[
            \ord[G]{g} = \abs*{\gen*{g}} \divides \abs*{G}.    
        \]
        \item Sia $n \deq \abs*{G}$ e $k \deq \ord[G]{g}$. Per il punto precedente vale che $k \divides n$, ovvero che esiste $m \in \Z$ tale che \[
            n = km.    
        \] Dunque segue che \begin{align*}
            g^{\abs*{G}} &= g^n \\
            &= (g^k)^m \tag{per def. di \hyperref[def:ord_grp]{ordine}}\\
            &= e^m \\
            &= e. \tag*{\qedhere}
        \end{align*}
    \end{enumerate}
\end{proof}

\begin{corollary}[I gruppi di ordine primo sono ciclici]
    Sia $(G, \cdot)$ un gruppo tale che $\abs*{G} = p$ per qualche $p \in \Z$, $p$ primo. Allora $G$ è ciclico ed in particolare \[
        G \isomorph \Zmod{p}.    
    \]
\end{corollary}
\begin{proof}
    Sia $x \in G$, $x \neq e_G$. Allora $\gen*{x} \neq \set{e_G}$, da cui segue che \[
        1 \neq \ord[G]{x} \divides p = \abs*{G}.
    \] Dunque per definizione di numero primo $\ord[G]{x} = p$, ma siccome l'ordine del sottogruppo $\gen*{x}$ è uguale all'ordine di $G$ segue che $G = \gen*{x}$.
    
    Dunque $G$ è ciclico e per il \autoref{th:iso_ciclico} è isomorfo a $\Zmod{p}$.
\end{proof}

Il teorema di Lagrange ci consente inoltre di dimostrare molto semplicemente il Teorema di Eulero-Fermat.
\begin{proof}
    Segue dal \autoref{cor:lagrange} (in particolare dal \hyperref[cor:x_alla_ordG=e_G]{punto (ii)}) considerando come gruppo $(\invert{\Zmod{n}}, \cdot)$: infatti per definizione $\phi(n) = \abs*{\invert{\Zmod{n}}}$, da cui la tesi.
\end{proof}

\subsection{Sottogruppi normali e gruppo quoziente}

\begin{definition}
    [Sottogruppo normale] \label{def:sgr_normale}
    Sia $(G, \cdot)$ un gruppo e sia $H \sgr G$. Allora si dice che $H$ è un \emph{sottogruppo normale} di $G$ se per ogni $g \in G$ vale che \begin{equation} \label{eq:def_normale}
        gH = Hg.
    \end{equation} 
    
    Se $H$ è normale si scrive $H \normal G$.
\end{definition}

\begin{remark}
    Se $G$ è abeliano allora tutti i suoi sottogruppi sono normali.
\end{remark}
\begin{remark}
    Se un sottogruppo $H$ è normale non significa che per ogni $h \in H$ vale che $gh = hg$, ma soltanto che per ogni $h \in H$ esiste un $h^\prime \in H$ tale che \[
        gh = h^\prime g.    
    \]
\end{remark}

\begin{proposition} \label{prop:normale_sse_chiuso_per_coniugio}
    Sia $(G, \cdot)$ un gruppo e $H \sgr G$.
    Allora $H$ è normale se e solo se è chiuso per coniugio, ovvero se e solo se per ogni $g \in G$ vale che \[
        gHg\inv \subseteq H.    
    \]
\end{proposition}
\begin{proof}
    Mostriamo entrambi i versi dell'implicazione.
    \begin{description}
        \item[($\implies$)] Supponiamo che $H \normal G$, ovvero che per ogni $g \in G$ vale che \[
            gH = Hg,
        \] ovvero per ogni $h \in H$ esiste un $h^\prime \in H$ tale che \[
            gh = h^\prime g.    
        \] Moltiplicando a destra per $g\inv$ si ottiene che \[
            ghg\inv = h^\prime \in H,   
        \] da cui $gHg\inv \subseteq H$.
        % \item[($\impliedby$)] Supponiamo che $ghg\inv \in H$, ovvero esista $h^\prime$ tale che $ghg\inv = h^\prime$, il che è equivalente ad affermare $gh = h^\prime g \in Hg$. Questo significa che $gH \subseteq Hg$. Mostriamo ora che vale anche l'inclusione contraria.
        
        % Dato che la relazione deve valere per quasliasi $g$, dovrà valere anche per $g\inv \in G$
    \end{description}
\end{proof}

\begin{proposition}[Il centro è un sottogruppo normale]
    \label{prop:centro_normale}
    Sia $(G, \cdot)$ un gruppo. Allora vale che \[
        Z(G) \normal G.    
    \]
\end{proposition}

\begin{definition}
    [Indice di un sottogruppo]
    Sia $(G, \cdot)$ un gruppo e sia $H \sgr G$. Allora si dice \emph{indice di $H$ in $G$} il numero di classi laterali sinistre di $H$, e si indica con \begin{equation}
        \grindex{G}{H}.
    \end{equation}
\end{definition}

\begin{proposition}\label{prop:norm_se_indice2}
    Sia $(G, \cdot)$ un gruppo, $H \sgr G$. Allora se $\grindex{G}{H} = 2$ segue che $H \normal G$.
\end{proposition}

\begin{proposition}
    [Nucleo di omomorfismi e normalità]
    \label{prop:rel_kernel_sgr_normali}
    Siano $(G, \cdot)$, $(G^\prime, *)$ due gruppi e sia $f : G \to G^\prime$ un omomorfismo. 
    
    Valgono le seguenti affermazioni.
    \begin{enumerate}[label={(\roman*)}]
        \item $\ker f \normal G$,
        \item per ogni $x, y \in G$ vale che $f(x) = f(y)$ se e solo se $x\ker f = y\ker f$, ovvero se $x, y$ appartengono alla stessa classe laterale del nucleo,
        \item se $z \in \Imm f$ (ovvero $f(x) = z$ per qualche $x \in G$) allora $f\inv(\set{z}) = x\ker f$.
    \end{enumerate}
\end{proposition}
\begin{proof}
    \begin{enumerate}[label={(\roman*)}]
        \item Per la \autoref{prop:normale_sse_chiuso_per_coniugio} la tesi è equivalente a dimostrare che \[
            g(\ker f) g\inv \subseteq \ker f   
        \] per ogni $g \in G$.
        
        Sia $x \in \ker f$ qualsiasi: mostriamo che $gxg\inv \in \ker f$. Per definizione di kernel, questo significa mostrare che $f(gxg\inv) = e_G$, ovvero (siccome $f$ è un omomorfismo) \[
            f(g) * f(x) * f(g\inv) = e_G.    
        \] Per ipotesi $x \in \ker f$, dunque $f(x) = e_G$; inoltre per la \hyperref[prop:prop_omo:inv_passa_dentro]{Proposizione \ref*{prop:prop_omo:inv_passa_dentro}} sappiamo che $f(g\inv) = f(g)\inv$.
        
        Dunque segue che \begin{align*}
            f(g) * f(x) * f(g\inv) &= f(g) * e_G * f(g)\inv\\
            &= f(g) * f(g)\inv \\
            &= e_G
        \end{align*} che è la tesi.
        \item Supponiamo $f(x) = f(y)$. Moltiplicando a destra per $f(y)\inv$ segue che \begin{align*}
            &f(x) * f(y)\inv = e_G \\
            \iff &f(x) * f(y\inv) = e_G \\
            \iff &f(xy\inv) = e_g\\
            \iff &xy\inv \in \ker f\\
            \iff &x \sim_L y.
        \end{align*}
        
        Dunque le classi di equivalenza di $x$ e $y$ sono uguali, ovvero \[
            x\ker f = y\ker f.    
        \]
        \item Per definizione di controimmagine: \begin{align*}
            f\inv(z) &= \set{g \in G \suchthat f(g) = z} \tag{hp: $f(x) = z$} \\
            &= \set{g \in G \suchthat f(g) = f(x)} \tag{per il punto (ii)}\\
            &= x\ker f. \tag*{\qedhere}
        \end{align*}
    \end{enumerate}
\end{proof}

Consideriamo ora l'insieme di tutte le possibili classi laterali sinistre di un sottogruppo $H \sgr G$ e chiamiamo questo insieme $\quot{G}{H}$: \begin{equation}
    \quot{G}{H} \deq \set{gH \suchthat g \in G}.
\end{equation}

Se $H \normal G$ possiamo definire un'operazione su $\quot{G}{H}$: \begin{align} \label{eq:op_gruppo_quoziente}
    \begin{split}
        \cdot : \quot{G}{H} \times \quot{G}{H} &\to \quot{G}{H} \\
        (aH, bH) &\mapsto abH.
    \end{split}
\end{align}

La struttura $(\quot{G}{H}, \cdot)$ si definisce \emph{gruppo quoziente}.

\begin{proposition}
    Sia $(G, \cdot)$ un gruppo e sia $N \normal G$. Allora la struttura $(\quot{G}{N}, \star)$ (dove l'operazione è definita come in \eqref{eq:op_gruppo_quoziente}) è un gruppo.
\end{proposition}
\begin{proof}
    Mostriamo innanzitutto che l'operazione $*$ è ben definita.
    Supponiamo che $xN = x^\prime N$ e $yN = y^\prime N$ e mostriamo che $xyN = x^\prime y^\prime N$.

    Siano $n_1, n_2$ tali che \[
        x^\prime = xn_1, \quad y^\prime = yn_2.    
    \] Allora vale che \begin{align*}
        x^\prime y^\prime &= xn_1 yn_2.\\
        \intertext{Siccome $N \normal G$ segue che $Ny = yN$, ovvero che esiste un $n_3 \in N$ tale che $n_1y = yn_3$. Dunque}
        &= xy n_3n_2 \tag{$N$ è chiuso rispetto a $\cdot$}\\
        &\in xyN.
    \end{align*}
    Per simmetria dunque $xyN = x^\prime y^\prime N$.

    Mostriamo ora che valgono gli assiomi di gruppo.
    \begin{description}
        \item[Associatività] Siano $xN, yN, zN \in \quot{G}{N}$. Mostriamo che vale la proprietà associativa.
        \begin{align*}
            xN \star (yN \star zN) &= xN \star yzN \\
            &= x(yz)N \tag{ass. in $G$}\\
            &= (xy)zN \\
            &= xyN \star zN \\
            &= (xN \star yN) \star zN.
        \end{align*}
        \item[Elemento neutro] L'elemento neutro del gruppo è $e_GN$. Infatti per qualsiasi $xN \in \quot{G}{N}$ \begin{align*}
            &e_GN \star xN = e_GxN = xN.\\
            &xN \star e_GN = xe_GN = xN.
        \end{align*}
        \item[Invertibilità] Sia $xN \in \quot{G}{N}$. Mostriamo che il suo inverso rispetto a $\star$ è $x\inv N$.
        \begin{align*}
            &xN \star x\inv N = xx\inv N = e_GN.\\
            &x\inv N \star xN = x\inv xN = e_GN.
        \end{align*} 
    \end{description}
    Dunque $(\quot{G}{N}, \star)$ è un gruppo.
\end{proof}

\begin{example}
    Se consideriamo il gruppo $\Z$ e il suo sottogruppo normale $n\Z$ il gruppo quoziente $\Zmod{n}$ è esattamente il gruppo delle classi resto modulo $n$.
\end{example}

\begin{proposition}\label{prop:kernel_proiezione}
    Sia $(G, \cdot)$ un gruppo e sia $N \normal G$. Allora la mappa \begin{align}
        \begin{split}
            \pi_N : G &\to \quot{G}{N}\\
            x &\mapsto xN
        \end{split}
    \end{align}
    è un omomorfismo di gruppi e $\ker \pi_N = N$.
\end{proposition}
\begin{proof}
    Mostriamo innanzitutto che $\pi_N$ è un omomorfismo. \begin{align*}
        \pi_N(xy) &= xyN \\
        &= xN \cdot yN\\
        &= \pi_N(x) \cdot \pi_N(y).
    \end{align*}
    Inoltre per definizione \begin{align*}
        \ker \pi_N &= \set{x \in G \suchthat \pi_N(x) = xN = N} \\
        &= \set{x \in G \suchthat x \in N}\\
        &= N,  
    \end{align*} dove il secondo segno di uguaglianza viene dalla \autoref{prop:cond_laterale_è_sgrp} (in particolare per l'equivalenza tra i punti (ii) e (iii)).
\end{proof}

\begin{corollary}
    I sottogruppi normali di $G$ sono tutti e solo i nuclei degli omomorfismi definiti su $G$.
\end{corollary}
\begin{proof}
    Infatti se $N \normal G$ allora per la \autoref{prop:kernel_proiezione} segue che $N = \ker \pi_N$; invece dato un omomorfismo di gruppi $\phi : G \to G^\prime$ vale che $\ker \phi$ è normale per la \autoref{prop:rel_kernel_sgr_normali}.
\end{proof}