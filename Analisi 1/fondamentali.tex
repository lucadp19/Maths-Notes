\chapter{Fondamentali}

\section{Relazioni}

\begin{definition}
    [Relazione su un insieme]
    Sia $X$ un insieme. Allora si dice \emph{relazione su $X$} un sottoinsieme $R \subseteq X \times X$. 
    
    Si indica che una coppia $(x, y)$ soddisfa $R$ scrivendo $xRy$. 
\end{definition}

\begin{definition}
    [Relazione di equivalenza]
    Sia $X$ un insieme e $\sim$ una relazione su $X$. Allora $\sim$ si dice \emph{relazione di equivalenza} se valgono i seguenti assiomi: \begin{enumerate}[label={(EQ\arabic*)}]
        \item La relazione $\sim$ è \emph{riflessiva}:
        
        per ogni $x \in X$ vale che $x \sim x$.
        \item La relazione $\sim$ è \emph{simmetrica}:
        
        per ogni $x, y \in X$, se $x \sim y$ allora necessariamente $y \sim x$.
        \item La relazione $\sim$ è \emph{transitiva}:
        
        per ogni $x, y, z \in X$, se $x \sim y$ e $y \sim z$ allora necessariamente $x \sim z$.
    \end{enumerate}
\end{definition}

\begin{definition}
    [Relazione di ordinamento]
    Sia $X$ un insieme e $\leq$ una relazione su $X$. Allora $\leq$ si dice \emph{relazione di ordinamento} se valgono i seguenti assiomi: \begin{enumerate}[label={(ORD\arabic*)}]
        \item La relazione $\leq$ è \emph{riflessiva}:
        
        per ogni $a \in \K$ vale che $a \leq a$.
        \item La relazione $\leq$ è \emph{antisimmetrica}:
        
        per ogni $a, b \in \K$, se $a \leq b$ e $b \leq a$ allora necessariamente $a = b$.
        \item La relazione $\leq$ è \emph{transitiva}:
        
        per ogni $a, b, c \in \K$, se $a \leq b$ e $b \leq c$ allora necessariamente $a \leq c$.
    \end{enumerate}

    In particolare l'ordinamento si dice \emph{totale} se vale anche che
    \begin{enumerate}[label={(O\arabic*)}, start=4]
        \item La relazione $\leq$ è \emph{totale}:
        
        per ogni $a, b \in \K$ vale che $a \leq b$ oppure $b \leq a$.
    \end{enumerate}
\end{definition}

\section{Funzioni}

\begin{definition}[Funzione (operativamente)]
    \label{def:funz_oper}
    Si dice funzione una terna $(A, B, f)$ formata da:
    \begin{itemize}
        \item un insieme $A$ detto \emph{dominio};
        \item un insieme $B$ detto \emph{codominio};
        \item una legge $f : A \to B$ che associa ad ogni $a \in A$ un elemento $f(a) \in B$.
    \end{itemize}
\end{definition}

\begin{definition}
    [Grafico di una funzione] \label{def:graph}
    Sia $f : A \to B$ una funzione. Allora si dice \emph{grafico della funzione} l'insieme \[
        \operatorname{graph}(f) = \set{(a, b) \in A \times B \suchthat b = f(a)}.    
    \]
\end{definition}

\begin{definition}
    [Funzione (rigorosamente)]
    Si dice funzione da $A$ a $B$ un qualunque sottoinsieme $G \subseteq A \times B$ tale che \[
        \forall a \in A.\quad \exists! b \in B. \quad (a, b) \in G.    
    \]

    Per ogni $a \in A$ si definisce quindi $f(a)$ come l'unico $b$ che rispetta la condizione sopra.
\end{definition}

\begin{definition}
    [Funzione composta]
    Siano $f : A \to B$, $g : B \to C$ due funzioni. Allora si dice \emph{funzione composta} la funzione $(g \circ f) : A \to C$ tale che per ogni $a \in A$ \[
        (g \circ f)(a) = g(f(a)).    
    \]
\end{definition}

\subsection{Iniettività, surgettività}

\begin{definition}
    [Funzione iniettiva]
    Sia $f : A \to B$ una funzione. La funzione $f$ si dice \emph{iniettiva} se \[
        \forall a_1, a_2 \in A. \quad a_1 \neq a_2 \implies f(a_1) \neq f(a_2)    
    \] o equivalentemente
    \[
        \forall a_1, a_2 \in A. \quad f(a_1) = f(a_2) \implies a_1 = a_2.  
    \]
\end{definition}

\begin{definition}
    [Funzione surgettiva]
    Sia $f : A \to B$ una funzione. Allora $f$ si dice \emph{surgettiva} se \[
        \forall b \in B. \quad \exists a \in A. \quad f(a) = b.    
    \]
\end{definition}

\begin{definition}
    [Funzione bigettiva]
    Sia $f : A \to B$ una funzione. Allora $f$ si dice \emph{bigettiva} se:
    \begin{enumerate}[label={(\roman*)}]
        \item $f$ è iniettiva;
        \item $f$ è surgettiva.
    \end{enumerate}

    In tal caso la funzione risulta \emph{invertibile}, ovvero esiste \[
        g : B \to A
    \] tale che per ogni $a \in A$, $b \in B$ vale che
    \begin{align*}
        g(f(a)) &= a, \\
        f(g(b)) &= b.
    \end{align*}
    Se l'inversa esiste, allora si indica con $f\inv : B \to A$.
\end{definition}

\begin{proposition}
    Siano $f : A \to B$, $g : B \to C$ due funzioni.
    Allora \begin{enumerate}[label={(\roman*)}]
        \item se $f$, $g$ sono iniettive, allora $(g \circ f)$ è iniettiva;
        \item se $f$, $g$ sono surgettive, allora $(g \circ f)$ è surgettiva;
        \item se $(g \circ f)$ è iniettiva, allora $f$ è iniettiva;
        \item se $(g \circ f)$ è surgettiva, allora $g$ è surgettiva.
    \end{enumerate}
\end{proposition}
\begin{proof}
    Dimostriamo le quattro affermazioni. \begin{enumerate}
        \item Siano $a_1, a_2 \in A$ con $a_1 \neq a_2$. Allora \[
            a_1 \neq a_2 \overbrace{\implies}^{f \text{ iniett.}} f(a_1) \neq f(a_2) \overbrace{\implies}^{g \text{ iniett.}}  g(f(a_1)) \neq g(f(a_2)).
        \]
    \end{enumerate}
\end{proof}

\subsection{Immagine e controimmagine}

\begin{definition}
    [Immagine di un sottoinsieme]
    Sia $f : A \to B$ una funzione e sia $E \subseteq A$. Allora si dice\emph {immagine di $E$ attraverso $f$} l'insieme\[
        f(E) \deq \set{f(a) \suchthat a \in E} \subseteq B    
    \]
    o equivalentemente \[
        f(E) \deq \set{b \in B \suchthat \exists a \in E \quad b = f(a)}.    
    \]
\end{definition}


\begin{definition}
    [Immagine di una funzione]
    Sia $f : A \to B$ una funzione. Allora si dice \emph{immagine di $f$} l'insieme \[
        \Imm{f} \deq f(A) = \set{f(a) \suchthat a \in A} \subseteq B.
    \]
\end{definition}

\begin{remark}
    Una funzione $f : A \to B$ è surgettiva se e solo se \[
        \Imm{f} = B.    
    \]
\end{remark}

\begin{definition}
    [Controimmagine di un sottoinsieme]
    Sia $f : A \to B$ una funzione e sia $F \subseteq B$. Allora si dice \emph{controimmagine di $F$ attraverso $f$} l'insieme \[
        f\inv(F) \deq \set{a \in A \suchthat f(a) \in F} \subseteq A.    
    \]
\end{definition}

\section{Numeri naturali e Induzione}

\begin{unnamed}
    [Assiomi di Peano]
    Si dice \emph{insieme dei numeri naturali} l'insieme $\N$ che rispetta i seguenti 5 assiomi: \begin{enumerate}[label={(P\arabic*)}, ref={(P\arabic*)}]
        \item Esiste il numero naturale $0$, ovvero $0 \in \N$.
        \item Esiste una funzione $\sigma : \N \to \N$, detta \emph{successore}, definita per ogni naturale.
        \item La funzione $\sigma$ è iniettiva, ovvero per ogni $n, m \in \N$, $n \neq m$ segue che $\sigma(n) \neq \sigma(m)$.
        \item Nessun numero ha $0$ come proprio successore, ovvero $\sigma(n) \neq 0$ per ogni $n \in \N$. 
        \item Se $A$ è un sottoinsieme di $\N$ tale che \begin{enumerate}
            \item $0 \in A$
            \item $n \in A$ implica $\sigma(n) \in A$
        \end{enumerate}
        allora $A = \N$.
    \end{enumerate}
\end{unnamed}

La terna $(\N, 0, \sigma)$ dove $\sigma(n) \deq n+1$ è l'unica terna che rispetta gli assiomi di Peano a meno di isomorfismi: ogni altra terna $(X, a, \Sigma)$ è isomorfa ad essa.

Il quinto assioma, detto \emph{Principio di Induzione}, può essere riformulato in modo da essere usato più comodamente nelle dimostrazioni. La formulazione alternativa è la seguente: 

\begin{unnamed}[Principio di Induzione]
    Sia $A = \set{n \in \N \suchthat n \geq n_0}$, $P : A \to \set{\TT, \FF}$ un predicato. Allora se \begin{enumerate}[label={(\roman*)}]
        \item vale $P(n_0)$
        \item $\forall n \geq n_0 \quad P(n) \implies P(n + 1)$
    \end{enumerate}
    segue che $P(n)$ vale per ogni $n \geq n_0$.
\end{unnamed}

\begin{example}
    Dimostrare che per ogni $n \geq 0$ \begin{align}
        &\sum_{k=0}^n k = \frac{n(n+1)}{2} \label{eq:somma_lineare}\\
        &\sum_{k=0}^n k^2 = \frac{n(n+1)(2n+1)}{6} \label{eq:somma_quadrati}\\
        &\sum_{k=0}^n a^k = \frac{a^{k+1}-1}{a-1} &&\forall a \neq 1. \label{eq:somma_geometrica}
    \end{align}
\end{example}
\begin{proof}
    Dimostriamo la $\ref{eq:somma_quadrati}$ per induzione su $n$.
    \begin{description}
        \item[Caso base] Sia $n = 0$. Allora \[
            \sum_{k=0}^0 k^2 = 0 = \frac{0 \cdot 1 \cdot 1}{2}.  
        \]
        \item[Passo induttivo] Supponiamo che valga la tesi per $n$ e dimostriamola per $n+1$.
        \begin{align*}
            \sum_{k=0}^{n+1} k^2 &= (n+1)^2 + \sum_{k=0}^n k^2 \\
            &= (n+1)^2 + \frac{n(n+1)(2n+1)}{6}\\
            &= \frac{(n+1)^2 + n(n+1)(2n+1)}{6}\\
            &= \frac{(n+1)(n+1 + 2n^2 + n)}{6}\\
            &= \frac{(n+1)(2n^2 + 2n + 1)}{6}\\
            &= \frac{(n+1)(n+2)(2n+2)}{6}.
        \end{align*} 
    \end{description}
    Dunque per il principio di induzione la tesi vale per ogni $n \in \N$.
\end{proof}

% \begin{proposition}
%     [Disuguaglianza di Bernoulli]\label{eq:bernoulli}
%     \[

%     \]
% \end{proposition}