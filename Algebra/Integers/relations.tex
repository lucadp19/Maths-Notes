\chapter{I numeri interi}

\section{Relazioni}

\begin{definition}
    [Relazione su un insieme]
    Sia $X$ un insieme. Allora si dice \emph{relazione su $X$} un sottoinsieme $R \subseteq X \times X$. 
    
    Le coppia $(x, y) \in R$ \emph{soddisfano} $R$, e si scrive anche $xRy$. 
\end{definition}

\begin{definition}
    [Relazione di equivalenza]
    Sia $X$ un insieme e $\sim$ una relazione su $X$. Allora $\sim$ si dice \emph{relazione di equivalenza} se valgono i seguenti assiomi: \begin{enumerate}[label={(EQ\arabic*)}]
        \item La relazione $\sim$ è \emph{riflessiva}:
        
        per ogni $x \in X$ vale che $x \sim x$.
        \item La relazione $\sim$ è \emph{simmetrica}:
        
        per ogni $x, y \in X$, se $x \sim y$ allora necessariamente $y \sim x$.
        \item La relazione $\sim$ è \emph{transitiva}:
        
        per ogni $x, y, z \in X$, se $x \sim y$ e $y \sim z$ allora necessariamente $x \sim z$.
    \end{enumerate}
\end{definition}

Un esempio di relazione di equivalenza è la relazione di uguaglianza tra numeri: diciamo che due numeri $a, b$ sono uguali (e lo scriviamo $a = b$) se sono lo stesso numero. Questa relazione verifica molto semplicemente tutti gli assiomi delle relazioni di equivalenza, ma ci sono altre relazioni di equivalenza che non siano l'uguaglianza.

\begin{definition}
    [Classi di equivalenza]
    Sia $X$ un insieme e $\sim$ una relazione di equivalenza su $X$. Sia inoltre $a \in X$ qualsiasi. Allora si dice \emph{classe di equivalenza di $a$} l'insieme di tutti gli elementi di $X$ che sono in relazione con $a$, ovvero: \begin{equation}
        \eqcl*{a} = \set{x \in X \suchthat a \sim x}.
    \end{equation}
\end{definition}

La relazione di equivalenza divide quindi l'insieme in classi di equivalenza, ognuna delle quali racchiude tutti gli elementi "identificabili tra loro", nel senso che sono in relazione l'uno con l'altro.

Mostriamo ora che le classi di equivalenza formano una partizione dell'insieme.

\begin{lemma}
    [Le classi sono o disgiunte o coincidenti]
    Sia $X$ un insieme, $a, b \in X$ e $\sim$ una relazione di equivalenza su $X$.

    Allora \begin{enumerate}
        \item se $a \nsim b$ segue che $\eqcl*{a} \inters \eqcl*{b} = \varnothing$;
        \item se $a \sim b$ segue che $\eqcl*{a} = \eqcl*{b}$.
    \end{enumerate}
\end{lemma}
\begin{proof}
    Supponiamo $a \nsim b$ e supponiamo per assurdo esista $x \in \eqcl*{a} \inters \eqcl*{b}$, ovvero $x \sim a$ e $x \sim b$. Per simmetria la prima delle due relazioni può essere scritta come $a \sim x$, dunque per transitività segue che $a \sim b$. Ma questo è assurdo per ipotesi, dunque le due classi sono disgiunte.

    Ora supponiamo $a \sim b$. Supponiamo per assurdo esista qualche $y \in X$ che appartiene alla classe di $a$ ma non alla classe di $b$. Allora $y \sim a$, ma dato che $a \sim b$ per transitività segue che $y \sim b$, il che è assurdo. Dunque le due classi coincidono.
\end{proof}

\begin{theorem}
    [Le classi di equivalenza partizionano l'insieme] 
    Sia $X$ un insieme e $\sim$ una relazione di equivalenza su $X$.

    Allora l'insieme delle classi di equivalenza forma una partizione dell'insieme, ovvero classi distinte sono disgiunte e la loro unione è l'intero insieme: \[
        X = \bigunion_{a \in X} \eqcl*{a}.    
    \]
\end{theorem}

Possiamo considerare quindi l'insieme formato da tutte le classi di equivalenza indotte dalla relazione $\sim$ su $X$.

\begin{definition}
    [Insieme quoziente]
    Sia $X$ un insieme e $\sim$ una relazione di equivalenza su $X$. Allora si definisce \emph{insieme quoziente} l'insieme \begin{equation}
        X/_{\!\sim} \deq \set{\eqcl*{a} \suchthat a \in X}.
    \end{equation}
\end{definition}

Notiamo che anche se alcune classi coincidono, dato che l'insieme quoziente è un insieme esse compariranno una singola volta.

Diamo ora un altro tipo di relazione su insiemi.

\begin{definition}
    [Relazione di ordinamento]
    Sia $X$ un insieme e $\leq$ una relazione su $X$. Allora $\leq$ si dice \emph{relazione di ordinamento} se valgono i seguenti assiomi: \begin{enumerate}[label={(ORD\arabic*)}]
        \item La relazione $\leq$ è \emph{riflessiva}:
        
        per ogni $a \in \K$ vale che $a \leq a$.
        \item La relazione $\leq$ è \emph{antisimmetrica}:
        
        per ogni $a, b \in \K$, se $a \leq b$ e $b \leq a$ allora necessariamente $a = b$.
        \item La relazione $\leq$ è \emph{transitiva}:
        
        per ogni $a, b, c \in \K$, se $a \leq b$ e $b \leq c$ allora necessariamente $a \leq c$.
    \end{enumerate}

    In particolare l'ordinamento si dice \emph{totale} se vale anche che
    \begin{enumerate}[label={(O\arabic*)}, start=4]
        \item La relazione $\leq$ è \emph{totale}:
        
        per ogni $a, b \in \K$ vale che $a \leq b$ oppure $b \leq a$.
    \end{enumerate}
\end{definition}

Esempi tipici di relazioni di ordinamento sono l'ordinamento tra numeri $\leq$ e l'inclusione tra insiemi $\subseteq$ (che è un \emph{ordinamento parziale}).