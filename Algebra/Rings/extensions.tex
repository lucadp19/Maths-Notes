\section{Estensioni di campi}

\begin{definition}
    [Estensione di campi] Siano $K, F$ campi con $K \subseteq F$. Allora $F$ si dice \emph{estensione} di $K$ e l'estensione si indica con $\ext{F}{K}$.
\end{definition}

\begin{definition}
    [Elementi algebrici e trascendenti] Sia $\ext{F}{K}$ un'estensione di campi. $\alpha \in F$ si dice \emph{algebrico su $K$} se esiste un polinomio $f \in \K[X] \setminus \set{0_{\K[X]}}$ tale che $f(\alpha) = 0$. Se $\alpha$ non è algebrico si dice \emph{trascendente}.
\end{definition}

\begin{definition}
    [Estensioni algebriche] Sia $\ext{F}{K}$ un'estensione di campi. $\ext{F}{K}$ si dice \emph{algebrica} se ogni $\alpha \in F$ è algebrico su $K$.
\end{definition}

Dato un valore $\alpha \in F$ possiamo valutare tutti i polinomi di $K[X]$ in $\alpha$ per verificare il loro valore: l'immagine di questa funzione è l'insieme \[
    K[\alpha] \deq \set{f(\alpha) \suchthat f \in K[X]}.    
\] Siccome $f(\alpha) \in F$ questo insieme è un sottoinsieme di $F$. In particolare essendo un sottoinsieme di un campo possiamo indurre due operazioni sugli elementi di $K[\alpha]$ (una somma e un prodotto) che si comportano esattamente come si comportano in $F$. Vale quindi la seguente proposizione.

\begin{proposition}
    Sia $\ext{F}{K}$ un'estensione di campi, $\alpha \in F$. Allora $(K[\alpha], +, \cdot)$ è un anello.
\end{proposition}

La funzione che porta ogni elemento di $K[X]$ nella sua valutazione in $\alpha$ si dice \emph{omomorfismo di valutazione}, ed è definito da \begin{align*}
    \phi_\alpha : K[X] &\to K[\alpha] \subseteq F\\   
    f(X) &\mapsto f(\alpha).
\end{align*} Esso è un omomorfismo tra l'anello dei polinomi $K[X]$ e l'anello $K[\alpha]$. Osserviamo inoltre che è un omomorfismo surgettivo, in quanto $K[\alpha] = \Imm \phi_\alpha$.

\begin{proposition}\label{prop:KX/ker_valutazione_isomorfo_Kalpha}
    Sia $\ext{F}{K}$ un'estensione di campi, $\alpha \in F$. Allora vale che \[
        \quot{K[X]}{\ker{\phi_\alpha}} \isomorph K[\alpha].    
    \]
\end{proposition}
\begin{proof}
    Consideriamo i gruppi additivi $K[X]$ e $K[\alpha]$ insieme all'omomorfismo $\phi_\alpha$ e alla proiezione canonica sul quoziente. Per il \hyperref[th:first_iso]{Primo Teorema degli Omomorfismi} vale che \[
        \begin{tikzcd}
            K[X] \arrow[d, swap, hook, "\pi_{\ker \phi_\alpha}"] \arrow[r, two heads, "\phi_\alpha"] & K[\alpha] \\
            \quot{K[X]}{\ker \phi_\alpha} \arrow[ur, hook, two heads, swap, "\bar\phi"] &
        \end{tikzcd}
    \]

    Innanzitutto osservo che \[
        \phi_\alpha(f(X)) = \parens*{\bar\phi \circ \pi_{\ker{\phi_\alpha}}}\parens*{f(X)} 
        = \bar\phi(f(X) + \ker\phi_\alpha) = \bar\phi\parens*{\eqcl{f(X)}}.
    \] Da questo possiamo verificare immediatamente che $\bar\phi$ è un omomorfismo di anelli:
    \begin{itemize}
        \item $\bar\phi\parens*{\eqcl 1} = \phi_\alpha(1) = 1$ poiché $\phi_\alpha$ è un omomorfismo di anelli;
        \item $\bar\phi \parens*{\eqcl{a(X)}\cdot\eqcl{b(X)}} = \bar\phi\parens*{\eqcl{a(X)}} \cdot \bar\phi\parens*{\eqcl{b(X)}}$ poiché: \begin{align*}
            \bar\phi \parens*{\eqcl{a(X)}\cdot\eqcl{b(X)}} &= \bar\phi\parens*{\eqcl{a(X) \cdot b(X)}} \\
            &= \phi_\alpha(a(X) \cdot b(X))\\
            &= \phi_\alpha(a(X)) \cdot \phi_\alpha(b(X))\\
            &= \bar\phi\parens*{\eqcl{a(X)}} \cdot \bar\phi\parens*{\eqcl{b(X)}}.
        \end{align*}
    \end{itemize} Notiamo che non c'è bisogno di verificare che $\bar\phi$ rispetti la struttura di gruppo additivo poiché sappiamo già che è un omomorfismo di gruppi.

    Siccome il quoziente è sul nucleo di $\phi_\alpha$ e $\phi_\alpha$ è surgettiva segue che $\bar\phi$ è bigettiva, dunque è un isomorfismo di anelli, da cui segue che la tesi.
\end{proof}

\begin{remark}
    L'omomorfismo di valutazione ci consente di descrivere gli elementi algebrici e quelli trascendenti sfruttando le proprietà degli omomorfismi. Infatti \[
        \ker \phi_\alpha = \set{f(X) \in K[X] \suchthat \phi_\alpha(f(X)) = f(\alpha) = 0}.    
    \] Dunque un elemento $\alpha \in F$ è algebrico su $K$ se e solo se $\ker \phi_\alpha \neq \set{0}$, ovvero se e solo se $\phi_\alpha$ non è iniettivo.

    In particolare se $\alpha$ è trascendente vale che $\quot{K[X]}{\ker \phi_\alpha} = K[X]$, dunque $K[\alpha] \isomorph K[X]$.
\end{remark}

\subsection{Polinomio minimo di un elemento algebrico}
Sia $\ext{F}{K}$ un'estensione di campi e sia $\alpha \in F$ un elemento algebrico su $K$, ovvero $\ker \phi_\alpha \neq \set{0}$.
Notiamo che siccome $\ker \phi_\alpha$ non è banale, esso contiene almeno un polinomio diverso dal polinomio nullo, dunque l'insieme dei gradi dei polinomi non nulli nel nucleo di $\phi_\alpha$ è un sottoinsieme di $\N$ non vuoto, perciò ha minimo.

\begin{proposition}\label{prop:caratt_polinomio_minimo}
    Sia $\mu_\alpha \in \ker \phi_\alpha$ un polinomio monico e di grado minimo tra i polinomi di $\ker \phi_\alpha$. Allora valgono le seguenti affermazioni:
    \begin{enumerate}[label={(\roman*)}]
        \item $\mu_\alpha$ è irriducibile in $K[X]$;
        \item $\ker \phi_\alpha = \ideal[\big]{\mu_\alpha(X)}$;
        \item $\mu_\alpha$ è l'unico polinomio monico irriducibile di $K[X]$ che si annulla in $\alpha$.
    \end{enumerate}
\end{proposition}
\begin{proof}
    Dimostriamo le tre affermazioni separensatamente.
    \begin{enumerate}[label={(\roman*)}]
        \item Per ipotesi $\mu_\alpha(\alpha) = 0$. Supponiamo per assurdo che $\mu_\alpha$ sia riducibile in $K[X]$, ovvero che esistano $a,\ b \in K[X]$ con $\deg a,\ \deg b < \deg \mu_\alpha$ tali che $\mu_\alpha(X) = a(X)b(X)$.
        Questo significa che \[
            \mu_\alpha(\alpha) = a(\alpha)b(\alpha) = 0 \in F.
        \] Siccome $F$ è un campo vale la \hyperref[prop:ann_prod_dominio]{legge di annullamento del prodotto}, dunque $a(\alpha) = 0$ oppure $b(\alpha) = 0$. Ma ciò è assurdo in quanto $\mu_\alpha$ è di grado minimo tra i polinomi che si annullano in $\alpha$, mentre $a$ e $b$ hanno grado minore. Dunque $\mu_\alpha$ è irriducibile.
        \item Per definizione l'ideale generato da $\mu_\alpha$ è \[
            \ideal[\big]{\mu_\alpha(X)} = \set{a(X)\mu_\alpha(X) \suchthat a(X) \in K[X].} 
        \] Siccome $\mu_\alpha \in \ker \phi_\alpha$ segue che $\ideal[\big]{\mu_\alpha(X)} \subseteq \ker \phi_\alpha$: infatti per ogni $a(X) \in K[X]$ vale che \begin{align*}
            \phi_\alpha\parens*{a(X)\mu_\alpha(X)} &= \phi_\alpha\parens*{a(X)}\phi_\alpha\parens*{\mu_\alpha(X)}\\
            &= a(\alpha)\mu_\alpha(\alpha)\\
            &= 0.
        \end{align*}
        Sia ora $f \in \ker \phi_\alpha$: dimostriamo che $f \in \ideal[\big]{\mu_\alpha}$. Per il \hyperref[th:divisione_euclidea_KX]{Teorema di Divisione Euclidea} esistono $q, r \in K[X]$ tali che \[
            f(X) = q(X)\mu_\alpha(X) + r(X),
        \] con $r = 0_{K[X]}$ oppure $\deg r < \deg f$.

        Applicando l'omomorfismo di valutazione ad entrambi i membri otteniamo che \[
            0 = f(\alpha) = q(\alpha)\mu_\alpha(\alpha) + r(\alpha) = r(\alpha),
        \] dove la prima uguaglianza viene dal fatto che $f \in \ker \phi_\alpha$, mentre l'ultima viene dal fatto che $\mu_\alpha$ si annulla in $\alpha$.

        Da questo segue che $r$ si annulla in $\alpha$, ma ciò è possibile se e solo se $r = 0_{K[X]}$, in quanto altrimenti sarebbe un polinomio che si annulla in $\alpha$ di grado minore di $\mu_\alpha$. Dunque \[
            f(X) = q(X)\mu_\alpha(X) \in \ideal[\big]{\mu_\alpha}, 
        \] da cui segue che $\ker \phi_\alpha = \mu_\alpha$.
        \item Sia $f \in K[X]$ un polinomio che si annulla in $\alpha$, monico e irriducibile: dimostriamo che $f = \mu_\alpha$.
        
        Siccome per il punto precedente tutti i polinomi che si annullano in $\alpha$ sono nell'ideale generato da $\mu_\alpha$, segue che $f(X) = g(X)\mu_\alpha(X)$ per qualche $g \in K[X]$. Tuttavia se $\deg g \geq 1$ allora $f$ sarebbe riducibile, dunque $\deg g = 0$, ovvero $g(X) = k_0$ per qualche $k_0 \in \invert{K}$. Ma $f$ deve essere monico, e siccome $\mu_\alpha$ è monico segue che $k_0 = 1$, da cui $f = \mu_\alpha$. \qedhere
    \end{enumerate}
\end{proof}

\begin{definition}
    [Polinomio minimo] Sia $\ext{F}{K}$ un'estensione di campi, $\alpha \in F$ algebrico su $K$. L'unico polinomio monico e irriducibile di $K[X]$ che si annulla in $\alpha$ viene detto \emph{polinomio minimo} di $\alpha$ su $K$.
\end{definition}

\begin{example}
    Data l'estensione $\ext{\R}{\Q}$, $\alpha = \sqrt[3]{2} \in \R$, vogliamo trovare il polinomio minimo $\mu_\alpha \in \Q[X]$.

    Sicuramente $X^3 - 2 \in \ideal[\big]{\mu_\alpha(X)}$ in quanto $\parens*{\sqrt[3]{2}}^3 - 2 = 0$, dunque $\mu_\alpha(X) \divides X^3 - 2$. Inoltre $X^3 - 2$ è monico ed irriducibile in $\Q[X]$, in quanto per il Criterio di Eisenstein (con $p = 2$) è irriducibile su $\Z$ e dunque per il Lemma di Gauss lo è su $\Q$. Da ciò segue che $\mu_\alpha(X) = X^3 - 2$.
\end{example}

\begin{proposition}
    Sia $\ext{F}{K}$ un'estensione di campi, $\alpha \in F$ algebrico su $K$ e $\mu_\alpha \in \K[X]$ il polinomio minimo di $\alpha$ su $K$. Allora vale che \[
        K[\alpha] \isomorph \quot{K[X]}{\ideal[\big]{\mu_\alpha}}
    \] e $K[\alpha]$ è un campo.
\end{proposition}
\begin{proof}
    Siccome $\mu_\alpha$ è irriducibile segue direttamente che il quoziente è un campo. Inoltre $K[\alpha]$ è isomorfo al quoziente per la \autoref{prop:KX/ker_valutazione_isomorfo_Kalpha} e per il secondo punto della \autoref{prop:caratt_polinomio_minimo}.
\end{proof}

\begin{remark}
    Sia $K(\alpha)$ l'insieme \[
        K(\alpha) \deq \set{\frac{f(\alpha)}{g(\alpha)} \suchthat f,\ g \in K[X],\ g(\alpha) \neq 0}.    
    \] Se $\alpha$ è algebrico su $K$ possiamo mostrare che $K[\alpha] = K(\alpha)$.

    Infatti innanzitutto esiste un'inclusione canonica \begin{align*}
        K[\alpha] &\to K(\alpha)\\
        f(\alpha) &\mapsto \frac{f(\alpha)}{1}.
    \end{align*} Inoltre per la proposizione precedente $K[\alpha]$ è un campo, dunque per ogni $g \in K[X]$ vale che $\frac{1}{g(\alpha)} \in K[\alpha]$, dunque per ogni $f \in K[X]$ vale che \[
        f(\alpha) \cdot \frac{1}{g(\alpha)} \in K[\alpha],    
    \] da cui $K[\alpha] = K(\alpha)$.
\end{remark}

\begin{definition}
    [Grado dell'estensione]
    Sia $\ext{F}{K}$ un'estensione di campi. Si dice \emph{grado di} $\ext{F}{K}$ il numero naturale \[
        \extdeg{F}{K} \deq \dim_K F.    
    \]
\end{definition}

\begin{proposition}
    Sia $\ext{F}{K}$ un'estensione di campi, $\alpha \in F$. Allora vale che \[
        \dim_K K[\alpha] = \begin{cases}
            +\infty, &\text{se $\alpha$ è trascendente su $K$}\\
            \deg \mu_\alpha, &\text{se $\alpha$ è algebrico su $K$.}
        \end{cases}    
    \]
\end{proposition}
\begin{proof}
    Se $\alpha$ è trascendente allora $K[\alpha] \isomorph K[X]$, dunque \[
        \extdeg{K[\alpha]}{K} = \extdeg{K[X]}{K} = \dim_K K[X] = +\infty.
    \] Invece se $\alpha$ è algebrico su $K$ vale che $K[\alpha] \isomorph \quot{K[X]}{\ideal[\big]{\mu_\alpha(X)}}$, da cui segue che \[
        \extdeg{K[\alpha]}{K} = \extdeg{\quot{K[X]}{\ideal[\big]{\mu_\alpha(X)}}}{K} = \deg \mu_\alpha
    \] per il secondo punto della \autoref{th:caratt_quot_polinomi}. In particolare una $K$-base di $K[\alpha]$ può essere ottenuta sfruttando una $K$-base del quoziente e l'isomorfismo: \[
        \parens*{\eqcl 1,\ \eqcl{x}, \dots,\ \eqcl{x^{n-1}}} \xmapsto{\bar\phi} \parens*{1,\ \alpha, \dots,\ \alpha^{n-1}}.    
    \]
\end{proof}