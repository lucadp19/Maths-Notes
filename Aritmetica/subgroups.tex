\section{Sottogruppi}

\begin{definition}[Sottogruppo]\label{def:sottogruppo}
    Sia $(G, *)$ un gruppo e sia $H \subseteq G$, $H \neq \emptyset$.
    
    Allora $H$ insieme ad un'operazione $*_H$ si dice \emph{sottogruppo} di $(G, *)$ se $(H, *_H)$ è un gruppo.

    Inoltre se l'operazione $*_H$ è l'operazione $*$, ovvero l'operazione del sottogruppo è indotta da $G$, allora si scrive $H \leq G$.
\end{definition}

\begin{proposition}[Condizione necessaria e sufficiente per i sottogruppi]\label{prop:cond_sgr}
    Sia $(G, *)$ un gruppo e sia $H \subseteq G$, $H \neq \emptyset$.

    Allora $H \leq G$ se e solo se \begin{enumerate}[label={(\roman*)}, ref={\theproposition: (\roman*)}]
        \item \label{prop:cond_sgr:op} $*$ è un'operazione su $H$, ovvero \begin{align*}
            a*b \in H &&\forall a, b \in H
        \end{align*}
        \item \label{prop:cond_sgr:inv} ogni elemento di $H$ è invertibile (in $H$), ovvero \begin{align*}
            h\inv \in H &&\forall h \in H
        \end{align*}
    \end{enumerate}
\end{proposition}
\begin{proof}
    Dimostriamo entrambi i versi dell'implicazione.
    \begin{description}
        \item[($\implies$)] Ovvio in quanto se $H \leq G$ allora $H$ è un gruppo.
        \item[($\impliedby$)] Sappiamo che $*$ è associativa poichè lo è in $G$; dobbiamo quindi mostrare solamente che $e_G \in H$.
        
        Per ipotesi $H \neq \emptyset$, dunque esiste un $h \in H$. Per l'ipotesi \ref{prop:cond_sgr:inv} dovrà esistere anche $h\inv \in H$, mentre per l'ipotesi \ref{prop:cond_sgr:op} deve valere che $h * h\inv \in H$. 

        Tuttavia $h * h\inv = e_G$, dunque $e_G \in H$ e quindi $H$ è un sottogruppo indotto da $G$. 
    \end{description}

    Da ciò viene la tesi.
\end{proof}

Un sottogruppo particolarmente importante di qualsiasi gruppo è il \emph{centro del gruppo}:

\begin{definition}
    [Centro di un gruppo] \label{def:centro}
    Sia $(G, *)$ un gruppo. Allora si definisce \emph{centro di $G$} l'insieme \[
        Z(G) \deq \set{x \in G \suchthat g*x = x*g \;\;\forall g \in G}.    
    \]
\end{definition}

Intuitivamente, il centro di un gruppo è l'insieme di tutti gli elementi per cui $*$ diventa commutativa.

Mostriamo che il centro di un gruppo è un sottogruppo tramite la prossima proposizione.

\begin{proposition}
    [Proprietà del centro di un gruppo]
    \label{prop:centro}
    Sia $(G, *)$ un gruppo e sia $Z(G)$ il suo centro.

    Allora vale che \begin{enumerate}[label={(\roman*)}, ref={\theproposition: (\roman*)}]
        \item $Z(G) \leq G$;
        \item $Z(G) = G$ se e solo se $G$ è abeliano.
    \end{enumerate}
\end{proposition}
\begin{proof} Mostriamo le due affermazioni separatamente

    \paragraph{$Z(G)$ è un sottogruppo} 
    Notiamo innanzitutto che $Z(G) \neq \varnothing$ poichè $e_G \in Z(G)$. Per la proposizione \ref{prop:cond_sgr} ci basta mostrare che $*$ è un'operazione su $Z(G)$ e che ogni elemento di $Z(G)$ è invertibile.

    \begin{enumerate}
        [label={(\arabic*)}]
        \item Siano $x, y \in Z(G)$ e mostriamo che $x*y \in Z(G)$, ovvero che per ogni $g \in G$ vale che $g*(x*y) = (x*y)*g$. 
        \begin{align*}
            &g*(x*y) \tag*{(per (G1))}\\
            =\ &(g*x)*y \tag*{(dato che $x \in \Z(G)$)}\\
            =\ &(x*g)*y \tag*{(per (G1))}\\
            =\ &x*(g*y) \tag*{(dato che $x \in \Z(G)$)} \\
            =\ &x*(y*g) \tag*{(per (G1))}\\
            =\ &(x*y)*g.
        \end{align*}
        \item Sia $x \in Z(G)$, mostriamo che $x\inv \in Z(G)$.
        
        Per ipotesi \begin{align*}
            &g*x = x*g \tag*{(moltiplico a sinistra per $x\inv$)} \\
            \iff\ &x\inv * g*x = x\inv * x*g\tag*{(dato che $x\inv * x = e$)} \\
            \iff\ &x\inv * g*x = g \tag*{(moltiplico a destra per $x\inv$)} \\
            \iff\ &x\inv * g*x*x\inv = g*x\inv \tag*{(dato che $x\inv * x = e$)} \\
            \iff\ &x\inv * g = g*x\inv
        \end{align*}
        da cui $x\inv \in Z(G)$.
    \end{enumerate}

    Per la proposizione \ref{prop:cond_sgr} segue che $Z(G) \leq G$.

    \paragraph{$Z(G) = G$ se e solo se $G$ abeliano} Dimostriamo entrambi i versi dell'implicazione.
    \begin{description}
        \item[($\implies$)] Ovvia: $Z(G)$ è un gruppo abeliano, dunque se $G = Z(G)$ allora $G$ è abeliano.
        \item[($\impliedby$)]  Ovvia: $Z(G)$ è l'insieme di tutti gli elementi di $G$ per cui $*$ commuta, ma se $G$ è abeliano questi sono tutti gli elementi di $G$, ovvero $Z(G) = G$.  \qedhere
    \end{description}
\end{proof}

Un altro esempio è dato dai sottogruppi di $(\Z, +)$.

\begin{definition}
    [Insieme dei multipli interi]
    Sia $n \in \Z$. Allora chiamo $n\Z$ l'insieme dei multipli interi di $n$ \[
         n\Z \deq \set{nk \suchthat k \in \Z}.
    \]
\end{definition}

È semplice verificare che $(n\Z, +)$ è un gruppo per ogni $n \in \Z$. In particolare vale la seguente proposizione.

\begin{proposition}
    [$n\Z$ è sottogruppo di $\Z$] \label{prop:nZ_sgr_Z}
    Consideriamo il gruppo $(\Z, +)$.
    Per ogni $n \in \Z$ vale che $n\Z \leq \Z$.
\end{proposition}
\begin{proof}
    Innanzitutto notiamo che $n\Z \neq \varnothing$ in quanto $n \cdot 0 = 0 \in n\Z$. 
    
    Mostriamo ora che $n\Z \leq \Z$.
    \begin{enumerate}[label={(\arabic*)}]
        \item Siano $x, y \in n\Z$ e mostriamo che $x+y \in \Z$. 
        
        Per definizione di $n\Z$ esisteranno $k, h \in \Z$ tali che $x = nk$, $y = nh$.
        
        Allora $x + y = nk + nh = n(k + h) \in n\Z$ in quanto $k + h \in \Z$.
        \item Sia $x \in n\Z$, mostriamo che $-x \in n\Z$.
        
        Per definizione di $n\Z$ esisterà $k \in \Z$ tale che $x = nk$.

        Allora affermo che $-x = n(-k) \in n\Z$. Infatti \[
            x + (-x) = nk + n(-k) = n(k - k) = 0    
        \] che è l'elemento neutro di $\Z$.
    \end{enumerate}

    Dunque per la proposizione \ref{prop:cond_sgr} segue che $n\Z \leq \Z$, ovvero la tesi.
\end{proof}

\begin{corollary}\label{cor:nZ_mZ}
    Siano $n, m \in \Z$. Allora valgono i due fatti seguenti:
    \begin{enumerate}[label={(\roman*)}, ref={\thecorollary: (\roman*)}]
        \item \label{cor:nZ_mZ:subset} $n\Z \subseteq m\Z \iff m \divides n$;
        \item \label{cor:nZ_mZ:eq} $n\Z = m\Z \iff n = \pm m$.
    \end{enumerate}
\end{corollary}
\begin{proof} Dimostriamo le due affermazioni separatamente.

    \paragraph{Parte 1.} Dimostriamo entrambi i versi dell'implicazione.

    \begin{description}
        \item[($\implies$)] Supponiamo $n\Z \subseteq m\Z$, ovvero che per ogni $x \in n\Z$ allora $x \in m\Z$.
        
        Sia $k \in \Z$ tale che $\mcd{k}{m} = 1$ e sia $x = nk$.
        
        Per definizione di $n\Z$ segue che $x \in n\Z$, dunque $x \in m\Z$. Allora dovrà esistere $h \in \Z$ tale che \begin{align*}
            &x = mh \\
            \iff\ &nk = mh \\
            \implies\ &m \divides nk
            \intertext{Ma abbiamo scelto $k$ tale che $\mcd{k}{m} = 1$, dunque}
            \implies\ &m \divides n.
        \end{align*}
        \item[($\impliedby$)] Supponiamo che $m \divides n$, ovvero $n = mh$ per qualche $h \in \Z$. Allora \[
            n\Z = (mh)\Z \subseteq m\Z    
        \] in quanto i multipli di $mh$ sono necessariamente anche multipli di $m$.
    \end{description}

    \paragraph{Parte 2.} Se $n\Z = m\Z$ allora vale che $n\Z \subseteq m\Z$ e $m\Z \subseteq n\Z$, dunque per \ref{cor:nZ_mZ:subset} $m \divides n$ e $n \divides m$, ovvero $n$ e $m$ sono uguali a meno del segno.
\end{proof}

\begin{proposition}
    [Intersezione di sottogruppi è un sottogruppo]
    Sia $(G, \cdot)$ un gruppo e siano $H, K \leq G$.

    Allora $H \inters K \leq G$.
\end{proposition}
\begin{proof}
    Innanzitutto dato che $e_G \in H$, $e_G \in K$ segue che $e_G \in H \inters K$, che quindi non può essere vuoto.

    Per la proposizione \ref{prop:cond_sgr} è sufficiente dimostrare che $H \inters K$ è chiuso rispetto all'operazione $\cdot$ e che ogni elemento è invertibile.

    \begin{enumerate}[label={(\roman*)}]
        \item Siano $x, y \in H \inters K$; mostriamo che $xy \in H \inters K$.
        
        Per definizione di intersezione sappiamo che $x, y \in H$ e $x, y \in K$. Dato che $H$ è un gruppo varrà che $xy \in H$; per lo stesso motivo $xy \in K$; dunque $xy \in H \inters K$.

        \item Sia $x \in H \inters K$; mostriamo che $x\inv \in H \inters K$.
        
        Per definizione di intersezione sappiamo che $x \in H$ e $x \in K$. Dato che $H$ è un gruppo varrà che $x\inv \in H$; per lo stesso motivo $x\inv \in K$; dunque $x\inv \in H \inters K$.
    \end{enumerate}

    Dunque per la proposizione \ref{prop:cond_sgr} segue che $H \inters K \leq G$.
\end{proof}